\documentclass{beamer}
\beamertemplatenavigationsymbolsempty % remove the default navigation symbols
% default colors
% used in standalone pictures and when no university is selected
\definecolor{black}{HTML}{000000}
\definecolor{green}{HTML}{33BB33}
\definecolor{red}{HTML}{BB3333}
\definecolor{orange}{HTML}{BB6600}
\definecolor{blue}{HTML}{3333BB}
\definecolor{uulmaccent}{HTML}{999999}

\definecolor{uulmlogoblue}{named}{blue}
\definecolor{uulmblue}{named}{blue}

\ifdarkmode
	\colorlet{green}{green!85!white}
	\colorlet{red}{red!85!white}
	\colorlet{orange}{orange!85!white}
	\colorlet{blue}{blue!85!white}
	\setbeamercolor{section in toc shaded}{fg=black}
	\setbeamertemplate{section in toc shaded}[default][50]
\fi


\makeatletter
    \setlength\beamer@paperwidth{120mm}%
    \setlength\beamer@paperheight{80mm}%
    \geometry{papersize={\beamer@paperwidth,\beamer@paperheight}}
\makeatother
\setbeamersize{text margin left=0mm,text margin right=0mm}

\usepackage{tikz}
\usepackage{pgfcalendar} % convert the dates to Julian integers
\usepackage{pgfplots}
\usepackage{pgfplotstable} % manipulate the data file/table
\usepgfplotslibrary{dateplot}
\pgfplotsset{
	width=.95\linewidth,
	height=.68\linewidth,
	date coordinates in=x,
	xmin={2003-01-01},
	xmax={2026-01-01},
	xlabel=Year,
	xlabel near ticks,
	ylabel near ticks,
	xtick={2005-01-01,2010-01-01,2015-01-01,2020-01-01,2025-01-01},
	xticklabel={\year},
	/pgf/number format/.cd,fixed,use comma,1000 sep={.},
}

% the following trick enables a trendline for a date axis
% adapted from https://tex.stackexchange.com/questions/367339/linear-regression-with-dates-on-x-axis-in-pgfplots
\pgfplotsset{compat=1.14}
\newcommand{\readtablewithtrendline}[2]{
	\pgfplotstableread[col sep=comma]{#1}#2
	% add new column with Julian integer numbers
	    % therefore a counter is needed
	    \newcount\julianday
	\pgfplotstablecreatecol[
	    create col/assign/.code={
	        % convert the number of the current row and save it to `\julianday'
	        \pgfcalendardatetojulian{\thisrow{release_date}}{\julianday}
	        % then give the entry of `\julianday' to `\entry' which is then
	        % given to the current cell
	        \edef\entry{\the\julianday}
	        \pgfkeyslet{/pgfplots/table/create col/next content}\entry
	    },
	]{JulianDay}{#2}
	% because the `dateplot' library shifts automatically all dates to 0 using
	% the first found coordinate we can't use the created `JulianDay' data
	% directly for `linear regression', but have to do the same first with
	% the data
	    % get the first coordinate of the column ...
	    \pgfplotstablegetelem{0}{JulianDay}\of{#2}
	    % ... and store it in `\xmin'
	    \pgfmathtruncatemacro{\xmin}{\pgfplotsretval}
	% now create another column with the shifted values
	\pgfplotstablecreatecol[
	    expr={\thisrow{JulianDay}-\xmin},
	]{JulianDayMod}{#2}
}
\readtablewithtrendline{linux-sloc.csv}{\loc}
\readtablewithtrendline{linux-features.csv}{\features}
\readtablewithtrendline{linux-products.csv}{\products}

\begin{document}

	% lines of code (SLOC) of Linux kernel
	\begin{frame}[plain]
		\centering
		\begin{tikzpicture}
			\begin{axis}[
				width=.89\linewidth,
				ymin=0,
				ymax=29000000,
				ylabel=Lines of Source Code (SLOC),
				scaled y ticks=false]
				\addplot table [col sep=comma,x=release_date,y=sloc] {\loc};
				\addplot table [mark=none,col sep=comma,x=release_date,y={create col/linear regression={x=JulianDayMod,y=sloc}}] {\loc};
				\node [coordinate,pin=above:{v2.6.12}] at (axis cs:2005-06-18,4361591) {};
				\node [coordinate,pin=above:{v3.0}] at (axis cs:2011-07-21,9734588) {};
				\node [coordinate,pin=above:{v4.0}] at (axis cs:2015-04-12,12852861) {};
				\node [coordinate,pin=above:{v5.0}] at (axis cs:2019-03-03,17847774) {};
				\node [coordinate,pin=above:{v5.19}] at (axis cs:2022-07-31,23970553) {};
			\end{axis}
		\end{tikzpicture}
	\end{frame}

	% number of features of Linux kernel
	\begin{frame}[plain]
		\centering
		\begin{tikzpicture}
			\begin{axis}[
				ylabel=Number of Features,
				ymin=0,
				ymax=21000,
				scaled y ticks=false]
				\addplot table [only marks,col sep=comma,x=release_date,y=features] {\features};
				\addplot table [mark=none,col sep=comma,x=release_date,y={create col/linear regression={x=JulianDayMod,y=features}}] {\features};
				\node [coordinate,pin=above:{v2.6.12}] at (axis cs:2005-06-18,3454) {};
				\node [coordinate,pin=above:{v3.0}] at (axis cs:2011-07-21,8517) {};
				\node [coordinate,pin=above:{v4.0}] at (axis cs:2015-04-12,11292) {};
				\node [coordinate,pin=above:{v4.18}] at (axis cs:2018-08-12,14049) {};
			\end{axis}
		\end{tikzpicture}
	\end{frame}

	% number of products of Linux kernel
	\begin{frame}[plain]
		\centering
		\begin{tikzpicture}
			\begin{axis}[
				ylabel=Number of Products,
				ymax=8500,
				ytick={1000,2000,3000,4000,5000,6000,7000,8000},
				yticklabels={$10^{1000}$,$10^{2000}$,$10^{3000}$,$10^{4000}$,$10^{5000}$,$10^{6000}$,$10^{7000}$,$10^{8000}$},
				scaled y ticks=false]
				\addplot table [only marks,col sep=comma,x=release_date,y=products_log10] {\products};
				\addplot table [mark=none,col sep=comma,x=release_date,y={create col/linear regression={x=JulianDayMod,y=products_log10}}] {\products};
				\node [coordinate,pin=above:{v2.6.12}] at (axis cs:2005-06-18,940.89) {};
				\node [coordinate,pin=above:{v3.0}] at (axis cs:2011-07-21,3069.90) {};
				\node [coordinate,pin=above:{v4.0}] at (axis cs:2015-04-12,4289.37) {};
				\node [coordinate,pin=above:{v4.8 ($10^{5019}$)}] at (axis cs:2016-09-25,5019.45) {};
			\end{axis}
		\end{tikzpicture}
	\end{frame}

	% model counting time of Linux kernel
	\begin{frame}[plain]
		\centering
		\begin{tikzpicture}
			\begin{semilogyaxis}[
				width=.93\linewidth,
				ylabel=Time for Counting (logarithmic),
				ymax=30000000000,
				ytick={1,1000,60000,3600000,86400000,2419200000,31536000000},
				yticklabels={1 ms,1 sec,1 min,1 hour,1 day,4 weeks,1 year},
				]
				\addplot table [only marks,col sep=comma,x=release_date,y=time_in_ms] {\products};
				\addplot table [red,mark=none,col sep=comma,x=release_date,y={create col/linear regression={x=JulianDayMod,y=time_in_ms}}] {\products};
				\node [coordinate,pin=above:{v2.6.12}] at (axis cs:2005-06-18,331) {};
				\node [coordinate,pin=above:{v3.0}] at (axis cs:2011-07-21,17698) {};
				\node [coordinate,pin=above:{v4.0}] at (axis cs:2015-04-12,748797) {};
				\node [coordinate,pin=above:{v4.8 (46 min)}] at (axis cs:2016-09-25,2740432) {};
			\end{semilogyaxis}
		\end{tikzpicture}
	\end{frame}

\end{document}
