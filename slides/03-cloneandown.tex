\documentclass[
	aspectratio=169, % default is 43
	8pt, % font size, default is 11pt
	handout, % < do not remove this comment, it is used by the Makefile >
]{beamer}
\def\university{} % < do not remove this comment, it is used by the Makefile >

\documentclass[
	aspectratio=169, % default is 43
	8pt, % font size, default is 11pt
	handout, % handout mode without animations, comment out to add animations
]{beamer}

\usepackage{../template/beamerthemeuulm} % use the inofficial uulm beamer theme
\setfaculty{infIngPsy} % set the color scheme for your faculty here [med/infIngPsy/math/nat]

% requires symbolic links
% git clone git@github.com:SoftVarE-Group/SlideTemplate.git C:\Users\...\SlideTemplate
% mklink /J template C:\Users\...\SlideTemplate
% git clone git@spgit.informatik.uni-ulm.de:thuem/slides.git C:\Users\...\ThomasSlides
% mklink /J thomasslides C:\Users\...\ThomasSlides
\graphicspath{{../template/pics/logos}{../template/pics/nature}{../template/pics/uulm}{../thomasslides/}{../pics/}}

%\usepackage[ngerman]{babel} % use this line for slides in German
%\recordingtrue % special recording mode for use with a greenscreen, gives you space to show yourself in a layer in front of the slides, has no effect in the handout mode

\title{Software Product Lines} % short title is used for the slide footer but optional

%
%
%% IMPORTED PACKAGES
%
%\usepackage{adjustbox} % used for partofpage
%\usepackage{tcolorbox} % used for mydefinition, mynote, myexample
\usepackage{multicol} % used temporarily for the lecture overview
%\usepackage{mathtools} % required for absolute value in modeling lecture
%
%% SLIDE TEMPLATE
%
%\beamertemplatenavigationsymbolsempty 
%
%% COMMANDS TO LAYOUT AND ANNIMATE SLIDES
%
\newcommand{\lessonslearned}[3]{
	\subsection{Summary}
	\begin{frame}{\insertsection -- \insertsubsection}
		\leftorright{
			\mydefinition{Lessons Learned}{
				\begin{itemize}
					#1
				\end{itemize}
			}
			\mynote{Further Reading}{
				\small % references take space, can be a little smaller
				\begin{itemize}
					#2
				\end{itemize}
			}
		}{
			\myexample{Practice}{
				#3
			}
		}
	\end{frame}
}

\renewcommand{\lectureoverview}{
%	\section*{Overview}
%	\subsection*{Overview}
	\begin{frame}{\insertsubtitle}
		\begin{multicols}{2}
			\tableofcontents
		\end{multicols}
	\end{frame}
}

%
%\newcommand{\onlyleft}[1]{
%	\halfpage{#1}
%}
%
%\newcommand{\onlyright}[1]{
%	~\hfill
%	\halfpage{#1}
%}
%
%\newcommand{\leftorright}[2]{
%	\uncover<1>{\halfpage{#1}}
%	\hfill
%	\uncover<3->{\halfpage{#2}}
%}
%
%\newcommand{\rightorleft}[2]{
%	\uncover<3->{\halfpage{#1}}
%	\hfill
%	\uncover<1>{\halfpage{#2}}
%}
%
%\newcommand{\leftthenright}[2]{
%	\halfpage{#1}
%	\hfill\pause
%	\halfpage{#2}
%}
%
%\newcommand{\leftandright}[2]{
%	\halfpage{#1}
%	\hfill
%	\halfpage{#2}
%}
%
%\newcommand{\leftmiddleandright}[3]{
%	\thirdpage{#1}
%	\hfill
%	\thirdpage{#2}
%	\hfill
%	\thirdpage{#3}
%}
%
%\newcommand{\leftmiddleorright}[3]{
%	\uncover<1>{\thirdpage{#1}}
%	\hfill
%	\uncover<3>{\thirdpage{#2}}
%	\hfill
%	\uncover<5->{\thirdpage{#3}}
%}
%
%\newcommand{\halfpage}[1]{\partofpage{48}{#1}}
%
%\newcommand{\thirdpage}[1]{\partofpage{31}{#1}}
%
%\newcommand{\partofpage}[2]{
%	\adjustbox{valign=t}{\begin{minipage}{0.#1\textwidth}
%			\begin{flushleft}
%				#2
%			\end{flushleft}
%	\end{minipage}}
%}
%
%\newcommand{\mydefinition}[2]{
%	\begin{tcolorbox}[title=#1,colback=orange!10,colframe=orange!30,coltitle=black,fonttitle=\bfseries,left=1mm,right=1mm,top=1mm,bottom=1mm]
%		\begin{flushleft}
%			#2
%		\end{flushleft}
%	\end{tcolorbox}
%}
%
%\newcommand{\mydefinitiontight}[2]{
%	\begin{tcolorbox}[title=#1,colback=white,colframe=orange!30,coltitle=black,fonttitle=\bfseries,left=0mm,right=0mm,top=0mm,bottom=0mm]
%		\begin{flushleft}
%			#2
%		\end{flushleft}
%	\end{tcolorbox}
%}
%
%\newcommand{\mynote}[2]{
%	\begin{tcolorbox}[title=#1,colback=red!10,colframe=red!30,coltitle=black,fonttitle=\bfseries,left=1mm,right=1mm,top=1mm,bottom=1mm]
%		\begin{flushleft}
%			#2
%		\end{flushleft}
%	\end{tcolorbox}
%}
%
%\newcommand{\myexample}[2]{
%	\begin{tcolorbox}[title=#1,colback=blue!10,colframe=blue!30,coltitle=black,fonttitle=\bfseries,left=1mm,right=1mm,top=1mm,bottom=1mm]
%		\begin{flushleft}
%			#2
%		\end{flushleft}
%	\end{tcolorbox}
%}
%
%\newcommand{\myexampletight}[2]{
%	\begin{tcolorbox}[title=#1,colback=white,colframe=blue!30,coltitle=black,fonttitle=\bfseries,left=0mm,right=0mm,top=0mm,bottom=0mm]
%		\begin{flushleft}
%			#2
%		\end{flushleft}
%	\end{tcolorbox}
%}

\subtitle{3. Compile-Time Variability with Clone-and-Own}
\author{Timo Kehrer, Thomas Thüm, Elias Kuiter}

\ifuniversity{ulm}{\setpicture[200]{apr21-south}}
\ifuniversity{magdeburg}{\setpicture[70]{ovgu-autumn2}\setcopyright{Photo: Hannah Theile (OVGU)}}

\begin{document}

% TITLE SLIDE

\maketitle

% SLIDE TEMPLATE

%\setbeamercolor{title}{fg=black}
%\setbeamercolor{frametitle}{fg=black}
\setbeamertemplate{frametitle}{{\huge~\\\insertsubsection~\insertframetitle}}
\setbeamertemplate{footline}[text line]{\parbox{\linewidth}{\vspace*{-10pt}\hspace{0pt}%
	\insertshortauthor\phantom{g\insertpagenumber}%
	\hfill%
	\inserttitle%
	\ifx \insertsubtitle \empty \else \ -- \insertsubtitle\fi%
	\ifx \insertsectionhead \empty \else \ -- \insertsectionhead\fi%
	\hfill%
	\phantom{g\insertshortauthor}\insertpagenumber%
}}
%\defbeamertemplate{footline}{\begin{beamercolorbox}[sep=1em]{author in head/foot}\insertshortauthor\hfill\insertsection\hfill\insertframenumber\end{beamercolorbox}}
%\defbeamertemplate*{footline}{mytheme}{\begin{beamercolorbox}[sep=1em]{author in head/foot}\insertshortauthor\hfill\insertsection\hfill\insertframenumber\end{beamercolorbox}}

% OVERVIEW SLIDES

\newcommand{\overview}{
	\section*{Overview}
	\subsection*{Overview}
	\begin{frame}{-- \insertsubtitle}
		\begin{multicols}{3}
			\tableofcontents
		\end{multicols}
	
		\begin{flushright}
			\footnotesize
			Author: \insertauthor
			
			Date: \insertdate
		\end{flushright}
	\end{frame}
}
% temporarily added slide to have a lecture overview 
\overview

% temporarily removed
%\begin{frame}{Lecture Overview -- \insertsubtitle}
%	\tableofcontents[hideallsubsections]
%\end{frame}

\AtBeginSection[]{%
	\begin{frame}{Lecture Overview -- \insertsubtitle}
		\tableofcontents[currentsection,hideothersubsections]
	\end{frame}
}

\newcommand{\sectionend}{\addtocontents{toc}{\newpage}}


\begin{frame}{\inserttitle}
	\lectureseriesoverview[3]
\end{frame}

\section{Compile-Time Variability and Clone-and-Own}

\subsection{Problems with Runtime Variability}

\begin{frame}<3>{Recap: How to Implement Software Product Lines?}
	\frameImplementSPLs
\end{frame}

\begin{frame}<1>{Recap: Variability and Binding Times}
	\frameVariabilityAndBindingTimes
\end{frame}

\begin{frame}<1-2>[label=PrinciplesAndProblemsOfRuntimeVariability]
	\frametitle<1-2>{Recap: Runtime Variability}
	\frametitle<3>{\myframetitle}
	\begin{fancycolumns}[t]
		\begin{note}{Basic Principles}
			{\bf Variability with Configuration Options:}
			\begin{itemize}
				\item Conditional statements controlled by configuration options
				\item Global variables vs.\ method parameters
			\end{itemize}
			{\bf Object-Orientation and Design Patterns:}
			\begin{itemize}
				\item Template Method
				\item Abstract Factory
				\item Decorator
			\end{itemize}	
		\end{note}
	\nextcolumn
		\frameRuntimeVariabilityProblems
	\end{fancycolumns}	
\end{frame}

\subsection{Compile-Time Variability}

\begin{frame}{\myframetitle}
	\begin{fancycolumns}[widths={50},animation=none]
		\frameRuntimeVariabilityProblems		
	\nextcolumn
		\uncover<2->{\begin{definition}{Compile-Time Variability\mysource{\fospl\mypage{49}}}
			\mycite{Compile-time variability is decided before or at compile time.}
		\end{definition}}
		\uncover<3->{\begin{note}{}
			{\bf Goals:}
			\begin{itemize}
				\item Only required source code is compiled
				\item Smaller and highly optimized variants
			\end{itemize}				
			{\bf Challenge:}
			\begin{itemize}
				\item How to implement options and alternatives (i.e.,~variability)?
			\end{itemize}	
		\end{note}}
		\uncover<4->{\begin{note}{In this Lecture}
			Simple concepts and techniques for a few variants
		\end{note}}
	\end{fancycolumns}	
\end{frame}

\subsection{Ad-Hoc Clone-and-Own}

\begin{frame}{Ad-Hoc Clone-and-Own}
	\begin{fancycolumns}[widths={47},animation=none]
		\begin{note}{Clone-and-Own\mysource{\seiwhitepaperspl\mypage{7}}}
			\mycite{Suppose you are developing a new system that seems very similar to one you have built before. You borrow what you can from your previous effort, modify it as necessary, add whatever it takes, and field the product, which then assumes its own maintenance trajectory separate from that of the first product. What you have done is what is called \emph{clone and own}. You certainly have taken economic advantage of previous work; you have reused a part of another system. But now you have two entirely different systems, not two systems built from the same base. This is again \emph{ad hoc reuse}.}
		\end{note}
	\nextcolumn
		\begin{example}{Cloning Whole Products}
			~\hfill\pic[width=.2\linewidth]{130}\hfill\pic[width=.2\linewidth]{230}\hfill~
		\end{example}
		\pause
		\begin{definition}{Clone-and-Own} % TODO source missing
			\begin{itemize}
				\item New variants of a software system are created by copying and adapting an existing variant
				\item Afterwards, cloned variants evolve independently of each other
			\end{itemize}	
		\end{definition}
	\end{fancycolumns}	
\end{frame}

\subsection{Variability with Clone-and-Own}

\begin{frame}[fragile]{Example for Ad-Hoc Clone-and-Own}
	\myframeicon{\pic[scale=0.25,page=6]{graphs}}%
	\begin{fancycolumns}[b,columns=3,widths={43,32},animation=none]
\begin{codetight}{}
class Graph {
	List nodes = new ArrayList();
	List edges = new ArrayList();

	Edge add(Node n, Node m) {
		Edge e = new Edge(n, m);
		e.weight = new Weight();
		nodes.add(n); nodes.add(m); edges.add(e);
		return e;
	}
	Edge add(Node n, Node m, Weight w) {
		Edge e = new Edge(n, m);
		e.weight = w;
		nodes.add(n); nodes.add(m); edges.add(e);
		return e;
	}
	void print() {
		for (int i = 0; i < edges.size(); i++) {
			((Edge) edges.get(i)).print();
		}
	}
}
\end{codetight}
	\nextcolumn
		\begin{example}{}
			initial graph implementation providing weighted graphs
		\end{example}
\begin{codetight}{}
class Edge {
	Node a, b;
	Weight weight = new Weight();

	Edge(Node a, Node b) {
		this.a = a; this.b = b;
	}
	void print() {
		a.print(); b.print();
		weight.print();
	}
}
\end{codetight}
\begin{codetight}{}
class Weight {
	void print() {...}
}
\end{codetight}
	\nextcolumn
\begin{codetight}{}
public class Node {
	int id = 0;

	void print() {
		System.out.print(id);
	}
}
\end{codetight}
	\end{fancycolumns}
\end{frame}

\begin{frame}[fragile]{Alice's Clone: Unweighted Graphs}
	\myframeicon{\picDark[scale=0.25]{alice} \pic[scale=0.25,page=2]{graphs}}%
	\begin{fancycolumns}[b,columns=3,widths={43,32},animation=none]
\begin{codetight}{}
class Graph {
	List nodes = new ArrayList();
	List edges = new ArrayList();

	Edge add(Node n, Node m) {
		Edge e = new Edge(n, m);
		@|e.weight = new Weight();|@
		nodes.add(n); nodes.add(m); edges.add(e);
		return e;
	}
	@|Edge add(Node n, Node m, Weight w) {
		Edge e = new Edge(n, m);
		e.weight = w;
		nodes.add(n); nodes.add(m); edges.add(e);
		return e;
	}|@
	void print() {
		for (int i = 0; i < edges.size(); i++) {
			((Edge) edges.get(i)).print();
		}
	}
}
\end{codetight}
	\nextcolumn
		\begin{example}{}
			Alice works with unweighted graphs: she copies and adapts the basic implementation
		\end{example}
\begin{codetight}{}
class Edge {
	Node a, b;
	@|Weight weight = new Weight();|@

	Edge(Node a, Node b) {
		this.a = a; this.b = b;
	}
	void print() {
		a.print(); b.print();
		@|weight.print();|@
	}
}
\end{codetight}
\begin{codetight}{}
@|class Weight {
	void print() {...}
}|@
\end{codetight}
	\nextcolumn
\begin{codetight}{}
public class Node {
	int id = 0;

	void print() {
		System.out.print(id);
	}
}
\end{codetight}
	\end{fancycolumns}
\end{frame}

\begin{frame}[fragile]{Alice's Clone: Unweighted Graphs}
	\myframeicon{\picDark[scale=0.25]{alice} \pic[scale=0.25,page=2]{graphs}}%
	\begin{fancycolumns}[b,columns=3,widths={43,32},animation=none]
\begin{codetight}{}
class Graph {
	List nodes = new ArrayList();
	List edges = new ArrayList();

	Edge add(Node n, Node m) {
		Edge e = new Edge(n, m);
		nodes.add(n); nodes.add(m); edges.add(e);
		return e;
	}
	void print() {
		for (int i = 0; i < edges.size(); i++) {
			((Edge) edges.get(i)).print();
		}
	}
}
\end{codetight}
	\nextcolumn
		\begin{example}{}
			Alice works with unweighted graphs: she copies and adapts the basic implementation
		\end{example}
\begin{codetight}{}
class Edge {
	Node a, b;

	Edge(Node a, Node b) {
		this.a = a; this.b = b;
	}
	void print() {
		a.print(); b.print();
	}
}
\end{codetight}
	\nextcolumn
\begin{codetight}{}
public class Node {
	int id = 0;

	void print() {
		System.out.print(id);
	}
}
\end{codetight}
	\end{fancycolumns}
\end{frame}

\begin{frame}[fragile]{Bob's Clone: Colored Graphs}
	\myframeicon{\picDark[scale=0.25]{bob} \pic[scale=0.25,page=12]{graphs}}%
	\begin{fancycolumns}[b,columns=3,widths={43,26},animation=none]
\begin{codetight}{}
public class Graph {
	List nodes = new ArrayList();
	List edges = new ArrayList();

	Edge add(Node n, Node m) {
		Edge e = new Edge(n, m);
		nodes.add(n); nodes.add(m); edges.add(e);
		return e;
	}
	void print() {
		for (int i = 0; i < edges.size(); i++) {
			((Edge) edges.get(i)).print();
		}
	}
}
\end{codetight}
	\nextcolumn
		\begin{example}{}
			Bob works with colored graphs: he is a colleague of Alice and knows her variant, so he copies and adapts Alice's variant
		\end{example}
\begin{codetight}{}
class Edge {
	Node a, b;

	Edge(Node a, Node b) {
		this.a = a; this.b = b;
	}
	void print() {
		a.print(); b.print();
	}
}
\end{codetight}
	\nextcolumn
\begin{codetight}{}
public class Node {
	int id = 0;
	~Color color = new Color();~

	void print() {
		~Color.setDisplayColor(color);~
		System.out.print(id);
	}
}
\end{codetight}
\begin{codetight}{}
~public class Color {
	static void setDisplayColor(Color c) {...}
}~
\end{codetight}
	\end{fancycolumns}
\end{frame}

\subsection{Problems of Clone-and-Own}

\begin{frame}{Why is Clone-and-Own Problematic?}
	~\hfill
	\pic[scale=0.25,page=6]{graphs}
	\hfill
	\picDark[scale=0.25]{alice}
	\pic[scale=0.25,page=2]{graphs}
	\hfill
	\picDark[scale=0.25]{bob}
	\pic[scale=0.25,page=12]{graphs}
	\hfill~
\end{frame}

\begin{frame}{Clone-and-Own Problems: Feature Combinations}
	\myframeicon{\picDark[scale=0.25]{eve} \pic[scale=0.25,page=16]{graphs}}%
	\vspace{5mm}

	\begin{fancycolumns}[widths={42}]
		\begin{example}{}
			Eve has a new requirement:
			she wants to work with graphs which are both colored and weighted
		\end{example}
	\nextcolumn
		\begin{note}{}
			\begin{itemize}
				\item Where to start from?
				\item Does Eve know about Bob's and Alice's variants?
				\item If so, how to avoid repeating the work that has been already done by Alice and Bob, respectively?
			\end{itemize}
		\end{note}
	\end{fancycolumns}

	~

	\uncover<2->{
		~\hfill
		\pic[scale=0.25,page=6]{graphs}
		\hfill
		\picDark[scale=0.25]{alice}
		\pic[scale=0.25,page=2]{graphs}
		\hfill
		\picDark[scale=0.25]{bob}
		\pic[scale=0.25,page=12]{graphs}
		\hfill~
	}
\end{frame}

\begin{frame}[fragile]{Clone-and-Own Problems: Evolution \& Maintenance}
	\myframeicon{\pic[scale=0.2,page=6]{graphs}\\[-9mm]\pic[scale=0.2,page=26]{graphs}~~~}%
	\begin{fancycolumns}[b,widths={43}]
		\begin{example}{}
			Maintainers of the initial variant refactor the code of the basic implementation
		\end{example}
\begin{codetight}{}
public class Graph {
	...
	Edge add(Node n, Node m) {
		Edge e = new Edge(n, m);
		nodes.add(n); nodes.add(m); edges.add(e);
		e.weight = new Weight();
		return e;
	}
	Edge add(Node n, Node m, Weight w) {
		@|Edge e = new Edge(n, m);|@
		@|nodes.add(n); nodes.add(m); edges.add(e);|@
		?Edge e = add(n, m);?
		e.weight = w;
		return e;
	}
	...
}
\end{codetight} % TODO order of statements not consistent with prior examples: shall we revert the changed order in prior examples?
	\nextcolumn
		\begin{note}{}
			\begin{itemize}
				\item Who informs Alice, Bob and Eve about the improvement?
				\item How do they know whether the improvement is relevant for them?
				\item If so, how to propagate the improvement to their variant?
			\end{itemize}
		\end{note}

		~

		\picDark[scale=0.15]{alice}
		\pic[scale=0.15,page=2]{graphs}
		\hfill
		\picDark[scale=0.15]{bob}
		\pic[scale=0.15,page=12]{graphs}
		\hfill
		\picDark[scale=0.15]{eve}
		\pic[scale=0.15,page=16]{graphs}
	\end{fancycolumns}
\end{frame}

\subsection{Software Clones}

\begin{frame}<2>{Recap: \myframetitle\ \mytitlesource{\lectureintroduction}}
	\frameSoftwareClones
\end{frame}

\begin{frame}[b]{Types of \myframetitle}
	\begin{fancycolumns}[b,height=\textheightwithtitle]
		\begin{definition}{Types of Software Clones\mysource{\softwareclonedetection\mypage{1167}}}
			\begin{itemize}
			\item Type 1: identical except whitespaces and comments
			\item Type 2: syntactically similar (e.g., changed identifiers, \ldots)
			\item Type 3: copied with modifications (e.g., inserted or removed statements)
			\item Type 4: similar functionality without textual similarities
			\end{itemize}
		\end{definition}
		\begin{example}{Cloning Parts of Software}
			~\hfill\pic[width=.2\linewidth]{pants-blue}\hfill\pic[width=.2\linewidth]{pants-red}\hfill~
		\end{example}
	\nextcolumn
		\begin{note}{Relevant Types for Clone-and-\emph{Own}?}
			\begin{itemize}
			\item Type 1: may happen if clones diverge and comments need to reflect actual changes
			\item Type 2: may happen if clones diverge and identifier names are not appropriate anymore
			\item Type 3: actually necessary for clone-and-own
			\item Type 4: may happen if same functionality is implemented again (simply unknown or merge/cherrypick infeasible) \mysource{see \lecturemodeling}
			\end{itemize}
			Every difference is an obstacle for future maintenance (cf.\ merge and cherrypick)
		\end{note}
		\begin{example}{Cloning Whole Products (Clone-and-Own)}
			~\hfill\pic[width=.2\linewidth]{130}\hfill\pic[width=.2\linewidth]{230}\hfill~
		\end{example}
	\end{fancycolumns}
\end{frame}

\subsection{Discussion}

\begin{frame}{Discussion of Clone-and-Own}
	\pic[scale=0.2,page=6]{graphs}
	\hfill
	\picDark[scale=0.2]{alice}
	\pic[scale=0.2,page=2]{graphs}
	\hfill
	\picDark[scale=0.2]{bob}
	\pic[scale=0.2,page=12]{graphs}
	\hfill
	\picDark[scale=0.2]{eve}
	\pic[scale=0.2,page=16]{graphs}
	\hfill
	\pic[scale=0.2,page=26]{graphs}

	\begin{fancycolumns}[columns=2,widths={45,55}]
		\begin{note}{Advantages}
			\begin{itemize}
				\item Simple and straightforward approach
				\item Rapid exploration of new ideas
				\item No upfront investments
			\end{itemize}
		\end{note}
	\nextcolumn
		\begin{note}{Disadvantages}
			\begin{itemize}
				\item No structured and systematic reuse (copy \& edit)
				\item No flexible combination of features
				\item Maintenance quickly becomes impractical
			\end{itemize}
		\end{note}
	\end{fancycolumns}	
	\begin{fancycolumns}[columns=3,widths={15,70}]
	\nextcolumn
		\uncover<3->{\begin{note}{Towards Managed Clone-and-Own}
			\begin{itemize}
				\item How can we better manage such clone-and-own development?
				\item The traditional answer: Software Configuration Management
				\item In the sequel: Software Configuration Management in practice
			\end{itemize}
		\end{note}}
	\nextcolumn
	\end{fancycolumns}
\end{frame}

% TODO add \pic[width=\linewidth,page=24]{lego}


\lessonslearned{
	\item Compile-time variability is decided before or at compile time
	\item In Clone-and-Own, new variants of a software system are created by copying and adapting an existing variant
	\item Simple paradigm, but suffering from maintenance problems in the long run
}{
	\item[] % TODO add literature pointers
}{
	\begin{itemize}
		\item What are the reasons why clone-and-own is very popular in practice?
		\item What is the order of magnitude of the number of variants that can be reasonably maintained in clone-and-own?
		\item Have you ever applied the principle of clone-and-own? If so, where and how? 
	\end{itemize}
}

\sectionend

\section{Clone-and-Own with Version Control}


\subsection{Software Configuration Management}

\begin{frame}{Excursus: Software Configuration Management}
	\begin{mycolumns}[columns=2,widths={35,63},animation=keep]
		\mydefinition{Software Configuration Management (SCM)}{	
			Policies, processes and tools for managing evolving software systems:
			\begin{itemize}
				\item Version control 
				\item System building
				\item Release management
				\item Change management
				\item Collaborative work
			\end{itemize}	
		}	
	\mynextcolumn
		\mydefinition{Basic Terms and Definitions}{	
			\begin{itemize}
				\item {\bf Software Item}: An (atomic) artifact that can be uniquely identified 
				\item {\bf Version}: A modified software item.
				\begin{itemize}
					\item {\bf Revision}: A new version that replaces an old one.
					\item {\bf Variant}: A version that co-exists with another one.
				\end{itemize}
				\item {\bf Configuration}: A set of software items that together form a functioning (partial) system.
				\item {\bf Baseline}: A stable configuration that represents a point of reference for further development.
				\item {\bf Release}: A baseline delivered to customers.				
			\end{itemize}
		}
	\end{mycolumns}	
\end{frame}

\begin{frame}{Excursus: Software Configuration Management}
	\vspace{2mm}
	\only<1|handout:0>{\pic[width=0.7\linewidth]{configuration-management-1}}%
	\only<2-|handout:1>{\pic[width=0.7\linewidth]{configuration-management-2}}%
\end{frame}

\begin{frame}{Example: A Conceptual Organization of our Graph Library}
	\vspace{2mm}
	\only<1|handout:0>{\pic[width=0.8\linewidth]{configuration-management-graphs-1}}%
	\only<2-|handout:1>{\pic[width=0.8\linewidth]{configuration-management-graphs-2}}%
\end{frame}

%\begin{frame}{Tool Support: Version Control Systems}
	%\only<1|handout:0>{\pic[width=\linewidth,page=1]{branching}}%
	%\only<2|handout:0>{\pic[width=\linewidth,page=2]{branching}}%
	%\only<3-|handout:1>{\pic[width=\linewidth,page=3]{branching}}%
%\end{frame}

\subsection{Version Control Systems}

\begin{frame}{Tool Support: Version Control Systems}
\begin{large}
	\hspace{70mm}
	\uncover<4->{$\text{cherrypick := patch}(\Delta(r8,r10),r11)$}\\
	\vspace{8mm}
	\only<1|handout:0>{\pic[width=0.8\linewidth]{versioncontrol-1}}%
	\only<2|handout:0>{\pic[width=0.8\linewidth]{versioncontrol-2}}%
	\only<3-|handout:1>{\pic[width=0.8\linewidth]{versioncontrol-3}}\\
	\vspace{8mm}
	\hspace{40mm}
	\uncover<4->{$\text{merge := 3-way-merge}(r4,\Delta(r4,r7),\Delta(r4,r9))$}
\end{large}
\end{frame}
% TODO no actual merges shown, only fast-forward

\subsection{Supporting Clone-and-Own Development}

\begin{frame}{Example: Graph Library under Version Control}
	\pic[width=0.7\linewidth]{versioncontrol-graphs-1.pdf}
\end{frame}

\begin{frame}{Example: Graph Library under Version Control}
	\pic[width=0.7\linewidth]{versioncontrol-graphs-2.pdf}
\end{frame}

\begin{frame}{Example: Graph Library under Version Control}
	\pic[width=0.7\linewidth]{versioncontrol-graphs-3.pdf}
\end{frame}

\subsection{Discussion}

\begin{frame}{Clone-and-Own with Version Control}
	\begin{mycolumns}[columns=2,widths={50,50},animation=keep]
		\mynote{Summary and Observations}{			
			Clone-and-Own with Provenance: % TODO provenance not introduced
			\begin{itemize}
				\item Supports keeping tracking of revisions and variants
				\item Creation of new variants is supported by merging of branches
				\item Propagation of changes between variants is supported by cherrypicking changes
			\end{itemize}	
			However:
			\begin{itemize}
				\item Versioning is typically limited to entire system variants (i.e., branches)
				\item No flexible combination of software items
			\end{itemize}
		}	
	\mynextcolumn
		\mynote{Advantages}{	
			\begin{itemize}
				\item Well-established and stable systems.
				\item Well-known known process.
				\item Good tool integration.	
			\end{itemize}
		}
		\mynote{Disadvantages}{	
			\begin{itemize}
				\item Development of variants, not features: flexible combination of features not directly possible.
				\item No structured reuse (copy \& edit).
				\item Merging and cherrypicking not fully automated.	
			\end{itemize}			
		}
	\end{mycolumns}	
\end{frame}

\lessonslearned{
	\item Software configuration management as a traditional discipline of managing the evolution of variability-intensive systems
	\item Version control systems as a widespread tool supporting clone-and-own in practice
}{
	\item \fospl, Chapter 5.1
}{
	\begin{itemize}
		\item Which software configuration management concepts are supported by version control systems?
		\item Do you know other version control systems than Git? 
		\item If so, in which way are they different from Git? 
	\end{itemize}
}

\sectionend

\section{Clone-and-Own with Build Systems}


\begin{frame}<2>{Recap: Software Configuration Management}
	\frameSoftwareConfigurationManagement
\end{frame}

\subsection{Build Systems}

\begin{frame}{Tool Support: \myframetitle}
	\begin{mycolumns}[widths={40},animation=none]
		\begin{definition}{Build Systems} % TODO source missing
			\begin{itemize}
				\item Automation of the build process through build scripts
				\item Multiple steps with dependencies/conditions
				\begin{itemize}
					\item Copy files, 
					\item call compiler, 
					\item start other tools, 
					\item \ldots
				\end{itemize}
				\item Tools: 
				\begin{itemize}
					\item make
					\item ant
					\item maven
					\ldots
				\end{itemize}
			\end{itemize}
		\end{definition}
	\mynextcolumn
		\pic[width=\linewidth]{ant-hello-world}
	\end{mycolumns}	
\end{frame}

\subsection{Variability with Build Scripts}

\begin{frame}{\myframetitle}
	\myframeicon{\mysource{\staplesandhill}}
	\vspace{-\textheightoftitle}
	\begin{mycolumns}
		\begin{note}{Basic Idea}
			\begin{itemize}
				\item One build script per variant
				\item Include/exclude files when translating
				\item Overwrite variant-specific files
			\end{itemize}
		\end{note}
	\mynextcolumn
		\centering\pic[height=\textheightwithouttitle]{buildsystems}
	\end{mycolumns}	
\end{frame}

% TODO the next two slides could profit from adding the main insights in a short but textual form
\begin{frame}[fragile]{Example: Graph Library}
	\myframeicon{\mysource{\fospl\mypage{107}}}
	\begin{mycolumns}[b,columns=3,widths={50,32},height=\textheightwithtitle,animation=none]
		\hfill\pic[height=\textheightwithtitle]{buildsystems-graphs-1} % TODO what about Node.java and Weight.java?
	\mynextcolumn
\begin{codetight}{}
class Edge {
	Node a, b;

	Edge(Node a, Node b) {
		this.a = a; this.b = b;
	}
	void print() {
		a.print(); b.print();
	}
}
\end{codetight}
\vspace{2mm}
\begin{codetight}{}
class Edge {
	Node a, b;
	@Weight weight = new Weight();@

	Edge(Node a, Node b) {
		this.a = a; this.b = b;
	}
	void print() {
		a.print(); b.print();
		@weight.print();@
	}
}
\end{codetight}
	\mynextcolumn
	\end{mycolumns}
\end{frame}

\begin{frame}[fragile]{Example: Graph Library}
	\begin{mycolumns}[b,columns=3,widths={50,43},height=\textheightwithtitle,animation=none]
		\hfill\pic[height=\textheightwithtitle]{buildsystems-graphs-2} % TODO what about Node.java, Weight.java, EdgeFactory.java?
	\mynextcolumn
\begin{codetight}{}
class Graph {
	@EdgeFactory edgeFactory;@
	...
	@Graph(EdgeFactory edgeFactory) {
		this.edgeFactory = edgeFactory;
	}@
	Edge add(Node n, Node m) {
		@Edge e = edgeFactory.createEdge(n, m);@
		nodes.add(n); nodes.add(m); edges.add(e);
		return e;
	}
}
\end{codetight}
\begin{codetight}{}
class Edge {
	Node a, b;
	...
}
\end{codetight}
\vspace{2mm}
\begin{codetight}{}
class WeightedEdge extends Edge {
	@Weight weight = new Weight();@
	...
}
\end{codetight}
	\mynextcolumn
	\end{mycolumns}
\end{frame}

% TODO new slide about experience report by \staplesandhill\ Section~3--6

\subsection{Discussion}

\begin{frame}{Clone-and-Own with Build Systems}
	\vspace{-\textheightoftitle}
	\begin{mycolumns}[b,widths={55},animation=none]
		\begin{note}{Comparison to Version Control Systems}
			\begin{itemize}
				\item Supports combination of more fine-grained software items (i.e., files)
				\item However: Only limited support for provenance
			\end{itemize}
		\end{note}
		\begin{note}{In General}
			\begin{itemize}
				\item Combination of items (i.e., files)\\$\neq$ combination of features
				\item Changes to the basic variant may have undesired side-effects
					\begin{itemize}
						\item Some variants are updated but do not need those changes
						\item Some variants are updated but incompatible to those changes
						\item Variants with copied files are not automatically updated
					\end{itemize}
			\end{itemize}
		\end{note}
	\mynextcolumn
		\centering\pic[height=\textheightwithouttitle]{buildsystems}
	\end{mycolumns}
\end{frame}

% TODO a general discussion of advantages and disadvantages is missing here, see \fospl\mypages{108--110} (but be aware that the book is mixing two ways of using build systems)


\lessonslearned{
	\item Variability through build scripts
	\item Granularity of clones: Individual files
	\item Combination of files $\neq$ combination of features
}{
	\item \fospl, Chapter 5.1
}{
	\begin{itemize}
		\item Which software configuration management concepts are supported by build systems?
		\item What are the commonalities and differences of clone-and-own with version control and clone-and-own with build systems?
		\item What are the strengths and weaknesses?
	\end{itemize}
}

% TODO Thomas: add exam questions here

\mode<beamer>{
	\begin{frame}{\inserttitle}
		\lectureseriesoverview
	\end{frame}

	\contentoverview
}


\end{document}
