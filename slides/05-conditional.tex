\documentclass[
	aspectratio=169, % default is 43
	8pt, % font size, default is 11pt
	handout, % < do not remove this comment, it is used by the Makefile >
]{beamer}
\def\university{} % < do not remove this comment, it is used by the Makefile >

\documentclass[
	aspectratio=169, % default is 43
	8pt, % font size, default is 11pt
	handout, % handout mode without animations, comment out to add animations
]{beamer}

\usepackage{../template/beamerthemeuulm} % use the inofficial uulm beamer theme
\setfaculty{infIngPsy} % set the color scheme for your faculty here [med/infIngPsy/math/nat]

% requires symbolic links
% git clone git@github.com:SoftVarE-Group/SlideTemplate.git C:\Users\...\SlideTemplate
% mklink /J template C:\Users\...\SlideTemplate
% git clone git@spgit.informatik.uni-ulm.de:thuem/slides.git C:\Users\...\ThomasSlides
% mklink /J thomasslides C:\Users\...\ThomasSlides
\graphicspath{{../template/pics/logos}{../template/pics/nature}{../template/pics/uulm}{../thomasslides/}{../pics/}}

%\usepackage[ngerman]{babel} % use this line for slides in German
%\recordingtrue % special recording mode for use with a greenscreen, gives you space to show yourself in a layer in front of the slides, has no effect in the handout mode

\title{Software Product Lines} % short title is used for the slide footer but optional

%
%
%% IMPORTED PACKAGES
%
%\usepackage{adjustbox} % used for partofpage
%\usepackage{tcolorbox} % used for mydefinition, mynote, myexample
\usepackage{multicol} % used temporarily for the lecture overview
%\usepackage{mathtools} % required for absolute value in modeling lecture
%
%% SLIDE TEMPLATE
%
%\beamertemplatenavigationsymbolsempty 
%
%% COMMANDS TO LAYOUT AND ANNIMATE SLIDES
%
\newcommand{\lessonslearned}[3]{
	\subsection{Summary}
	\begin{frame}{\insertsection -- \insertsubsection}
		\leftorright{
			\mydefinition{Lessons Learned}{
				\begin{itemize}
					#1
				\end{itemize}
			}
			\mynote{Further Reading}{
				\small % references take space, can be a little smaller
				\begin{itemize}
					#2
				\end{itemize}
			}
		}{
			\myexample{Practice}{
				#3
			}
		}
	\end{frame}
}

\renewcommand{\lectureoverview}{
%	\section*{Overview}
%	\subsection*{Overview}
	\begin{frame}{\insertsubtitle}
		\begin{multicols}{2}
			\tableofcontents
		\end{multicols}
	\end{frame}
}

%
%\newcommand{\onlyleft}[1]{
%	\halfpage{#1}
%}
%
%\newcommand{\onlyright}[1]{
%	~\hfill
%	\halfpage{#1}
%}
%
%\newcommand{\leftorright}[2]{
%	\uncover<1>{\halfpage{#1}}
%	\hfill
%	\uncover<3->{\halfpage{#2}}
%}
%
%\newcommand{\rightorleft}[2]{
%	\uncover<3->{\halfpage{#1}}
%	\hfill
%	\uncover<1>{\halfpage{#2}}
%}
%
%\newcommand{\leftthenright}[2]{
%	\halfpage{#1}
%	\hfill\pause
%	\halfpage{#2}
%}
%
%\newcommand{\leftandright}[2]{
%	\halfpage{#1}
%	\hfill
%	\halfpage{#2}
%}
%
%\newcommand{\leftmiddleandright}[3]{
%	\thirdpage{#1}
%	\hfill
%	\thirdpage{#2}
%	\hfill
%	\thirdpage{#3}
%}
%
%\newcommand{\leftmiddleorright}[3]{
%	\uncover<1>{\thirdpage{#1}}
%	\hfill
%	\uncover<3>{\thirdpage{#2}}
%	\hfill
%	\uncover<5->{\thirdpage{#3}}
%}
%
%\newcommand{\halfpage}[1]{\partofpage{48}{#1}}
%
%\newcommand{\thirdpage}[1]{\partofpage{31}{#1}}
%
%\newcommand{\partofpage}[2]{
%	\adjustbox{valign=t}{\begin{minipage}{0.#1\textwidth}
%			\begin{flushleft}
%				#2
%			\end{flushleft}
%	\end{minipage}}
%}
%
%\newcommand{\mydefinition}[2]{
%	\begin{tcolorbox}[title=#1,colback=orange!10,colframe=orange!30,coltitle=black,fonttitle=\bfseries,left=1mm,right=1mm,top=1mm,bottom=1mm]
%		\begin{flushleft}
%			#2
%		\end{flushleft}
%	\end{tcolorbox}
%}
%
%\newcommand{\mydefinitiontight}[2]{
%	\begin{tcolorbox}[title=#1,colback=white,colframe=orange!30,coltitle=black,fonttitle=\bfseries,left=0mm,right=0mm,top=0mm,bottom=0mm]
%		\begin{flushleft}
%			#2
%		\end{flushleft}
%	\end{tcolorbox}
%}
%
%\newcommand{\mynote}[2]{
%	\begin{tcolorbox}[title=#1,colback=red!10,colframe=red!30,coltitle=black,fonttitle=\bfseries,left=1mm,right=1mm,top=1mm,bottom=1mm]
%		\begin{flushleft}
%			#2
%		\end{flushleft}
%	\end{tcolorbox}
%}
%
%\newcommand{\myexample}[2]{
%	\begin{tcolorbox}[title=#1,colback=blue!10,colframe=blue!30,coltitle=black,fonttitle=\bfseries,left=1mm,right=1mm,top=1mm,bottom=1mm]
%		\begin{flushleft}
%			#2
%		\end{flushleft}
%	\end{tcolorbox}
%}
%
%\newcommand{\myexampletight}[2]{
%	\begin{tcolorbox}[title=#1,colback=white,colframe=blue!30,coltitle=black,fonttitle=\bfseries,left=0mm,right=0mm,top=0mm,bottom=0mm]
%		\begin{flushleft}
%			#2
%		\end{flushleft}
%	\end{tcolorbox}
%}

\subtitle{5. Conditional Compilation}
\author{Thomas Thüm, Elias Kuiter, Timo Kehrer}

\ifuniversity{ulm}{\setpicture[75]{may21-north2}}
\ifuniversity{magdeburg}{\setpicture[300]{magdeburg-river}}

\begin{document}

% TITLE SLIDE

\maketitle

% SLIDE TEMPLATE

%\setbeamercolor{title}{fg=black}
%\setbeamercolor{frametitle}{fg=black}
\setbeamertemplate{frametitle}{{\huge~\\\insertsubsection~\insertframetitle}}
\setbeamertemplate{footline}[text line]{\parbox{\linewidth}{\vspace*{-10pt}\hspace{0pt}%
	\insertshortauthor\phantom{g\insertpagenumber}%
	\hfill%
	\inserttitle%
	\ifx \insertsubtitle \empty \else \ -- \insertsubtitle\fi%
	\ifx \insertsectionhead \empty \else \ -- \insertsectionhead\fi%
	\hfill%
	\phantom{g\insertshortauthor}\insertpagenumber%
}}
%\defbeamertemplate{footline}{\begin{beamercolorbox}[sep=1em]{author in head/foot}\insertshortauthor\hfill\insertsection\hfill\insertframenumber\end{beamercolorbox}}
%\defbeamertemplate*{footline}{mytheme}{\begin{beamercolorbox}[sep=1em]{author in head/foot}\insertshortauthor\hfill\insertsection\hfill\insertframenumber\end{beamercolorbox}}

% OVERVIEW SLIDES

\newcommand{\overview}{
	\section*{Overview}
	\subsection*{Overview}
	\begin{frame}{-- \insertsubtitle}
		\begin{multicols}{3}
			\tableofcontents
		\end{multicols}
	
		\begin{flushright}
			\footnotesize
			Author: \insertauthor
			
			Date: \insertdate
		\end{flushright}
	\end{frame}
}
% temporarily added slide to have a lecture overview 
\overview

% temporarily removed
%\begin{frame}{Lecture Overview -- \insertsubtitle}
%	\tableofcontents[hideallsubsections]
%\end{frame}

\AtBeginSection[]{%
	\begin{frame}{Lecture Overview -- \insertsubtitle}
		\tableofcontents[currentsection,hideothersubsections]
	\end{frame}
}

\newcommand{\sectionend}{\addtocontents{toc}{\newpage}}


\begin{frame}{\inserttitle}
	\lectureseriesoverview[5]
\end{frame}

\section{Features with Build Systems}

\subsection{How to Implement Features?}
\begin{frame}{\myframetitle}
	\begin{mycolumns}
		\begin{exampletight}{Given a feature model for graphs \ldots}
			\centering\featureDiagramGraphs
			%\featureDiagramLegend
		\end{exampletight}
		\begin{example}{\ldots\ we can derive a valid configuration}
			\small
			\begin{mycolumns}[columns=3,animation=none]
				$\{G\}$\\
				$\{G,C\}$\\
				$\{G,D\}$\\
				$\{G,C,D\}$\\
			\mynextcolumn
				$\{G,W\}$\\
				$\{G,C,W\}$\\
				$\{G,D,W\}$\\
				$\{G,C,D,W\}$\\
			\mynextcolumn
				$\{G,W,S\}$\\
				$\{G,C,W,S\}$\\
				$\{G,D,W,S\}$\\
				$\{G,C,D,W,S\}$\\
			\end{mycolumns}
		\end{example}
	\mynextcolumn
		\vspace{-10mm}\pause
		\begin{exampletight}{How to Generate Products Automatically?}
			\centering\foreach \page in {2,12,4,14,6,16,8,18,10,20,42,44}{\pic[width=.23\linewidth,page=\page]{graphs} }
		\end{exampletight}
		\begin{note}{Goals}
			\begin{itemize}
				\item descriptive specification of a product (i.e., a configuration, a selection of features)
				\item automated generation of a product with compile-time variability
			\end{itemize}
			Focus of the next three lectures \ldots
		\end{note}
	\end{mycolumns}
\end{frame}

\subsection{Problems of Ad-Hoc Approaches for Variability}

\subsubsection{Features with Runtime Variability?}
\againframe<2>{GraphWithGlobalVariables}

\begin{frame}{\myframetitle}
	\begin{mycolumns}[b,widths={36}]
		\picDark[width=\linewidth]{preferences-eclipse}

		\begin{definition}{How to? -- Preference Dialog}
			\begin{itemize}
				\item implement runtime variability
				\item compile the program
				\item run the program
				\item \emph{manually adjust preferences based on configuration}
			\end{itemize}
		\end{definition}
	\mynextcolumn
		\begin{mycolumns}[widths={41},animation=none]
			\pic[width=\linewidth]{runtime-parameters-win10-cmd-dir}
		\mynextcolumn
			\picDark[width=\linewidth]{configfile-eclipse-ini}
		\end{mycolumns}

		\begin{definition}{How to? -- Command-Line Options / Configuration Files}
			\begin{itemize}
				\item implement runtime variability
				\item compile the program
				\item \emph{automatically generate command-line options / configuration files based on configuration}
				\item run the program
			\end{itemize}
		\end{definition}
	\end{mycolumns}
\end{frame}

\begin{frame}[fragile]{\myframetitle}
	\begin{mycolumns}[widths={48}]
\begin{codetight}[basicstyle=\small]{}
public class Config {
	~public final static boolean COLORED = true;~
	@public final static boolean WEIGHTED = false;@
}
\end{codetight}
		\begin{definition}{How to? -- Immutable Global Variables}
			\begin{itemize}
				\item implement runtime variability
				\item automatically generate class with global variables based on configuration
				\item compile and run the program
			\end{itemize}
		\end{definition}
	\mynextcolumn
		\begin{note}{What is missing?}
			\begin{itemize}
				\item automated generation:\\\hfill for preference dialogs
				\item no compile-time variability / same large binary:\\\hfill for all except immutable global variables
				\item very limited compile-time variability:\\\hfill for immutable global variables
			\end{itemize}
		\end{note}
	\end{mycolumns}
\end{frame}

\subsubsection{Features with Clone-and-Own?}
\begin{frame}{\myframetitle}
	\begin{mycolumns}[widths={30},animation=none]
		\centering~

		\picDark[scale=0.2]{alice}
		\pic[scale=0.26,page=2]{graphs}

		\picDark[scale=0.2]{bob}
		\pic[scale=0.26,page=12]{graphs}

		\picDark[scale=0.2]{eve}
		\pic[scale=0.26,page=16]{graphs}
	\mynextcolumn
		\begin{definition}{How to?}
			\begin{itemize}
				\item implement separate project for each product\\(i.e., branch with version control)
				\item download project / checkout branch based on configuration
				\item run build script, if existent
				\item compile and run the program
			\end{itemize}
		\end{definition}
		\pause
		\begin{note}{What is missing?}
			\begin{itemize}
				\item compile-time variability only for implemented products
				\item no automated generation:\\\hfill for clone-and-own (with version control systems)
				\item automated generation based on build script and extra files:\\\hfill for clone-and-own with build systems
				\item no free feature selection (i.e., configuration)
			\end{itemize}
		\end{note}
	\end{mycolumns}
\end{frame}

\subsection{Recap: Clone-and-Own with Build Systems}
\begin{frame}{\myframetitle\  \mytitlesource{\href{https://dl.acm.org/doi/10.1145/3461002.3473950}{Kuiter~et~al.~2021}}}
	\begin{mycolumns}[columns=2,widths={76},animation=none]
		\begin{mycolumns}[widths={50},animation=none] % TODO widths needed due to bug #56 in slide template
			\begin{example}{Case Study: Anesthesia Device}
				\begin{itemize}
					\item C application
					\item targets embedded devices (ESP32)
					\item configurations are hard-coded as build scripts
				\end{itemize}
			\end{example}
		\mynextcolumn
			\uncover<4->{
				\begin{exampletight}{{\color{red}Production Device}: OLED, Clock}
					\centering\pic[height=25mm]{pignap-cfg-production-oled}
				\end{exampletight}
			}
		\end{mycolumns}
		\begin{mycolumns}[widths={50},animation=none] % TODO widths needed due to bug #56 in slide template
			\uncover<2->{
				\begin{exampletight}{{\color{green}Prototype}: OLED Display}
					\centering\pic[height=25mm]{pignap-cfg-prototype-oled}
				\end{exampletight}
			}
		\mynextcolumn
			\uncover<3->{
				\begin{exampletight}{{\color{blue}Prototype}: LCD, Real-Time Clock}
					\centering\pic[height=25mm]{pignap-cfg-prototype-lcd-rtc}
				\end{exampletight}
			}
		\end{mycolumns}
	\mynextcolumn
		\only<1|handout:0>{\pic[width=\linewidth,page=1]{pignap-variants}}%
		\only<2|handout:0>{\pic[width=\linewidth,page=2]{pignap-variants}}%
		\only<3|handout:0>{\pic[width=\linewidth,page=3]{pignap-variants}}%
		\only<4->{\pic[width=\linewidth,page=4]{pignap-variants}}
	\end{mycolumns}
\end{frame}

\subsection{Introducing Features to Build Systems}
\begin{frame}{\myframetitle\  \mytitlesource{\href{https://dl.acm.org/doi/10.1145/3461002.3473950}{Kuiter~et~al.~2021}}}
	\begin{mycolumns}[widths={60}]
		\begin{definition}{How to Implement Features with Build Systems?}
			\begin{itemize}
				\item step 1: model variability in a feature model
				\item step 2: in build scripts, in- and exclude files based on feature selection
				\item step 3: pass a feature selection at build time
			\end{itemize}
			$\Rightarrow$ one build script per group of related features
		\end{definition}
		\begin{center}
			\featureDiagram{Anesthesia Device,abstract
			[Monitoring,abstract,mandatory
				[Display,abstract,or[LCD,concrete,alternative][OLED,concrete]]
				[Wi-Fi,abstract
					[HTTP Server,concrete,mandatory]]]
			[History,optional,concrete]
			[Libraries,abstract,mandatory
				[Storage,abstract,optional[NVS,concrete,alternative][FRAM,concrete]]
				[RTC,concrete,optional[Battery,concrete,mandatory]]]}

			$History \pimplies Storage \pand RTC$
		\end{center}
	\mynextcolumn
		\centering\pic[height=\textheightwithtitle]{pignap-features}
	\end{mycolumns}
\end{frame}

\subsection{The Linux Kernel}

\xkcdframe{619} % linux features 20s

\subsubsection*{KConfig for Feature Modeling}

\begin{frame}[fragile]{\myframetitle}
	\begin{mycolumns}
		\begin{kconfigtight}[basicstyle=\small]{Part of the x86 Architecture \mysource{\href{https://github.com/torvalds/linux/blob/0326074/arch/x86/Kconfig}{linux/arch/x86/Kconfig}}}
config 64BIT
	bool "64-bit kernel" if "$(ARCH)" = "x86"
	default "$(ARCH)" != "i386"
	help
		Say yes to build a 64-bit kernel (x86_64)
		Say no to build a 32-bit kernel (i386)

config X86_32
	def_bool y
	depends on !64BIT
	# Options that are inherently 32-bit kernel only:
	select GENERIC_VDSO_32
	select ARCH_SPLIT_ARG64

config X86_64
	def_bool y
	depends on 64BIT
	# Options that are inherently 64-bit kernel only:
	select ARCH_HAS_GIGANTIC_PAGE
	select ARCH_SUPPORTS_INT128 if CC_HAS_INT128
\end{kconfigtight}
	\mynextcolumn
		\begin{definition}{KConfig Language\mysource{\href{https://www.kernel.org/doc/html/latest/kbuild/kconfig-language.html}{kernel.org}}}
			\begin{itemize}
				\item configuration language used in embedded/OS development (e.g., Linux, Zephyr, ESP32)
				\item similar to UVL, but has many quirks (e.g., tristate features, \texttt{select})
				\item transformation into formula or feature model possible, but not trivial \mysource{\href{https://dl.acm.org/doi/abs/10.1145/3468264.3468578}{Oh~et~al.~2021}}
			\end{itemize}
		\end{definition}
		\hspace*{-0.07253886\linewidth}%=2*0.035/(1-0.035)
		\linuxbddlink{\pic[width=1.3\linewidth,trim=220 510 60 180,clip]{2020/2020-SPLC-Thuem}}
	\end{mycolumns}
\end{frame}

\subsubsection*{MenuConfig for Configuration}

\begin{frame}{\myframetitle}
	\begin{mycolumns}[widths={60,40}]
		\begin{exampletight}{}
			\pic[width=\textwidth]{linux-menuconfig} % sudo apt install make flex bison libncurses5-dev && make menuconfig
		\end{exampletight}
	\mynextcolumn
		\begin{note}{make menuconfig}
			\begin{itemize}
				\item configures KConfig models
				\item generates a \texttt{.config} file
				\item widely used to configure Linux
				\item still: it is possible to create invalid configurations and products % TODO does not fit into the storyline anymore. remove?
			\end{itemize}
		\end{note}
	\end{mycolumns}
\end{frame}

\subsubsection*{KBuild as Build System}

\begin{frame}[fragile]{\myframetitle}
	\begin{mycolumns}
		\begin{kconfigtight}[basicstyle=\footnotesize]{Feature Model with KConfig\mysource{\href{https://github.com/torvalds/linux/blob/0326074/arch/x86/Kconfig}{linux/arch/x86/Kconfig}}}
config X86_32 ...
config X86_64 ...

config IA32_EMULATION
	bool "IA32 Emulation"
	depends on X86_64
	help Include code to run legacy 32-bit programs under a 64-bit kernel. You should likely enable this, unless you're 100% sure that you don't have any 32-bit programs left.
\end{kconfigtight}
		\begin{definition}{KBuild\mysource{\href{https://www.kernel.org/doc/html/latest/kbuild/makefiles.html}{kernel.org}}}
			\begin{itemize}
				\item a style for writing Makefiles in Linux
				\item defines goals with Make variables
				\begin{itemize}
					\item \texttt{obj-y}: static linkage (= include feature)
					\item \texttt{obj-m}: dynamic linkage (= as module)
					\item \texttt{obj-}: no linkage (= exclude feature)
				\end{itemize}
				\item full power of Make $\Rightarrow$ hard to comprehend
			\end{itemize}
		\end{definition}
	\mynextcolumn
		\begin{kbuildtight}[basicstyle=\small]{Feature Mapping with KBuild \mysource{\href{https://github.com/torvalds/linux/blob/0326074/arch/x86/Kbuild}{linux/arch/x86/Kbuild}}}
# link these subdirectories statically:
obj-y += entry/ # entry routines
obj-y += realmode/ # 16-bit support
obj-y += kernel/ # x86 kernel
obj-y += mm/ # memory management

# link these depending on a configuration option:
obj-~$(CONFIG_IA32_EMULATION)~ += ia32/
obj-~$(CONFIG_XEN)~ += xen/ # paravirtualization

# the KConfig feature model can even be overridden:
obj-~$(subst m,y,$(CONFIG_HYPERV))~ += hyperv/
		\end{kbuildtight}

		\begin{kbuildtight}[basicstyle=\small]{Recurse into Subsystems\mysource{\href{https://github.com/torvalds/linux/blob/0326074/arch/x86/ia32/Makefile}{linux/arch/x86/ia32/Makefile}}}
# ia32 kernel emulation subsystem
obj-~$(CONFIG_IA32_EMULATION)~ := ia32_signal.o
audit-class-~$(CONFIG_AUDIT)~ := audit.o

# IA32_EMULATION and AUDIT required for audit.o:
obj-~$(CONFIG_IA32_EMULATION)~ += ~$(audit-class-y)~
		\end{kbuildtight}
	\end{mycolumns}
\end{frame}

\begin{frame}[fragile]{\myframetitle}
	\begin{mycolumns}
		\begin{example}{Interactive Linux Kernel Configurator}
			\pic[width=\linewidth]{linux-menuconfig-emulation}
		\end{example}

		\begin{kconfigtight}[basicstyle=\footnotesize]{Feature Model and Example Configuration}
			config AUDIT ... # configured as NO
			config IA32_EMULATION ... # configured as YES
			config HYPERV ... # configured as MODULE
			config XEN ... # configured as NO
\end{kconfigtight}
	\mynextcolumn
		\begin{kbuildtight}[basicstyle=\small]{Feature Mapping}
obj-y += entry/ realmode/ kernel/ mm/
obj-~$(CONFIG_IA32_EMULATION)~ += ia32/
obj-~$(CONFIG_XEN)~ += xen/
obj-~$(subst m,y,$(CONFIG_HYPERV))~ += hyperv/
obj-~$(CONFIG_IA32_EMULATION)~ := ia32_signal.o
audit-class-~$(CONFIG_AUDIT)~ := audit.o
obj-~$(CONFIG_IA32_EMULATION)~ += ~$(audit-class-y)~
		\end{kbuildtight}

		\begin{kbuildtight}[basicstyle=\small]{Feature Mapping for Example Configuration}
obj-y += entry/ realmode/ kernel/ mm/
obj-?y? += ia32/
obj- += xen/
obj-?y? += hyperv/
obj-?y? := ia32_signal.o
audit-class- := audit.o
obj-?y? +=
		\end{kbuildtight}

		\begin{note}{}
			\small i.e., \texttt{entry, realmode, kernel, mm, ia32, hyperv, ia32\_signal.o} are compiled
		\end{note}
	\end{mycolumns}
\end{frame}

% TODO add quote? \mycite{As Kbuild relies on a Turing-complete language, complex conditions can be encoded.} \mysource{\fospl\mypage{108}}

\subsection{Discussion of Features with Build Systems}
\begin{frame}{\myframetitle}
	\begin{mycolumns}
		\begin{note}{Advantages}
			\begin{itemize}
				\item compile-time variability\\
					$\Rightarrow$ \emph{fast, small binaries} with smaller attack surface and without disclosing secrets
				\item automated generation of arbitrary products\\
					$\Rightarrow$ \emph{free feature selection}
				\item allows in- and exclusion of individual files or even entire subsystems\\
					$\Rightarrow$ high-level, \emph{modular variability}
			\end{itemize}
		\end{note}
	\mynextcolumn
		\begin{note}{Challenges}
			\begin{itemize}
				\item not easily reconfigurable at run- or load-time
				\item build scripts may become complex, there is no limit to what can be done (e.g., you can run arbitrary shell commands on files)\\
					$\Rightarrow$ \emph{hard to understand and analyze}
				\item no simple in- and exclusion of individual lines or chunks of code\\
				$\Rightarrow$ high-level use \emph{only}!
			\end{itemize}
		\end{note}
	\end{mycolumns}
\end{frame}


\lessonslearned{
	\item ad-hoc variability is lacking
	\item features with build systems allow for automated generation of products and free feature selection
	\item build systems include entire files, not lines or chunks
}{
	\item \fospl\mychapter{5.2.1}\mypages{105--106}\\--- brief introduction to variability in build scripts
	% TODO add literature? van der Storm T (2004).Variability and component composition. In: Proc. Int’l Conf. Software Reuse (ICSR) Lecture notes in computer science, vol 3107. Springer, pp 157–166
	\item \fospl\mychapter{5.2.3}\mypages{107--108}\\--- variability in build scripts of the Linux kernel
	% TODO add literature? Berger T, She S, Czarnecki K, Wa ̨sowski A (2010a) Feature-to-code mapping in two large product lines. In: Proc. Int’l Software Product Line Conference (SPLC). Lecture Notes in Computer Science, vol 6287. Springer, pp 498–499
	% TODO add literature? Dietrich C, Tartler R, Schröder-Preikschat W, Lohmann D (2012b) Understanding Linux feature distribution. In: Proc. AOSD Workshop on Modularity In Systems Software (MISS). ACM Press, pp 15–20
	% TODO add work on KernelHaven
}{
	Choose a build system and sketch how it may be used to implement this SPL (file structure and build script):

	\begin{center}
		\small\featureDiagramConfigurableDatabase
	\end{center}

	\uploadpractice
}

\sectionend

\section{Features with Preprocessors}

\subsection{Granularity of Variability}
\begin{frame}{\myframetitle}
	\begin{mycolumns}
		\todots
	\mynextcolumn
		\todots
	\end{mycolumns}
\end{frame}
% Christian's paper?
% empirical studies?
% essence: file-level variability is not engough

\subsection{What is a Preprocessor?}
\begin{frame}{\myframetitle\mysource{\fospl\mypages{110--111}}}
	\begin{mycolumns}
		\begin{definition}{Preprocessor}
			\begin{itemize}
				\item tool manipulating source code before compilation (i.e., at compile time)
				\item preprocessors are used:
					\begin{itemize}
						\item to inline files\hfill(e.g., header files)
						\item to define and expand macros\\\hfill(cf.\ metaprogramming)
						\item for \textbf{conditional compilation}\\\hfill(e.g., remove debug code for release)
					\end{itemize}
			\end{itemize}
		\end{definition}
	\mynextcolumn
		\begin{note}{Preprocessor}
			\begin{itemize}
				\item the C Preprocessor is used in almost every C/C++ project
				\item preprocessors are typically oblivious to the target language as they operate on text files (e.g., the C Preprocessor can also used for Fortran or Java)
				\item conditional compilation is a very common technique to implement product lines
			\end{itemize}
		\end{note}
	\end{mycolumns}
\end{frame}

% in-place vs outa-place preprocessors
% how to select features? parameters passed to the preprocessor, define in source code

\subsection{CPP -- The C Preprocessor}
\begin{frame}{\myframetitle\ -- In a Nutshell \mytitlesource{\featureide}}
	\leftorright{
		\myexampletight{Example Input to the Preprocessor}{\includegraphics[scale=.3]{preprocessor-c}}
	}{
		\myexampletight{Example Output (Simplified)}{\includegraphics[scale=.15]{preprocessor-c-output}}
		\begin{note}{}
			preprocessors typically do not remove line breaks to not influence of line numbers reported by compilers
		\end{note}
	}
\end{frame}
% TODO example contains "Beautiful" but should be "beautiful"
% TODO keywords: #ifdef #endif #else
% TODO illustrate parameters and the call of the preprocessor

\begin{frame}{\myframetitle\ -- In a Coconutshell}
	\todots
\end{frame}
% TODO explain the most important commands: #if #defined or and not ... #elif
% TODO #include

% TODO #error
% TODO parameters, #define, even combinations

% TODO single characters
% TODO discipliced, undisciplined

\subsection{Preprocessors for Java}
\begin{frame}{Munge -- A Simple Preprocessor for Java \mytitlesource{\featureide}}
	\begin{mycolumns}[widths={55}]
		\myexampletight{Example Input and Output}{\includegraphics[width=\linewidth]{preprocessor-munge}}
	\mynextcolumn
		\myexampletight{Calling the Preprocessor}{
			\centering
			\includegraphics[width=\linewidth]{preprocessor-munge-call}
			
			\includegraphics[width=.7\linewidth]{preprocessor-munge-idea}
		}
	\end{mycolumns}
\end{frame}

\begin{frame}{Antenna -- An In-Place Preprocessor for Java \mytitlesource{\featureide}}
	\begin{mycolumns}[widths={55}]
		\myexampletight{Example Input and Output}{\includegraphics[width=\linewidth]{preprocessor-antenna}}
	\mynextcolumn
		\myexampletight{Calling the Preprocessor}{
			\centering
			\includegraphics[width=\linewidth]{preprocessor-antenna-call}
			
			~
			
			\includegraphics[width=.7\linewidth]{preprocessor-antenna-idea}
		}
	\end{mycolumns}
\end{frame}

\subsection{Preprocessors in FeatureIDE}
\begin{frame}{\myframetitle\mysource{\featureide}}
	\begin{mycolumns}[animation=none,widths={75}]
		\only<1|handout:0>{\pic[width=\linewidth]{featureide-antenna-featuremodel}}%
		\only<2|handout:1>{\pic[width=\linewidth]{featureide-antenna-configuration}}%
		\only<3|handout:0>{\pic[width=\linewidth]{featureide-antenna-warning}}%
		\only<4|handout:0>{\pic[width=\linewidth]{featureide-antenna-contentassist}}%
	\mynextcolumn
%		\begin{definition}{FeatureIDE\mysource{\featureide}}
%			\begin{itemize}
%				\item 
%			\end{itemize}
%		\end{definition}
		\begin{note}{\href{https://www.youtube.com/watch?v=jVe7f32mLCQ}{Demo Video}}
			\begin{itemize}
				\item preprocessing with Antenna on command line
				\item feature modeling
				\item warnings for unreferenced features
				\item content assist proposing feature names
				\item configuration and automated regeneration
				\item (first 2 min relevant here)
			\end{itemize}
		\end{note}
	\end{mycolumns}
\end{frame}
% TODO has FeatureIDE been shown/discussed before? ideally, here only present preprocessor integration

\subsection{Discussion of Preprocessors}
\begin{frame}{\myframetitle\ \mytitlesource{\featureide}}
	\leftorright{
		\myexampletight{A Slightly More Complex Example}{\includegraphics[width=\linewidth]{preprocessor-antenna-elevator}}
	}{
		\todots
	}
\end{frame}
% pros: fine granular, language-independent
% cons: IDE support, easy to create mistakes

\xkcdframe{619} % linux features 20s

\subsection{Preprocessor-Based Product Lines in the Wild}
\begin{frame}{\myframetitle\mysource{\fortyproductlines}}
	\leftorright{
		\begin{exampletight}{Number of Features}
			\centering\pic[width=.8\linewidth,page=1]{fortyproductlines}\\Lines of Code
		\end{exampletight}
	}{
		\begin{exampletight}{Percentage of Variable Code}
			\centering\pic[width=.8\linewidth,page=2]{fortyproductlines}\\Lines of Code
		\end{exampletight}
	}
\end{frame}

\begin{frame}{\myframetitle\mysource{\fortyproductlines}}
	\leftorright{
		\begin{exampletight}{Lines of Variable Code}
			\centering\pic[width=.8\linewidth,page=3]{fortyproductlines}\\Number of Features
		\end{exampletight}
	}{
		\begin{exampletight}{Average Nesting Depth}
			\centering\pic[width=.8\linewidth,page=6]{fortyproductlines}\\Number of Features
		\end{exampletight}
	}
\end{frame}

\begin{frame}{\myframetitle\mysource{\fortyproductlines}}
	\leftorright{
		\begin{exampletight}{Average Number of Feature References}
			\centering\pic[width=.8\linewidth,page=4]{fortyproductlines}\\Number of Features
		\end{exampletight}
	}{
		\begin{exampletight}{Average Number of Features per Annotation}
			\centering\pic[width=.8\linewidth,page=5]{fortyproductlines}\\Number of Features
		\end{exampletight}
	}
\end{frame}
% TODO create own plots for the data?
% TODO add pictures from Rodrigues et al. @ INFSOF’16 (Assessing fine-grained feature dependencies): methods with directives vs product lines



\lessonslearned{
	\item granularity of variability at file level is not sufficient
	\item preprocessors facilitate fine-grained variability within files
	\item a widely applied preprocessor is the C Preprocessor
	\item industrial systems often combine preprocessors and build systems for features
}{
	\item \fospl, Section~5.3 Preprocessors
}{
	\begin{enumerate}
		\item Antenna performs an in-place transformation on implementation artifacts.
		What might be the benefits of using an in-place approach? Do you see any drawbacks?
		\item The preprocessors we have seen so far are also called lexical preprocessors.
		What is emphasized by the notion of lexical and can you think of other preprocessing approaches?
		\item The literature on software product lines has coined the term ``\#ifdef hell''.
		What could be meant with this?
	\end{enumerate}
}

\sectionend

\section{Feature Traceability}

\subsection{Recap: Code Scattering and Tangling}
\begin{frame}[fragile]{\myframetitle}
	\small\begin{mycolumns}[columns=3,T,animation=none,widths={43,32}]
\begin{codetight}{}
public class Graph {
	List nodes = new ArrayList();
	List edges = new ArrayList();

	Edge add(Node n, Node m) {
		Edge e = new Edge(n, m);
		nodes.add(n); nodes.add(m); edges.add(e);
		e.weight = new Weight();
		return e;
	}
	Edge add(Node n, Node m, Weight w) {
		Edge e = new Edge(n, m);
		nodes.add(n); nodes.add(m); edges.add(e);
		e.weight = w;
		return e;
	}
	void print() {
		for (int i = 0; i < edges.size(); i++) {
			((Edge) edges.get(i)).print();
		}
	}
}
\end{codetight}
		\mynextcolumn
\begin{codetight}{}
public class Node {
	int id = 0;
	Color color = new Color();

	void print() {
		Color.setDisplayColor(color);
		System.out.print(id);
	}
}
\end{codetight}
\begin{codetight}{}
public class Edge {
	Node a, b;
	Weight weight = new Weight();

	Edge(Node a, Node b) {
		this.a = a; this.b = b;
	}
	void print() {
		a.print(); b.print();
		weight.print();
	}
}
\end{codetight}
		\mynextcolumn
\begin{codetight}{}
public class Color {
	static void setDisplayColor(Color c) {...}
}
\end{codetight}	
\begin{codetight}{}
public class Weight {
	void print() {...}
}
\end{codetight}
		\hspace{10mm}
		\begin{note}{What is \ldots}
			\setlength\leftmargini{3mm}
			\begin{itemize}
				\item<+-> code scattering? 
				\item<+-> code tangling? 
				\item<+-> feature traceability? 
			\end{itemize}
		\end{note}
	\end{mycolumns}
\end{frame}
% note to editors: it is on purpose that no feature traces are shown on this slide, as the students are supposed to trace features manually

% TODO example copied from Lecture 2.2 and modified, integrate changes again? ideally we would have this code somewhere and reference it (rather than copying it)
% TODO would be nice to avoid this example clone (below) by ignoring @ and ~ on certain slides/animations

\begin{frame}[fragile]{\myframetitle}
	\small\begin{mycolumns}[columns=3,T,animation=none,widths={43,32}]
\begin{codetight}{}
public class Graph {
	List nodes = new ArrayList();
	List edges = new ArrayList();

	Edge add(Node n, Node m) {
		Edge e = new Edge(n, m);
		nodes.add(n); nodes.add(m); edges.add(e);
		@e.weight = new Weight();@
		return e;
	}
	@Edge add(Node n, Node m, Weight w) {
		Edge e = new Edge(n, m);
		nodes.add(n); nodes.add(m); edges.add(e);
		e.weight = w;
		return e;
	}@
	void print() {
		for (int i = 0; i < edges.size(); i++) {
			((Edge) edges.get(i)).print();
		}
	}
}
\end{codetight}
		\mynextcolumn
\begin{codetight}{}
public class Node {
	int id = 0;
	~Color color = new Color();~

	void print() {
		~Color.setDisplayColor(color);~
		System.out.print(id);
	}
}
\end{codetight}
\begin{codetight}{}
public class Edge {
	Node a, b;
	@Weight weight = new Weight();@

	Edge(Node a, Node b) {
		this.a = a; this.b = b;
	}
	void print() {
		a.print(); b.print();
		@weight.print();@
	}
}
\end{codetight}
		\mynextcolumn
\begin{codetight}{}
~public class Color {
	static void setDisplayColor(Color c) {...}
}~
\end{codetight}	
\begin{codetight}{}
@public class Weight {
	void print() {...}
}@
\end{codetight}
		\pause
		\begin{note}{Is it a problem for \ldots}
			\setlength\leftmargini{4mm}
			\begin{enumerate}[(a)]
				\item single systems? 
				\item runtime variability? {\tiny\lectureruntime}
				\item clone-and-own? {\tiny\lecturecloneandown}
				\item conditional compilation? {\tiny\lecturefeatures}
			\end{enumerate}
		\end{note}
	\end{mycolumns}
\end{frame}

\subsection{The Feature Traceability Problem}
\begin{frame}{\myframetitle}
	\begin{mycolumns}
		\todots
	\mynextcolumn
		\todots
	\end{mycolumns}
\end{frame}

\subsection{Feature Location}
\begin{frame}{\myframetitle}
	\begin{mycolumns}
		\todots
	\mynextcolumn
		\todots
	\end{mycolumns}
\end{frame}

\subsection{Feature Traceability with Colors}

\subsubsection*{Feature Commander}
\begin{frame}{\myframetitle}
	\begin{mycolumns}[animation=none,widths={35}]
		\begin{definition}{\featurecommander}
			\begin{itemize}
				\item each feature can be assigned to a color
				\item color used to support feature traceability
				\item features not assigned to a color shown in a shade of grade
				\item visualizations based on preprocessor macros % TODO check wording
			\end{itemize}
		\end{definition}
		\begin{note}{\href{https://www.tu-chemnitz.de/informatik/ST/research/material/xenomai/video.wmv}{Demo Video}}
			 (there is no sound)
		\end{note}
	\mynextcolumn
		\only<1,6-|handout:1>{\pic[width=\linewidth]{feature-commander1-cropped}}%
		\only<2|handout:0>{\pic[width=\linewidth,trim=0 450 600 0,clip]{feature-commander1-cropped}}%
		\only<3|handout:0>{\pic[width=\linewidth,trim=750 400 -150 50,clip]{feature-commander1-cropped}}%
		\only<4|handout:0>{\pic[width=\linewidth,trim=175 175 425 275,clip]{feature-commander1-cropped}}%
		\only<5|handout:0>{\pic[width=\linewidth,trim=600 0 0 450,clip]{feature-commander1-cropped}}%
	\end{mycolumns}
\end{frame}

\begin{frame}{\myframetitle}
	\begin{mycolumns}[animation=none,widths={65}]
		\only<1,5-|handout:1>{\pic[width=\linewidth]{feature-commander2-cropped}}%
		\only<2|handout:0>{\pic[width=\linewidth,trim=0 450 600 0,clip]{feature-commander2-cropped}}%
		\only<3|handout:0>{\pic[width=\linewidth,trim=600 450 0 0,clip]{feature-commander2-cropped}}%
		\only<4|handout:0>{\pic[width=\linewidth,trim=600 0 0 450,clip]{feature-commander2-cropped}}%
	\mynextcolumn
		\begin{definition}{\featurecommander}
			\begin{itemize}
				\item research prototype (last update August 2010)
				\item only static view on the source code
				\item only works for \href{https://source.denx.de/Xenomai/xenomai/-/wikis/home}{Xenomai} (a real-time core for Linux)
				\item further reading on experiments with developers: \backgroundcolors
			\end{itemize}
		\end{definition}
	\end{mycolumns}
\end{frame}

\subsubsection*{FeatureIDE}
\begin{frame}{\myframetitle}
	\begin{mycolumns}[animation=none,widths={65}]
		\only<1,4-|handout:1>{\pic[width=\linewidth]{feature-traceability}}%
		\only<2|handout:0>{\pic[width=\linewidth,trim=0 160 350 120,clip]{feature-traceability}}%
		\only<3|handout:0>{\pic[width=\linewidth,trim=210 65 140 295,clip]{feature-traceability}}%
	\mynextcolumn
		\begin{definition}{FeatureIDE\mysource{\featureide}}
			\begin{itemize}
				\item tool support for feature traceability
				\item inspired by Feature~Commander
				\item color can be assigned to features
				\item colors used in feature model, configurations, package explorer, and source code
			\end{itemize}
		\end{definition}
		\begin{note}{\href{https://youtu.be/jVe7f32mLCQ?t=240}{Demo Video} (last minute only)}
			\begin{itemize}
				\item collaboration view
				\item support for colors
			\end{itemize}
		\end{note}
	\end{mycolumns}
\end{frame}
% \href{https://www.youtube.com/watch?v=S32Cy2LXB8o}{assigning colors from everywhere}
% \href{https://www.youtube.com/watch?v=9QNDu0rMuQQ}{overview on support for colors with FeatureHouse (even in generated code)}

\subsection{Virtual Separation of Concerns}
\begin{frame}{\myframetitle\mysource{\virtualseparation}}
	\begin{mycolumns}
		\begin{definition}{Virtual Separation of Concerns}
			\begin{itemize}
				\item annotations of code based on the underlying structure (i.e., abstract syntax)
				\item \textbf{disciplined annotations}: only optional nodes in the abstract syntax tree can be annotated
				\item tool support used to provide views and navigate in source code
				\item \textbf{syntactic correctness} guaranteed for all generated program variants
			\end{itemize}
		\end{definition}
		% TODO show abstract syntax tree here? maybe for expressions with addition, multiplication, and values. or classes, class name, methods, fields, parameters, return type?
	\mynextcolumn
		\begin{note}{What is different with preprocessors?}
			\begin{itemize}
				\item annotation of characters in plain text
				\item undisciplined annotations possible
				\item can lead to generation of syntactically invalid program variants
			\end{itemize}
		\end{note}
		\begin{note}{What is different with physical separation?}
			\begin{itemize}
				\item features physically separated from each other
				\item dedicated components, services, plug-ins \lecturemodules
				\item dedicated modules, folders, files \lecturelanguages
			\end{itemize}
		\end{note}
	\end{mycolumns}
\end{frame}

\subsubsection*{CIDE}
\begin{frame}{\myframetitle\mysource{\cide}}
	\begin{mycolumns}[widths={60},animation=none]
		\pic[width=\linewidth]{cide-open-editor}
	\mynextcolumn
		\begin{example}{What is CIDE?}
			\begin{itemize}
				\item stands for Colored Integrated Development Environment
				\item colors used to mark features
				\item based on Eclipse~3.5 and FeatureIDE
				\item research prototype (last update in May 2012)
				\item special editors available for several languages:
				ANTLR, Bali, C (experimental), C++ (experimental), C\#, ECMAScript (JavaScript), Featherweight Java, \textbf{Java 1.5}, gCIDE, Haskell, HTML, JavaCC, OSGi Manifest, Properties, Python, and XHTML
			\end{itemize}
		\end{example}
	\end{mycolumns}
\end{frame}
%Featherweight Java 	.fj 	
%Java 	.java 	
%Java (JDT) 	.java 	recommended, supports type checking
%C 	.c, .h 	exact parser, does not understand preprocessor statements
%C (approx) 	.c, .h 	pseudo-parser, does not recognize full structure and not all preprocessor statements; use only one C extension at a time
%C# 	.cs 	
%ECMAScript (JavaScript) 	.js 	
%Haskell 	.hs 	pseudo-parser, skips over expressions
%Bali 	.b 	supports type checking
%ANTLR 	.g 	simple productions only, no options or semantic extensions.
%JavaCC 	.cc 	
%gCIDE 	.gcide 	bootstrapped grammar for gCIDE itself
%Properties 	.properties 	for line-based property files
%HTML and XML 	.html; .xml 	XML parser does not parse doctype declarations yet (not compatible with Eclipse's WST plugins for some reason)
%XHTML 	.xhtml 	version 1.0 strict, does not understand doctype declaration yet; generated from dtd
%XML-People 	.xml 	simple proof-of-concept parser for the following DTD: [people.dtd]; do not use together with XML extension
%Python 	.py 	
%OSGi Manifest 	MANIFEST.MF 	simple, with simple type system

\begin{frame}{\myframetitle\mysource{\cide}}
	\begin{mycolumns}[widths={70},animation=none]
		\pic[width=\linewidth]{cide-editor-and-outline}
	\mynextcolumn
		\begin{example}{Why colors?}
			\begin{itemize}
				\item colors replace preprocessor directives
				\item features relevant for development task are assigned a color
				\item code annotated to a feature by selection and context menu
				\item features visualized by background colors
				\item annotations stored externally (no changes outside the special editor feasible)
			\end{itemize}
		\end{example}
	\end{mycolumns}
\end{frame}

\begin{frame}{\myframetitle\mysource{\cide}}
	\begin{mycolumns}[widths={65},animation=none]
		\pic[width=\linewidth]{cide}
	\mynextcolumn
		\begin{example}{Why virtual separation?}
			\begin{itemize}
				\item source code is a view on the abstract syntax tree (AST)
				\item possible to hide irrelevant features
				\item possible to show overlapping features
				\item supporting development despite scattering and tangling
				\item no need to handle separators and logical connectors:\\\mycite{\texttt{,}}, \mycite{\texttt{||}}
				\item efficient detection of type errors \lectureanalyses
			\end{itemize}
		\end{example}
	\end{mycolumns}
\end{frame}

\begin{frame}{\myframetitle\mysource{\cide}}
	\pic[width=\linewidth]{cide-show-single-feature}
	\begin{example}{}\centering
		\textbf{view on a feature}: possible to only show a single feature -- in its surrounded code
	\end{example}
\end{frame}

\begin{frame}{\myframetitle\mysource{\cide}}
	\begin{mycolumns}[widths={45},animation=none]
		\begin{example}{Why configuration?}
			\begin{itemize}
				\item features specified in FeatureIDE feature model
				\item configuration created in FeatureIDE configuration editor
				\item configuration used to generate and visualize variant
				\item \ldots
			\end{itemize}
		\end{example}
	\mynextcolumn
		\pic[width=\linewidth]{cide-feature-selection}
	\end{mycolumns}
\end{frame}

\begin{frame}{\myframetitle\mysource{\cide}}
	\begin{mycolumns}[widths={45},animation=none]
		\begin{example}{Why configuration?}
			\begin{itemize}
				\item \ldots
				\item \textbf{view on a variant}: variant visualized in source code and project explorer
				\item only necessary to press CIDE button in project explorer
				\item pressing it again returns to the view of the product line
			\end{itemize}
		\end{example}
	\mynextcolumn
		\pic[width=\linewidth]{cide-variant-view-in-project-explorer}
	\end{mycolumns}
\end{frame}
% CIDE literature
% forward reference to physical separation of concerns in next two lectures


\lessonslearned{
	\item preprocessor variability suffers from scattering and tangling
	\item feature traceability can be established with tool support (e.g., Feature Commander)
	\item virtual separation of concerns is an alternative to preprocessors (e.g., CIDE)
}{
	\item \fospl, Section~3.2.2 Feature Traceability
	\item \fospl, Section~7 Advanced, Tool-Driven Variability Mechanisms
}{
	\begin{enumerate}
		\item Why is it beneficial to have feature traceability even for single systems?
		\item Using disciplined annotations, only optional nodes in the abstract syntax tree can be annotated (i.e., assigned to features); if we remove them, no syntax error occurs. What are examples of optional and non-optional (i.e., mandatory) types of nodes in Java?
	\end{enumerate}
}

% TODO Thomas: add exam questions here

\mode<beamer>{
	\begin{frame}{\inserttitle}
		\lectureseriesoverview
	\end{frame}

	\contentoverview
}


\end{document}
