\documentclass[
	aspectratio=169, % default is 43
	8pt, % font size, default is 11pt
	handout, % < do not remove this comment, it is used by the Makefile >
]{beamer}
\def\university{} % < do not remove this comment, it is used by the Makefile >

\documentclass[
	aspectratio=169, % default is 43
	8pt, % font size, default is 11pt
	handout, % handout mode without animations, comment out to add animations
]{beamer}

\usepackage{../template/beamerthemeuulm} % use the inofficial uulm beamer theme
\setfaculty{infIngPsy} % set the color scheme for your faculty here [med/infIngPsy/math/nat]

% requires symbolic links
% git clone git@github.com:SoftVarE-Group/SlideTemplate.git C:\Users\...\SlideTemplate
% mklink /J template C:\Users\...\SlideTemplate
% git clone git@spgit.informatik.uni-ulm.de:thuem/slides.git C:\Users\...\ThomasSlides
% mklink /J thomasslides C:\Users\...\ThomasSlides
\graphicspath{{../template/pics/logos}{../template/pics/nature}{../template/pics/uulm}{../thomasslides/}{../pics/}}

%\usepackage[ngerman]{babel} % use this line for slides in German
%\recordingtrue % special recording mode for use with a greenscreen, gives you space to show yourself in a layer in front of the slides, has no effect in the handout mode

\title{Software Product Lines} % short title is used for the slide footer but optional

%
%
%% IMPORTED PACKAGES
%
%\usepackage{adjustbox} % used for partofpage
%\usepackage{tcolorbox} % used for mydefinition, mynote, myexample
\usepackage{multicol} % used temporarily for the lecture overview
%\usepackage{mathtools} % required for absolute value in modeling lecture
%
%% SLIDE TEMPLATE
%
%\beamertemplatenavigationsymbolsempty 
%
%% COMMANDS TO LAYOUT AND ANNIMATE SLIDES
%
\newcommand{\lessonslearned}[3]{
	\subsection{Summary}
	\begin{frame}{\insertsection -- \insertsubsection}
		\leftorright{
			\mydefinition{Lessons Learned}{
				\begin{itemize}
					#1
				\end{itemize}
			}
			\mynote{Further Reading}{
				\small % references take space, can be a little smaller
				\begin{itemize}
					#2
				\end{itemize}
			}
		}{
			\myexample{Practice}{
				#3
			}
		}
	\end{frame}
}

\renewcommand{\lectureoverview}{
%	\section*{Overview}
%	\subsection*{Overview}
	\begin{frame}{\insertsubtitle}
		\begin{multicols}{2}
			\tableofcontents
		\end{multicols}
	\end{frame}
}

%
%\newcommand{\onlyleft}[1]{
%	\halfpage{#1}
%}
%
%\newcommand{\onlyright}[1]{
%	~\hfill
%	\halfpage{#1}
%}
%
%\newcommand{\leftorright}[2]{
%	\uncover<1>{\halfpage{#1}}
%	\hfill
%	\uncover<3->{\halfpage{#2}}
%}
%
%\newcommand{\rightorleft}[2]{
%	\uncover<3->{\halfpage{#1}}
%	\hfill
%	\uncover<1>{\halfpage{#2}}
%}
%
%\newcommand{\leftthenright}[2]{
%	\halfpage{#1}
%	\hfill\pause
%	\halfpage{#2}
%}
%
%\newcommand{\leftandright}[2]{
%	\halfpage{#1}
%	\hfill
%	\halfpage{#2}
%}
%
%\newcommand{\leftmiddleandright}[3]{
%	\thirdpage{#1}
%	\hfill
%	\thirdpage{#2}
%	\hfill
%	\thirdpage{#3}
%}
%
%\newcommand{\leftmiddleorright}[3]{
%	\uncover<1>{\thirdpage{#1}}
%	\hfill
%	\uncover<3>{\thirdpage{#2}}
%	\hfill
%	\uncover<5->{\thirdpage{#3}}
%}
%
%\newcommand{\halfpage}[1]{\partofpage{48}{#1}}
%
%\newcommand{\thirdpage}[1]{\partofpage{31}{#1}}
%
%\newcommand{\partofpage}[2]{
%	\adjustbox{valign=t}{\begin{minipage}{0.#1\textwidth}
%			\begin{flushleft}
%				#2
%			\end{flushleft}
%	\end{minipage}}
%}
%
%\newcommand{\mydefinition}[2]{
%	\begin{tcolorbox}[title=#1,colback=orange!10,colframe=orange!30,coltitle=black,fonttitle=\bfseries,left=1mm,right=1mm,top=1mm,bottom=1mm]
%		\begin{flushleft}
%			#2
%		\end{flushleft}
%	\end{tcolorbox}
%}
%
%\newcommand{\mydefinitiontight}[2]{
%	\begin{tcolorbox}[title=#1,colback=white,colframe=orange!30,coltitle=black,fonttitle=\bfseries,left=0mm,right=0mm,top=0mm,bottom=0mm]
%		\begin{flushleft}
%			#2
%		\end{flushleft}
%	\end{tcolorbox}
%}
%
%\newcommand{\mynote}[2]{
%	\begin{tcolorbox}[title=#1,colback=red!10,colframe=red!30,coltitle=black,fonttitle=\bfseries,left=1mm,right=1mm,top=1mm,bottom=1mm]
%		\begin{flushleft}
%			#2
%		\end{flushleft}
%	\end{tcolorbox}
%}
%
%\newcommand{\myexample}[2]{
%	\begin{tcolorbox}[title=#1,colback=blue!10,colframe=blue!30,coltitle=black,fonttitle=\bfseries,left=1mm,right=1mm,top=1mm,bottom=1mm]
%		\begin{flushleft}
%			#2
%		\end{flushleft}
%	\end{tcolorbox}
%}
%
%\newcommand{\myexampletight}[2]{
%	\begin{tcolorbox}[title=#1,colback=white,colframe=blue!30,coltitle=black,fonttitle=\bfseries,left=0mm,right=0mm,top=0mm,bottom=0mm]
%		\begin{flushleft}
%			#2
%		\end{flushleft}
%	\end{tcolorbox}
%}

\subtitle{6. Modular Features}
\author{Timo Kehrer, Thomas Thüm, Elias Kuiter}

\ifuniversity{ulm}{\setpicture[250]{may21-q37}}
\ifuniversity{magdeburg}{\setpicture[35]{ovgu-autumn4}\setcopyright{Photo: Hannah Theile (OVGU)}}

\begin{document}

% TITLE SLIDE

\maketitle

% SLIDE TEMPLATE

%\setbeamercolor{title}{fg=black}
%\setbeamercolor{frametitle}{fg=black}
\setbeamertemplate{frametitle}{{\huge~\\\insertsubsection~\insertframetitle}}
\setbeamertemplate{footline}[text line]{\parbox{\linewidth}{\vspace*{-10pt}\hspace{0pt}%
	\insertshortauthor\phantom{g\insertpagenumber}%
	\hfill%
	\inserttitle%
	\ifx \insertsubtitle \empty \else \ -- \insertsubtitle\fi%
	\ifx \insertsectionhead \empty \else \ -- \insertsectionhead\fi%
	\hfill%
	\phantom{g\insertshortauthor}\insertpagenumber%
}}
%\defbeamertemplate{footline}{\begin{beamercolorbox}[sep=1em]{author in head/foot}\insertshortauthor\hfill\insertsection\hfill\insertframenumber\end{beamercolorbox}}
%\defbeamertemplate*{footline}{mytheme}{\begin{beamercolorbox}[sep=1em]{author in head/foot}\insertshortauthor\hfill\insertsection\hfill\insertframenumber\end{beamercolorbox}}

% OVERVIEW SLIDES

\newcommand{\overview}{
	\section*{Overview}
	\subsection*{Overview}
	\begin{frame}{-- \insertsubtitle}
		\begin{multicols}{3}
			\tableofcontents
		\end{multicols}
	
		\begin{flushright}
			\footnotesize
			Author: \insertauthor
			
			Date: \insertdate
		\end{flushright}
	\end{frame}
}
% temporarily added slide to have a lecture overview 
\overview

% temporarily removed
%\begin{frame}{Lecture Overview -- \insertsubtitle}
%	\tableofcontents[hideallsubsections]
%\end{frame}

\AtBeginSection[]{%
	\begin{frame}{Lecture Overview -- \insertsubtitle}
		\tableofcontents[currentsection,hideothersubsections]
	\end{frame}
}

\newcommand{\sectionend}{\addtocontents{toc}{\newpage}}


\begin{frame}{\inserttitle}
	\lectureseriesoverview[6]
\end{frame}

\section{Components}

\subsection{Recap: How to Implement Features?}
\begin{frame}{\myframetitle}
	\begin{mycolumns}
		\myexampletight{Given a feature model for graphs \ldots}{
			\centering\featureDiagramGraphs
			%\featureDiagramLegend
		}
		\myexample{\ldots\ we can derive a valid configuration}{
			\small
			\leftmiddleandright{
				$\{G\}$\\
				$\{G,C\}$\\
				$\{G,D\}$\\
				$\{G,C,D\}$\\
			}{
				$\{G,W\}$\\
				$\{G,C,W\}$\\
				$\{G,D,W\}$\\
				$\{G,C,D,W\}$\\
			}{
				$\{G,W,S\}$\\
				$\{G,C,W,S\}$\\
				$\{G,D,W,S\}$\\
				$\{G,C,D,W,S\}$\\
			}
		}
	\mynextcolumn
		\vspace{-10mm}
		\myexampletight{How to Generate Products Automatically?}{
			\centering\foreach \page in {2,12,4,14,6,16,8,18,10,20,42,44}{\includegraphics[width=.23\linewidth,page=\page]{graphs} }
		}
		\mynote{Goals}{
			\begin{itemize}
				\item descriptive specification of a product (i.e., a configuration, a selection of features)
				\item automated generation of a product with compile-time variability
			\end{itemize}
			Focus of the next three lectures \ldots
		}
	\end{mycolumns}
\end{frame}




\subsection{Recap: UML Component Diagram}
\subsection{Vision: Component Markets}
\subsection{The Library Scaling Problem}
\subsection{Components in Java}
\subsection{Components for Features}
\subsection{Discussion}



\lessonslearned{
	\item Modularity = information hiding and data encapsulation
	\item Components foster a modular software architecture and design
	\item Reuse within and beyond product lines
	\item No automated product derivation, glue code is necessary
}{
	\item \szyperski
	\item \fospl, Chapter 4.4
}{
	\item Consider the feature traceability problem in a component-based implementation. How would you localize a feature? Does it still suffer from scattering, tangling and replication? 
	\item We have introduced a simple color component which we reused in our graph implementation. Could we consider the graph library itself as another component? What about its reusability?
	\item How can product-line engineering help to find the right trade-offs regarding the library scaling problem?
}

\sectionend

\section{Services and Microservices}

\subsection{Services vs. Components}
\subsection{Microservices vs. Services}

\subsection{(German Slides on Microservices)}
\begin{frame}{Microservices und Microservice Architekturen}
	\mycite{A \emph{microservice} is a cohesive, independent process interacting via messages. A \emph{microservice architecture} is a distributed application where all its modules are microservices.}{Dragoni et al., 2017}
	
	\pause
	\mycite{[A \emph{microservice} is a module] implemented and operated as a small yet independent system, offering access to its internal logic and data through a well-defined network interface.}{Jamshidi et al., 2018}
	
	\begin{center}
		\vspace{-12mm}
		\pic[width=.45\linewidth]{peer-to-peer}
	\end{center}
	
	%\zitat{The microservices architecture gained popularity relatively recently and can be considered to be in its infancy since there is still a lack of consensus on what microservices actually are.}{Dragoni et al., 2017}
	
	%\zitat{\emph{Microservices} are small, autonomous services that work together.}{Sam Newman, 2015}
\end{frame}

\begin{frame}{Wie klein sind Microservices?}
	\mycite{[A microservice] could be re-written in \emph{two weeks}.}{\href{https://www.rea-group.com/blog/micro-services-what-even-are-they/}{Jon Eaves, 2014}}
	
	\pause
	\mycite{If the codebase is too big to be \emph{managed by a small team}, looking to break it down is very sensible. [...] The smaller the service, the more you \emph{maximize the benefits and downsides} of microservice architecture.}{Sam Newman, 2015}% [...] The golden rule: can you make a change to a service and deploy it by itself without changing anything else?
	
	\pause
	\begin{columns}
		\column{.4\linewidth}
		\hfill\pic[width=.8\linewidth]{pizza}
		\column{.55\linewidth}
		\href{https://www.theguardian.com/technology/2018/apr/24/the-two-pizza-rule-and-the-secret-of-amazons-success}{\emph{Two-Pizza Rule} by Jeff Bezos}:\\Every team should be small enough that it can be fed with two pizzas.
	\end{columns}
\end{frame}

\begin{frame}{Beispiel: Microservices bei Netflix}
	\large\bigskip\bigskip\bigskip\bigskip\bigskip
	\begin{center}
		\begin{tabular}{rl}
			100s & of microservices\\
			1,000s & of daily production changes\\
			10,000s & of instances\\
			100,000s & of customer interactions per minute\\
			1,000,000s & of customers\\
			%1,000,000,000s & of metrics\\
			10,000,000,000\phantom{s} & hours streamed\\
			10s & of operations engineers
		\end{tabular}
	\end{center}
	
	\bigskip\bigskip\bigskip\bigskip\hfill\tiny Quelle: \href{https://fr.slideshare.net/AmazonWebServices/dvo203-the-life-of-a-netflix-engineer-using-37-of-the-internet}{https://fr.slideshare.net/AmazonWebServices/dvo203-the-life-of-a-netflix-engineer-using-37-of-the-internet}
\end{frame}

\begin{frame}{Prinzipien von Microservices}
	\emph{Gesetz von Conway}: \mycite{Organizations which design systems [...] are constrained to produce designs which are copies of the communication structures of these organizations.}{\href{https://www.melconway.com/Home/pdf/committees.pdf}{Melvin Edward Conway, 1968}}
	
	\pause
	\emph{Single Responsibility Principle}: \mycite{Gather together the things that change for the same reasons. Separate those things that change for different reasons.}{\href{https://blog.cleancoder.com/uncle-bob/2014/05/08/SingleReponsibilityPrinciple.html}{Robert C. Martin, 2014}}
	
	\pause
	\mycite{\emph{You build it, you run it.}}{Amazon CTO Werner Vogel, 2006}
	\begin{itemize}
		\item Team übernimmt Entwicklung, Betrieb und Wartung (vgl. DevOps)
		\item Besseres Verständnis für Potential und Leistungsfähigkeit
	\end{itemize}
\end{frame}

\begin{frame}{Vorteile von Microservices}
	\begin{itemize}
		\item Heterogenität: einfache Anwendung von neuen Technologien/Programmiersprachen
		\item Continuous Integration/Deployment: bei Änderungen müssen nur einzelne Microservices ausgeliefert oder bei Fehlfunktionen zurückgesetzt werden (Rollback)
		\item Skalierbarkeit: Anzahl von Instanzen wählbar nach Rechenaufwand pro Microservice
		\pause
		\item Effizienz und Performance: Konfiguration von Ausführungsumgebung kann pro Microservice optimiert werden
		\item Wiederverwendung: Microservices können in beliebig vielen anderen Microservices verwendet werden
		\item Modernisierung: Ersetzung oder Entfernung einfacher möglich
	\end{itemize}
\end{frame}

\begin{frame}{Nachteile von Microservices}
	\begin{itemize}
		\item Latenz im Netzwerk größer als die von gemeinsam genutztem Arbeitsspeicher
		\item Netzwerksecurity erfordert Mehraufwand für Verschlüsselung und Authentifizierung
		\item Integrationstest kann sehr aufwändig werden
		\pause
		\item Verschieben von Code kann neue Programmiersprache erfordern
		\item Änderungen an Teams schwierig durch Heterogenität
		\item Ggfs. zu viele Microservices
		\item Datenkonsistenz schwierig herzustellen wenn mehrere Microservices verändert werden
	\end{itemize}
\end{frame}

\begin{frame}{Der Hype um Microservices}
	\pic[width=\linewidth]{microservices}
\end{frame}

\begin{frame}{Der Hype um Microservices}
	\pic[width=\linewidth]{microservices-comparison}
\end{frame}

\begin{frame}{Vergleich mit Serviceorientierten Architekturen}
	\begin{columns}[t]
		\column{.35\linewidth}
		Gemeinsamkeiten
		\begin{itemize}
			\item Services werden im Betriebsystem als eigenständige Prozess angesehen
			\item Kommunikation über Netzwerke
		\end{itemize}
		\column{.55\linewidth}
		Unterschiede
		\begin{itemize}
			\item Microservices sind ein spezifischer Ansatz für SOA
			\item Microservices sind meist kleiner als Services (feingranulares SOA)
			\item Orchestrierung: SOA vernetzt sich über zentralen Service
			\item Choreographie: Microservices kommunizieren durch Publish-Subscribe-Prinzip
		\end{itemize}
	\end{columns}
\end{frame}

\begin{frame}{Vergleich mit Komponenten/Bibliotheken/Plugins}
	\begin{columns}[t]
		\column{.35\linewidth}
		Gemeinsamkeiten
		\begin{itemize}
			\item Modularisierung
			\item Wiederverwendung
		\end{itemize}
		\column{.55\linewidth}
		Unterschiede
		\begin{itemize}
			\item Microservices unterstützen Heterogenität (z.B. verschiedene Programmiersprachen)
			\item Microservices haben keinen gemeinsamen Arbeitsspeicher
			\item Update von Microservices erfordert keinen Neustart der Gesamtapplikation
		\end{itemize}
	\end{columns}
\end{frame}



\lessonslearned{
	\item Services are another kind of module implemented and operated independently of each other 
	\item Microservices have a clear philosophy regarding their size, driven by organizational constraints
	\item Reuse within and beyond product lines
	\item No automated product derivation, orchestration or choreography is necessary
}{
	\item \fospl, Chapter 4.4
}{
	\item We have talked a lot about promises of microservices. Do you also see any drawbacks?
	\item In a component-based product-line implementation, practitioners often rely on clone-and-own for glue code. How could we handle variability in service orchestrations?
}

\sectionend

\section{Frameworks with Plug-Ins}

\subsection{Hot Spots and Plug-Ins}
\begin{frame}{Framework with Plug-Ins}
	\begin{mycolumns}[widths={50,50}]
		\mydefinition{Framework and Hot Spot \mysource{\fospl\mypages{80--81}}}{
			A \emph{framework} is a set of classes that embodies an abstract design for solutions to a family of related problems, and supports reuse at a larger granularity than classes. A framework is open for \emph{extension} at explicit \emph{hot spots}.				
		}
		\mydefinition{Plug-In \mysource{\fospl\mypages{80--81}}}{
			A \emph{plug-in} extends hot spots of a [...] framework with custom behavior. A plug-in can be separately compiled and deployed.	
		}
		\mynote{}{
			Frameworks with plug-ins are also called "`blackbox frameworks"': Developers need to understand interfaces, but not the internal framework implementation.
		}
	\mynextcolumn
		\vspace{-0.7cm}
		\mynote{Inversion of Control}{
			Hollywood principle: ``Don't call us, we call you''\\
			\centering 
			\pic[width=0.75\linewidth]{library_vs_framework}
		}
		\mynote{}{
			\begin{itemize}
				\item Can be understood in terms of the observer and/or strategy pattern: The framework exposes explicit hot spots, at which plug-ins can register observers and strategies.
				\item Requires \emph{preplanning} for possible future extensions
			\end{itemize}
		}		
	\end{mycolumns}
\end{frame}

\subsection{Preplanned Extensions}
\begin{frame}{Real-World Example: Preplanned Bike Extensions}
	\begin{mycolumns}[columns=4,T]
		\centering\pic[width=\linewidth]{bike-extensionpoint1}

		bike lock
	\mynextcolumn
		\centering\pic[width=\linewidth]{bike-extensionpoint2}

		front wheel brake
	\mynextcolumn
		\centering\pic[width=\linewidth]{bike-extensionpoint3}

		rear wheel brake
	\mynextcolumn
		\centering\pic[width=\linewidth]{bike-extensionpoint4}

		kickstand
	\end{mycolumns}
\end{frame}

\begin{frame}{Real-World Example: Preplanned Bike Extensions}
	\begin{mycolumns}[columns=5,widths={30,5,30,5,30}]
		\centering\pic[width=\linewidth]{bike-extensionpoint1-withoutplugin}

		framework with extension~points
	\mynextcolumn
		\centering\Huge +
	\mynextcolumn
		\centering\pic[width=\linewidth]{bike-lock}

		plug-ins\\~
	\mynextcolumn
		\centering\Huge =
	\mynextcolumn
		\centering\pic[width=\linewidth]{bike-extensionpoint1}

		framework with plug-ins\\~
	\end{mycolumns}
\end{frame}

\subsection{Basic Design Principles}
\begin{frame}{Simple Java Example: A Family of Dialogs \mytitlesource{\fospl}}
	\begin{mycolumns}[widths={50,50}]
		\myexampletight{}{
			\centering \pic[width=0.8\linewidth]{dialog1}
		}
		\myexampletight{}{
			\centering \pic[width=0.8\linewidth]{dialog2}
		}
		\myexampletight{}{
			\centering \pic[width=\linewidth]{dialog3}
		}
	\mynextcolumn		
		\mynote{}{
			\begin{itemize}
				\item All dialogs have a similar structure (basically textfield + button)
				\item Large parts of the source code are identical:
				\begin{itemize}
					\item Main method,
					\item Initialization of windows, textfield and button
					\item layouting,
					\item window closing and disposal,
					\item etc.
				\end{itemize}
			\end{itemize}
		}		
	\end{mycolumns}
\end{frame}
	
\begin{frame}[fragile]{Simple Java Example: A Family of Dialogs}
	\small\begin{mycolumns}[columns=2,widths={50,50}]
\begin{codetight}{}
public class Calc extends JFrame {
	private JTextField textfield;
	public static void main(String[] args) {
		new Calc().setVisible(true);
	}
	public Calc() { init(); }
	protected void init() {
		JPanel p = new JPanel(new BorderLayout());
		p.setBorder(new BevelBorder(/* ... */));
		JButton button = new JButton();
		button.setText(@"calculate"@);
		p.add(button, BorderLayout.EAST);
		textfield = new JTextField("");
		textfield.setText(@"10 / 2 + 6"@);
		textfield.setPreferredSize(new Dimension(350, 40));
		p.add(textfield, BorderLayout.WEST);
		button.addActionListener(new ActionListener() 
			{ @/* calculate */@ });
		this.setContentPane(p);
		this.setTitle(@"My Great Calculator"@);
		this.pack();
		// ...
	}
}
\end{codetight}
		\mynextcolumn
\begin{codetight}{}
public class Ping extends JFrame {
	private JTextField textfield;
	public static void main(String[] args) {
		new Calc().setVisible(true);
	}
	public Calc() { init(); }
	protected void init() {
		JPanel p = new JPanel(new BorderLayout());
		p.setBorder(new BevelBorder(/* ... */));
		JButton button = new JButton();
		button.setText(@"ping"@);
		p.add(button, BorderLayout.EAST);
		textfield = new JTextField("");
		textfield.setText(@"127.0.0.1"@);
		textfield.setPreferredSize(new Dimension(350, 40));
		p.add(textfield, BorderLayout.WEST);
		button.addActionListener(new ActionListener() 
			{ @/* calculate */@ });
		this.setContentPane(p);
		this.setTitle(@"Ping"@);
		this.pack();	
		// ...
	}
}
\end{codetight}
	\end{mycolumns}
\end{frame}


\begin{frame}[fragile]{Simple Java Example: A Family of Dialogs}
	\begin{mycolumns}[columns=2,widths={50,50}]
		\mynote{}{Plug-in implementation hidden from application.}
\tiny
\begin{codetight}{}
public class Application extends JFrame {
	private Plugin plugin;
	// ...
	public Application(@Plugin plugin@) {
		@this.plugin = plugin;
		plugin.setApplication(this);@
		init();
	}
	protected void init() {
		JPanel p = new JPanel(new BorderLayout());
		p.setBorder(new BevelBorder(/*...*/);
		JButton button = new JButton();
		button.setText(@plugin.getButtonText()@);
		p.add(button, BorderLayout.EAST);
		textfield = new JTextField("");
		textfield.setText(@plugin.getInitialText()@);
		textfield.setPreferredSize(new Dimension(200, 20));
		p.add(textfield, BorderLayout.WEST);		
		button.addActionListener(/*... @plugin.buttonClicked();@ ...*/);
		this.setContentPane(p);		
		this.setTitle(@plugin.getApplicationTitle()@);
		// ...
	}
	@public String getInput() {
		return textfield.getText();
	}@
}
\end{codetight}
		\mynextcolumn
{\tiny
\begin{codetight}{}
public interface Plugin {
	String getApplicationTitle();
	String getButtonText();
	String getInitialText();
	void buttonClicked();
	void setApplication(Application app);
}
\end{codetight}
\begin{codetight}{}
public class CalcPlugin implements Plugin {
	private Application application;

	public String getApplicationTitle() {
		return "My Great Calculator";
	}
	public String getButtonText() {
		return "calculate";
	}
	public String getInitialText() {
		return "10 / 2 + 6";
	}
	public void buttonClicked() {
		calculate(application.getInput());
	}
	public void setApplication(Application app) {
		application = app;
	}
	private void calculate(String expression) {
		/* calculate */
	}
}
\end{codetight}
}
	\end{mycolumns}
\end{frame}

\begin{frame}[fragile]{Simple Example: A Family of Dialogs}
	\begin{mycolumns}[columns=2,widths={50,50}]
		\mynote{}{Application implementation hidden from plug-in.}
\tiny
\begin{codetight}{}
public class Application extends JFrame @implements InputProvider@ {
	private Plugin plugin;
	// ...
	public Application(Plugin plugin) {
		this.plugin = plugin;
		@plugin.setInputProvider(this);@
		init();
	}
	protected void init() {
		JPanel p = new JPanel(new BorderLayout());
		p.setBorder(new BevelBorder(/*...*/);
		JButton button = new JButton();
		button.setText(plugin.getButtonText());
		p.add(button, BorderLayout.EAST);
		textfield = new JTextField("");
		textfield.setText(plugin.getInitialText());
		textfield.setPreferredSize(new Dimension(200, 20));
		p.add(textfield, BorderLayout.WEST);		
		button.addActionListener(/* . . . plugin.buttonClicked(); . . . */);		
		// ...
	}
	public String getInput() {
		return textfield.getText();
	}
}
\end{codetight}
		\mynextcolumn
{\tiny
\begin{codetight}{}
public interface InputProvider {
	public String getInput();
}
\end{codetight}
\begin{codetight}{}
public interface Plugin {
	String getApplicationTitle();
	String getButtonText();
	String getInitialText();
	void buttonClicked();
	@void setInputProvider(InputProvider p);@
}
\end{codetight}
\begin{codetight}{}
public class CalcPlugin implements Plugin {
	@private InputProvider inputProvider;@

	public String getApplicationTitle() { /*...*/ }
	public String getButtonText() { /*...*/ }
	public String getInitialText() { /*...*/ }
	public void buttonClicked() {
		calculate(@inputProvider.getInput()@);
	}
	@public void setInputProvider(InputProvider p) {
		inputProvider = p;
	}@
	private void calculate(String expression) {
		/* calculate */
	}
}
\end{codetight}
}
	\end{mycolumns}
\end{frame}

\subsection{Plug-In Loading and Management}
\begin{frame}{Plug-In Loading and Management}
	\begin{mycolumns}[widths={50,50}]
		\mynote{Simple example vs.\ reality}{
			Simplification in our previous example:
			\begin{itemize}
				\item A single extension point supporting the registration of a single plug-in 
				\item Plug-in implementation known at compile-time 
			\end{itemize}
			Typical requirements in practice:
			\begin{itemize}
				\item Many extension points supporting the registration of arbitrarily many plug-ins 
				\item Plug-in implementation provided by third parties (requires load-time variability)
			\end{itemize}
		}		
	\mynextcolumn
		\mydefinition{Plug-in Loader}{
			\begin{itemize}
				\item Searches in a dedicated directory for DLL/JAR/XML files
				\item Tests whether file implements a plug-in
				\item Checks dependencies
				\item Initializes plug-ins
			\end{itemize}
		}
		\mydefinition{Plug-In Manager}{
			\begin{itemize}
				\item GUI and/or console interface for plug-in administration and configuration.
			\end{itemize}
		}
	\end{mycolumns}
\end{frame}

\begin{frame}[fragile]{Example: Plug-In Loading and Management}
	\small\begin{mycolumns}[columns=2,widths={50,50}]
\begin{codetight}{Plug-In Loader using Java Reflection}
public class Starter {
	public static void main(String[] args) {
		if (args.length != 1)
			System.out.println("Plugin name not specified");
		else {
			String pluginName = args[0];
			try {
				Class pluginClass = Class.forName(pluginName);
				new Application((Plugin) 
					pluginClass.newInstance()).setVisible(true);
			} catch (Exception e) {
				System.out.println("Cannot load plugin " + 
					pluginName + ", reason: " + e);
			}
		}
	}
}
\end{codetight}
		\mynextcolumn
\begin{codetight}{Handling multiple Plug-Ins}
public class Application {
	private List<Plugin> plugins;

	public Application(List<Plugin> plugins) {
		this.plugins = plugins;
		for (Plugin plugin : plugins) {
			plugin.init();
		}
	}

	public Message processMsg(Message msg) {
		for (Plugin plugin : plugins) {
			msg = plugin.process(msg);
			// ...
		}
		return msg;
	}
}
\end{codetight}
	\end{mycolumns}
\end{frame}

\subsection{Frameworks in the Wild}
\begin{frame}{Frameworks in the Wild: Eclipse}
	\begin{mycolumns}[widths={70,30},animation=none]
		\pic[width=\linewidth]{eclipse_overview}
	\mynextcolumn
		\vspace{-0.7cm}
		\myexample{Versatile IDE}{
			\begin{itemize}
				\item Lots of common functionality required by any IDE
					(e.g., editors, incremental project build, etc.) 
				\item Only language-specific extensions need to be registered
					(e.g., syntax highlighting, compiler, etc.)
			\end{itemize}
		}
		\pause
		\mynote{Specifically in Eclipse}{
			\begin{itemize}
				\item Actually a set of (recursively nested) frameworks
				\item Largely declarative description of extension points
			\end{itemize}
		}
	\end{mycolumns}
\end{frame}

\begin{frame}{Frameworks in the Wild: Eclipse \mytitlesource{\featureide}}
	\leftorright{
		\pic[width=\linewidth]{framework-with-plugins}
	}{
		\pic[width=\linewidth]{framework-a-plugin}
	}
\end{frame}

\begin{frame}{Frameworks in the Wild: Further Examples}
	\begin{mycolumns}[widths={75,25},animation=none]
		\myexample{}{
			\begin{itemize}
				\item Other IDEs (e.g. IntelliJ, ...)
				\item Unit test frameworks (e.g., JUnit, ...)
				\item Frontend frameworks (e.g., React, Angular ...)
				\item Backend frameworks (e.g., Spring Boot, Ruby on Rails, Django, ...) 
				\item Multimedia frameworks (e.g., DirectX, ...)
				\item Raster/vector graphics editors (e.g., Adobe Photoshop, MS Visio, ...)
				\item Instant messenger frameworks (z.B. Miranda, Trillian)
				\item Compiler frameworks (z.B. LLVM, Polyglot, abc, JustAddJ)
				\item Web browsers (e.g., Firefox, ...)
				\item E-Mail clients (e.g., Thunderbird, ...)
				\item etc.
			\end{itemize}
		}		
	\mynextcolumn
		~
	\end{mycolumns}
\end{frame}

\subsection{Implementation of Product Lines}
\begin{frame}{Framework-Based Implementation of Software Product Lines}
	\begin{mycolumns}[widths={40,60},animation=none]
		\myexample{Recap: Service-Based Implementation}{
			\pic[width=.23\linewidth,height=1.0cm]{lego_components} 
				\vspace*{\fill}
					$+$ 
				\vspace*{\fill}	
			\pic[width=.23\linewidth,height=1.0cm]{lego_orchestration}
				\vspace*{\fill}
					$=$ 
				\vspace*{\fill}	
			\pic[width=.3\linewidth,height=1.0cm]{lego_product}
		}		
		\myexample{Still needs some specification of ``composition'' (cf.\ orchestration vs.\ choreography)}{
			\centering
			\pic[width=.65\linewidth,height=2.5cm]{lego_orchestration}				
		}				
	\mynextcolumn		
		\pause
		\mydefinition{Same Idea}{
			\begin{itemize}
				\item Features are implemented by different plug-ins
				\item Feature selection determines the plug-ins to be loaded and registered 
			\end{itemize}
		}
		\pause
		\mynote{}{
				However: Neither glue code nor explicit service composition required.
		}
		\myexample{}{
			\pic[width=.31\linewidth,height=1.75cm]{lego_product_partial} 
				\vspace*{\fill}
					$+$ 
				\vspace*{\fill}	
			\pic[width=.27\linewidth,height=1.75cm]{lego_components}
				\vspace*{\fill}
					$=$ 
				\vspace*{\fill}	
			\pic[width=.31\linewidth,height=1.75cm]{lego_product}
		}	
		\pause
		\mynote{}{
				But: Full automation comes at a price (s.\ preplanning problem).
		}			
	\end{mycolumns}	
\end{frame}

\begin{frame}[fragile]{Example: Extending Basic Graphs by Plug-Ins?}
	\tiny\begin{mycolumns}[columns=2,widths={50,50}]
\begin{codetight}{}
public class Graph {
	@private List<GraphPlugin> plugins = new ArrayList<GraphPlugin>();@
	// ...	
	@public void registerPlugin(GraphPlugin p){
		plugins.add(p);
	}@
	public void addNode(int id, Color c){
		Node n = new Node(id);
		@notifyAdd(n, c);@
		nodes.add(n);
	}
	public void print() {
		for (Node n : nodes) {
			@notifyPrint(n);@
			// ...
		}
		// ...
	}
	@private void notifyAdd(Node n, Color c) {
		for (GraphPlugin p : plugins) {
			p.aboutToAdd(n, c);
		}
	}
	private void notifyPrint(Node n) {
		for (GraphPlugin p : plugins) {
			p.aboutToPrint(n);
		}
	}@
	// ...
}
\end{codetight}
		\mynextcolumn
\begin{codetight}{}
public interface GraphPlugin {
	public void aboutToAdd(Node n, Color c);
	public void aboutToAdd(Edge e, Weight w);
	public void aboutToPrint(Node n);
	public void aboutToPrint(Edge e);
}
\end{codetight}
\begin{codetight}{}
public class ColorPlugin implements GraphPlugin {
	private Map<Node, Color> map = new HashMap<Node, Color>();

	public void aboutToAdd(Node n, Color c) {
		map.put(n, c);
	}
	
	public void aboutToAdd(Edge e, Weight w) {
		~// do nothing~
	}
	
	public void aboutToPrint(Node n) {
		Color c = map.get(n);
		Color.setDisplayColor(c);
	}
	
	public void aboutToPrint(Edge e) {
		~// do nothing~
	}
}
\end{codetight}
	\end{mycolumns}
\end{frame}

\subsection{Discussion}
\begin{frame}{Challenges and Problems}
	\begin{mycolumns}[widths={50,50}]
		\myexample{}{
			In our example, we can observe that:
			\begin{itemize}
				\item There are lots of empty methods in the ColorPlugin 
				\item The Framework consults all registered plug-ins before printing a node or edge
			\end{itemize}
		}		
		\mydefinition{General Challenge: Cross-cutting Concerns}{
			Implementing cross-cutting concerns as plug-ins 
			\begin{itemize}				
				\item typically leads to huge interfaces, large parts of which are irrelevant for a dedicated plug-in 
				\item causes lots of communication overhead between plug-ins and framework
			\end{itemize}
		}
	\mynextcolumn
		\myexample{}{
			If we were not familiar with our graph library, would we anticipate that:
			\begin{itemize}
				\item Colors and weights should be part of the Plugin interface?
				\item Every plug-in needs to be notified that the framework is about to print a node or edge? 
			\end{itemize}
		}
		\mydefinition{Generally known as Preplanning Problem}{
			\begin{itemize}
				\item Hard to identify and foresee the relevant hot spots and nature of extensions
				\item Developing a framework needs lots of expertise and excellent domain knowledge 
			\end{itemize}
		}	
	\end{mycolumns}
\end{frame}


\lessonslearned{
	\item A framework is open to extension by integrating plug-ins at explicit hot spots.
	\item The framework takes full control over all plug-ins (load-time and run-time).
	\item Enables full automation but requires preplanning effort.
}{
	\item \fospl, Chapter 4.3
}{
	\item Remember two fundamental principles in object-oriented programming; how do they relate to frameworks with plug-ins?
	\begin{itemize}
		\item Interface Segregation Principle: ``Clients should not be forced to depend upon interfaces that they do not use''.
		\item Open-Closed Principle: ``Classes should be open for extension but closed for modification''.
	\end{itemize}
	\item In general: Where are the limitations of a modular approach?
}

\mode<beamer>{
	\begin{frame}{\inserttitle}
		\lectureseriesoverview
	\end{frame}

	\contentoverview
}


\end{document}
