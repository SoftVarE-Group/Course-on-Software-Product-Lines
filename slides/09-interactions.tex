\documentclass[
	aspectratio=169, % default is 43
	8pt, % font size, default is 11pt
	handout, % handout mode without animations, comment out to add animations
]{beamer}

\documentclass[
	aspectratio=169, % default is 43
	8pt, % font size, default is 11pt
	handout, % handout mode without animations, comment out to add animations
]{beamer}

\usepackage{../template/beamerthemeuulm} % use the inofficial uulm beamer theme
\setfaculty{infIngPsy} % set the color scheme for your faculty here [med/infIngPsy/math/nat]

% requires symbolic links
% git clone git@github.com:SoftVarE-Group/SlideTemplate.git C:\Users\...\SlideTemplate
% mklink /J template C:\Users\...\SlideTemplate
% git clone git@spgit.informatik.uni-ulm.de:thuem/slides.git C:\Users\...\ThomasSlides
% mklink /J thomasslides C:\Users\...\ThomasSlides
\graphicspath{{../template/pics/logos}{../template/pics/nature}{../template/pics/uulm}{../thomasslides/}{../pics/}}

%\usepackage[ngerman]{babel} % use this line for slides in German
%\recordingtrue % special recording mode for use with a greenscreen, gives you space to show yourself in a layer in front of the slides, has no effect in the handout mode

\title{Software Product Lines} % short title is used for the slide footer but optional

%
%
%% IMPORTED PACKAGES
%
%\usepackage{adjustbox} % used for partofpage
%\usepackage{tcolorbox} % used for mydefinition, mynote, myexample
\usepackage{multicol} % used temporarily for the lecture overview
%\usepackage{mathtools} % required for absolute value in modeling lecture
%
%% SLIDE TEMPLATE
%
%\beamertemplatenavigationsymbolsempty 
%
%% COMMANDS TO LAYOUT AND ANNIMATE SLIDES
%
\newcommand{\lessonslearned}[3]{
	\subsection{Summary}
	\begin{frame}{\insertsection -- \insertsubsection}
		\leftorright{
			\mydefinition{Lessons Learned}{
				\begin{itemize}
					#1
				\end{itemize}
			}
			\mynote{Further Reading}{
				\small % references take space, can be a little smaller
				\begin{itemize}
					#2
				\end{itemize}
			}
		}{
			\myexample{Practice}{
				#3
			}
		}
	\end{frame}
}

\renewcommand{\lectureoverview}{
%	\section*{Overview}
%	\subsection*{Overview}
	\begin{frame}{\insertsubtitle}
		\begin{multicols}{2}
			\tableofcontents
		\end{multicols}
	\end{frame}
}

%
%\newcommand{\onlyleft}[1]{
%	\halfpage{#1}
%}
%
%\newcommand{\onlyright}[1]{
%	~\hfill
%	\halfpage{#1}
%}
%
%\newcommand{\leftorright}[2]{
%	\uncover<1>{\halfpage{#1}}
%	\hfill
%	\uncover<3->{\halfpage{#2}}
%}
%
%\newcommand{\rightorleft}[2]{
%	\uncover<3->{\halfpage{#1}}
%	\hfill
%	\uncover<1>{\halfpage{#2}}
%}
%
%\newcommand{\leftthenright}[2]{
%	\halfpage{#1}
%	\hfill\pause
%	\halfpage{#2}
%}
%
%\newcommand{\leftandright}[2]{
%	\halfpage{#1}
%	\hfill
%	\halfpage{#2}
%}
%
%\newcommand{\leftmiddleandright}[3]{
%	\thirdpage{#1}
%	\hfill
%	\thirdpage{#2}
%	\hfill
%	\thirdpage{#3}
%}
%
%\newcommand{\leftmiddleorright}[3]{
%	\uncover<1>{\thirdpage{#1}}
%	\hfill
%	\uncover<3>{\thirdpage{#2}}
%	\hfill
%	\uncover<5->{\thirdpage{#3}}
%}
%
%\newcommand{\halfpage}[1]{\partofpage{48}{#1}}
%
%\newcommand{\thirdpage}[1]{\partofpage{31}{#1}}
%
%\newcommand{\partofpage}[2]{
%	\adjustbox{valign=t}{\begin{minipage}{0.#1\textwidth}
%			\begin{flushleft}
%				#2
%			\end{flushleft}
%	\end{minipage}}
%}
%
%\newcommand{\mydefinition}[2]{
%	\begin{tcolorbox}[title=#1,colback=orange!10,colframe=orange!30,coltitle=black,fonttitle=\bfseries,left=1mm,right=1mm,top=1mm,bottom=1mm]
%		\begin{flushleft}
%			#2
%		\end{flushleft}
%	\end{tcolorbox}
%}
%
%\newcommand{\mydefinitiontight}[2]{
%	\begin{tcolorbox}[title=#1,colback=white,colframe=orange!30,coltitle=black,fonttitle=\bfseries,left=0mm,right=0mm,top=0mm,bottom=0mm]
%		\begin{flushleft}
%			#2
%		\end{flushleft}
%	\end{tcolorbox}
%}
%
%\newcommand{\mynote}[2]{
%	\begin{tcolorbox}[title=#1,colback=red!10,colframe=red!30,coltitle=black,fonttitle=\bfseries,left=1mm,right=1mm,top=1mm,bottom=1mm]
%		\begin{flushleft}
%			#2
%		\end{flushleft}
%	\end{tcolorbox}
%}
%
%\newcommand{\myexample}[2]{
%	\begin{tcolorbox}[title=#1,colback=blue!10,colframe=blue!30,coltitle=black,fonttitle=\bfseries,left=1mm,right=1mm,top=1mm,bottom=1mm]
%		\begin{flushleft}
%			#2
%		\end{flushleft}
%	\end{tcolorbox}
%}
%
%\newcommand{\myexampletight}[2]{
%	\begin{tcolorbox}[title=#1,colback=white,colframe=blue!30,coltitle=black,fonttitle=\bfseries,left=0mm,right=0mm,top=0mm,bottom=0mm]
%		\begin{flushleft}
%			#2
%		\end{flushleft}
%	\end{tcolorbox}
%}

\subtitle{9. Feature Interactions}
\author{Thomas Thüm}

\begin{document}

% TITLE SLIDE

\maketitle

% SLIDE TEMPLATE

%\setbeamercolor{title}{fg=black}
%\setbeamercolor{frametitle}{fg=black}
\setbeamertemplate{frametitle}{{\huge~\\\insertsubsection~\insertframetitle}}
\setbeamertemplate{footline}[text line]{\parbox{\linewidth}{\vspace*{-10pt}\hspace{0pt}%
	\insertshortauthor\phantom{g\insertpagenumber}%
	\hfill%
	\inserttitle%
	\ifx \insertsubtitle \empty \else \ -- \insertsubtitle\fi%
	\ifx \insertsectionhead \empty \else \ -- \insertsectionhead\fi%
	\hfill%
	\phantom{g\insertshortauthor}\insertpagenumber%
}}
%\defbeamertemplate{footline}{\begin{beamercolorbox}[sep=1em]{author in head/foot}\insertshortauthor\hfill\insertsection\hfill\insertframenumber\end{beamercolorbox}}
%\defbeamertemplate*{footline}{mytheme}{\begin{beamercolorbox}[sep=1em]{author in head/foot}\insertshortauthor\hfill\insertsection\hfill\insertframenumber\end{beamercolorbox}}

% OVERVIEW SLIDES

\newcommand{\overview}{
	\section*{Overview}
	\subsection*{Overview}
	\begin{frame}{-- \insertsubtitle}
		\begin{multicols}{3}
			\tableofcontents
		\end{multicols}
	
		\begin{flushright}
			\footnotesize
			Author: \insertauthor
			
			Date: \insertdate
		\end{flushright}
	\end{frame}
}
% temporarily added slide to have a lecture overview 
\overview

% temporarily removed
%\begin{frame}{Lecture Overview -- \insertsubtitle}
%	\tableofcontents[hideallsubsections]
%\end{frame}

\AtBeginSection[]{%
	\begin{frame}{Lecture Overview -- \insertsubtitle}
		\tableofcontents[currentsection,hideothersubsections]
	\end{frame}
}

\newcommand{\sectionend}{\addtocontents{toc}{\newpage}}


\section{What is a Feature Interaction?}

\subsection{Examples for Feature Interactions}
% skype+baby monitor
% beamer+notebook
% esp+abs: both control brakes and motor
\begin{frame}{A Common Interaction of Toasters}
	\leftandright{
		\only<1|handout:0>{\includegraphics[width=\linewidth]{toast1}}%
		\only<2|handout:0>{\includegraphics[width=\linewidth]{toast2}}%
		\only<3-|handout:1>{\includegraphics[width=\linewidth]{toast3}}%
		\uncover<5->{\myexample{}{\centering no interaction for \emph{$T_1 \pand T_2$} and  $\pnot T_1 \pand \pnot T_2$}}
	}{
		\only<4->{\includegraphics[width=\linewidth]{toast4}}%
		\uncover<6->{\myexample{}{\centering unwanted interaction for \emph{$T_1 \pand \pnot T_2$} and  $\pnot T_1 \pand T_2$}}
	}
\end{frame}

\begin{frame}{An Interaction of Clothes}
	\centering\includegraphics[width=\linewidth,page=14,trim=40 30 40 70,clip]{2021/2021-09-08-SPLC-Keynote}
\end{frame}
\begin{frame}{An Interaction of Clothes}
	\centering\includegraphics[width=\linewidth,page=15,trim=40 30 40 70,clip]{2021/2021-09-08-SPLC-Keynote}
\end{frame}
\begin{frame}{An Interaction of Clothes}
	\centering\includegraphics[width=\linewidth,page=16,trim=40 30 40 70,clip]{2021/2021-09-08-SPLC-Keynote}
\end{frame}

\begin{frame}{An Interaction of Android Apps}
	\centering\includegraphics[width=\linewidth,page=11,trim=40 30 40 70,clip]{2021/2021-09-08-SPLC-Keynote}
\end{frame}

\begin{frame}{Can We Trust Our Scans?}
	\centering\includegraphics[width=\linewidth,page=28,trim=40 30 40 70,clip]{2021/2021-09-08-SPLC-Keynote}
\end{frame}

\begin{frame}{Known Interactions of Lenovo Hardware}
	\centering\includegraphics[height=\textheightwithtitle,page=19]{2021/2021-09-08-SPLC-Keynote}
\end{frame}
\begin{frame}{Known Interactions of Lenovo Hardware}
	\centering\includegraphics[height=\textheightwithtitle,page=20]{2021/2021-09-08-SPLC-Keynote}
\end{frame}
\begin{frame}{Known Interactions of Lenovo Hardware}
	\centering\includegraphics[height=\textheightwithtitle,page=21]{2021/2021-09-08-SPLC-Keynote}
\end{frame}
\begin{frame}
	\centering\includegraphics[width=\linewidth,page=22,trim=40 20 40 0,clip]{2021/2021-09-08-SPLC-Keynote}
\end{frame}

\subsection{Static and Dynamic Interactions}
\begin{frame}{Interactions with Preprocessors}
	\centering
	\includegraphics[height=\textheightwithtitle,page=7,trim=20 20 20 40,clip]{FeatureIDE with Graphs/2016-05-19 ICSE FeatureIDEdemo}
\end{frame}
\begin{frame}{No Interaction}
	\centering
	\includegraphics[height=\textheightwithtitle,page=8,trim=20 20 20 40,clip]{FeatureIDE with Graphs/2016-05-19 ICSE FeatureIDEdemo}
\end{frame}
\begin{frame}{A Static Interaction}
	\centering
	\includegraphics[height=\textheightwithtitle,page=9,trim=20 20 20 40,clip]{FeatureIDE with Graphs/2016-05-19 ICSE FeatureIDEdemo}
\end{frame}
\begin{frame}{Another Static Interaction}
	\centering
	\includegraphics[height=\textheightwithtitle,page=10,trim=20 20 20 40,clip]{FeatureIDE with Graphs/2016-05-19 ICSE FeatureIDEdemo}
\end{frame}
\begin{frame}{A Dynamic Interaction}
	\centering
	\includegraphics[height=\textheightwithtitle,page=11,trim=20 20 20 40,clip]{FeatureIDE with Graphs/2016-05-19 ICSE FeatureIDEdemo}
\end{frame}

\subsection{Unwanted and Wanted Interactions} % Desired + Undesired
% \href{https://github.com/SoftVarE-Group/Slides/blob/main/2021/2021-09-08-SPLC-Keynote.pdf}{\mycite{Every unwanted feature interaction waits to be fixed or at least documented in form of a constraint.}} T:SPLC21 

\subsection{Pairwise Interactions}
\subsection{Higher-Order Interactions}

\begin{frame}{Interaction on Data and Control Flow \mytitlesource{\essentialconfigurationcomplexity}}
	\centering
	\essentialconfigurationcomplexitylink{\includegraphics[height=\textheightwithtitle,page=2,trim=55 495 225 75,clip]{2016/2016-ASE-Meinicke}}
\end{frame}
\begin{frame}{Execution Traces in Configurable Systems \mytitlesource{\essentialconfigurationcomplexity}}
	\essentialconfigurationcomplexitylink{\includegraphics[width=\linewidth,page=8,trim=55 520 55 55,clip]{2016/2016-ASE-Meinicke}}
\end{frame}

% example from the The Variability Bug Database

% TODO explain duality between partial configurations and conjunctions of literals (cf. elevator product line by Varshosaz et al.)

%\subsection{Slide Title 1}
\begin{frame}{~}
	\ldots
\end{frame}

\subsection{Slide Title 2}
\begin{frame}{-- Example Subtitle}
	\ldots
\end{frame}

\subsection{Slide Title 3}
\begin{frame}{~}
	\ldots
\end{frame}



\lessonslearned{
	\item \ldots
}{
	\item \ldots
}{
	\ldots
}

\sectionend

\section{How to Handle Feature Interactions?}

% discuss techniques and refer to previous examples (message: it depends when what is feasible)
\subsection{S1: Exclude Feature Combinations}
% example: leonovo option compatibility matrix (even if not strictly enforced)
% example: web configurators (Thinkpad)
\subsection{S2: Orthogonal Implementation}
\subsection{S3: Duplicate Implementations}
\subsection{S4: Move Source Code}
\subsection{S5: Preprocessors}
\subsection{S6: Derivative Modules}
\subsection{Overview on all Strategies}
\subsection{When to Handle?}
% spreadshirt example: fixing foreground+background colors for every order

%\subsection{Slide Title 1}
\begin{frame}{~}
	\ldots
\end{frame}

\subsection{Slide Title 2}
\begin{frame}{-- Example Subtitle}
	\ldots
\end{frame}

\subsection{Slide Title 3}
\begin{frame}{~}
	\ldots
\end{frame}



\lessonslearned{
	\item \ldots
}{
	\item \ldots
}{
	\ldots
}

\sectionend

\section{How to Avoid Feature Interactions?}

\subsection{Recap: Size of Configuration Spaces}
\subsection{Costs of Variability}
\subsection{Common Interaction Patterns}
% here or together with Variability Bug Database?
\subsection{Interactions on Data and Control Flow}
% MWK+:ASE16
\subsection{Interactions in Automotive Product Lines}
\subsection{Reduction of Variants as Solution}
\begin{frame}{\myframetitle}
	\leftorright{
		\href{https://commons.wikimedia.org/wiki/File:1910Ford-T.jpg}{\includegraphics[width=\linewidth]{ford-t-1910}}
	}{
		\mynote{Henry Ford, 1909}{\mycite{Any customer can have a car painted any color that he wants so long as it is black.}}
		\myexample{Why only black?\mysource{\fospl}}{
			\begin{itemize}
				\item black color dried faster
				\item faster production
				\item more products and cheaper production
			\end{itemize}
		}
	}
\end{frame}

\subsection{In Practice: Increase of Features and Variants}
% features: Linux
% variants: Automotive02-05, KfW?
\subsection{The Choice of Features}
\subsection{Choose Features Wisely}
\begin{frame}{\myframetitle}
	\leftorright{
		\myexampletight{John Ferguson Smart (2017)}{\centering\href{https://twitter.com/wakaleo/status/854702550469234692}{\includegraphics[width=.98\linewidth,angle=2,trim=0 0 5 0,clip]{unnecessary-features}}}
		% 
	}{
		\centering\href{https://commons.wikimedia.org/wiki/File:John_Carmack_at_GDCA_2017_--_1_March_2017_(cropped).jpeg}{\includegraphics[width=.47\linewidth,trim=0 0 0 0,clip]{john-carmack}}
		\vspace{-7mm}
		
		\mynote{John Carmack (born 1970) \mysource{\href{https://www.ics.uci.edu/~pattis/quotations.html\#C}{uci.edu}}}{\mycite{The important point is that the cost of adding a feature isn't just the time it takes to code it. The cost also includes the addition of an obstacle to future expansion. %Sure, any given feature list can be implemented, given enough coding time. But in addition to coming out late, you will usually wind up with a codebase that is so fragile that new ideas that should be dead-simple wind up taking longer and longer to work into the tangled existing web. 
		[...] The trick is to pick the features that don't fight each other.}}
		% video game developer, co-founder of a video game company
	}
\end{frame}

% TODO \subsection{Choose Features Wisely}
%\begin{frame}{\myframetitle}
%	\leftorright{
%		% TODO \myexampletight{John Ferguson Smart (2017)}{\centering\href{https://twitter.com/wakaleo/status/854702550469234692}{\includegraphics[width=.98\linewidth,angle=2,trim=0 0 5 0,clip]{unnecessary-features}}}
%		% 
%	}{
%		\centering\href{https://commons.wikimedia.org/wiki/File:John_Carmack_at_GDCA_2017_--_1_March_2017_(cropped).jpeg}{\includegraphics[width=.47\linewidth,trim=0 0 0 0,clip]{john-carmack}}
%		\vspace{-7mm}
%		
%		\mynote{John Carmack (born 1970) \mysource{\href{https://www.ics.uci.edu/~pattis/quotations.html\#C}{uci.edu}}}{\mycite{The important point is that the cost of adding a feature isn't just the time it takes to code it. The cost also includes the addition of an obstacle to future expansion. %Sure, any given feature list can be implemented, given enough coding time. But in addition to coming out late, you will usually wind up with a codebase that is so fragile that new ideas that should be dead-simple wind up taking longer and longer to work into the tangled existing web. 
%		[...] The trick is to pick the features that don't fight each other.}}
%		% video game developer, co-founder of a video game company
%	}
%\end{frame}

% TODO \subsection{Number of Features in Linux}
%\begin{frame}{\myframetitle\ \mytitlesource{\href{https://www4.cs.fau.de/Ausarbeitung/MA-I4-2015-04-Hengelein.pdf}{Hengelein 2015}}}
%	\partofpage{70}{
%		% TODO \myexampletight{{2005--2015: Number of Features Tripled}}{\includegraphics[width=\linewidth]{linux-features}}
%	}
%\end{frame}

% TODO variant reduction, prevent the explosion. marketing wants them all. engineering and quality assurance too expensive.

\lessonslearned{
	\item \ldots
}{
	\item \ldots
}{
	\ldots
}

\mode<beamer>{
	\begin{frame}{\inserttitle}
		\lectureseriesoverview
	\end{frame}

	\contentoverview
}


\end{document}
