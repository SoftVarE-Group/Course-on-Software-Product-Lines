\documentclass[
	aspectratio=169, % default is 43
	8pt, % font size, default is 11pt
	handout, % handout mode without animations, comment out to add animations
]{beamer}
\def\university{}

\documentclass[
	aspectratio=169, % default is 43
	8pt, % font size, default is 11pt
	handout, % handout mode without animations, comment out to add animations
]{beamer}

\usepackage{../template/beamerthemeuulm} % use the inofficial uulm beamer theme
\setfaculty{infIngPsy} % set the color scheme for your faculty here [med/infIngPsy/math/nat]

% requires symbolic links
% git clone git@github.com:SoftVarE-Group/SlideTemplate.git C:\Users\...\SlideTemplate
% mklink /J template C:\Users\...\SlideTemplate
% git clone git@spgit.informatik.uni-ulm.de:thuem/slides.git C:\Users\...\ThomasSlides
% mklink /J thomasslides C:\Users\...\ThomasSlides
\graphicspath{{../template/pics/logos}{../template/pics/nature}{../template/pics/uulm}{../thomasslides/}{../pics/}}

%\usepackage[ngerman]{babel} % use this line for slides in German
%\recordingtrue % special recording mode for use with a greenscreen, gives you space to show yourself in a layer in front of the slides, has no effect in the handout mode

\title{Software Product Lines} % short title is used for the slide footer but optional

%
%
%% IMPORTED PACKAGES
%
%\usepackage{adjustbox} % used for partofpage
%\usepackage{tcolorbox} % used for mydefinition, mynote, myexample
\usepackage{multicol} % used temporarily for the lecture overview
%\usepackage{mathtools} % required for absolute value in modeling lecture
%
%% SLIDE TEMPLATE
%
%\beamertemplatenavigationsymbolsempty 
%
%% COMMANDS TO LAYOUT AND ANNIMATE SLIDES
%
\newcommand{\lessonslearned}[3]{
	\subsection{Summary}
	\begin{frame}{\insertsection -- \insertsubsection}
		\leftorright{
			\mydefinition{Lessons Learned}{
				\begin{itemize}
					#1
				\end{itemize}
			}
			\mynote{Further Reading}{
				\small % references take space, can be a little smaller
				\begin{itemize}
					#2
				\end{itemize}
			}
		}{
			\myexample{Practice}{
				#3
			}
		}
	\end{frame}
}

\renewcommand{\lectureoverview}{
%	\section*{Overview}
%	\subsection*{Overview}
	\begin{frame}{\insertsubtitle}
		\begin{multicols}{2}
			\tableofcontents
		\end{multicols}
	\end{frame}
}

%
%\newcommand{\onlyleft}[1]{
%	\halfpage{#1}
%}
%
%\newcommand{\onlyright}[1]{
%	~\hfill
%	\halfpage{#1}
%}
%
%\newcommand{\leftorright}[2]{
%	\uncover<1>{\halfpage{#1}}
%	\hfill
%	\uncover<3->{\halfpage{#2}}
%}
%
%\newcommand{\rightorleft}[2]{
%	\uncover<3->{\halfpage{#1}}
%	\hfill
%	\uncover<1>{\halfpage{#2}}
%}
%
%\newcommand{\leftthenright}[2]{
%	\halfpage{#1}
%	\hfill\pause
%	\halfpage{#2}
%}
%
%\newcommand{\leftandright}[2]{
%	\halfpage{#1}
%	\hfill
%	\halfpage{#2}
%}
%
%\newcommand{\leftmiddleandright}[3]{
%	\thirdpage{#1}
%	\hfill
%	\thirdpage{#2}
%	\hfill
%	\thirdpage{#3}
%}
%
%\newcommand{\leftmiddleorright}[3]{
%	\uncover<1>{\thirdpage{#1}}
%	\hfill
%	\uncover<3>{\thirdpage{#2}}
%	\hfill
%	\uncover<5->{\thirdpage{#3}}
%}
%
%\newcommand{\halfpage}[1]{\partofpage{48}{#1}}
%
%\newcommand{\thirdpage}[1]{\partofpage{31}{#1}}
%
%\newcommand{\partofpage}[2]{
%	\adjustbox{valign=t}{\begin{minipage}{0.#1\textwidth}
%			\begin{flushleft}
%				#2
%			\end{flushleft}
%	\end{minipage}}
%}
%
%\newcommand{\mydefinition}[2]{
%	\begin{tcolorbox}[title=#1,colback=orange!10,colframe=orange!30,coltitle=black,fonttitle=\bfseries,left=1mm,right=1mm,top=1mm,bottom=1mm]
%		\begin{flushleft}
%			#2
%		\end{flushleft}
%	\end{tcolorbox}
%}
%
%\newcommand{\mydefinitiontight}[2]{
%	\begin{tcolorbox}[title=#1,colback=white,colframe=orange!30,coltitle=black,fonttitle=\bfseries,left=0mm,right=0mm,top=0mm,bottom=0mm]
%		\begin{flushleft}
%			#2
%		\end{flushleft}
%	\end{tcolorbox}
%}
%
%\newcommand{\mynote}[2]{
%	\begin{tcolorbox}[title=#1,colback=red!10,colframe=red!30,coltitle=black,fonttitle=\bfseries,left=1mm,right=1mm,top=1mm,bottom=1mm]
%		\begin{flushleft}
%			#2
%		\end{flushleft}
%	\end{tcolorbox}
%}
%
%\newcommand{\myexample}[2]{
%	\begin{tcolorbox}[title=#1,colback=blue!10,colframe=blue!30,coltitle=black,fonttitle=\bfseries,left=1mm,right=1mm,top=1mm,bottom=1mm]
%		\begin{flushleft}
%			#2
%		\end{flushleft}
%	\end{tcolorbox}
%}
%
%\newcommand{\myexampletight}[2]{
%	\begin{tcolorbox}[title=#1,colback=white,colframe=blue!30,coltitle=black,fonttitle=\bfseries,left=0mm,right=0mm,top=0mm,bottom=0mm]
%		\begin{flushleft}
%			#2
%		\end{flushleft}
%	\end{tcolorbox}
%}

\subtitle{11. Product-Line Testing}
\author{Thomas Thüm, Timo Kehrer, Elias Kuiter}

\begin{document}

% TITLE SLIDE

\maketitle

% SLIDE TEMPLATE

%\setbeamercolor{title}{fg=black}
%\setbeamercolor{frametitle}{fg=black}
\setbeamertemplate{frametitle}{{\huge~\\\insertsubsection~\insertframetitle}}
\setbeamertemplate{footline}[text line]{\parbox{\linewidth}{\vspace*{-10pt}\hspace{0pt}%
	\insertshortauthor\phantom{g\insertpagenumber}%
	\hfill%
	\inserttitle%
	\ifx \insertsubtitle \empty \else \ -- \insertsubtitle\fi%
	\ifx \insertsectionhead \empty \else \ -- \insertsectionhead\fi%
	\hfill%
	\phantom{g\insertshortauthor}\insertpagenumber%
}}
%\defbeamertemplate{footline}{\begin{beamercolorbox}[sep=1em]{author in head/foot}\insertshortauthor\hfill\insertsection\hfill\insertframenumber\end{beamercolorbox}}
%\defbeamertemplate*{footline}{mytheme}{\begin{beamercolorbox}[sep=1em]{author in head/foot}\insertshortauthor\hfill\insertsection\hfill\insertframenumber\end{beamercolorbox}}

% OVERVIEW SLIDES

\newcommand{\overview}{
	\section*{Overview}
	\subsection*{Overview}
	\begin{frame}{-- \insertsubtitle}
		\begin{multicols}{3}
			\tableofcontents
		\end{multicols}
	
		\begin{flushright}
			\footnotesize
			Author: \insertauthor
			
			Date: \insertdate
		\end{flushright}
	\end{frame}
}
% temporarily added slide to have a lecture overview 
\overview

% temporarily removed
%\begin{frame}{Lecture Overview -- \insertsubtitle}
%	\tableofcontents[hideallsubsections]
%\end{frame}

\AtBeginSection[]{%
	\begin{frame}{Lecture Overview -- \insertsubtitle}
		\tableofcontents[currentsection,hideothersubsections]
	\end{frame}
}

\newcommand{\sectionend}{\addtocontents{toc}{\newpage}}


\begin{frame}{\inserttitle}
	\lectureseriesoverview[11]
\end{frame}

\begin{frame}{Recap: Quality Assurance \mytitlesource{\ludewiglichter}}
	\begin{mycolumns}[widths={60},animation=none]
		%\only<1|handout:0>{\includegraphics[width=\linewidth,page=1]{quality-assurance}}%
		\only<1|handout:0>{\includegraphics[width=\linewidth,page=8]{quality-assurance}}%
		\only<2|handout:0>{\includegraphics[width=\linewidth,page=3]{quality-assurance}}%
		\only<3|handout:0>{\includegraphics[width=\linewidth,page=4]{quality-assurance}}%
		\only<4|handout:1>{\includegraphics[width=\linewidth,page=5]{quality-assurance}}%
		\only<5-|handout:0>{\includegraphics[width=\linewidth,page=7]{quality-assurance}}%
	\mynextcolumn
		\begin{note}{Lectures on Quality Assurance}
			how to \emph{avoid} variability bugs (esp. feature interactions) \ldots
			\begin{itemize}
				\item<+-> with processes (e.g., domain scoping) \lectureprocess
				\item<+-> with guidelines \lectureinteractions
			\end{itemize}

			how to \emph{find} variability bugs \ldots
			\begin{itemize}
				\item<+-> \emph{statically} \lectureanalyses
				\item<+-> \emph{dynamically} \lecturetesting
				\begin{itemize}
					\item<+-> challenges of product-line testing in Part I
					\item<+-> black-box testing in Part II
					\item<+-> white-box testing in Part III
				\end{itemize}
			\end{itemize}
		\end{note}
	\end{mycolumns}
\end{frame}

% TODO add xkcd from SWT? \widexkcd{974} % salt 12s

\section{Product-Line Testing}
% TODO \section{Product-Line Testing in Practice} practice part still missing

% TODO famous example for a product line failure?

\subsection{Recap: Software Testing}
\begin{frame}{\myframetitle{}}
	\begin{fancycolumns}
		\uncover<1->{
			\begin{definition}{Software Testing \mysource{\sommerville}}
				\mycite{Testing is intended to show that a program does what it is intended to do and to discover program defects before it is put into use.}
			\end{definition}
		}
		\uncover<2->{
			\begin{definition}{Validation Testing \mysource{\sommerville}}
				\mycite{Demonstrate to the developer and the customer that the software meets its requirements.}
			\end{definition}
		}
		\uncover<3->{
			\begin{definition}{Defect Testing \mysource{\sommerville}}
				\mycite{Find inputs or input sequences where the behavior of the software is incorrect, undesirable, or does not conform to its specification.}
			\end{definition}
		}
	\nextcolumn
		\uncover<4->{
			\begin{note}{Stages of Testing \mysource{\sommerville}}
				\begin{itemize}
					\setlength\itemsep{.1em}
					\item[1.] \mycite{\emph{Development testing}, where the system is tested during development to discover
					bugs and defects}
					\item[2.] \mycite{\emph{Release testing}, where a separate testing team tests a complete version of the
					system before it is released to users}
					\item[3.] \mycite{\emph{User testing}, where users or potential users of a system test the system in their
					own environment}
				\end{itemize}
			\end{note}
		}
		\uncover<5->{
			\begin{note}{Manual vs Automated Testing \mysource{\sommerville}}
				\mycite{In \emph{manual testing}, a tester runs the program with some test data and
				compares the results to their expectations. [...] In \emph{automated testing}, the tests are encoded in a program that is run each time the system under development is to be tested.}
			\end{note}
		}
	\end{fancycolumns}
\end{frame}

\subsection{Test-Case Design in Single-System Engineering}
\begin{frame}{Recap: Test-Case Design \deutschertitel{Testfallentwurf}}
	\begin{fancycolumns}
		\begin{definition}{Test Case \mysource{\ludewiglichter}}
			In a test, a number of test cases are executed, whereas each test case consists \emph{input values} for a single execution and \emph{expected outputs}. An \emph{exhaustive test} refers a test in which the test cases exercise all the possible inputs.
		\end{definition}
		\pause
		\begin{definition}{Systematic Test \mysource{\ludewiglichter}}
			A systematic test is a test, in which
			\begin{itemize}
				\setlength\itemsep{.1em}
				\item[1.] the \emph{setup} is defined,
				\item[2.] the \emph{inputs} are chosen systematically,
				\item[3.] the \emph{results} are documented and evaluated by criteria being defined prior to the test. 
			\end{itemize}
		\end{definition}
	\nextcolumn
		\pause
		\begin{note}{Goal \mysource{\ludewiglichter}}%
			Detect a large number of failures with a low number of test cases. A test case (execution) is \emph{positive}, if it detects a failure, and \emph{successful} if it detects an unknown failure.
		\end{note}
		\pause
		\begin{definition}{An ideal test case is \ldots \mysource{\ludewiglichter}}
			\begin{itemize}
				\setlength\itemsep{.1em}
				\item representative: represents a large number of feasible test cases
				\item failure sensitive: has a high probability to detect a failure
				\item non-redundant: does not check what other test cases already check
			\end{itemize}
		\end{definition}
	\end{fancycolumns}
\end{frame}

% TODO recap analysis strategies for product lines and discuss why feature based and family based is typically not feasible for testing

\subsection{Testing All Configurations}
\begin{frame}{\myframetitle{}}
	\begin{fancycolumns}
		\centering\featureDiagramConfigurableDatabase
		
		\begin{example}{Recap: 26 Valid Configurations\mysource{\lecturemodeling}}
			\footnotesize
			\begin{fancycolumns}[animation=none]
				$\{C,G,W\}$\\
				$\{C,P,W\}$\\
				$\{C,G,P,W\}$\\
				$\{C,D,W\}$\\
				$\{C,G,D,W\}$\\
				$\{C,P,D,W\}$\\
				$\{C,G,P,D,W\}$\\
				$\{C,P,T,W\}$\\
				$\{C,G,P,T,W\}$\\
				$\{C,D,T,W\}$\\
				$\{C,G,D,T,W\}$\\
				$\{C,P,D,T,W\}$\\
				$\{C,G,P,D,T,W\}$
			\nextcolumn
				$\{C,G,L\}$\\
				$\{C,P,L\}$\\
				$\{C,G,P,L\}$\\
				$\{C,D,L\}$\\
				$\{C,G,D,L\}$\\
				$\{C,P,D,L\}$\\
				$\{C,G,P,D,L\}$\\
				$\{C,P,T,L\}$\\
				$\{C,G,P,T,L\}$\\
				$\{C,D,T,L\}$\\
				$\{C,G,D,T,L\}$\\
				$\{C,P,D,T,L\}$\\
				$\{C,G,P,D,T,L\}$
			\end{fancycolumns}
		\end{example}
	\nextcolumn
		\vspace{-7mm}
		\pause
		\begin{note}{Discussion}
			\begin{itemize}
				\setlength\itemsep{.5em}
				\item only feasible for \emph{small} product lines\\(few valid configurations)
				\item redundant test effort
				\vspace*{1ex}
				\item large product lines: not feasible to generate and compile all configurations
				\begin{itemize}
					\item (some) large product lines: even number of valid configurations is unknown
				\end{itemize}
			\end{itemize}
		\end{note}
	\end{fancycolumns}
\end{frame}

\begin{frame}{Recap: Feature Model of the Linux Kernel}
	\vspace{1mm}~\hspace{-15mm}\href{https://dl.acm.org/doi/abs/10.1145/3382025.3414943}{\pic[width=1.2\linewidth,page=1,trim=100 510 100 170,clip]{2020/2020-SPLC-Thuem}}
\end{frame}

\begin{frame}{\myframetitle{}}
	\begin{fancycolumns}[widths={60}]
		\begin{exampletight}{Recap: Industrial Configuration Spaces\mysource{\lectureintroduction}}
			\centering\evaluatingsharpsatsolverslink{\pic[width=\linewidth,page=6,trim=50 210 320 440,clip]{2020/2020-VaMoS-Sundermann}}
		\end{exampletight}
		\uncover<3->{
			\begin{note}{}
				Why being complete on the configurations then?
			\end{note}
		}
	\nextcolumn
		\vspace{-7mm}
		\href{https://commons.wikimedia.org/wiki/File:Edsger_Wybe_Dijkstra.jpg}{\pic[width=\linewidth,trim=0 425 0 75,clip]{edsger-dijkstra}}
		\vspace{-7mm}
		
		\begin{note}{Edsger W. Dijkstra (1972)}
			\mycite{Program testing can be a very effective way to show the presence of bugs, but it is hopelessly inadequate for showing their absence.}\mysource{\thehumbleprogrammer}
		\end{note}
		% 1930-2002, ACM Turing Award winner
	\end{fancycolumns}
\end{frame}

\newcommand{\eemph}[1]{{\color{red}\textbf{#1}}}

\subsection{Testing One Configuration}
\begin{frame}[b]{\myframetitle{}}
	\begin{fancycolumns}[b]
		\centering\featureDiagramConfigurableDatabase
		
		\begin{example}{Recap: 26 Valid Configurations\mysource{\lecturemodeling}}
			\footnotesize
			\begin{fancycolumns}[animation=none]
				$\{C,G,W\}$\\
				$\{C,P,W\}$\\
				$\{C,G,P,W\}$\\
				$\{C,D,W\}$\\
				$\{C,G,D,W\}$\\
				$\{C,P,D,W\}$\\
				$\{C,G,P,D,W\}$\\
				$\{C,P,T,W\}$\\
				$\{C,G,P,T,W\}$\\
				$\{C,D,T,W\}$\\
				$\{C,G,D,T,W\}$\\
				$\{C,P,D,T,W\}$\\
				\emph{$\{C,G,P,D,T,W\}$}
			\nextcolumn
				$\{C,G,L\}$\\
				$\{C,P,L\}$\\
				$\{C,G,P,L\}$\\
				$\{C,D,L\}$\\
				$\{C,G,D,L\}$\\
				$\{C,P,D,L\}$\\
				$\{C,G,P,D,L\}$\\
				$\{C,P,T,L\}$\\
				$\{C,G,P,T,L\}$\\
				$\{C,D,T,L\}$\\
				$\{C,G,D,T,L\}$\\
				$\{C,P,D,T,L\}$\\
				$\{C,G,P,D,T,L\}$
			\end{fancycolumns}
		\end{example}
	\nextcolumn
		\pause\vspace{-10mm}
		\begin{note}{Discussion}
			\begin{itemize}
				\setlength\itemsep{.4em}
				\item applicable to large product lines
				\item no redundant test effort (from configurations)
				\item strategy in practice: all-yes-config (configuration with many features selected)
				\vspace*{1ex}
				\item often unfeasible to test all features wit a single configuration (e.g., \texttt{Windows} and \texttt{Linux})
				\item[$\Rightarrow$] unnoticed feature interactions\mysource{\lectureinteractions}
			\end{itemize}
		\end{note}
		\pause
		\begin{example}{What about interactions with missing features?}
			\centering\pic[width=.5\linewidth]{toast4}
		\end{example}
	\end{fancycolumns}
\end{frame}

\subsection{Sample-Based Testing}
\begin{frame}{\myframetitle{} \deutschertitel{Stichprobenbasiertes Testen}}
	\begin{fancycolumns}
		\begin{definition}{Intuition}
			\begin{itemize}
				\item to analyze the product line, just analyze \emph{some products}
				\item sample \deutsch{Stichprobe} refers to a subset of all valid configurations
				\item common technique to test a product line
				\item sample configurations chosen by experts, randomly, or systematically
			\end{itemize}
		\end{definition}
		\pause
		\begin{note}{Advantages and Challenges}
			\begin{itemize}
				\item[+] lower effort than testing all configurations
				\item[+] higher chance to detect defects than testing one configuration
				\item[--] how many configurations to test?\\which configurations to test?
			\end{itemize}
		\end{note}
	\nextcolumn
		\pause
		\pic[width=\linewidth,page=10]{lego-analyses}
	\end{fancycolumns}
\end{frame}

% TODO \subsection{Expert Knowledge in Sampling}

% TODO \subsection{Random Sampling}

% TODO probably too much content anyway: \subsection{Excursus: Uniform Random Sampling}

% TODO \subsection{Testing the Linux Kernel}
% allyesconfig

%\subsection{Automation in Product Sampling} ???

%\subsection{Missing: Test-Case Selection/Generation}
% What to test for those configurations? Variable unit tests? Avoid redundant testing?

% TODO how Linux is developed: patches on mailing list, only considered if not rejected by CI, what happens in CI



\lessonslearned{
	\item recap on software testing
	\item testing all configurations
	\item testing one configuration
	\item sample-based testing
}{
	\item[]
}{
	Recap on feature interactions: What are examples of interactions that cannot be detected statically \lectureanalyses\ and could be missed when testing a single configuration only?
}

\sectionend

\section{Combinatorial Interaction Testing}

\subsection{Test-Case Design in Single-System Engineering}
\begin{frame}{Recap: Test-Case Design \deutschertitel{Testfallentwurf}}
	\leftorright{
		\uncover<1>{\mydefinition{Systematic Test \mysource{\ludewiglichter}}{A systematic test is a test, in which
			\begin{itemize}
				\setlength\itemsep{.1em}
				\item[1.] the setup is defined,
				\item[2.] the inputs are chosen systematically,
				\item[3.] the results are documented and evaluated by criteria being defined prior to the test. 
			\end{itemize}
		}{}}
		\uncover<2>{\mydefinition{Test Case \mysource{\ludewiglichter}}{In a test, a number of test cases are executed, whereas each test case consists \emph{input values} for a single execution and \emph{expected outputs}. An \emph{exhaustive test} refers a test in which the test cases exercise all the possible inputs.}{}}
	}{
		\mynote{Goal \mysource{\ludewiglichter}}{Detect a large number of failures with a low number of test cases. A test case (execution) is \emph{positive}, if it detects a failure, and \emph{successful} if it detects an unknown failure.}
		\mydefinition{An ideal test case is \ldots \mysource{\ludewiglichter}}{
			\begin{itemize}
				\setlength\itemsep{.1em}
				\item representative: represents a large number of feasible test cases
				\item failure sensitive: has a high probability to detect a failure
				\item non-redundant: does not check what other test cases already check
			\end{itemize}
		}{}
	}
\end{frame}

\begin{frame}{Recap: Black-Box Testing \deutschertitel{Funktionstest}}
	\begin{mycolumns}
		\begin{note}{Motivation \mysource{\ludewiglichter}}
			\begin{itemize}
				\setlength\itemsep{.1em}
				\item source code not always available (e.g., outsourced components, obfuscated code)
				%\item specific test cases derived from logical ones using arbitrary values
				%\item specification not incorporated so far (only for expected results)
				%\item invalid inputs not tested
				\item errors are not equally distributed
			\end{itemize}
		\end{note}
		\begin{definition}{Black-Box Testing \mysource{\ludewiglichter}}
			\begin{itemize}
				\setlength\itemsep{.1em}
				\item test-case design based on specification
				\item source code and its inner structure is ignored (assumed as a black-box)
			\end{itemize}
		\end{definition}
	\mynextcolumn
		\begin{note}{Sample Configuration $\neq$ Test Case}
			\begin{itemize}
				\item test case: concrete inputs and expected outputs for a program
				\item sample configuration: selection of features to derive the program
				\item both needed when testing product lines
				\item often confused in the literature
				\item test case derivation
					\begin{itemize}
						\item out of scope here % TODO add pointers to literature
						\item global tests (i.e., identical for all configurations)
						\item product-line implementation technique used to automatically derive configuration-specific tests \lectureprocess
					\end{itemize}
				\item on next slides: idea of black-box testing applied to derive sample configuration
			\end{itemize}
		\end{note}
	\end{mycolumns}
\end{frame}

\subsection{Pairwise Interaction Testing}
\begin{frame}{\myframetitle}
	\leftorright{
		\mydefinition{Pairwise Interaction Testing}{
			\begin{itemize}
				\setlength\itemsep{.5em}
				\item test a sample set $S \subseteq C$ of all valid configurations $C$ with pairwise coverage
				\item every pairwise interaction is covered by at least one configuration in the sample $S$
			\end{itemize}
		}
		\begin{example}{Configurations with the Interaction Get $\wedge$ Put}
			\footnotesize
			\leftandright{
				$\{C,G,W\}$\\
				$\{C,P,W\}$\\
				\emph{$\{C,G,P,W\}$}\\
				$\{C,D,W\}$\\
				$\{C,G,D,W\}$\\
				$\{C,P,D,W\}$\\
				\emph{$\{C,G,P,D,W\}$}\\
				$\{C,P,T,W\}$\\
				\emph{$\{C,G,P,T,W\}$}\\
				$\{C,D,T,W\}$\\
				$\{C,G,D,T,W\}$\\
				$\{C,P,D,T,W\}$\\
				\emph{$\{C,G,P,D,T,W\}$}
			}{
				$\{C,G,L\}$\\
				$\{C,P,L\}$\\
				\emph{$\{C,G,P,L\}$}\\
				$\{C,D,L\}$\\
				$\{C,G,D,L\}$\\
				$\{C,P,D,L\}$\\
				\emph{$\{C,G,P,D,L\}$}\\
				$\{C,P,T,L\}$\\
				\emph{$\{C,G,P,T,L\}$}\\
				$\{C,D,T,L\}$\\
				$\{C,G,D,T,L\}$\\
				$\{C,P,D,T,L\}$\\
				\emph{$\{C,G,P,D,T,L\}$}
			}
		\end{example}
	}{
		\mynote{Discussion}{
			\begin{itemize}
				\setlength\itemsep{.4em}
				\item applicable to large product lines
				\item reduced redundant effort compared to testing all configurations
				\item coverage guarantee (opposed to random configurations)
				\item still requires good test cases (program inputs)
				\item hard to compute small sample sets
			\end{itemize}
		}
		\mydefinition{Pairwise Interactions}{
			\begin{itemize}
				\setlength\itemsep{.5em}
				\item \emph{up-to} four interactions between $A$ and $B$
				\item both selected: $A \wedge B$
				\item one selected: $\neg A \wedge B$, $A \wedge \neg B$
				\item none selected: $\neg A \wedge \neg B$
			\end{itemize}
		}
	}
\end{frame}

\newcommand{\pair}[2]{$#1 \wedge #2$ & $#1 \wedge \neg #2$ & $\neg #1 \wedge #2$ & $\neg #1 \wedge \neg #2$\\}
\newcommand{\redandgray}[1]{\only<#1-| handout:#1->{\color{black}}\only<#1| handout:#1>{\color{blue}}}
\newcommand{\epair}[6]{
	{\redandgray{#3}$#1 \wedge #2$} & 
	{\redandgray{#4}$#1 \wedge \neg #2$} & 
	{\redandgray{#5}$\neg #1 \wedge #2$} & 
	{\redandgray{#6}$\neg #1 \wedge \neg #2$}\\
}

\begin{frame}{Pairwise Coverage}
	\leftorright{
		\centering\featureDiagramConfigurableDatabase

		\mydefinition{Interactions to Cover}{
			\begin{itemize}
				\setlength\itemsep{.5em}
				\item exclude invalid combinations (e.g., $W \wedge L$)
				\item exclude abstract features (e.g., $API$, $OS$)
				\item exclude features contained in every configuration (e.g., $C$)
			\end{itemize}
			% TODO \todo{add formal definitions based on \lecturemodeling}
		}
	}{
		\vspace{-10mm}
		\myexample{Pairwise Interactions}{
			\centering\footnotesize\color{lightgray}
			\begin{tabular}{llll}
				\epair{G}{P}{3}{2}{1}{6}
				\epair{G}{D}{2}{3}{1}{5}
				\epair{G}{T}{3}{2}{1}{5}
				\epair{G}{W}{4}{2}{1}{6}
				\epair{G}{L}{2}{4}{6}{1}
				\epair{P}{D}{1}{3}{2}{4}
				\epair{P}{T}{1}{5}{6}{2}
				\epair{P}{W}{1}{3}{4}{2}
				\epair{P}{L}{3}{1}{2}{4}
				\epair{D}{T}{1}{2}{3}{4}
				\epair{D}{W}{1}{2}{4}{3}
				\epair{D}{L}{2}{1}{3}{4}
				\epair{T}{W}{1}{3}{4}{2}
				\epair{T}{L}{3}{1}{2}{4}
				& {\redandgray{1}$W \wedge \neg L$} & {\redandgray{2}$\neg W \wedge L$} & \\
			\end{tabular} 
		}
		\myexample{Pairwise Coverage with Six Configurations}{
			\footnotesize\color{lightgray}
			{\redandgray{1}$\{C,P,D,T,W\}$}\\
			{\redandgray{2}$\{C,G,D,L\}$}\\
			{\redandgray{3}$\{C,G,P,T,L\}$}\\
			{\redandgray{4}$\{C,G,W\}$}\\
			{\redandgray{5}$\{C,P,W\}$}\\
			{\redandgray{6}$\{C,D,T,L\}$}\\
		}
	}
\end{frame}
% TODO use different colors for the different configurations (instead of separate handouts)

\subsection{T-Wise Interaction Testing}
\begin{frame}{\myframetitle}
	\begin{mycolumns}[widths={65}]
		\mydefinition{T-Wise Interaction Testing}{
			\begin{itemize}
				\setlength\itemsep{.5em}
				\item generalization of pairwise interaction testing
				\item t-wise coverage: every t-wise interaction is covered by at least one configuration in the sample
				\item t=1: every feature is selected and also deselected
				\item t=2: pairwise interaction coverage
				\item t=3: every combination of three features covered
			\end{itemize}
		}
	\mynextcolumn
		\begin{example}{{T=3 Interactions}}
			for the features $G$, $P$, and $D$:

			\leftandright{
				$G \wedge P \wedge D$\\
				$G \wedge P \wedge \neg D$\\
				$G \wedge \neg P \wedge D$\\
				$G \wedge \neg P \wedge \neg D$
			}{
				$\neg G \wedge P \wedge D$\\
				$\neg G \wedge P \wedge \neg D$\\
				$\neg G \wedge \neg P \wedge D$\\
				$\neg G \wedge \neg P \wedge \neg D$
			}
		\end{example}
	\end{mycolumns}
\end{frame}

\subsection{Algorithms for Combinatorial Interaction Testing}
\begin{frame}{\myframetitle}
	\begin{mycolumns}[widths={63}]
		\begin{definition}{A Greedy Algorithm}
			idea: select configuration that cover most of the missing interactions in each step
			\begin{enumerate}
				\item randomly choose first configuration (each covers one interaction of every line)
				\item find next optimal configuration
				\item repeat steps 2--3 until all interactions covered
			\end{enumerate}
		\end{definition}
		\pause
		\begin{note}{Challenges and Optimizations}
			\begin{itemize}
				\item non-deterministic: different sample for each run (cf.\ Step 1)
				\item starting with all-yes-config? covers more code
				\item Step 2: iterating over all valid configurations does not scale
				\item greedy strategy: optimal configuration in each step does not guarantee optimal sample
			\end{itemize}
		\end{note}
	\mynextcolumn
		\pause
		\begin{definition}{ICPL\mysource{\icpl}}
			\begin{itemize}
				\item widespread greedy algorithm
				\item maintains several partial configurations
				\item iterates over all interactions
				\item tries to integrate interaction into existing partial configuration
				\item create new partial configuration if not possible
				\item avoids extensive Step 2
				\item performance shown on next slides
			\end{itemize}
		\end{definition}
	\end{mycolumns}
\end{frame}

\subsection{Efficiency of Combinatorial Interaction Testing}
%\subsection{Combinatorial Interaction Testing with ICPL}
\begin{frame}{\myframetitle\ \mytitlesource{\icpl}}
	\leftorright{
		\myexampletight{Assumption: All Features are Optional}{
			\centering\footnotesize\featureDiagramEightOptionalFeatures
		}
		
		\myexampletight{Number of Configurations in Pairwise Sample}{
			\includegraphics[width=\linewidth,page=4]{cit-plots}
		}
	}{%
		\myexampletight{Assumption: All Features are Optional}{
			\centering\footnotesize\featureDiagramEightOptionalFeatures
		}
		
		\myexampletight{Number of Configurations in T-Wise Sample}{
			\includegraphics[width=\linewidth,page=5]{cit-plots}
		}
	}
\end{frame}

\begin{frame}{\myframetitle\ \mytitlesource{\icpl}}
	\leftorright{
		\myexampletight{Time in Minutes to Compute Sample}{
			\includegraphics[width=\linewidth,page=2]{cit-plots}

			\begin{itemize}
				\setlength\itemsep{.5em}
				\item about 9h for Linux
				\item 480 configuration in pairwise sample
			\end{itemize}
		}
	}{
		\myexampletight{Number of Configurations in Sample}{
			\includegraphics[width=\linewidth,page=3]{cit-plots}

			\begin{itemize}
				\setlength\itemsep{.5em}
				\item Linux kernel v2.6.28.6 (February 2009)
				\item 6,888 features
				\item 187,193 clauses in conjunctive normal form
			\end{itemize}
		}
	}
\end{frame}

% TODO distinguish testing efficiency and sampling efficiency

\subsection{Effectiveness of Combinatorial Interaction Testing}
\begin{frame}{\myframetitle}
	\leftorright{
		\myexampletight{Effectiveness of Interaction Testing \mysource{\href{https://ieeexplore.ieee.org/document/1321063}{Kuhn et al.\ 2004}}}{
			\includegraphics[width=\linewidth,page=1]{cit-plots}
		}
	}{
		\mynote{Trade-Off}{large t: high coverage (more effective)
			
			small t: low testing effort (more efficient)}
	}
\end{frame}


\lessonslearned{
	\item \ldots
}{
	\item \ldots
}{
	\ldots
}

\sectionend

\section{Solution-Space Sampling}

\subsection{Coverage in Single-System Engineering}
\begin{frame}{Recap: Coverage in White-Box Testing{} \deutschertitel{Testüberdeckung für Strukturtests} \mytitlesource{\ludewiglichter}}
	\begin{fancycolumns}
		\begin{definition}{White-Box Testing \deutsch{Strukturtest}}
			\begin{itemize}
%				\setlength\itemsep{.1em}
				\item inner structure of test object is used
				\item idea: coverage of structural elements
				\begin{itemize}
					\item code translated into control flow graph
					\item specific test case (concrete inputs)\\derived from logical test case (conditions)\\derived from path in control flow graph
				\end{itemize}	
			\end{itemize}
		\end{definition}
	\nextcolumn
		\begin{definition}{Coverage Criteria \deutsch{Überdeckungskriterien}}
			\begin{itemize}
%				\setlength\itemsep{.1em}
				\item[1.] \emph{statement coverage} \deutsch{Anweisungsüberdeck.}:\\all statements are executed for at least one test case
				\item<3->[2.] \emph{branching coverage} \deutsch{Zweigüberdeckung}: statement coverage and all branches of branching statement are executed %TODO not so easy to define as percentage
				\item<4->[3.] \emph{term coverage} \deutsch{Termüberdeckung}:\\branching coverage and all terms used in a branching statement ($n$) are combined exhaustively ($2^n$)\hfill(simplified)
				% TODO discuss path coverage?
			\end{itemize}
		\end{definition}
	\end{fancycolumns}
\end{frame}

\subsection{Coverage of Ifdef Blocks}
\begin{frame}[b]
	\mywhite{Can You Spot Problems in the Elevator Product Line?\mysource{\samplingsurvey}}{
		\centering\pic[width=.8\linewidth,page=3,trim=40 530 70 100,clip]{2018/2018-SPLC-Varshosaz}
	}
	\pause
	\begin{itemize}
		\item<+-> Line~29: compiler error when $FIFO$: field \texttt{dreq} undefined
		\item<+-> Line~29: compiler error when $FIFO \pand \pnot DirectedCall$: method \texttt{callButtonsNextState} undefined
		\item<+-> Line~8: runtime error when $DirectedCall \pand \pnot FIFO$: null pointer exception
		\item<+-> both problems detectable with pairwise coverage, but presence conditions are more complicated in practice
		\item<+-> also: pairwise coverage often too much effort for large configuration spaces / continuous integration
	\end{itemize}
	\onslide<+->{\begin{definition}{\myframetitle{} \deutschertitel{Überdeckung von Ifdef-Blöcken} \mysource{\tartlerconfigurationcoverage}}
	\begin{itemize}
		\item every block selected for at least one configuration in the sample (cf.\ statement coverage)
	\end{itemize}
	\end{definition}}
\end{frame}

\subsection{Presence-Condition Coverage}
\begin{frame}{\myframetitle{} \deutschertitel{Presence-Condition-Überdeckung}}
	\begin{fancycolumns}[widths={48}]
		\begin{definition}{Presence-Condition Coverage\mysource{\krieterpresenceconditioncoverage}}
			\begin{itemize}
				\item application of t-wise interaction testing to presence conditions
				\item recap presence condition: formula specifying exactly those configurations under which a block is present
				\item t-wise presence condition coverage: every t-wise interaction of presence conditions is covered by at least one configuration in the sample
				\item $t=1$: every block is selected and also deselected (i.e., more than Tartler's coverage of ifdef blocks)
				\item $t=2$: every combination of two blocks covered
				\item $t=3$: every combination of three blocks covered
			\end{itemize}
		\end{definition}
	\nextcolumn
		\pause
		\begin{example}{{T=3 Presence-Condition Interactions}}
			for the blocks $a$, $b$, and $c$ with presence conditions $A$, $B$, and $C$:

			\begin{fancycolumns}[animation=none]
				$A \wedge B \wedge C$\\
				$A \wedge B \wedge \neg C$\\
				$A \wedge \neg B \wedge C$\\
				$A \wedge \neg B \wedge \neg C$
			\nextcolumn
				$\neg A \wedge B \wedge C$\\
				$\neg A \wedge B \wedge \neg C$\\
				$\neg A \wedge \neg B \wedge C$\\
				$\neg A \wedge \neg B \wedge \neg C$
			\end{fancycolumns}
		\end{example}
		\pause
		\begin{note}{Presence-Condition Coverage\mysource{\krieterpresenceconditioncoverage}}
			\begin{itemize}
				\item coverage of solution space (not problem space)
				\item aka.\ solution-space sampling
				\item for same t: often fewer configurations and similar effectiveness than feature interaction coverage
				\item also feasible by translating presence conditions into feature model \mysource{\hentzesolutionspacesampling}
			\end{itemize}
		\end{note}
	\end{fancycolumns}
\end{frame}

\subsection{Overview on Coverage Criteria}
\begin{frame}{\myframetitle{} \deutschertitel{Überblick über Überdeckungskriterien}}
	\begin{definition}{Techniques \& Coverage Criteria \mysource{\samplingsurvey}}
		\pic[width=\linewidth,page=4,trim=50 100 310 610,clip]{2018/2018-SPLC-Varshosaz}
	\end{definition}
\end{frame}

\subsection{Input for Sampling Algorithms}
\begin{frame}{\myframetitle{}}
	\begin{fancycolumns}
		\begin{definition}{Input Data \mysource{\samplingsurvey}}
			\pic[width=\linewidth,page=3,trim=350 430 90 310,clip]{2018/2018-SPLC-Varshosaz}
		\end{definition}
		\pause
		\begin{example}{Further Domain Knowledge \mysource{\samplingsurvey}}
			\begin{itemize}
				\item in addition to feature model
				\item e.g., configurations chosen by experts
				\item e.g., specialized feature model for sampling
			\end{itemize}
		\end{example}
	\nextcolumn
		\pause
		\vspace{-5mm}
		\begin{example}{Part 2: Combinatorial Interaction Testing}
			\begin{itemize}
				\item (Problem-Space Sampling)
				\item \emph{feature model} used to consider only valid configurations
			\end{itemize}
		\end{example}
		\pause
		\begin{example}{Part 3: Solution-Space Sampling}
			\begin{itemize}
				\item mapping from features to \emph{implementation artifacts}
				\item \emph{feature model} used to consider only valid configurations
			\end{itemize}
		\end{example}
		\pause
		\begin{example}{Combinatorial Reduction of Tests \mysource{\reducingconfigurations}}
			\begin{itemize}
				\item which configurations matter for each test?
				\item analyze \emph{unit tests} and \emph{impl. artifacts}
				\item \emph{feature model} used to consider only valid configurations
			\end{itemize}
		\end{example}
	\end{fancycolumns}
\end{frame}


\lessonslearned{
	\item \ldots
}{
	\item \ldots
}{
	\ldots
}

%\lessonslearned{
%	\item Testing software product lines
%	\item All configurations? one configuration? feature interactions!
%	\item Combinatorial interaction testing: pairwise and t-wise
%	\item Sample coverage (effectiveness) vs sample size (testing efficiency) vs time to compute the sample (sampling efficiency)
%	\item Further Reading: \icpl
%}{
%	\item Quiz on the complete product-line lecture
%	\item Post questions to Moodle, if any
%	\\\hfill\qrcode{https://moodle.uni-ulm.de/mod/moodleoverflow/discussion.php?d=4496} % TODO in 2023 update link
%}

\mode<beamer>{\begin{frame}{\inserttitle}
	\lectureseriesoverview[11]
\end{frame}}

\mode<beamer>{
	\begin{frame}{\inserttitle}
		\lectureseriesoverview
	\end{frame}

	\contentoverview
}


\end{document}
