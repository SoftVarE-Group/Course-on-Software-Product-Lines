\documentclass[
	aspectratio=169, % default is 43
	8pt, % font size, default is 11pt
	handout, % handout mode without animations, comment out to add animations
]{beamer}
\def\university{magdeburg}

\documentclass[
	aspectratio=169, % default is 43
	8pt, % font size, default is 11pt
	handout, % handout mode without animations, comment out to add animations
]{beamer}

\usepackage{../template/beamerthemeuulm} % use the inofficial uulm beamer theme
\setfaculty{infIngPsy} % set the color scheme for your faculty here [med/infIngPsy/math/nat]

% requires symbolic links
% git clone git@github.com:SoftVarE-Group/SlideTemplate.git C:\Users\...\SlideTemplate
% mklink /J template C:\Users\...\SlideTemplate
% git clone git@spgit.informatik.uni-ulm.de:thuem/slides.git C:\Users\...\ThomasSlides
% mklink /J thomasslides C:\Users\...\ThomasSlides
\graphicspath{{../template/pics/logos}{../template/pics/nature}{../template/pics/uulm}{../thomasslides/}{../pics/}}

%\usepackage[ngerman]{babel} % use this line for slides in German
%\recordingtrue % special recording mode for use with a greenscreen, gives you space to show yourself in a layer in front of the slides, has no effect in the handout mode

\title{Software Product Lines} % short title is used for the slide footer but optional

%
%
%% IMPORTED PACKAGES
%
%\usepackage{adjustbox} % used for partofpage
%\usepackage{tcolorbox} % used for mydefinition, mynote, myexample
\usepackage{multicol} % used temporarily for the lecture overview
%\usepackage{mathtools} % required for absolute value in modeling lecture
%
%% SLIDE TEMPLATE
%
%\beamertemplatenavigationsymbolsempty 
%
%% COMMANDS TO LAYOUT AND ANNIMATE SLIDES
%
\newcommand{\lessonslearned}[3]{
	\subsection{Summary}
	\begin{frame}{\insertsection -- \insertsubsection}
		\leftorright{
			\mydefinition{Lessons Learned}{
				\begin{itemize}
					#1
				\end{itemize}
			}
			\mynote{Further Reading}{
				\small % references take space, can be a little smaller
				\begin{itemize}
					#2
				\end{itemize}
			}
		}{
			\myexample{Practice}{
				#3
			}
		}
	\end{frame}
}

\renewcommand{\lectureoverview}{
%	\section*{Overview}
%	\subsection*{Overview}
	\begin{frame}{\insertsubtitle}
		\begin{multicols}{2}
			\tableofcontents
		\end{multicols}
	\end{frame}
}

%
%\newcommand{\onlyleft}[1]{
%	\halfpage{#1}
%}
%
%\newcommand{\onlyright}[1]{
%	~\hfill
%	\halfpage{#1}
%}
%
%\newcommand{\leftorright}[2]{
%	\uncover<1>{\halfpage{#1}}
%	\hfill
%	\uncover<3->{\halfpage{#2}}
%}
%
%\newcommand{\rightorleft}[2]{
%	\uncover<3->{\halfpage{#1}}
%	\hfill
%	\uncover<1>{\halfpage{#2}}
%}
%
%\newcommand{\leftthenright}[2]{
%	\halfpage{#1}
%	\hfill\pause
%	\halfpage{#2}
%}
%
%\newcommand{\leftandright}[2]{
%	\halfpage{#1}
%	\hfill
%	\halfpage{#2}
%}
%
%\newcommand{\leftmiddleandright}[3]{
%	\thirdpage{#1}
%	\hfill
%	\thirdpage{#2}
%	\hfill
%	\thirdpage{#3}
%}
%
%\newcommand{\leftmiddleorright}[3]{
%	\uncover<1>{\thirdpage{#1}}
%	\hfill
%	\uncover<3>{\thirdpage{#2}}
%	\hfill
%	\uncover<5->{\thirdpage{#3}}
%}
%
%\newcommand{\halfpage}[1]{\partofpage{48}{#1}}
%
%\newcommand{\thirdpage}[1]{\partofpage{31}{#1}}
%
%\newcommand{\partofpage}[2]{
%	\adjustbox{valign=t}{\begin{minipage}{0.#1\textwidth}
%			\begin{flushleft}
%				#2
%			\end{flushleft}
%	\end{minipage}}
%}
%
%\newcommand{\mydefinition}[2]{
%	\begin{tcolorbox}[title=#1,colback=orange!10,colframe=orange!30,coltitle=black,fonttitle=\bfseries,left=1mm,right=1mm,top=1mm,bottom=1mm]
%		\begin{flushleft}
%			#2
%		\end{flushleft}
%	\end{tcolorbox}
%}
%
%\newcommand{\mydefinitiontight}[2]{
%	\begin{tcolorbox}[title=#1,colback=white,colframe=orange!30,coltitle=black,fonttitle=\bfseries,left=0mm,right=0mm,top=0mm,bottom=0mm]
%		\begin{flushleft}
%			#2
%		\end{flushleft}
%	\end{tcolorbox}
%}
%
%\newcommand{\mynote}[2]{
%	\begin{tcolorbox}[title=#1,colback=red!10,colframe=red!30,coltitle=black,fonttitle=\bfseries,left=1mm,right=1mm,top=1mm,bottom=1mm]
%		\begin{flushleft}
%			#2
%		\end{flushleft}
%	\end{tcolorbox}
%}
%
%\newcommand{\myexample}[2]{
%	\begin{tcolorbox}[title=#1,colback=blue!10,colframe=blue!30,coltitle=black,fonttitle=\bfseries,left=1mm,right=1mm,top=1mm,bottom=1mm]
%		\begin{flushleft}
%			#2
%		\end{flushleft}
%	\end{tcolorbox}
%}
%
%\newcommand{\myexampletight}[2]{
%	\begin{tcolorbox}[title=#1,colback=white,colframe=blue!30,coltitle=black,fonttitle=\bfseries,left=0mm,right=0mm,top=0mm,bottom=0mm]
%		\begin{flushleft}
%			#2
%		\end{flushleft}
%	\end{tcolorbox}
%}

\subtitle{14. Conclusion}
\author{Elias Kuiter, Thomas Thüm, Timo Kehrer}
\foruniversity{}
	{\setpicture[10]{ovgu-winter7}\setcopyright{Photo: Ulrich Arendt (OVGU)}}
	{\setpicture{oct20-south4}}

\newcommand{\addsummary}[5][]{
	\subsection{#3}
	\begin{frame}{\inserttitle}
		\lectureseriesoverview[#4]
	\end{frame}
	\foreach \x in {#1}{
		\begin{frame}{}\hspace*{-1cm}\includegraphics[page=\x]{#2}\end{frame}
	}
	\begin{frame}{}
		\myexample{Selected Questions and Tasks for Lecture #4}{
			\begin{itemize}\item #5 \item \ldots\end{itemize}
		}
	\end{frame}
}
\newcommand{\also}{\item}

\begin{document}

% TITLE SLIDE

\maketitle

% SLIDE TEMPLATE

%\setbeamercolor{title}{fg=black}
%\setbeamercolor{frametitle}{fg=black}
\setbeamertemplate{frametitle}{{\huge~\\\insertsubsection~\insertframetitle}}
\setbeamertemplate{footline}[text line]{\parbox{\linewidth}{\vspace*{-10pt}\hspace{0pt}%
	\insertshortauthor\phantom{g\insertpagenumber}%
	\hfill%
	\inserttitle%
	\ifx \insertsubtitle \empty \else \ -- \insertsubtitle\fi%
	\ifx \insertsectionhead \empty \else \ -- \insertsectionhead\fi%
	\hfill%
	\phantom{g\insertshortauthor}\insertpagenumber%
}}
%\defbeamertemplate{footline}{\begin{beamercolorbox}[sep=1em]{author in head/foot}\insertshortauthor\hfill\insertsection\hfill\insertframenumber\end{beamercolorbox}}
%\defbeamertemplate*{footline}{mytheme}{\begin{beamercolorbox}[sep=1em]{author in head/foot}\insertshortauthor\hfill\insertsection\hfill\insertframenumber\end{beamercolorbox}}

% OVERVIEW SLIDES

\newcommand{\overview}{
	\section*{Overview}
	\subsection*{Overview}
	\begin{frame}{-- \insertsubtitle}
		\begin{multicols}{3}
			\tableofcontents
		\end{multicols}
	
		\begin{flushright}
			\footnotesize
			Author: \insertauthor
			
			Date: \insertdate
		\end{flushright}
	\end{frame}
}
% temporarily added slide to have a lecture overview 
\overview

% temporarily removed
%\begin{frame}{Lecture Overview -- \insertsubtitle}
%	\tableofcontents[hideallsubsections]
%\end{frame}

\AtBeginSection[]{%
	\begin{frame}{Lecture Overview -- \insertsubtitle}
		\tableofcontents[currentsection,hideothersubsections]
	\end{frame}
}

\newcommand{\sectionend}{\addtocontents{toc}{\newpage}}


\section{Ad-Hoc Approaches for Variability}

\addsummary[3,16]{01-introduction}{Introduction}{1}{
	what is a software product line (feature, product, domain)?
	\also who needs software variability, and why?
	\also is XY a software product line? why, why not?
	\also what is challenging about SPL engineering (compared to traditional software engineering)?
	\also name your own (non-)example for a software product line
}

\addsummary[9,13]{02-runtime}{Runtime Variability and Design Patterns}{2}{
	what is runtime variability (binding time)?
	\also how can configuration parameters be supplied?
	\also (when) does the validity of parameters need to be checked?
	\also how can runtime variability be realized in the code? which way is preferred when?
	\also implement feature XY in a given excerpt of source code
	\also what problems are associated with runtime variability?
	\also how can design patterns be used to implement variability? which design patterns for variability are there?
	\also when should runtime variability (not) be used?
}

\addsummary[7,21]{03-cloneandown}{Compile-Time Variability with Clone-and-Own}{3}{
	what is compile-time variability (clone-and-own, software configuration management)?
	\also when should clone-and-own by unmanaged, managed with version control, or managed with build systems?
	\also create a variant for feature XY in the given version control history
	\also what problems are associated with clone-and-own?
	\also when should clone-and-own (not) be used?
}

\sectionend

\section{Modeling and Implementing Features}

\addsummary[11,49]{04-modeling}{Feature Modeling}{4}{
	what is feature modeling? when is it needed?
	\also what representations of feature models are there? are they equivalent?
	\also would you recommend Excel for feature modeling? why, why not?
	\also given a feature model, explain its syntax and semantics
	\also what are the pros and cons of feature modeling?
	\also create a feature model with XY valid configurations
	\also how can feature models by analyzed? what analyses are there?
	\also transform a given feature model into a propositional formula
	\also what is a (\#)SAT solver? what is it used for?
}

\addsummary[9,17]{05-conditional}{Conditional Compilation}{5}{
	what is conditional compilation (build system, preprocessor, granularity, feature traceability)?
	\also explain the given CPP/Munge/Antenna code
	\also what is the difference between build system and preprocessor variability? can they be combined?
	\also implement feature XY in a given excerpt of source code
	\also what problems are associated with conditional compilation?
	\also how can features be traced?
	\also when should conditional compilation (not) be used?
}

\addsummary{06-modular}{Modular Features}{6}{

}

\addsummary{07-languages}{Languages for Features}{7}{

}

\addsummary{08-process}{Development Process}{8}{

}

\sectionend

\section{Quality Assurance and Maintenance}

\addsummary{09-interactions}{Feature Interactions}{9}{

}

\addsummary{10-analyses}{Product-Line Analyses}{10}{

}

\addsummary{11-testing}{Product-Line Testing}{11}{

}

\addsummary{12-evonance}{Evolution and Maintenance}{12}{

}

\sectionend

\section{What Next?}

\subsection{Prepare for the Exam}

\begin{frame}{\myframetitle}
	\begin{mycolumns}
		\mydefinition{Our Oral Exam}{
			\begin{itemize}
				\item takes $\approx 20$ minutes
				\item starts with questions on lectures 1 and 4
				\item for lectures 2, 3, and 5--7, you draw a random implementation technique, which you have to explain and discuss
				\item then you draw another technique, which you have to compare to the first
				\item finally, some questions on lectures 8--12
			\end{itemize}
		}
	\mynextcolumn
		\mynote{Tips}{
			\begin{itemize}
				\item if you have questions, ask them now
				\item for practicing, answer the questions on the slides above
				\item also use the exam preparation material posted on Elearning
				\item review the exercise tasks
				\item ``roleplay'' an oral exam with a fellow student
			\end{itemize}
		}
	\end{mycolumns}
\end{frame}

\subsection{Take More Courses}

\begin{frame}{\myframetitle}
	\begin{mycolumns}[t]
		\mynote{Student Conference}{
			\begin{itemize}
				\item Master course in the summer term
				\item you will learn how to write a (short) scientific paper on a given topic
				\item usually a literature survey on database or software engineering topics
				\item three draft milestones, two review rounds, two presentations
				\item lectures on academic writing, the publication process, and research ethics
				\item recommended for those who pursue an academic career or want to practice for writing a thesis
			\end{itemize}
		}
	\mynextcolumn
		\mynote{Scientific Team Project}{
			\begin{itemize}
				\item Bachelor/Master course in the summer term
				\item you will conduct a small research project on a database topic
				\item includes a literature survey and implementation
				\item summarized in a project report
				\item difference to student conference: less focus on academic (thesis) writing, more on implementation and team collaboration
			\end{itemize}
		}
		\mynote{}{individual projects also possible}
	\end{mycolumns}
\end{frame}

\subsection{Write Your Thesis}

\begin{frame}{\myframetitle}
	\begin{mycolumns}[t]
		\mynote{Openings for Bachelor/Master Theses}{
			\begin{itemize}
				\item our thesis topics focus on potential new research contributions
				\item industry collaborations possible if there is a relation to our research
				\item example thesis topics \href{https://www.uni-ulm.de/fileadmin/website_uni_ulm/iui.inst.170/projects/spldev/thesis_topics_22-06-29.pdf}{here} (most taken)
				\item if interested, contact Elias at \href{mailto:kuiter@ovgu.de}{kuiter@ovgu.de}
			\end{itemize}
		}
	\mynextcolumn
		\mynote{Open Position for Research Assistant}{
			\begin{itemize}
				\item we are searching for a student interested in contributing to the new FeatureIDE library
				\item you are: an excellent Java developer familiar with (or willing to familiarize with) modern Java constructs (lambdas, streams, futures)
				\item if interested, contact Elias at \href{mailto:kuiter@ovgu.de}{kuiter@ovgu.de}
				\item programming test required
			\end{itemize}
		}
	\end{mycolumns}
\end{frame}

\mode<beamer>{
	\begin{frame}{\inserttitle}
		\lectureseriesoverview
	\end{frame}

	\contentoverview
}


\end{document}
