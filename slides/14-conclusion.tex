\documentclass[
	aspectratio=169, % default is 43
	8pt, % font size, default is 11pt
	handout, % < do not remove this comment, it is used by the Makefile >
]{beamer}
\def\university{} % < do not remove this comment, it is used by the Makefile >

\documentclass[
	aspectratio=169, % default is 43
	8pt, % font size, default is 11pt
	handout, % handout mode without animations, comment out to add animations
]{beamer}

\usepackage{../template/beamerthemeuulm} % use the inofficial uulm beamer theme
\setfaculty{infIngPsy} % set the color scheme for your faculty here [med/infIngPsy/math/nat]

% requires symbolic links
% git clone git@github.com:SoftVarE-Group/SlideTemplate.git C:\Users\...\SlideTemplate
% mklink /J template C:\Users\...\SlideTemplate
% git clone git@spgit.informatik.uni-ulm.de:thuem/slides.git C:\Users\...\ThomasSlides
% mklink /J thomasslides C:\Users\...\ThomasSlides
\graphicspath{{../template/pics/logos}{../template/pics/nature}{../template/pics/uulm}{../thomasslides/}{../pics/people/}{../pics/xkcd/}}

%\usepackage[ngerman]{babel} % use this line for slides in German
%\recordingtrue % special recording mode for use with a greenscreen, gives you space to show yourself in a layer in front of the slides, has no effect in the handout mode

\title{Software Product Lines} % short title is used for the slide footer but optional

% LINKED LITERATURE

\newcommand{\ludewiglichter}{\href{https://learning.oreilly.com/library/view/-/9781457184932/?ar}{Ludewig and Lichter}}
\newcommand{\seeconomics}{\href{https://rds-ulm.ibs-bw.de/link?kid=027381854}{SE Economics}}
\newcommand{\sommervillelink}[1]{\href{https://ulm.ibs-bw.de/aDISWeb/app?service=direct/0/Home/$DirectLink\&sp=SOPAC00\&sp=SAKSWB-IdNr1615420983}{#1}}
\newcommand{\sommerville}{\sommervillelink{Sommerville}}
\newcommand{\thehumbleprogrammer}{\href{https://dl.acm.org/doi/10.1145/1283920.1283927}{The Humble Programmer}}
\newcommand{\thepragmaticprogrammer}{\href{https://learning.oreilly.com/library/view/the-pragmatic-programmer/9780135956977/}{The Pragmatic Programmer}}

% TYPICAL COMMANDS FOR LECTURES

\renewcommand{\emph}[1]{{\color{blue}\textbf{#1}}}

\newcommand{\deutsch}[1]{{\color{blue}(#1)}}
\newcommand{\deutschertitel}[1]{{\tiny\deutsch{#1}}}

\newcommand{\mycite}[1]{``#1''}
\newcommand{\mytitlesource}[1]{{\tiny\normalfont\mbox{[#1]}}}
\newcommand{\mysource}[1]{\ifthenelse{\equal{#1}{}}{}{\phantom{.}~\hfill~\mytitlesource{#1}}}

\newcommand{\todo}[1]{{\color{red}\textbf{[#1]}}}
\newcommand{\fodo}[1]{\todo{\footnote{\todo{#1}}}}
\newcommand{\todots}{\todo{\ldots}}

% IMPORTED PACKAGES

%\usepackage{adjustbox} % used for partofpage
%\usepackage{tcolorbox} % used for mydefinition, mynote, myexample
\usepackage{multicol} % used temporarily for the lecture overview
\usepackage{mathtools} % required for absolute value in modeling lecture

% COMMANDS TO LAYOUT AND ANNIMATE SLIDES

\newcommand{\lessonslearned}[3]{
	\subsection{Summary}
	\begin{frame}{\insertsection -- \insertsubsection}
		\leftorright{
			\mydefinition{Lessons Learned}{
				\begin{itemize}
					#1
				\end{itemize}
			}
			\mynote{Further Reading}{
				\small % references take space, can be a little smaller
				\begin{itemize}
					#2
				\end{itemize}
			}
		}{
			\myexample{Practice}{
				#3
			}
		}
	\end{frame}
}

% TODO temporary hack to layout the slide overview in two colums
\renewcommand{\lectureoverview}{
%	\section*{Overview}
%	\subsection*{Overview}
	\begin{frame}{\insertsubtitle}
		\begin{multicols}{2}
			\tableofcontents
		\end{multicols}
	\end{frame}
}

\renewcommandx{\maketitle}[2][1=apr21-o25a,2=150]{
    {
	\usebackgroundtemplate{} % TODO temporary hack to enable missing pictures at title slide
	%\ifx {#1} \empty \else {\usebackgroundtemplate{\includegraphics[trim=0 0 0 #2,clip,width=\paperwidth]{#1}}} \fi     
	%\usebackgroundtemplate{\includegraphics[trim=0 0 0 #2,clip,width=\paperwidth]{#1}}
    \begin{frame}[plain]
        \vskip0pt plus 1filll
        \begin{beamercolorbox}[wd=\paperwidth,ht=4.5ex,dp=2ex,right]{titlebox}
            \LARGE\textbf{\inserttitle}\hspace*{20pt}
        \end{beamercolorbox}%
        \nointerlineskip%
        \begin{beamercolorbox}[wd=\paperwidth,ht=2.25ex,dp=1ex,right]{subtitlebox}
            \small 
            \ifx \insertsubtitle \empty \else \insertsubtitle\ $\vert$ \fi
            \insertauthor\
            \ifx \insertdate \empty \else $\vert$ \insertdate \fi
            \hspace*{20pt}
        \end{beamercolorbox}%
        \nointerlineskip%
        \begin{beamercolorbox}[wd=\paperwidth,ht=4.5ex,dp=2ex,left]{logobox}
            \centering
            \vspace{-1ex}
            \hspace{10pt}
            \includegraphics[height=4.5ex]{sp} % SPECIFY INSTITUTE LOGO HERE
            \hfill
            \includegraphics[height=4.5ex]{uulm}
            \hspace{10pt}
        \end{beamercolorbox}%
    \end{frame}
    }  
}

%
%\newcommand{\onlyleft}[1]{
%	\halfpage{#1}
%}
%
%\newcommand{\onlyright}[1]{
%	~\hfill
%	\halfpage{#1}
%}
%
%\newcommand{\leftorright}[2]{
%	\uncover<1>{\halfpage{#1}}
%	\hfill
%	\uncover<3->{\halfpage{#2}}
%}
%
%\newcommand{\rightorleft}[2]{
%	\uncover<3->{\halfpage{#1}}
%	\hfill
%	\uncover<1>{\halfpage{#2}}
%}
%
%\newcommand{\leftthenright}[2]{
%	\halfpage{#1}
%	\hfill\pause
%	\halfpage{#2}
%}
%
%\newcommand{\leftandright}[2]{
%	\halfpage{#1}
%	\hfill
%	\halfpage{#2}
%}
%
%\newcommand{\leftmiddleandright}[3]{
%	\thirdpage{#1}
%	\hfill
%	\thirdpage{#2}
%	\hfill
%	\thirdpage{#3}
%}
%
%\newcommand{\leftmiddleorright}[3]{
%	\uncover<1>{\thirdpage{#1}}
%	\hfill
%	\uncover<3>{\thirdpage{#2}}
%	\hfill
%	\uncover<5->{\thirdpage{#3}}
%}
%
%\newcommand{\halfpage}[1]{\partofpage{48}{#1}}
%
%\newcommand{\thirdpage}[1]{\partofpage{31}{#1}}
%
%\newcommand{\partofpage}[2]{
%	\adjustbox{valign=t}{\begin{minipage}{0.#1\textwidth}
%			\begin{flushleft}
%				#2
%			\end{flushleft}
%	\end{minipage}}
%}
%
%\newcommand{\mydefinition}[2]{
%	\begin{tcolorbox}[title=#1,colback=orange!10,colframe=orange!30,coltitle=black,fonttitle=\bfseries,left=1mm,right=1mm,top=1mm,bottom=1mm]
%		\begin{flushleft}
%			#2
%		\end{flushleft}
%	\end{tcolorbox}
%}
%
%\newcommand{\mydefinitiontight}[2]{
%	\begin{tcolorbox}[title=#1,colback=white,colframe=orange!30,coltitle=black,fonttitle=\bfseries,left=0mm,right=0mm,top=0mm,bottom=0mm]
%		\begin{flushleft}
%			#2
%		\end{flushleft}
%	\end{tcolorbox}
%}
%
%\newcommand{\mynote}[2]{
%	\begin{tcolorbox}[title=#1,colback=red!10,colframe=red!30,coltitle=black,fonttitle=\bfseries,left=1mm,right=1mm,top=1mm,bottom=1mm]
%		\begin{flushleft}
%			#2
%		\end{flushleft}
%	\end{tcolorbox}
%}
%
%\newcommand{\myexample}[2]{
%	\begin{tcolorbox}[title=#1,colback=blue!10,colframe=blue!30,coltitle=black,fonttitle=\bfseries,left=1mm,right=1mm,top=1mm,bottom=1mm]
%		\begin{flushleft}
%			#2
%		\end{flushleft}
%	\end{tcolorbox}
%}
%
%\newcommand{\myexampletight}[2]{
%	\begin{tcolorbox}[title=#1,colback=white,colframe=blue!30,coltitle=black,fonttitle=\bfseries,left=0mm,right=0mm,top=0mm,bottom=0mm]
%		\begin{flushleft}
%			#2
%		\end{flushleft}
%	\end{tcolorbox}
%}

\subtitle{14. Conclusion}
\author{Elias Kuiter, Thomas Thüm, Timo Kehrer}

\ifuniversity{ulm}{\setpicture[250]{may21-ulm}}
\ifuniversity{magdeburg}{\setpicture[10]{ovgu-winter7}\setcopyright{Photo: Ulrich Arendt (OVGU)}}

\newcommand{\addsummary}[5][]{
	\subsection{#3}
	\begin{frame}{\inserttitle}
		\lectureseriesoverview[#4]
	\end{frame}
	\foreach \x in {#1}{
		\begin{frame}{}\hspace*{-1cm}\includegraphics[page=\x]{#2}\end{frame}
	}
	\begin{frame}{}
		\myexample{Selected Questions and Tasks for Lecture #4}{
			\begin{itemize}\item #5 \item \ldots\end{itemize}
		}
	\end{frame}
}
\newcommand{\also}{\item}

\begin{document}

\mode<handout>{\contentoverview}

\mode<beamer>{
	\ifdefined\thepicture
		\maketitle[\thepicture][\thepictureoffset]
	\else
		\maketitle[]
	\fi
}

% shared slide content

% introduced: 02a-configuration
% reused: 03a-intro
\newcommand{\frameImplementSPLs}{
	\begin{mycolumns}[widths={45},animation=none]
		\pic[width=\linewidth]{metaproduct2}
	\mynextcolumn
		\begin{note}{Key Issues}
			\begin{itemize}
			\item Systematic reuse of implementation artifacts
			\item Explicit handling of variability
			\end{itemize}
		\end{note}
		\uncover<2->{\begin{definition}{Variability\mysource{\fospl\mypage{48}}}
			\mycite{\emph{Variability} is the ability to derive different products from a common set of artifacts.}
		\end{definition}}
		~
		\uncover<3->{\begin{note}{Variability-Intensive System}
			Any software product line is a variability-intensive system. % TODO Timo: do we really need this term? where does this definition come from?
		\end{note}}
	\end{mycolumns}
}

% introduced: 02a-configuration
% reused: 02b-implementation, 03a-intro
\newcommand{\frameVariabilityAndBindingTimes}{
	\begin{mycolumns}[widths={55},animation=none]
		\begin{definition}{Binding Time \deutsch{Bindungszeitpunkt}\mysource{\fospl\mypage{48}}}
			\begin{itemize}
				\item Variability offers choices
				\item Derivation of a product requires to make decisions (aka. binding)
				\item Decisions may be bound at different binding times
			\end{itemize}
		\end{definition}
		~
		\uncover<2->{\begin{note}{When? By whom? How?}
			\lectureruntime\parta: \emph{when} and \emph{by whom}

			\lectureruntime\partb: \emph{how}
		\end{note}}
	\mynextcolumn
		\pic[width=\linewidth]{metaproduct2}
	\end{mycolumns}
}

% introduced: 03a-intro
% reused: 03a-intro
\newcommand{\frameRuntimeVariabilityProblems}{
	\begin{note}{Problems of Runtime Variability}
		{\bf Conditional Statements:}
		\begin{itemize}
			\item Code scattering, tangling, and replication
		\end{itemize}
		{\bf Design Patterns for Variability:}
		\begin{itemize}
			\item Trade-offs and potential negative side effects
			\item Constraints that may restrict their usage
		\end{itemize}
		{\bf In General:}
		\begin{itemize}
			\item Variable parts are always delivered
			\item Not well-suited for compile-time binding
		\end{itemize}
	\end{note}
}

% introduced: 03a-intro
% reused: 03a-intro
\newcommand{\frameSoftwareConfigurationManagement}{
	\begin{mycolumns}
		\begin{definition}{Software Configuration Management} % TODO source missing
			Policies, processes, and tools for managing evolving software systems:
			\begin{itemize}
				\item Version control
				\item System building
				\item Release management
				\item Change management
				\item Collaborative work
			\end{itemize}
		\end{definition}
	\mynextcolumn
		\begin{note}{No Software Configuration Management}
			\lecturecloneandown\parta: Ad-Hoc Clone-and-Own

			aka.\ unmanaged clone-and-own
		\end{note}
		\begin{note}{Version Control}
			\lecturecloneandown\partb: Clone-and-Own with Version Control

			instance of managed clone-and-own
		\end{note}
		\begin{note}{System Building}
			\lecturecloneandown\partc: Clone-and-Own with Build Systems

			instance of managed clone-and-own
		\end{note}
	\end{mycolumns}
}


\begin{frame}{\inserttitle}
	\lectureseriesoverview[14]
\end{frame}

\section{Ad-Hoc Approaches for Variability}

\addsummary[3,16]{01-introduction}{Introduction}{1}{
	what is a software product line (feature, product, domain)?
	\also who needs software variability, and why?
	\also is XY a software product line? why, why not?
	\also what is challenging about SPL engineering (compared to traditional software engineering)?
	\also name your own (non-)example for a software product line
}

\addsummary[9,13]{02-runtime}{Runtime Variability and Design Patterns}{2}{
	what is runtime variability (binding time)?
	\also how can configuration parameters be supplied?
	\also (when) does the validity of parameters need to be checked?
	\also how can runtime variability be realized in the code? which way is preferred when?
	\also implement feature XY in a given excerpt of source code
	\also what problems are associated with runtime variability?
	\also how can design patterns be used to implement variability? which design patterns for variability are there?
	\also when should runtime variability (not) be used?
}

\addsummary[7,21]{03-cloneandown}{Compile-Time Variability with Clone-and-Own}{3}{
	what is compile-time variability (clone-and-own, software configuration management)?
	\also when should clone-and-own by unmanaged, managed with version control, or managed with build systems?
	\also create a variant for feature XY in the given version control history
	\also what problems are associated with clone-and-own?
	\also when should clone-and-own (not) be used?
}

\sectionend

\section{Modeling and Implementing Features}

\addsummary[11,49]{04-modeling}{Feature Modeling}{4}{
	what is feature modeling? when is it needed?
	\also what representations of feature models are there? are they equivalent?
	\also would you recommend Excel for feature modeling? why, why not?
	\also given a feature model, explain its syntax and semantics
	\also what are the pros and cons of feature modeling?
	\also create a feature model with XY valid configurations
	\also how can feature models by analyzed? what analyses are there?
	\also transform a given feature model into a propositional formula
	\also what is a (\#)SAT solver? what is it used for?
}

\addsummary[9,17]{05-conditional}{Conditional Compilation}{5}{
	what is conditional compilation (build system, preprocessor, granularity, feature traceability)?
	\also explain the given CPP/Munge/Antenna code
	\also what is the difference between build system and preprocessor variability? can they be combined?
	\also implement feature XY in a given excerpt of source code
	\also what problems are associated with conditional compilation?
	\also how can features be traced?
	\also when should conditional compilation (not) be used?
}

\addsummary[12,25]{06-modular}{Modular Features}{6}{
	what is modularity (cohesion, coupling, encapsulation, plug-in)?
	\also what is the difference between using components, services, and frameworks for implementing variability? when should which be preferred?
	\also what is a microservice architecture?
	\also what problems are associated with these techniques?
	\also when should these techniques (not) be used?
}

\addsummary[20,45]{07-languages}{Languages for Features}{7}{
	what is a feature module (collaboration, role, aspect, join point, pointcut, crosscutting concern, obliviousness)?
	\also what is feature-oriented (aspect-oriented) programming? when to use it?
	\also explain the tyranny of the dominant decomposition at a given example
	\also refactor given FOP code into AOP and vice versa
	\also does composition order matter for FOP/AOP? why, why not?
	\also what problems are associated with FOP and AOP, respectively?
	\also when should these techniques (not) be used?
}

\addsummary[7,34]{08-process}{Development Process}{8}{
	what is SPL enginering (problem space, solution space, domain engineering, application engineering, scoping)?
	\also explain each part of the SPL development process at a given example
	\also compare two given implementation techniques
	\also how to introduce a product line in practice? what adoption strategies are there?
}

\sectionend

\section{Quality Assurance and Maintenance}

\addsummary[8,36]{09-interactions}{Feature Interactions}{9}{
	what is a ($t$-wise) feature interaction ($t \in \mathbb{N^+}$)?
	\also what kinds of interactions are there?
	\also for $t$ features, how many potential interactions are there? for which values of $t$ do interactions occur in practice?
	\also how to find feature interactions?
	\also how to handle feature interactions?
	\also apply a feature interaction handling strategy to a given example
	\also how to avoid feature interactions?
}

\addsummary[14,35]{10-analyses}{Product-Line Analyses}{10}{
	what is a program analysis (product-line analysis)?
	\also what strategies for analyzing SPLs are there?
	\also given an algorithm, classify its analysis strategy
	\also given CPP source code, determine its presence conditions, dead code, and superfluous annotations
	\also (how) can the problem and solution space be combined?
	\also give examples for easy and difficult SPLs in terms of analysis effort
}

\addsummary[10,32]{11-testing}{Product-Line Testing}{11}{
	what is sample-based testing (combinatorial interaction testing, t-wise interaction coverage, solution-space sampling)?
	\also given a feature model, list all potential t-wise feature interactions
	\also given a feature model and t-wise sample, verify that the sample fulfills t-wise interaction coverage
	\also how efficient and effective are sampling algorithms? is there a tradeoff?
	\also how can presence conditions be employed for improving the sampling efficiency?
}

\subsection{Refactoring}
\begin{frame}{}
	\myexample{Selected Questions and Tasks for Lecture 12}{
		\begin{itemize}
			\item what is an (SPL) refactoring?
			\item why are refactorings important? when are they needed?
			\item what kinds of code refactorings are there?
			\item what kinds of feature-model refactorings are there? how can they be tested in an automated fashion?
			\item \ldots
		\end{itemize}
	}
\end{frame}

\sectionend

\section{What Next?}

\subsection{Prepare for the Exam}

\begin{frame}{\myframetitle}
	\begin{mycolumns}
		\mydefinition{Oral Exam}{
			\begin{itemize}
				\item takes $\approx 20$ minutes
				\item starts with questions on lectures 1 and 4
				\item for lectures 2, 3, and 5--7, we ask you to draw a random implementation technique, which you have to explain and discuss
				\item then you draw another technique, which you have to compare to the first
				\item finally, some questions on lectures 8--12
			\end{itemize}
		}
	\mynextcolumn
		\mynote{Tips}{
			\begin{itemize}
				\item for practicing, answer the questions on the slides above
				\item also use the exam preparation material posted on Elearning
				\item review the tasks from the exercise
				\item revise your diagram from exercise 11
				\item ``roleplay'' an oral exam with a fellow student
				\item if you have questions, ask them now
			\end{itemize}
		}
	\end{mycolumns}
\end{frame}

\subsection{Take More Courses}

\begin{frame}{\myframetitle}
	\begin{mycolumns}
		\mynote{Student Conference}{
			\begin{itemize}
				\item Master course in the summer term
				\item you will learn how to write a (short) scientific paper on a given topic
				\item usually a literature survey on database or software engineering topics
				\item three draft milestones, two review rounds, two presentations
				\item lectures on academic writing, the publication process, and research ethics
				\item recommended for those who pursue an academic career or want to practice for writing a thesis
			\end{itemize}
		}
		\mynote{}{planned: seminar \deutsch{Proseminar/wiss. Seminar} in the summer term, ``lite'' version of student conf.}
	\mynextcolumn
		\mynote{Scientific Team Project \deutsch{Wiss. Teamprojekt}}{
			\begin{itemize}
				\item Bachelor/Master course in the summer term
				\item you will conduct a small research project on a database topic
				\item includes a literature survey and implementation
				\item summarized in a project report
				\item difference to student conference: less focus on academic (thesis) writing, more on implementation and team collaboration
			\end{itemize}
		}
		\mynote{}{individual project \deutsch{Individualprojekt} also possible}
	\end{mycolumns}
\end{frame}

\subsection{Write Your Thesis}

\begin{frame}{\myframetitle}
	\begin{mycolumns}[t]
		\mynote{Openings for Bachelor/Master Theses}{
			\begin{itemize}
				\item our thesis topics focus on potential new research contributions
				\item industry collaborations only if there is a relation to our research
				\item general area: feature-model analysis, CNF transformation, (\#)SAT solving
				\item example thesis topics \href{https://www.uni-ulm.de/fileadmin/website_uni_ulm/iui.inst.170/projects/spldev/thesis_topics_22-06-29.pdf}{here} (most taken)
				\item if interested, contact Elias at \href{mailto:kuiter@ovgu.de}{kuiter@ovgu.de}
			\end{itemize}
		}
	\mynextcolumn
		\mynote{Open Position for Research Assistant}{
			\begin{itemize}
				\item we are searching for a student interested in contributing to the new FeatureIDE library
				\item you are: an excellent Java developer familiar with (or willing to familiarize with) modern Java constructs (lambdas, streams, futures)
				\item if interested, contact Elias at \href{mailto:kuiter@ovgu.de}{kuiter@ovgu.de}
				\item programming test required
			\end{itemize}
		}
	\end{mycolumns}
\end{frame}

\begin{frame}{}
	\centering\Huge Thank you for participating!
\end{frame}

\mode<beamer>{
	\begin{frame}{\inserttitle}
		\lectureseriesoverview
	\end{frame}

	\contentoverview
}


\end{document}
