\begin{frame}{Recap: Software Product Lines}
	\begin{mycolumns}
		\mydefinition{Software Product Line \mysource{\seiwhitepaperspl\mypage{5}}}{\mycitebegin A \emph{software product line} is 
			\begin{itemize}
				\item a set of software-intensive systems (aka.\ products or variants)
				\item that share a common, managed set of features 
				\item satisfying the specific needs of a particular market segment or mission (aka.\ domain)
				\item and that are developed from a common set of core assets in a prescribed way.\myciteend
			\end{itemize}
			\mysource{\href{https://resources.sei.cmu.edu/library/asset-view.cfm?assetID=513819}{Software Engineering Institute, Carnegie Mellon University}}
		}
	\mynextcolumn
		\pic[width=\linewidth,page=24]{lego}
	\end{mycolumns}
\end{frame}

\subsection{Variability and Binding Time}

\begin{frame}{How to Implement Software Product Lines?}
	\begin{mycolumns}[widths={45},animation=none]
		\includegraphics[width=\linewidth]{metaproduct2}	
	\mynextcolumn
		\mynote{Key Issues}{
			\begin{itemize}
			\item Systematic reuse of implementation artifacts.
			\item Explicit handling of variability.
			\end{itemize}
		}
		\uncover<2->{\mydefinition{Variability\mysource{\fospl\mypage{48}}}{
			Variability is the ability to derive different products from a common set of artifacts.
		}}
		~
		\uncover<3->{\mynote{Variability-Intensive System}{
			Any software product line is a variability-intensive system.
		}}
	\end{mycolumns}
\end{frame}

\begin{frame}{Variability and Binding Times}
	\begin{mycolumns}[widths={45},animation=none]
		\includegraphics[width=\linewidth]{metaproduct2}	
	\mynextcolumn
		\mydefinition{Binding Time}{
			\begin{itemize}
				\item Variability offers choices.
				\item Derivation of a product requires to make decisions (aka. binding).
				\item Decisions may be bound at different binding times.
			\end{itemize}
		}	
		\uncover<2->{\mynote{When, how and by whom?}{
			In the sequel: Focus on when and by whom\ldots
		}}
	\end{mycolumns}
\end{frame}

\subsection{Command-Line Options}

\begin{frame}{Example: Command-Line Options}
	\leftorright{
		\includegraphics[width=\linewidth]{runtime-parameters-win10-cmd-dir}
	}{
		\includegraphics[width=\linewidth]{runtime-parameters-win10-cmd-dir2}
	}
\end{frame}

\subsection{Configuration Files}

\begin{frame}{Example: Configuration Files}
	\includegraphics[width=0.75\linewidth]{configfile-eclipse-ini}
\end{frame}

\subsection{Preference Dialogs}

\begin{frame}{Example: Preference Dialogs}
	\includegraphics[width=0.75\linewidth]{preferences-eclipse}
\end{frame}

\subsection{Runtime Variability}

\begin{frame}{What do these examples have in common?}
	\begin{mycolumns}[columns=3,widths={26,36,36},animation=none]
		\pic[width=\linewidth]{runtime-parameters-win10-cmd-dir}
	\mynextcolumn
		\pic[width=\linewidth]{configfile-eclipse-ini}
	\mynextcolumn
		\pic[width=\linewidth]{preferences-eclipse}
	\end{mycolumns}
		
	\uncover<2->{
		\begin{mycolumns}[columns=2,widths={55,40},animation=none]
			\mynote{Configuration Parameters}{
				\begin{itemize}
					\item Behavior of a program is determined by configuration parameters being interpreted at runtime.
					\item In other words: Choices offered by variability are decided at runtime.
					\item Configuration may happen non-interactively (at startup) or interactively (through dialogs).
				\end{itemize}
			}			
		\mynextcolumn	
			\mydefinition{Runtime Variability\mysource{\fospl\mypage{49}}}{
				Runtime variability is decided after compilation when the program is started (aka. load-time variability) or during program execution.
			}	
		\end{mycolumns}
	}
\end{frame}

\subsection{Configuration Parameters}

\begin{frame}{Example: A Graph Library}
	A simple library providing \ldots \\
	\vspace{5mm}
	\leftorright{
		\myexample{\ldots graph data structures}{
			\begin{itemize}
				\item Directed/undirected edges
				\item Weighted/unweighted edges 
				\item Colored/uncolored nodes
				\item etc.
			\end{itemize}
		}		
	}{
		\myexample{\ldots and algorithms}{
			\begin{itemize}
				\item Vertex numbering
				\item Vertex coloring 
				\item Shortest path
				\item Minimum spanning tree 
				\item etc.
			\end{itemize}
		}
	}
\end{frame}

\begin{frame}{Features of a Graph}
	\begin{mycolumns}[columns=4,widths={25},animation=none]
		\includegraphics[width=\linewidth,page=1]{graphs}
		\includegraphics[width=\linewidth,page=3]{graphs}
		\ldots etc.
	\mynextcolumn
		\includegraphics[width=\linewidth,page=5]{graphs}
		\includegraphics[width=\linewidth,page=7]{graphs}
	\mynextcolumn
		\includegraphics[width=\linewidth,page=11]{graphs}
		\includegraphics[width=\linewidth,page=13]{graphs}
	\mynextcolumn
		\includegraphics[width=\linewidth,page=15]{graphs}
		\includegraphics[width=\linewidth,page=17]{graphs}		
	\end{mycolumns}
\end{frame}

\begin{frame}{Features as Configuration Parameters}
	\begin{mycolumns}[columns=3,widths={25,25,45},animation=none]
		\myexampletight{Directed}{\includegraphics[width=\linewidth,page=3]{graphs}}	
		\myexampletight{Colored}{\includegraphics[width=\linewidth,page=11]{graphs}}	
	\mynextcolumn
		\myexampletight{Weighted}{\includegraphics[width=\linewidth,page=5]{graphs}}	
		\myexampletight{Weighted, Colored}{\includegraphics[width=\linewidth,page=15]{graphs}}			
	\mynextcolumn
		\includegraphics[width=0.45\linewidth,page=6]{graphs}
		\hfill
		\includegraphics[width=0.45\linewidth,page=12]{graphs}
		
		~
		
		\mynote{Configuration of graph data structures}{
			\begin{itemize}
				\item Typically, configuration parameters are {\em flags}.
				\item Their boolean value determines which {\em features} are {\em activated} and which ones are {\em deactivated}.
			\end{itemize}
		}
	\end{mycolumns}
\end{frame}

\subsection{Validity of Parameter Settings}

\begin{frame}{\myframetitle}
	\begin{mycolumns}[columns=4,widths={65,5,15,15},animation=none]
		\begin{tabular}{llll}
			\toprule
			{\bf Algorithm} 							& {\bf Graph type} 	& {\bf Weights} & {\bf Coloring}  \\ \midrule
			{\em Vertex numbering}			  & *          				& *        			& *         			\\
			{\em Vertex coloring}       	& undirected 				& *        			& colored   			\\
			{\em Shortest path}        		& directed   				& weighted 			& *         			\\
			{\em Minimum spanning tree} 	& undirected 				& weighted 			& *         			\\
			\ldots         					& \ldots 			 			& \ldots 		  	& \ldots 					\\ \bottomrule
		\end{tabular}
		\vspace{5mm}	
		\uncover<2->{\mynote{Dependencies between features must be checked}{
			\begin{itemize}
				\item When the parameters are configured at startup, or
				\item whenever parameters are changed at runtime.
			\end{itemize}
		}}
	\mynextcolumn
		~
	\mynextcolumn
		\myexampletight{Directed}{\includegraphics[width=\linewidth,page=3]{graphs}}	
		\myexampletight{Colored}{
			\hfill\\
			\hfill\\
			\includegraphics[width=\linewidth,page=11]{graphs}
		}	
	\mynextcolumn
		\myexampletight{Weighted}{\includegraphics[width=\linewidth,page=5]{graphs}}	
		\myexampletight{Weighted, Colored}{\includegraphics[width=\linewidth,page=15]{graphs}}	
	\end{mycolumns}
\end{frame}