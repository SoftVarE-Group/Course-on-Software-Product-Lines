\subsection{Software Configuration Management}

\begin{frame}{Excursus: Software Configuration Management}
	\frameSoftwareConfigurationManagement
\end{frame}

\begin{frame}{Excursus: Software Configuration Management}
	\myframeicon{\mysource{\fospl\mypages{100--101}}}% TODO source missing for configuration and baseline
	\begin{fancycolumns}[widths={45},animation=none]
		\begin{definition}{Basic Terms and Definitions}
			\begin{itemize}
				\item {\bf Software Item}: An (atomic) artifact that can be uniquely identified
				\item {\bf Version}: A modified software item
				\begin{itemize}
					\item {\bf Revision}: A new version that replaces an old one
					\item {\bf Variant}: A version that co-exists with another one
				\end{itemize}
				\uncover<2->{
					\item {\bf Configuration}: A set of software items that together form a functioning (partial) system % TODO term is confusing and overloaded in next lecture + it appears to me that this was only needed in the good old days with CVS with versioning on a per file basis
					\item {\bf Baseline}: A stable configuration that represents a point of reference for further development
					\item {\bf Release}: A baseline delivered to customers
				}
			\end{itemize}
		\end{definition}
	\nextcolumn
		\only<1|handout:0>{\pic[width=\linewidth]{configuration-management-1}}%
		\only<2-|handout:1>{\pic[width=\linewidth]{configuration-management-2}}%
	\end{fancycolumns}	
\end{frame}
% TODO introduce terms variation in time and variation in space? intentional versioning?
% TODO introduce what branches are designed for? development branches, release branches, feature branches, bug fixes \fospl\mypages{101--102}

\begin{frame}{Example: A Conceptual Organization of our Graph Library}
	\centering
	\only<1|handout:0>{\pic[width=0.8\linewidth]{configuration-management-graphs-1}}%
	\only<2-|handout:1>{\pic[width=0.8\linewidth]{configuration-management-graphs-2}}%
\end{frame}
% TODO axis labels missing on this slide

%\begin{frame}{Tool Support: Version Control Systems}
	%\only<1|handout:0>{\pic[width=\linewidth,page=1]{branching}}%
	%\only<2|handout:0>{\pic[width=\linewidth,page=2]{branching}}%
	%\only<3-|handout:1>{\pic[width=\linewidth,page=3]{branching}}%
%\end{frame}

\subsection{Version Control Systems}

\begin{frame}{Tool Support: \myframetitle}
	\centering
	\uncover<4->{\hspace{30mm}$\text{cherrypick := patch}(\Delta(r8,r10),r11)$}

	\vfill
	\only<1|handout:0>{\pic[width=0.6\linewidth]{versioncontrol-1}}%
	\only<2|handout:0>{\pic[width=0.6\linewidth]{versioncontrol-2}}%
	\only<3-|handout:1>{\pic[width=0.6\linewidth]{versioncontrol-3}}\\

	\vfill
	\uncover<5->{$\text{merge := 3-way-merge}(r4,\Delta(r4,r7),\Delta(r4,r9))$}
\end{frame}
% TODO no actual merges shown, only fast-forward
% TODO cherry pick needs to point to an arrow, not a node

\subsection{Variability with Version Control}

\begin{frame}{Example: Graph Library under Version Control}
	\centering\pic[width=0.7\linewidth]{versioncontrol-graphs-1}
\end{frame}

\begin{frame}{Example: Graph Library under Version Control}
	\centering\pic[width=0.7\linewidth]{versioncontrol-graphs-2}
\end{frame}

\begin{frame}{Example: Graph Library under Version Control}
	\centering\pic[width=0.7\linewidth]{versioncontrol-graphs-3}
\end{frame}

% TODO we have not yet discussed how features could be developed in separate branches \fospl\mypage{103}

\subsection{Discussion}

\begin{frame}{Clone-and-Own with Version Control}
	\begin{fancycolumns}[widths={60}]
		\begin{note}{Observations}
			\begin{itemize}
				\item Aka.\ \emph{managed clone-and-own} (opposed to ad-hoc clone-and-own)
				\item Supports keeping track of revisions and variants\\-- aka.\ \emph{provenance} \deutsch{Herkunft}
				\item Creation of new variants is partially supported by merging of branches
				\item Propagation of changes between variants is supported by cherrypicking changes
			\end{itemize}
			However:
			\begin{itemize}
				\item Versioning is typically limited to entire system variants (i.e., branches)
				\item No flexible combination of software items
			\end{itemize}
		\end{note}
	\nextcolumn
		\begin{note}{Advantages\mysource{\fospl\mypage{104}}}
			\begin{itemize}
				\item Well-established and stable systems
				\item Well-known known process
				\item Good tool integration	
			\end{itemize}
		\end{note}
		\begin{note}{Disadvantages\mysource{\fospl\mypages{104--105}}}
			\begin{itemize}
				\item Development of variants, not features: flexible combination of features not directly possible
				\item No structured reuse (copy \& edit)
				\item Merging and cherrypicking not fully automated
			\end{itemize}	
		\end{note}
	\end{fancycolumns}
\end{frame}

\begin{frame}{No Version Control at All?}
	\begin{fancycolumns}[columns=3,widths={20,60},animation=none]
	\nextcolumn
		\begin{note}{Revision Control $\subset$ Version Control}
			\mycite{Unless only few small variations are required for few customers, the use of version control systems should be restricted to \emph{revision control}.}\mysource{\fospl\mypage{104}}
		\end{note}
	\nextcolumn
	\end{fancycolumns}
\end{frame}
