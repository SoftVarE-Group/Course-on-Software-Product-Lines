
\begin{frame}<2>{Recap: Software Configuration Management}
	\frameSoftwareConfigurationManagement
\end{frame}

\subsection{Build Systems}

\begin{frame}{Tool Support: \myframetitle}
	\begin{fancycolumns}[widths={40},animation=none]
		\begin{definition}{Build Systems} % TODO source missing
			\begin{itemize}
				\item Automation of the build process through build scripts
				\item Multiple steps with dependencies/conditions
				\begin{itemize}
					\item Copy files, 
					\item call compiler, 
					\item start other tools, 
					\item \ldots
				\end{itemize}
				\item Tools: 
				\begin{itemize}
					\item make
					\item ant
					\item maven
					\ldots
				\end{itemize}
			\end{itemize}
		\end{definition}
	\nextcolumn
		\picDark[width=\linewidth]{ant-hello-world}
	\end{fancycolumns}	
\end{frame}

\subsection{Variability with Build Systems}

\begin{frame}{\myframetitle}
	\myframeicon{\mysource{\staplesandhill}}%
	\begin{fancycolumns}
		\begin{note}{Basic Idea}
			\begin{itemize}
				\item One build script per variant
				\item Include/exclude files when translating
				\item Overwrite variant-specific files
			\end{itemize}
		\end{note}
	\nextcolumn
		\vspace{-\textheightoftitle}

		\centering\pic[height=\textheightwithouttitle]{buildsystems} % TODO looks weird in dark mode
	\end{fancycolumns}	
\end{frame}

% TODO the next two slides could profit from adding the main insights in a short but textual form
\begin{frame}[fragile]{Example: Graph Library}
	\myframeicon{\mysource{\fospl\mypage{107}}}%
	\begin{fancycolumns}[b,columns=3,widths={50,32},height=\textheightwithtitle,animation=none]
		\hfill\pic[height=\textheightwithtitle]{buildsystems-graphs-1} % TODO what about Node.java and Weight.java? + looks weird in dark mode
	\nextcolumn
\begin{codetight}{}
class Edge {
	Node a, b;

	Edge(Node a, Node b) {
		this.a = a; this.b = b;
	}
	void print() {
		a.print(); b.print();
	}
}
\end{codetight}
\vspace{2mm}
\begin{codetight}{}
class Edge {
	Node a, b;
	@Weight weight = new Weight();@

	Edge(Node a, Node b) {
		this.a = a; this.b = b;
	}
	void print() {
		a.print(); b.print();
		@weight.print();@
	}
}
\end{codetight}
	\nextcolumn
	\end{fancycolumns}
\end{frame}

\begin{frame}[fragile]{Example: Graph Library}
	\begin{fancycolumns}[b,columns=3,widths={50,43},height=\textheightwithtitle,animation=none]
		\hfill\pic[height=\textheightwithtitle]{buildsystems-graphs-2} % TODO what about Node.java, Weight.java, EdgeFactory.java? + looks weird in dark mode
	\nextcolumn
\begin{codetight}{}
class Graph {
	@EdgeFactory edgeFactory;@
	...
	@Graph(EdgeFactory edgeFactory) {
		this.edgeFactory = edgeFactory;
	}@
	Edge add(Node n, Node m) {
		@Edge e = edgeFactory.createEdge(n, m);@
		nodes.add(n); nodes.add(m); edges.add(e);
		return e;
	}
}
\end{codetight}
\begin{codetight}{}
class Edge {
	Node a, b;
	...
}
\end{codetight}
\vspace{2mm}
\begin{codetight}{}
class WeightedEdge extends Edge {
	@Weight weight = new Weight();@
	...
}
\end{codetight}
	\nextcolumn
	\end{fancycolumns}
\end{frame}

% TODO new slide about experience report by \staplesandhill\ Section~3--6

\subsection{Discussion}

\begin{frame}{Clone-and-Own with Build Systems}
	\vspace{-\textheightoftitle}
	\begin{fancycolumns}[b,widths={55},animation=none]
		\begin{note}{Comparison to Version Control Systems}
			\begin{itemize}
				\item Supports combination of more fine-grained software items (i.e., files)
				\item However: Only limited support for provenance
			\end{itemize}
		\end{note}
		\begin{note}{In General}
			\begin{itemize}
				\item Combination of items (i.e., files)\\$\neq$ combination of features
				\item Changes to the basic variant may have undesired side-effects
					\begin{itemize}
						\item Some variants are updated but do not need those changes
						\item Some variants are updated but incompatible to those changes
						\item Variants with copied files are not automatically updated
					\end{itemize}
			\end{itemize}
		\end{note}
	\nextcolumn
		\centering\pic[height=\textheightwithouttitle]{buildsystems} % TODO looks weird in dark mode
	\end{fancycolumns}
\end{frame}

% TODO a general discussion of advantages and disadvantages is missing here, see \fospl\mypages{108--110} (but be aware that the book is mixing two ways of using build systems)
