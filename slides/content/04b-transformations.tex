\subsection{Representations and Transformations}

\begin{frame}[label=FeatureModelRepresentations]{\myframetitle}
	\begin{fancycolumns}[b]
		\begin{example}{Natural Language}
			\tiny ``A \feat{configurable database} has an API that allows for at least one of the request types \feat{Get}, \feat{Put}, or \feat{Delete}.
			Optionally, the database can support \feat{transactions}, provided that the API allows for Put or Delete requests.
			Also, the database targets a supported operating system, which is either \feat{Windows} or \feat{Linux}.''
		\end{example}
		\begin{example}{Configuration Map}
			\tiny
			\begin{fancycolumns}[animation=none]
				$\{C,G,W\}$\\
				\hspace{4mm}\vdots\\[1ex]
				$\{C,G,P,D,T,W\}$
			\nextcolumn
				$\{C,G,L\}$\\
				\hspace{4mm}\vdots\\[1ex]
				$\{C,G,P,D,T,L\}$
			\end{fancycolumns}
		\end{example}
		\begin{exampletight}{Feature Diagram (Graphical Feature Model)}
			\centering\tiny
			\featureDiagramConfigurableDatabase
		\end{exampletight}
	\nextcolumn
		\centering
		\sffamily
		\begin{tikzpicture}
			\tikzstyle{every edge}=[font=\tiny,draw,color=blue]
			\node (fd) at (2,0) [align=center] {Feature Model\\[-1ex]{\uncover<3->{\tiny\color{lecturered}Feature Diagram, \textbf{P1}}}};
			\node (nat) at (0,-2) [align=center] {Natural Language\\[-1ex]{\tiny\color{lecturered}Thoughts, Plain Text}};
			\node (cfg) at (4,-2) [align=center] {Configuration Map\\[-1ex]{\tiny\color{lecturered}Excel, Set of Sets}};

			\path [->] (fd) edge[bend left] node[sloped,yshift=1mm] {toString} (nat);
			\uncover<5->{\path [dotted, ->] (nat) edge[bend left] node[sloped,yshift=1mm] {\textbf{P3}} (fd);}
			
			\uncover<4->{\path [->] (fd) edge[bend left] node[sloped,yshift=1mm] {\textbf{P2}} (cfg);}
			\uncover<5->{\path [dotted, ->] (cfg) edge[bend left] node[sloped,yshift=1mm] {\textbf{P3}} (fd);}

			\node (trans) at (-1,-2.8) {};
			\node (trans2) at (2,-2.8) {};
			\node (trans3) at (-1,-3.1) {};
			\node (trans4) at (2,-3.1) {};
			\path [->] (trans) edge node[yshift=1mm] {Automated Transformation} (trans2);
			\path [dotted, ->] (trans3) edge[yshift=5mm] node[yshift=1mm] {Semi-Automated Transformation} (trans4);
		
			\node (bottomleft2) at (-0.2,-3.3) {\tiny\color{lecturered}Concrete Format};
		\end{tikzpicture}

		\begin{note}{Problems}
			\begin{enumerate}
				\item<3->[P1] How to express feature models \emph{textually}?
				\item<4->[P2] How to (a) validate configurations and (b) get all valid configurations \emph{automatically}?
				\item<5->[P3] \color{gray}{(How to reverse engineer feature models?)}
			\end{enumerate}
		\end{note}
	\end{fancycolumns}
\end{frame}

\subsection{UVL, the Universal Variability Language}

\begin{frame}[fragile]{\myframetitle\ \mytitlesource{\uvlwebsite}}
	\begin{fancycolumns}
\begin{uvltight}[basicstyle=\normalsize]{}
features
	ConfigDB
		mandatory
			API {abstract}
				or
					Get
					Put
					Delete

		optional
			Transactions
		mandatory
			OS {abstract}
				alternative
					Windows
					Linux

constraints
	Transactions => Put | Delete
\end{uvltight}
	\nextcolumn
		\begin{exampletight}{A Feature Model ``Sideways''}
			\centering
			\pic[width=\linewidth]{varied-model}
			$Transactions \pimplies Put \por Delete$
		\end{exampletight}
		\begin{note}{Universal Variability Language (UVL)}
			\begin{itemize}
				\item textual language for feature modeling
				\item adds advanced modeling constructs (e.g.,~attributes, cardinalities, submodels, \ldots)
			\end{itemize}
		\end{note}
	\end{fancycolumns}
\end{frame}

\begin{frame}{Representations and Transformations}
	\begin{fancycolumns}
		\centering
		\sffamily
		\begin{tikzpicture}
			\tikzstyle{every edge}=[font=\tiny,draw,color=blue]
			\node (fd) at (2,0) [align=center] {Feature Model\\[-1ex]{\tiny\color{lecturered}Feature Diagram, UVL}};
			\node (nat) at (0,-2) [align=center] {Natural Language\\[-1ex]{\tiny\color{lecturered}Thoughts, Plain Text}};
			\node (cfg) at (4,-2) [align=center] {Configuration Map\\[-1ex]{\tiny\color{lecturered}Excel, Set of Sets}};

			\path [->] (fd) edge[bend left] node[sloped,yshift=1mm] {toString} (nat);
			\path [dotted, ->] (nat) edge[bend left] node[sloped,yshift=1mm] {\textbf{P3}} (fd);
			
			\path [->] (fd) edge[bend left] node[sloped,yshift=1mm] {\textbf{P2}} (cfg);
			\path [dotted, ->] (cfg) edge[bend left] node[sloped,yshift=1mm] {\textbf{P3}} (fd);

			\node (trans) at (-1,-2.8) {};
			\node (trans2) at (2,-2.8) {};
			\node (trans3) at (-1,-3.1) {};
			\node (trans4) at (2,-3.1) {};
			\path [->] (trans) edge node[yshift=1mm] {Automated Transformation} (trans2);
			\path [dotted, ->] (trans3) edge[yshift=5mm] node[yshift=1mm] {Semi-Automated Transformation} (trans4);
		
			\node (bottomleft2) at (-0.2,-3.3) {\tiny\color{lecturered}Concrete Format};
		\end{tikzpicture}
	\nextcolumn
		\begin{note}{Problems}
			\begin{enumerate}
				\item[P1] How to express feature models \emph{textually}?
				\item[P2] How to (a) validate configurations and (b) get all valid configurations \emph{automatically}?
				\item[P3] \color{gray}{(How to reverse engineer feature models?)}
			\end{enumerate}
		\end{note}
		\begin{note}{Solutions}
			\begin{enumerate}
				\item[P1] Universal Variability Language $\Rightarrow$ \emph{Syntax}
				\item[P2] \emph{Semantics}?
				\item[P3] \color{gray}{--}
			\end{enumerate}
		\end{note}
	\end{fancycolumns}
\end{frame}
% TODO first show the simpler version of the tikz picture, then animate towards the above, animate together with text
% TODO do we really need this slide? semantics was already defined in Part 4a for feature models

\subsection{Propositional Formulas}

\subsubsection*{Recap}

\begin{frame}{\myframetitle}
	\begin{fancycolumns}
		\begin{definition}{Syntax of Propositional Formulas}
			Inductive definition of \emph{propositional formulas} \deutsch{aussagenlogische Formeln}
			\begin{itemize}
				\item the \emph{Boolean truth values} $\top$, $\bot$
				\item any \emph{Boolean variable} $X$
    			\item any \emph{negation} $\pnot \phi$ of a formula $\phi$
    			\item any \emph{conjunction} $(\phi \pand \psi)$ of formulas $\phi$ and $\psi$
				\item any \emph{disjunction} $(\phi \por \psi)$, \emph{implication} $(\phi \pimplies \psi)$, or \emph{biimplication} $(\phi \pequals \psi)$
			\end{itemize}
		\end{definition}
	\nextcolumn
		\begin{definition}{Informal Semantics of Propositional Formulas}
			\vspace*{-3ex}
			\begin{equation*}
				\begin{rcases}
					\top                \\
					\bot                \\
					\pnot \phi          \\
					\phi \pand \psi     \\
					\phi \por \psi      \\
					\phi \pimplies \psi \\
					\phi \pequals \psi
				\end{rcases} \text{ means } \begin{cases}
					\text{``true'' (or \emph{tautology})} \\
					\text{``false'' (or \emph{contradiction})} \\
					\text{``not $\phi$''} \\
					\text{``$\phi$ and $\psi$''} \\
					\text{``$\phi$ or $\psi$'' (inclusive or!)} \\
					\text{``if $\phi$, then $\psi$'' (and else?)} \\
					\text{``$\phi$ if and only if $\psi$''}
				\end{cases}
			\end{equation*}
		\end{definition}
		\begin{example}{Operator Precedence: $\pnot$, $\pand$, $\por$, $\pimplies$, $\pequals$}
			\vspace*{-3ex} % TODO Benno: why is this hack needed?
			\begin{align*}
				           &~Transactions \pimplies (Put \por Delete) \\
				\equiv     &~Transactions \pimplies Put \por Delete \\
				\not\equiv &~(Transactions \pimplies Put) \por Delete
			\end{align*}
		\end{example}
	\end{fancycolumns}
\end{frame}

\subsubsection*{Example}

\begin{frame}{\myframetitle}
	\begin{fancycolumns}[animation=none]
		\only<13|handout:2>{\addtocounter{framenumber}{+1}}\only<1-|handout:1-2>{
			\begin{exampletight}{A Feature Model $FM$ \ldots}
				\centering\tiny
				\featureDiagramConfigurableDatabase[_phi]
				\featureDiagramOverlay{
					\only<1|handout:0>{
						\featureDeemph{(API_phi),(Get_phi),(Put_phi),(Delete_phi),(Transactions_phi),(OS_phi),(Windows_phi),(Linux_phi)}
						\featureEmph{(ConfigDB_phi)}
					}
					\only<2|handout:0>{
						\featureDeemph{(Get_phi),(Put_phi),(Delete_phi),(Transactions_phi),(OS_phi),(Windows_phi),(Linux_phi)}
						\featureEmph{(ConfigDB_phi)(API_phi)}
					}
					\only<3|handout:0>{
						\featureDeemph{(Get_phi),(Put_phi),(Delete_phi),(API_phi),(OS_phi),(Windows_phi),(Linux_phi)}
						\featureEmph{(ConfigDB_phi)(Transactions_phi)}
					}
					\only<4|handout:0>{
						\featureDeemph{(Get_phi),(Put_phi),(Delete_phi),(API_phi),(Transactions_phi),(Windows_phi),(Linux_phi)}
						\featureEmph{(ConfigDB_phi)(OS_phi)}
					}
					\only<5|handout:0>{
						\featureDeemph{(ConfigDB_phi),(Transactions_phi),(Windows_phi),(Linux_phi),(OS_phi)}
						\featureEmph{(Get_phi)(Put_phi)(Delete_phi)(API_phi)}
					}
					\only<6-7|handout:0>{
						\featureDeemph{(Get_phi),(Put_phi),(Delete_phi),(API_phi),(Transactions_phi),(ConfigDB_phi)}
						\featureEmph{(Windows_phi)(Linux_phi)(OS_phi)}
					}
					\only<8|handout:0>{
						\featureDeemph{(ConfigDB_phi)(Transactions_phi)(Windows_phi)(Linux_phi)(OS_phi)(Get_phi)(Put_phi)(Delete_phi)(API_phi)}
					}
					\only<9-12|handout:1>{
						\featureSelected{(ConfigDB_phi),(API_phi),(Get_phi),(OS_phi),(Windows_phi)}
						\featureDeselected{(Put_phi)(Delete_phi),(Transactions_phi),(Linux_phi)}
					}
					\only<13-|handout:2>{
						\featureSelected{(ConfigDB_phi),(API_phi),(Get_phi)}
						\featureDeselected{(Put_phi)(Delete_phi)(Windows_phi)(Linux_phi),(Transactions_phi)(OS_phi)}
					}
				}
			\end{exampletight}
			\begin{exampletight}{\ldots as a Propositional Formula $\Phi(FM)$}
				\vspace*{-1.5ex}
				\small
				\begin{align*}
					\Phi(FM) = &~ConfigDB \\
					\uncover<2->{\pand &~(API \pequals ConfigDB) \\}
					\uncover<3->{\pand &~(Transactions \pimplies ConfigDB) \\}
					\uncover<4->{\pand &~(OS \pequals ConfigDB) \\}
					\uncover<5->{\pand &~(Get \por Put \por Delete \pequals API) \\}
					\uncover<6->{\pand &~(Windows \por Linux \pequals OS) \\}
					\uncover<7->{\pand &~\pnot (Windows \pand Linux) \\}
					\uncover<8->{\pand &~(Transactions \pimplies Put \por Delete)}
				\end{align*}
			\end{exampletight}
		}
	\nextcolumn
		\only<9-|handout:1-2>{
			\begin{example}{Is This a Valid Configuration?}
				\vspace*{-3ex}
				\only<9-12|handout:1>{
					\begin{align*}
								&~\Phi(FM)({\{C, A, G, O, W\}}) \\
						\equiv	&~\Phi(FM)(\cfg[2-]{C, A, G, O, W}{P, D, T, L}) \\
						\uncover<10->{\equiv &~\fs{C} \pand (\fs{A} \pequals \fs{C}) \pand (\fd{T} \pimplies \fs{C}) \pand (\fs{O} \pequals \fs{C}) \\
								&~\pand (\fs{G} \por \fd{P} \por \fd{D} \pequals \fs{A}) \pand (\fs{W} \por \fd{L} \pequals \fs{O}) \\
								&~\pand \pnot (\fs{W} \pand \fd{L}) \pand (\fd{T} \pimplies \fd{P} \por \fd{D}) \\}
						\uncover<11->{\equiv &~\fs{\top} \pand (\fs{\top} \pequals \fs{\top}) \pand (\fd{\bot} \pimplies \fs{\top}) \pand (\fs{\top} \pequals \fs{\top}) \\
								&~\pand (\fs{\top} \por \fd{\bot} \por \fd{\bot} \pequals \fs{\top}) \pand (\fs{\top} \por \fd{\bot} \pequals \fs{\top}) \\
								&~\pand \pnot (\fs{\top} \pand \fd{\bot}) \pand (\fd{\bot} \pimplies \fd{\bot} \por \fd{\bot}) \\}
						\uncover<12->{\equiv &~\fs{\top} \pand \fs{\top} \pand \fs{\top} \pand \fs{\top} \pand \fs{\top} \pand \fs{\top} \pand \fs{\top} \pand \fs{\top} \\
						\equiv	&~\fs{\top}}
					\end{align*}
					\uncover<12->{\emph{$\leadsto$ configuration is valid}\\
					\phantom{$\leadsto$ }(read-only database on Windows)}
				}
				\only<13-|handout:2>{
					\begin{align*}
								&~\Phi(FM)({\{C, A, G\}}) \\
						\equiv	&~\Phi(FM)(\cfg[2-]{C, A, G}{P, D, T, O, W, L}) \\
						\uncover<14->{\equiv&~\fs{C} \pand (\fs{A} \pequals \fs{C}) \pand (\fd{T} \pimplies \fs{C}) \pand (\fd{O} \pequals \fs{C}) \\
								&~\pand (\fs{G} \por \fd{P} \por \fd{D} \pequals \fs{A}) \pand (\fd{W} \por \fd{L} \pequals \fd{O}) \\
								&~\pand \pnot (\fd{W} \pand \fd{L}) \pand (\fd{T} \pimplies \fd{P} \por \fd{D}) \\}
						\uncover<15->{\equiv&~\fs{\top} \pand (\fs{\top} \pequals \fs{\top}) \pand (\fd{\bot} \pimplies \fs{\top}) \pand (\fd{\bot} \pequals \fs{\top}) \\
								&~\pand (\fs{\top} \por \fd{\bot} \por \fd{\bot} \pequals \fs{\top}) \pand (\fd{\bot} \por \fd{\bot} \pequals \fd{\bot}) \\
								&~\pand \pnot (\fd{\bot} \pand \fd{\bot}) \pand (\fd{\bot} \pimplies \fd{\bot} \por \fd{\bot}) \\}
						\uncover<16->{\equiv&~\fs{\top} \pand \fs{\top} \pand \fs{\top} \pand \fd{\bot} \pand \fs{\top} \pand \fs{\top} \pand \fs{\top} \pand \fs{\top} \\
						\equiv	&~\fd{\bot}} % TODO: only show \pand and phi(fm) = and parentheses at the end
					\end{align*}
					\uncover<16->{\emph{$\leadsto$ configuration is invalid}\\
					\phantom{$\leadsto$ }($\lightning$ no operating system)}
				}
			\end{example}
		}
	\end{fancycolumns}
\end{frame}

\subsubsection*{Algorithm}

\newcommand{\featureDiagramFn}[4]{#1\left(~\parbox{#2}{\centering\scalebox{0.8}{\featureDiagram{#3}}}~\right) &= #4}
\newcommand{\featureDiagramPhantom}[4]{\vphantom{#1\left(~\parbox{#2}{\centering\scalebox{0.8}{\featureDiagram{#3}}}~\right)}}

\begin{frame}{\myframetitle}
	\begin{fancycolumns}[animation=none]
		\begin{definition}{Algorithm: Translate $FM$ Into $\Phi(FM)$}
			\begin{enumerate}
				\item<2-> translate each tree constraint
				\begin{itemize}
					\item<3-> \emph{Root feature}: $R$ is always required
						\item<4-> \emph{Optional feature}: $C$ requires $P$
						\item<5-> \emph{Mandatory feature}:\\
							Optional + $P$ requires $C$
						\item<6-> \emph{Or group}:\\
							Optional + $P$ requires at least one $C_i$
						\item<7-> \emph{Alternative group}:\\
							Optional + $P$ requires exactly one $C_i$
				\end{itemize}
				\item<8-> conjoin translated tree constraints\\
					$\Phi(TC) \gets \bigwedge_{tc \in TC} \Phi(tc)$
				\item<9-> conjoin \emph{cross-tree constraints}\\
					$\Phi(CTC) \gets \bigwedge_{ctc \in CTC} ctc$
				\item<10-> $\Phi(FM) \gets \Phi(TC) \pand \Phi(CTC)$
			\end{enumerate}
		\end{definition}
	\nextcolumn
		\begin{definition}{}
			\vspace*{-3ex}
			\begin{align*}
				\uncover<3->{\featureDiagramFn{\Phi}{6ex}{Root,concrete}{Root}}\\
				\uncover<4->{\featureDiagramFn{\Phi}{6ex}{P,concrete[C,optional,concrete]}{C \pimplies P}}\\
				\uncover<5->{\featureDiagramFn{\Phi}{6ex}{P,concrete[C,mandatory,concrete]}{C \pequals P}}\\
				\uncover<6->{\featureDiagramFn{\Phi}{15ex}{P,concrete[$C_1$,or,concrete][\ldots,concrete][$C_n$,concrete]}{\bigvee_{1 \leq i \leq n} C_i \pequals P}}\\
				\uncover<7->{\featureDiagramFn{\Phi}{15ex}{P,concrete[$C_1$,alternative,concrete][\ldots,concrete][$C_n$,concrete]}{\bigvee_{1 \leq i \leq n} C_i \pequals P}}\\
				\uncover<7->{& \pand \bigwedge_{1 \leq i < j \leq n} \pnot (C_i \pand C_j)}
			\end{align*}
		\end{definition}
	\end{fancycolumns}
\end{frame}

\subsection{CNF as a Universal Formula Language}

\begin{frame}{\myframetitle} % TODO unmotivated topic switch after prior slide?
	\begin{fancycolumns}[animation=none]
		\begin{definition}{Recap: Conjunctive Normal Form}
			\begin{itemize}
				\item a \emph{literal} $L$ is a variable $X$ or its negation $\pnot X$
				\item a \emph{clause} $\clause{C}$ is a disjunction of literals $\clause{\bigvee_{j} L_j}$
				\item a \emph{conjunctive normal form (CNF)} is a conjunction of clauses $\bigwedge_{i} \clause{C_i} = \bigwedge_{i} \clause{\bigvee_{j} L_j}$
				\item intuitively: a set of ``rules'' to be satisfied
				\item any formula $\phi$ can be transformed into a CNF $\phi'$ that is logically equivalent ($\phi \mequals \phi'$)
			\end{itemize}
		\end{definition}
		\uncover<2->{
			\begin{definition}{Recap: Laws of Propositional Logic}
				\begin{itemize}
					\item<3-> implication: $\phi \pimplies \psi \mequals \pnot \phi \por \psi$
					\item<3-> biimplication: $\phi \pequals \psi \mequals (\pnot \phi \por \psi) \pand (\pnot \psi \por \phi)$
					\item<4-> De Morgan's laws: $\pnot (\phi \pand \psi) \mequals \pnot \phi \por \pnot \psi$
					\item<5-> distributivity:\,$(\phi \pand \psi) \por \chi \mequals (\phi \por \chi) \pand (\psi \por \chi)$
				\end{itemize}
			\end{definition}
		}
	\nextcolumn
		\uncover<2->{
			\begin{example}{Transforming Part of $\Phi(FM)$ into $CNF(\Phi(FM))$}
				\vspace*{-1.5ex}
				\small
				% TODO: use color-blind friendly colors
				\begin{fancycolumns}[animation=none]
					\begin{align*}
						\uncover<2->{&~\clause{C} \\
						\pand &~(T \pimplies C) \\
						\pand &~(O \pequals C) \\
						\pand &~(W \por L \pequals O) \\
						\pand &~\pnot (W \pand L)}
					\end{align*}
				\nextcolumn
					\begin{align*}
						\uncover<3->{&~\clause{C} \\
						\pand &~\clause{(\pnot T \por C)} \\
						\pand &~\clause{(\pnot O \por C)} \pand \clause{(\pnot C \por O)} \\
						\pand &~(\pnot (W \por L) \por O) \\
						\pand &~\clause{(\pnot O \por W \por L)} \\
						\pand &~\pnot (W \pand L)}
					\end{align*}
				\end{fancycolumns}
				\begin{fancycolumns}[animation=none]
					\begin{align*}
						\uncover<4->{&~\clause{C} \\
						\pand &~\clause{(\pnot T \por C)} \\
						\pand &~\clause{(\pnot O \por C)} \pand \clause{(\pnot C \por O)} \\
						\pand &~((\pnot W \pand \pnot L) \por O) \\
						\pand &~\clause{(\pnot O \por W \por L)} \\
						\pand &~\clause{(\pnot W \por \pnot L)}}
					\end{align*}
				\nextcolumn
					\begin{align*}
						\uncover<5->{&~\clause{C} \\
						\pand &~\clause{(\pnot T \por C)} \\
						\pand &~\clause{(\pnot O \por C)} \pand \clause{(\pnot C \por O)} \\
						\pand &~\clause{(\pnot W \por O)} \pand \clause{(\pnot L \por O)} \\
						\pand &~\clause{(\pnot O \por W \por L)} \\
						\pand &~\clause{(\pnot W \por \pnot L)}}
					\end{align*}
				\end{fancycolumns}
			\end{example}
		}
	\end{fancycolumns}
\end{frame} % TODO would be more intuitive to me if clauses are green (as they are good to go)

\subsubsection*{DIMACS}

\begin{frame}[fragile]{\myframetitle}
 	\begin{fancycolumns}[columns=3,widths={25,25,50}]
		\vspace*{14ex}
		\begin{align*}
			&~C \\
			\pand &~(\pnot T \por C) \\
			\pand &~(\pnot O \por C) \pand (\pnot C \por O) \\
			\pand &~(\pnot W \por O) \pand (\pnot L \por O) \\
			\pand &~(\pnot O \por W \por L) \\
			\pand &~(\pnot W \por \pnot L)
		\end{align*}
 	\nextcolumn
\begin{dimacstight}[basicstyle=\large]{}
c 1 C
c 2 T
c 3 O
c 4 W
c 5 L
p cnf 5 8
1 0
-2 1 0
-3 1 0 -1 3 0
-4 3 0 -5 3 0
-3 4 5 0
-4 5 0
\end{dimacstight}
	\nextcolumn
		\begin{definition}{DIMACS Format\mysource{\dimacsformat}}
			\begin{itemize}
				\item de facto industry standard for storing CNF
				\item machine-readable, automated analyses, \ldots
				\item comments start with \texttt{c \ldots}
				\item problem line:\\
					\texttt{p cnf \#variables \#clauses}
				\item clause $\bigvee_{i} L_i$ translates to \texttt{L1 \ldots\ Ln 0}
				\item intuitively:
					\begin{equation*}
						\begin{rcases}
							\texttt{0}\\
							\texttt{\textvisiblespace}\\
							\texttt{-}
						\end{rcases} \text{ means } \begin{cases}
							\pand \\
							\por \\
							\pnot
						\end{cases}
					\end{equation*}
			\end{itemize}
		\end{definition}
 	\end{fancycolumns}
\end{frame}

\subsection*{Representations and Transformations}
\begin{frame}[label=FeatureModelTransformations]{\myframetitle}
	\begin{fancycolumns}[widths={52,48}]
		\centering
		\sffamily
		\begin{tikzpicture}
			\tikzstyle{every edge}=[font=\tiny,draw,color=blue]

			\node (topleft) at (-1.4,0) {};
			\node (bottomleft) at (-1.4,-4) {};
			
			\node (fd) at (2,0) [align=center] {Feature Model\\[-1ex]{\tiny\color{lecturered}Feature Diagram, UVL}};
			\node (nat) at (0,-2) [align=center] {Natural Language\\[-1ex]{\tiny\color{lecturered}Thoughts, Plain Text}};
			\node (phi) at (2,-2) [align=center] {Formula\\[-1ex]{\tiny\color{lecturered}Infix Notation}};
			\node (cfg) at (4,-4) [align=center] {Configuration Map\\[-1ex]{\tiny\color{lecturered}Excel, Set of Sets}};
			\node (cnf) at (2,-4) [align=center] {CNF\\[-1ex]{\tiny\color{lecturered}DIMACS}};

			\path [dashed, thick, ->] (topleft) edge node[left, rotate=90, yshift=2mm, xshift=10mm] {Loss of Structure} (bottomleft);
		
			\path [->] (fd.south west) edge[bend left=15] node[sloped,yshift=1mm] {toString} (nat);
			\path [dotted, ->] (nat) edge[bend left=15] node[sloped,yshift=1mm] {(i)} (fd.south west);
			
			\path [->] (fd.south) edge[bend left=15] node[sloped,yshift=1mm,rotate=90] {$\Phi$} (phi);
			\path [dotted, ->] (phi) edge[bend left=15] node[sloped,yshift=2mm,rotate=270] {(ii)} (fd.south);

			\path [->] (phi) edge[bend left=15] node[sloped,yshift=1mm] {CNF} (cnf);
			\path [dotted, ->] (cnf) edge[bend left=15] node[sloped,yshift=2mm,rotate=270] {(ii)} (phi);
		
			\path [->] (cfg) edge[bend right=15] node[sloped,yshift=1mm] {CNF} (cnf);
			\path [->] (cnf.south) edge[bend right=15] node[sloped,yshift=1mm] {\textbf{P2(b)}} (cfg);

			\node (trans) at (-1,-4.6) {};
			\node (trans2) at (2,-4.6) {};
			\node (trans3) at (-1,-4.9) {};
			\node (trans4) at (2,-4.9) {};
			\path [->] (trans) edge node[yshift=1mm] {Automated Transformation} (trans2);
			\path [dotted, ->] (trans3) edge[yshift=5mm] node[yshift=1mm] {Semi-Automated Transformation} (trans4);
		
			\node (bottomleft2) at (-0.2,-5) {\tiny\color{lecturered}Concrete Format};
		\end{tikzpicture} % TODO would prefer another color for concrete formats, as they have nothing to do with the red boxes on thoses slides. what about green? needs to be changed consistently in all slides of this lecture
	\nextcolumn
		\begin{note}{Problems}
			\begin{enumerate}
				\item[P1] How to express feature models \emph{textually}?
				\item[P2] How to 
				\begin{itemize}
					\item[(a)] validate configurations and
					\item[(b)] get all valid configurations
				\end{itemize}
				\emph{automatically}?
				\item[P3] \color{gray}{(How to reverse engineer feature models?)}
			\end{enumerate}
		\end{note}
		\begin{note}{Solutions}
			\begin{enumerate}
				\item[P1] Universal Variability Language $\Rightarrow$ \emph{Syntax}
				\item[P2] Propositional Formulas $\Rightarrow$ \emph{Semantics}
				\begin{itemize}
					\item[(a)] evaluate feature-model formula
					\item[(b)] \lecturemodeling\partc{}
				\end{itemize}
				\item[P3] \color{gray}{(i) e.g., \bakarnaturallanguage\\(ii) e.g., \czarneckithereandbackagain}
			\end{enumerate}
		\end{note}
	\end{fancycolumns}
\end{frame}

% removed the following slide. could be confusing to use Excel in two different ways.
% \begin{frame}{\myframetitle}
% 	\centering
% 	\picDark[width=.8\linewidth,trim=10 10 0 10,clip]{constraints-in-excel}

% 	\textbf{``It's installed anyway.''}
% \end{frame}
