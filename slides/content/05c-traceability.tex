\subsection{The Feature Traceability Problem}
\begin{frame}{\myframetitle}
	\begin{mycolumns}
		\todots
	\mynextcolumn
		\todots
	\end{mycolumns}
\end{frame}

\subsection{Feature Location}
\begin{frame}{\myframetitle}
	\begin{mycolumns}
		\todots
	\mynextcolumn
		\todots
	\end{mycolumns}
\end{frame}

\subsection{Feature Traceability with Colors}

\subsubsection*{FeatureCommander}
\begin{frame}{\myframetitle}
	\centering\pic[height=\textheightwithtitle]{feature-commander1}
\end{frame}
\begin{frame}{\myframetitle}
	\centering\pic[height=\textheightwithtitle]{feature-commander2}
\end{frame}
% TODO keywords on feature commander + references

\subsubsection*{FeatureIDE}
\begin{frame}{\myframetitle}
	\begin{mycolumns}
		\myexampletight{Tool Support for Feature Traceability}{\pic[width=\linewidth]{feature-traceability}}
	\mynextcolumn
		\todots
	\end{mycolumns}
\end{frame}

\subsection{Recap: Scattering, Tangling, Replication}
\begin{frame}{\myframetitle}
	\begin{mycolumns}
		\todots
	\mynextcolumn
		\todots
	\end{mycolumns}
\end{frame}

\subsection{Virtual Separation of Concerns}
\begin{frame}{\myframetitle}
	\begin{mycolumns}
		\todots
	\mynextcolumn
		\todots
	\end{mycolumns}
\end{frame}

\subsubsection*{CIDE}
\begin{frame}{\myframetitle}
	\begin{mycolumns}[widths={70},animation=none]
		\pic[width=\linewidth,trim=0 10 0 0,clip]{cide-open-editor}
	\mynextcolumn
		\begin{example}{What is CIDE?}
			\begin{itemize}
				\item stands for Colored IDE
				\item based on Eclipse and FeatureIDE
				\item special editors available for several languages:
			\end{itemize}
			\begin{flushleft}
				ANTLR, C (experimental), C++ (experimental), C\#, ECMAScript (JavaScript), Featherweight Java, \textbf{Java 1.5}, gCIDE, Haskell, JavaCC, and Python
			\end{flushleft}
		\end{example}
	\end{mycolumns}
\end{frame}

\begin{frame}{\myframetitle}
	\begin{mycolumns}[widths={70},animation=none]
		\pic[width=\linewidth]{cide-editor-and-outline}
	\mynextcolumn
		\begin{example}{Why colors?}
			\begin{itemize}
				\item colors replace preprocessor directives
				\item annotation based on syntax of the language
				\item easy annotation with selection and context menu
				\item no need to handle separators and logical connectors:\\\mycite{\texttt{,}}, \mycite{\texttt{||}}
			\end{itemize}
		\end{example}
	\end{mycolumns}
\end{frame}

\begin{frame}{\myframetitle}
	\begin{mycolumns}[widths={70},animation=none]
		\pic[width=\linewidth]{cide}
	\mynextcolumn
		\begin{example}{Why virtual separation?}
			\begin{itemize}
				\item source code is a view on the abstract syntax tree (AST)
				\item possible to hide irrelevant features
				\item possible to show overlapping features
				\item supporting development despite scattering and tangling
			\end{itemize}
		\end{example}
	\end{mycolumns}
\end{frame}

\begin{frame}{\myframetitle}
	\pic[width=\linewidth]{cide-show-single-feature}
	\begin{example}{}\centering
		possible to only show a single feature -- in its surrounded code
	\end{example}
\end{frame}

\begin{frame}{\myframetitle}
	\begin{mycolumns}[widths={45},animation=none]
		\begin{example}{Why configuration?}
			\begin{itemize}
				\item features specified in FeatureIDE feature model
				\item configuration created in FeatureIDE configuration editor
				\item configuration used to generate and visualize variant
				\item \ldots
			\end{itemize}
		\end{example}
	\mynextcolumn
		\pic[width=\linewidth]{cide-feature-selection}
	\end{mycolumns}
\end{frame}

\begin{frame}{\myframetitle}
	\begin{mycolumns}[widths={45},animation=none]
		\begin{example}{Why configuration?}
			\begin{itemize}
				\item \ldots
				\item variant visualized in source code and project explorer
				\item only necessary to press CIDE button in project explorer
				\item pressing it again returns to the view of the product line
			\end{itemize}
		\end{example}
	\mynextcolumn
		\pic[width=\linewidth]{cide-variant-view-in-project-explorer}
	\end{mycolumns}
\end{frame}
% CIDE literature
% forward reference to physical separation of concerns in next two lectures
