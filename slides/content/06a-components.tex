\subsection[Recap]{Recap: How to Implement Features?}
\againframe{HowToImplementFeatures}

\begin{frame}{Recap: Features with Build Systems}
	\begin{mycolumns}[widths={33},animation=none]
		\centering\pic[width=\linewidth]{pignap-features}
	\mynextcolumn
		\begin{definition}{Conditional Compilation with Build Systems}
			\begin{itemize}
				\item exploit the expressiveness of a build system's configuration language
				\item include and exclude individual files or entire directories based on feature selection
			\end{itemize}
		\end{definition}
		\uncover<2->{
			\begin{note}{Major Challenges\mysource{\lecturefeatures}}
				\begin{itemize}
					\MajorChallengesOfBuildSystems
				\end{itemize}
			\end{note}
		}
	\end{mycolumns}
\end{frame}

\begin{frame}{Recap: Features with Preprocessors}
	\begin{mycolumns}[widths={33},animation=none]
		\pic[width=\linewidth,page=1,trim=15 20 185 40,clip]{preprocessor-wilderness}
	\mynextcolumn
		\begin{definition}{Conditional Compilation with Preprocessors}
			\begin{itemize}
				\item use conditional compilation facilities provided by preprocessors 
				\item annotate and potentially remove code fragments based on feature selection
			\end{itemize}
		\end{definition}
		\uncover<2->{
			\begin{note}{Major Challenges\mysource{\lecturefeatures}}
				\begin{itemize}
					\MajorChallengesOfPreprocessors
				\end{itemize}
			\end{note}
		}
	\end{mycolumns}
\end{frame}

\subsection{Modularity}
\begin{frame}{Modularity} % TODO references missing
	\begin{mycolumns}[widths={50,50},animation=none]
		\begin{definition}{Modularization}
			consistent application of \emph{information hiding} and \emph{data encapsulation} to achieve:
			\begin{itemize}
				\item strong logical connection between the inner parts of a module (high cohesion)
				\item precisely defined, minimal interfaces (low coupling)
			\end{itemize}						
		\end{definition}
		\pause
		\begin{definition}{Coupling and Cohesion}
			\begin{itemize}
				\item \emph{cohesion}: measure of how well the parts of a module work together\\ $Rightarrow$ intra-module communication
				\item \emph{coupling}: measure of the complexity of communication across modules\\ $Rightarrow$ inter-module~communication % removed term interface as the term is too narrow
			\end{itemize}
		\end{definition}
	\mynextcolumn
		\vspace{-1cm}
		\begin{exampletight}{High Coupling, Low Cohesion}
			\centering
			\picDark[width=0.5\textwidth]{cohesion_coupling_1} % TODO picture source?
		\end{exampletight}
		\begin{exampletight}{Low Coupling, High Cohesion}
			\centering
			\picDark[width=0.5\textwidth]{cohesion_coupling_2} % TODO picture source?
		\end{exampletight}
	\end{mycolumns}
	\begin{mycolumns}[widths={50,50},animation=none]	
	\end{mycolumns}
\end{frame}

\begin{frame}{Why Modularity?} % TODO references missing
	\begin{mycolumns}[widths={45,55},animation=none]
		\begin{note}{Traditional Reasons}
			\begin{itemize}
				\item modules can be developed independently of each other (collaborative work)
				\item easier to maintain because changes can be made locally
				\item data encapsulation promotes stability and reliability
				\item software is easier to understand
				\item hiding complexity behind interfaces
				\item decomposition = divide and conquer
			\end{itemize}						
		\end{note}
		\pause
	\mynextcolumn
		\begin{note}{Modularization and Software Product Lines}
			\begin{itemize}
				\item \emph{reuse}: parts of the software can be {\em reused}
				\item \emph{alternatives}: modules can be {\em exchanged by alternative implementations}
				\item \emph{variability}: modules can be {\em reassembled in a new context} (e.g., in other projects)
			\end{itemize}
		\end{note}
		\begin{exampletight}{}
			\pic[width=\linewidth,page=18,trim=5 115 5 5,clip]{lego} 
		\end{exampletight}
	\end{mycolumns}	
\end{frame}

\subsection{Software Components}
\begin{frame}{Components}
	\begin{mycolumns}[b,animation=none]
		\begin{definition}{Component \mysource{\szyperski}}		
			A software component is a unit of composition with contractually specified \emph{interfaces} and explicit \emph{context dependencies} only. 
			A software component can be deployed independently and is \emph{subject to composition} by third parties. % TODO real quote? then use \mycite{}
		\end{definition}
		\begin{exampletight}{}
			\centering
			\pic[width=.7\linewidth]{component_uml}
		\end{exampletight}
		\begin{note}{Context/Deployment Dependencies} % TODO references missing
			typically container or middleware (e.g., JavaEE, CORBA, OSGi, etc.)
		\end{note}
	\mynextcolumn
		\vspace{-12mm}
		\begin{note}{Composition and Reuse} % TODO references missing
			Components \ldots
			\begin{itemize}
				\item are composed with other components to form software systems
				\item are supposed to be re-usable in other software systems
				\item may stem from third-party vendors: markets for components, make-or-buy-decisions
			\end{itemize}			
		\end{note}
		\pause
		\begin{exampletight}{UML Component Diagrams}
			\centering
			\pic[width=\linewidth]{component_diagram_uml}
		\end{exampletight}
		\pause
	\end{mycolumns}	
\end{frame}

\begin{frame}{Components vs. Objects/Classes} % TODO references missing
	\begin{mycolumns}[widths={33},animation=none]
		\begin{note}{Commonalities}	
			Numerous similar principles:
			\begin{itemize}
				\item encapsulation and information hiding
				\item accessibility through public interfaces
				\item (de-)composition and nested objects/components
				\item etc.
			\end{itemize}
		\end{note}
		\pause
	\mynextcolumn
		\begin{note}{Differences}
			\begin{itemize}
				\item \emph{objects are smaller} than components by focusing on detailed implementation problems (components aim at abstracting from implementation details)
				\item \emph{object are less cohesive and stronger coupled} than components due to (deliberately) delegating lots of responsibilities to other objects whereas components aim at maximizing cohesion and minimizing coupling
				\item \emph{classes are reused through inheritance and polymorphism} whereas components are reused by being integrated into a component architecture
			\end{itemize}			
		\end{note}
	\end{mycolumns}	
\end{frame}

\subsection[Example Component]{Example: A Color Component}
\begin{frame}[fragile]{\myframetitle\ \mytitlesource{\fospl}}
	\begin{mycolumns}[widths={40}]
			\begin{example}{A Reusable Component in Java}
				\begin{itemize}
					\item assume that storing and printing colors is non-trivial (e.g., in our graph library)
					\item implement color management as a reusable component, using Java's visibility mechanism to enforce encapsulation
				\end{itemize}
			\end{example}
		\mynextcolumn
\begin{codetight}{}
package components.color;

// public API
public class ColorComponent {
	public Color createColor(int r, int g, int b) { /* ... */ }
	public void printColor(Color color) { /* ... */ }
	public void mapColor(Object o, Color c) { /* ... */ }
	public Color getColor(Object o) { /* ... */ }
	
	// just one component instance
	public static ColorComponent getInstance() { return instance; }
	private static ColorComponent instance = new ColorComponent();
	private ColorComponent() { super(); }
}
public interface Color { /* ... */ }

// hidden implementation
class ColorImpl implements Color { /* ... */ }
class ColorPrinter { /* ... */ }
class ColorMapping { /* ... */ }
\end{codetight}
	\end{mycolumns}
\end{frame}

\begin{frame}[fragile]{\myframetitle}
	\begin{mycolumns}[b,widths={60},animation=none]
\begin{codetight}{}
public class Graph {
	private List<Node> nodes = new ArrayList<Node>();
	public Node add() {
		Node n = null;
		@if (Config.COLORED) {
			Color c = ColorComponent.getInstance().createColor(0, 0, 0);
			n = new Node(nodes.size(), c);
		} else @ {
			n = new Node(nodes.size());
		}
		nodes.add(n);
		return n;
	}
	// ...
}
\end{codetight}
		\mynextcolumn
		\vspace{-10mm}
\begin{codetight}{}
public class Node {
	private int id;
	@private ColorComponent colorComp =
		ColorComponent.getInstance();@
	public Node(int id) { this.id = id; }
	public Node(int id, @Color c@) {
		this(id);
		@colorComp.mapColor(this, c);@
	}
	public void print() {
		@if (Config.COLORED) {
			Color c = colorComp.getColor(this);
			colorComp.printColor(c);
		}@
		System.out.print(id);
	}
}
\end{codetight}
	\end{mycolumns}
	\begin{note}{}
		\begin{itemize}
			\item we can \emph{reuse} the color component when implementing the color feature for the graph library and also for other applications
			\item however: we need to write custom code to connect our implementation with the component\\
				$\Rightarrow$ \emph{glue code}
		\end{itemize}
	\end{note}
\end{frame}

\subsection{Component-Based Product Lines}
\begin{frame}{\myframetitle} % TODO check license of pictures
	\begin{mycolumns}[t,widths={40}]
		\begin{definition}{General Idea}					
			\begin{itemize}
				\item every feature is implemented by a dedicated component
				\item feature selection determines which components shall be integrated to form an application
			\end{itemize}
		\end{definition}
		\begin{example}{Vision}
			\pic[height=1.75cm]{lego_components}
				\vspace*{\fill}
					$\Longrightarrow$ 
				\vspace*{\fill}	
			\pic[height=1.75cm]{lego_product}
		\end{example}
	\mynextcolumn
		\begin{example}{Reality}
			\pic[height=1.75cm]{lego_components}
				\vspace*{\fill}
					$+$ 
				\vspace*{\fill}	
			\pic[height=1.75cm]{lego_glue}
				\vspace*{\fill}
					$=$ 
				\vspace*{\fill}	
			\pic[height=1.75cm]{lego_product}
		\end{example}		
		\begin{note}{Glue Code and Customization}
			\begin{itemize}
				\item developers must connect components through glue code\\exception: components are only exchanged against alternative components with identical interface
				\item components may contain run-time variability\\e.g., color manager in our example may be parameterized by color model RGB or CMYK
			\end{itemize}
		\end{note}
	\end{mycolumns}	
\end{frame}

\subsection[Discussion]{Discussion of Components}
\begin{frame}{The Library Scaling Problem}
	\begin{mycolumns}[animation=none]
		\begin{exampletight}{Decomposition and Reuse}
			\picDark[width=\linewidth]{tangram} % TODO source? license?
		\end{exampletight}
	\mynextcolumn
		\begin{definition}{What is the optimal size of a component?}
			\begin{itemize}
				\item large components (vertical scaling):\\typically not widely reusable
				\item small components (horizontal scaling):\\limited payoff for component integrators
			\end{itemize}
		\end{definition}
		\pause
		\begin{note}{Practical Compromise}
			\begin{itemize}
				\item mostly vertically-scaled components within a few important, narrow domains (e.g., user interface construction systems)
				\item in other words: there is no general market for arbitrary components, but a few specialized market segments % TODO unclear (Thomas)
			\end{itemize}
		\end{note}
	\end{mycolumns}
\end{frame}

\newcommand{\MajorChallengesOfComponents}{
	\item requires \emph{glue code} for every product: clone-and-own for glue code? glue code with runtime variability?
	\item no automated generation based on feature selection
	\item requires preplanning to identify good level of granularity (cf.\ library scaling problem)
	\item no support for fine-granular variability\\ $\Rightarrow$ often combined with runtime variability within components
}
\begin{frame}{\myframetitle}
	\begin{mycolumns}[animation=none]
		\begin{note}{Advantages}
			\begin{itemize}
				\item supports \emph{compile-time variability}
				\item modular implementation with reduced scattering and tangling (compared to runtime variability and preprocessors)
			\end{itemize}
		\end{note}
	\mynextcolumn
		\pause
		\begin{note}{Challenges}
			\begin{itemize}
				\MajorChallengesOfComponents
			\end{itemize}
		\end{note}
	\end{mycolumns}
\end{frame}
