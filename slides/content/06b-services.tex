\subsection{Services vs. Components}
\subsection{Microservices vs. Services}

\subsection{(German Slides on Microservices)}
\begin{frame}{Microservices und Microservice Architekturen}
	\mycite{A \emph{microservice} is a cohesive, independent process interacting via messages. A \emph{microservice architecture} is a distributed application where all its modules are microservices.}{Dragoni et al., 2017}
	
	\pause
	\mycite{[A \emph{microservice} is a module] implemented and operated as a small yet independent system, offering access to its internal logic and data through a well-defined network interface.}{Jamshidi et al., 2018}
	
	\begin{center}
		\vspace{-12mm}
		\pic[width=.45\linewidth]{peer-to-peer}
	\end{center}
	
	%\zitat{The microservices architecture gained popularity relatively recently and can be considered to be in its infancy since there is still a lack of consensus on what microservices actually are.}{Dragoni et al., 2017}
	
	%\zitat{\emph{Microservices} are small, autonomous services that work together.}{Sam Newman, 2015}
\end{frame}

\begin{frame}{Wie klein sind Microservices?}
	\mycite{[A microservice] could be re-written in \emph{two weeks}.}{\href{https://www.rea-group.com/blog/micro-services-what-even-are-they/}{Jon Eaves, 2014}}
	
	\pause
	\mycite{If the codebase is too big to be \emph{managed by a small team}, looking to break it down is very sensible. [...] The smaller the service, the more you \emph{maximize the benefits and downsides} of microservice architecture.}{Sam Newman, 2015}% [...] The golden rule: can you make a change to a service and deploy it by itself without changing anything else?
	
	\pause
	\begin{columns}
		\column{.4\linewidth}
		\hfill\pic[width=.8\linewidth]{pizza}
		\column{.55\linewidth}
		\href{https://www.theguardian.com/technology/2018/apr/24/the-two-pizza-rule-and-the-secret-of-amazons-success}{\emph{Two-Pizza Rule} by Jeff Bezos}:\\Every team should be small enough that it can be fed with two pizzas.
	\end{columns}
\end{frame}

\begin{frame}{Beispiel: Microservices bei Netflix}
	\large\bigskip\bigskip\bigskip\bigskip\bigskip
	\begin{center}
		\begin{tabular}{rl}
			100s & of microservices\\
			1,000s & of daily production changes\\
			10,000s & of instances\\
			100,000s & of customer interactions per minute\\
			1,000,000s & of customers\\
			%1,000,000,000s & of metrics\\
			10,000,000,000\phantom{s} & hours streamed\\
			10s & of operations engineers
		\end{tabular}
	\end{center}
	
	\bigskip\bigskip\bigskip\bigskip\hfill\tiny Quelle: \href{https://fr.slideshare.net/AmazonWebServices/dvo203-the-life-of-a-netflix-engineer-using-37-of-the-internet}{https://fr.slideshare.net/AmazonWebServices/dvo203-the-life-of-a-netflix-engineer-using-37-of-the-internet}
\end{frame}

\begin{frame}{Prinzipien von Microservices}
	\emph{Gesetz von Conway}: \mycite{Organizations which design systems [...] are constrained to produce designs which are copies of the communication structures of these organizations.}{\href{https://www.melconway.com/Home/pdf/committees.pdf}{Melvin Edward Conway, 1968}}
	
	\pause
	\emph{Single Responsibility Principle}: \mycite{Gather together the things that change for the same reasons. Separate those things that change for different reasons.}{\href{https://blog.cleancoder.com/uncle-bob/2014/05/08/SingleReponsibilityPrinciple.html}{Robert C. Martin, 2014}}
	
	\pause
	\mycite{\emph{You build it, you run it.}}{Amazon CTO Werner Vogel, 2006}
	\begin{itemize}
		\item Team übernimmt Entwicklung, Betrieb und Wartung (vgl. DevOps)
		\item Besseres Verständnis für Potential und Leistungsfähigkeit
	\end{itemize}
\end{frame}

\begin{frame}{Vorteile von Microservices}
	\begin{itemize}
		\item Heterogenität: einfache Anwendung von neuen Technologien/Programmiersprachen
		\item Continuous Integration/Deployment: bei Änderungen müssen nur einzelne Microservices ausgeliefert oder bei Fehlfunktionen zurückgesetzt werden (Rollback)
		\item Skalierbarkeit: Anzahl von Instanzen wählbar nach Rechenaufwand pro Microservice
		\pause
		\item Effizienz und Performance: Konfiguration von Ausführungsumgebung kann pro Microservice optimiert werden
		\item Wiederverwendung: Microservices können in beliebig vielen anderen Microservices verwendet werden
		\item Modernisierung: Ersetzung oder Entfernung einfacher möglich
	\end{itemize}
\end{frame}

\begin{frame}{Nachteile von Microservices}
	\begin{itemize}
		\item Latenz im Netzwerk größer als die von gemeinsam genutztem Arbeitsspeicher
		\item Netzwerksecurity erfordert Mehraufwand für Verschlüsselung und Authentifizierung
		\item Integrationstest kann sehr aufwändig werden
		\pause
		\item Verschieben von Code kann neue Programmiersprache erfordern
		\item Änderungen an Teams schwierig durch Heterogenität
		\item Ggfs. zu viele Microservices
		\item Datenkonsistenz schwierig herzustellen wenn mehrere Microservices verändert werden
	\end{itemize}
\end{frame}

\begin{frame}{Der Hype um Microservices}
	\pic[width=\linewidth]{microservices}
\end{frame}

\begin{frame}{Der Hype um Microservices}
	\pic[width=\linewidth]{microservices-comparison}
\end{frame}

\begin{frame}{Vergleich mit Serviceorientierten Architekturen}
	\begin{columns}[t]
		\column{.35\linewidth}
		Gemeinsamkeiten
		\begin{itemize}
			\item Services werden im Betriebsystem als eigenständige Prozess angesehen
			\item Kommunikation über Netzwerke
		\end{itemize}
		\column{.55\linewidth}
		Unterschiede
		\begin{itemize}
			\item Microservices sind ein spezifischer Ansatz für SOA
			\item Microservices sind meist kleiner als Services (feingranulares SOA)
			\item Orchestrierung: SOA vernetzt sich über zentralen Service
			\item Choreographie: Microservices kommunizieren durch Publish-Subscribe-Prinzip
		\end{itemize}
	\end{columns}
\end{frame}

\begin{frame}{Vergleich mit Komponenten/Bibliotheken/Plugins}
	\begin{columns}[t]
		\column{.35\linewidth}
		Gemeinsamkeiten
		\begin{itemize}
			\item Modularisierung
			\item Wiederverwendung
		\end{itemize}
		\column{.55\linewidth}
		Unterschiede
		\begin{itemize}
			\item Microservices unterstützen Heterogenität (z.B. verschiedene Programmiersprachen)
			\item Microservices haben keinen gemeinsamen Arbeitsspeicher
			\item Update von Microservices erfordert keinen Neustart der Gesamtapplikation
		\end{itemize}
	\end{columns}
\end{frame}

