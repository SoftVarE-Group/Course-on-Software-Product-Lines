\subsection{(Micro-)Services}
\begin{frame}{(Micro-)Services}
	\begin{fancycolumns}[animation=none]
		\begin{definition}{(Micro-)Service}
			\mycite{[A (micro-)service is a module] implemented and operated as a small yet independent system, offering access to its internal logic and data through a well-defined network interface.}
				{Jamshidi et al., 2018}
		\end{definition}
		\begin{definition}{(Micro-)Service Architecture}
			\mycite{A [micro]service is a cohesive, independent process interacting via messages. 
				A microservice architecture [service-oriented architecture] is a distributed application where all its modules are microservices.}
				{Dragoni et al., 2017}
		\end{definition}
	\nextcolumn		
		\pic[width=\linewidth]{peer-to-peer}
	\end{fancycolumns}
\end{frame}
% TODO hard to talk on this slide. we should avoid using all those brackets and rather state the differences between services and microservices explicitly.

\subsection{Services vs. Components}
\begin{frame}{Services vs.\ Components}
	\begin{fancycolumns}[animation=none]
		\begin{definition}{Component}
			Intra-process communication (i.e., method calls)
			\begin{center}
				\pic[width=0.65\linewidth]{component_vs_service_1}
			\end{center}			
		\end{definition}
	\nextcolumn		
		\begin{definition}{Service}
			Inter-process communication (e.g., REST API)
			\begin{center}
				\pic[width=0.65\linewidth]{component_vs_service_2}
			\end{center}	
		\end{definition}
	\end{fancycolumns}
	~
	\pause
	\begin{note}{}
		As a consequence, each (micro-)service can be implemented using different technology stacks, whereas components are bound to the same technology (given by container or middleware).
	\end{note}
\end{frame}

\subsection{Microservice Architectures}
\begin{frame}{Microservice Architecture: Motivation}
	\begin{fancycolumns}[animation=none]
		\begin{note}{Recap: The Library Scaling Problem}
			\centering
			\pic[width=.95\linewidth]{tangram}			
		\end{note}
	\nextcolumn		
		\pause
		\begin{note}{How small is a Microservice?}
			\begin{itemize}
				\item Components are very unspecific of how to deal with the general requirement ``not too small but not too big''.
				\item On the contrary, there is a clear philosophy behind microservice architectures, largely driven by organizational constraints wrt.\ agile teams and continuous delivery.
			\end{itemize}
		\end{note}
		\pause
		\begin{definition}{}
			\mycite{If the codebase is too big to be \emph{managed by a small team}, looking to break it down is very sensible. [...] The smaller the service, the more you \emph{maximize the benefits and downsides} of microservice architecture.}{Sam Newman, 2015}% [...] The golden rule: can you make a change to a service and deploy it by itself without changing anything else?
		\end{definition}
	\end{fancycolumns}
\end{frame}

\begin{frame}{Microservice Architecture: Philosophy and Principles}
	\begin{fancycolumns}[animation=none]
		\begin{definition}{Conway's Law}
			\mycite{Organizations which design systems [...] are constrained to produce designs which are copies of the communication structures of these organizations.}{\href{https://www.melconway.com/Home/pdf/committees.pdf}{Melvin Edward Conway, 1968}}			
		\end{definition}
		\begin{definition}{Single Responsibility Principle}
			\mycite{Gather together the things that change for the same reasons. Separate those things that change for different reasons.}{\href{https://blog.cleancoder.com/uncle-bob/2014/05/08/SingleReponsibilityPrinciple.html}{Robert C. Martin, 2014}}			
		\end{definition}
		\begin{definition}{}
			\mycite{You build it, you run it.}{Amazon CTO Werner Vogel, 2006}			
		\end{definition}
	\nextcolumn		
		\pause
		\begin{note}{Consequences}
			\begin{itemize}
				\item Microservices are supposed to be split along business capabilities (e.g., purchase, sale, ...) instead of technical concerns (e.g., UI, persistence, ...)
				\item Each microservice is built (full stack) {\em and} operated by a small agile team that takes over full responsibility (cf. DevOps) 
			\end{itemize}
		\end{note}
		\pause
		\begin{definition}{}
			\mycite{[A microservice] could be re-written in two weeks.}{\href{https://www.rea-group.com/blog/micro-services-what-even-are-they/}{Jon Eaves, 2014}}
		\end{definition}
		\pause
		\begin{definition}{}
			\mycite{Every team should be small enough that it can be fed with two pizzas.}{\href{https://www.theguardian.com/technology/2018/apr/24/the-two-pizza-rule-and-the-secret-of-amazons-success}{Jeff Bezos, 2018}}
		\end{definition}
	\end{fancycolumns}
\end{frame}

\begin{frame}{Traditional Promises of Microservices}
	\begin{fancycolumns}
		\begin{note}{}
			\begin{itemize}
				\item \emph{Scalability}: Microservices are small enough to be developed by a small, agile team. %(cf. typical size of a Scrum team).
				\item \emph{Continuous integration/deployment}: Microservices can be deployed independently of each other. %(i.e., each microservice has its own CI/CD pipeline).
				\item \emph{Heterogeneity}: Each microservice can be implemented using its own technology stack. %(team preferences, adoption of new technologies, etc.).
				\item \emph{Fault tolerance}: The crash of a single microservice should not lead to a crash of the entire system. %(cf. design for failure).
				\item \emph{Efficiency}: Configuration of execution environment can be optimized per microservice. % TODO [ek] this is hard to understand. what is the difference to heterogeneity above?
				\item \emph{Modernization}: A microservice can be easily replaced by an alternative one (even re-implemented from scratch).
			\end{itemize}
		\end{note}
	\nextcolumn
		\begin{exampletight}{The Microservice ``Hype''}
			\pic[width=\linewidth]{microservices}

			\pic[width=\linewidth]{microservices-comparison}
		\end{exampletight}
	\end{fancycolumns}	
\end{frame}

\subsection{Implementation of Product Lines}
\begin{frame}<1-3>[label=SPLwithServices]
	\frametitle<1-3>{Service-Oriented Implementation of Software Product Lines}
	\frametitle<4>{\myframetitle}
	\begin{fancycolumns}[widths={40},animation=none]
		\begin{example}{Recap: Component-Based Implementation}
				\raisebox{-0.4cm}{\pic[width=.24\linewidth,height=1.0cm]{lego_components}} 
				$+$ 
				\raisebox{-0.4cm}{\pic[width=.24\linewidth,height=1.0cm]{lego_glue}}
				$=$ 
				\raisebox{-0.4cm}{\pic[width=.3\linewidth,height=1.0cm]{lego_product}}
		\end{example}	
		\begin{example}{Plenty of Glue Code}
			\centering
			\pic[width=.65\linewidth,height=2.5cm]{lego_glue}				
		\end{example}
	\nextcolumn
		\pause
		\begin{definition}{Same Idea}
			\begin{itemize}
				\item Features are implemented as services.
				\item Feature selection determines the services to be composed.
			\end{itemize}
		\end{definition}
		\pause
		\begin{note}{However}
			``Standardized'' service composition instead of highly individual glue code.
		\end{note}
		\begin{example}{}
			\raisebox{-0.8cm}{\pic[width=.27\linewidth,height=1.75cm]{lego_components}}
			$+$
			\raisebox{-0.8cm}{\pic[width=.27\linewidth,height=1.75cm]{lego_orchestration}}
			$=$
			\raisebox{-0.8cm}{\pic[width=.31\linewidth,height=1.75cm]{lego_product}}
		\end{example}
	\end{fancycolumns}	
\end{frame}

\subsection{Service Composition}
\begin{frame}{\myframetitle}
	\begin{fancycolumns}[t]
		\begin{definition}{Orchestration \deutsch{Orchestrierung}}
			Description of an executable (business-)process as a combination of services (centralized perspective).
		\end{definition}
		\begin{example}{}
			\href{https://docs.oasis-open.org/wsbpel/2.0/wsbpel-v2.0.html}{Web Services Business Process Execution Language (WS-BPEL)}
		\end{example}

		\centering\pic[height=35mm]{trafficlights}
	\nextcolumn
		\begin{definition}{Choreography \deutsch{Choreographie}}
			Each service describes its own task within a service composition (decentralized perspective).
		\end{definition}
		\begin{example}{}
			\href{https://www.w3.org/TR/ws-cdl-10/}{Web Services Choreography Description Language (WS-CDL)}
		\end{example}

		\centering\pic[height=35mm]{crossroad}
	\end{fancycolumns}	
\end{frame}