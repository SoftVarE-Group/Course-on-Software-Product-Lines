\subsection{(Micro-)Services}
\begin{frame}{(Micro-)Services}
	\begin{mycolumns}[widths={50,50},animation=none]
		\mydefinition{(Micro-)Service}{
			\mycite{[A (micro-)service is a module] implemented and operated as a small yet independent system, offering access to its internal logic and data through a well-defined network interface.}
				{Jamshidi et al., 2018}
		}
		\mydefinition{(Micro-)Service Architecture}{
			\mycite{A [micro]service is a cohesive, independent process interacting via messages. 
				A microservice architecture [service-oriented architecture] is a distributed application where all its modules are microservices.}
				{Dragoni et al., 2017}
		}	
	\mynextcolumn		
		\pic[width=\linewidth]{peer-to-peer}
	\end{mycolumns}
\end{frame}

\subsection{Services vs. Components}
\begin{frame}{Services vs.\ Components}
	\begin{mycolumns}[widths={50,50},animation=none]
		\mydefinition{Component}{
			Intra-process communication (i.e., method calls)
			\begin{center}
				\pic[width=0.65\linewidth]{component_vs_service_1}
			\end{center}			
		}
	\mynextcolumn		
		\mydefinition{Service}{
			Inter-process communication (e.g., REST API)
			\begin{center}
				\pic[width=0.65\linewidth]{component_vs_service_2}
			\end{center}	
		}
	\end{mycolumns}
	~
	\pause
	\mynote{}{
		As a consequence, each (micro-)service can be implemented using different technology stacks, whereas components are bound to the same technology (given by container or middleware).
	}
\end{frame}

\subsection{Microservice Architectures}
\begin{frame}{Microservice Architecture: Motivation}
	\begin{mycolumns}[widths={50,50},animation=none]
		\mynote{Remember: The Library Scaling Problem}{
			\centering
			\pic[width=.95\linewidth]{tangram}			
		}
	\mynextcolumn		
		\pause
		\mynote{How small is a Microservice?}{
			\begin{itemize}
				\item Components are very unspecific of how to deal with the general requirement ``not too small but not too big''.
				\item On the contrary, there is a clear philosophy behind microservice architectures, largely driven by organizational constraints wrt.\ agile teams and continuous delivery.
			\end{itemize}
		}
		\pause
		\mydefinition{}{
			\mycite{If the codebase is too big to be \emph{managed by a small team}, looking to break it down is very sensible. [...] The smaller the service, the more you \emph{maximize the benefits and downsides} of microservice architecture.}{Sam Newman, 2015}% [...] The golden rule: can you make a change to a service and deploy it by itself without changing anything else?
		}		
	\end{mycolumns}
\end{frame}

\begin{frame}{Microservice Architecture: Philosophy and Principles}
	\begin{mycolumns}[widths={50,50},animation=none]
		\mydefinition{Conway's Law}{
			\mycite{Organizations which design systems [...] are constrained to produce designs which are copies of the communication structures of these organizations.}{\href{https://www.melconway.com/Home/pdf/committees.pdf}{Melvin Edward Conway, 1968}}			
		}
		\mydefinition{Single Responsibility Principle}{
			\mycite{Gather together the things that change for the same reasons. Separate those things that change for different reasons.}{\href{https://blog.cleancoder.com/uncle-bob/2014/05/08/SingleReponsibilityPrinciple.html}{Robert C. Martin, 2014}}			
		}
		\mydefinition{}{
			\mycite{You build it, you run it.}{Amazon CTO Werner Vogel, 2006}			
		}				
	\mynextcolumn		
		\pause
		\mynote{Consequences}{
			\begin{itemize}
				\item Microservices are supposed to be split along business capabilities (e.g., purchase, sale, ...) instead of technical concerns (e.g., UI, persistence, ...)
				\item Each microservice is built (full stack) {\em and} operated by a small agile team that takes over full responsibility (cf. DevOps) 
			\end{itemize}
		}
		\pause
		\mydefinition{}{
			\mycite{[A microservice] could be re-written in two weeks.}{\href{https://www.rea-group.com/blog/micro-services-what-even-are-they/}{Jon Eaves, 2014}}
		}
		\pause
		\mydefinition{}{
			\mycite{Every team should be small enough that it can be fed with two pizzas.}{\href{https://www.theguardian.com/technology/2018/apr/24/the-two-pizza-rule-and-the-secret-of-amazons-success}{Jeff Bezos, 2018}}
		}
	\end{mycolumns}
\end{frame}

\begin{frame}{Traditional Promises of Microservices}
	\begin{mycolumns}[widths={50,50},animation=none]
		\mynote{}{
			\begin{itemize}
				\item \emph{Scalability}: Microservices are small enough to be developed by a small, agile team. %(cf. typical size of a Scrum team).
				\item \emph{Continuous integration/deployment}: Microservices can be deployed independently of each other. %(i.e., each microservice has its own CI/CD pipeline).
				\item \emph{Heterogeneity}: Each microservice can be implemented using its own technology stack. %(team preferences, adoption of new technologies, etc.).
				\item \emph{Fault tolerance}: The crash of a single microservice should not lead to a crash of the entire system. %(cf. design for failure).
				\item \emph{Efficiency}: Configuration of execution environment can be optimized per microservice.
				\item \emph{Modernization}: A microservice can be easily replaced by an alternative one (even re-implemented from scratch).
			\end{itemize}
		}
	\mynextcolumn
		\pause
		\myexampletight{The Microservice ``Hype''}{
			\pic[width=\linewidth]{microservices}\\
			\vspace{0.3cm}
			\pause
			\pic[width=\linewidth]{microservices-comparison}
		}
	\end{mycolumns}	
\end{frame}

\subsection{Implementation of Product Lines}
\begin{frame}{Service-Oriented Implementation of Software Product Lines}
	\begin{mycolumns}[widths={40,60},animation=none]
		\myexample{Recap: Component-Based Implementation}{
			\pic[width=.24\linewidth,height=1.0cm]{lego_components} 
				\vspace*{\fill}
					$+$ 
				\vspace*{\fill}	
			\pic[width=.24\linewidth,height=1.0cm]{lego_glue}
				\vspace*{\fill}
					$=$ 
				\vspace*{\fill}	
			\pic[width=.3\linewidth,height=1.0cm]{lego_product}
		}		
		\myexample{Plenty of Glue Code}{
			\centering
			\pic[width=.65\linewidth,height=2.5cm]{lego_glue}				
		}				
	\mynextcolumn
		\pause
		\mydefinition{Same Idea}{
			\begin{itemize}
				\item Features are implemented as services.
				\item Feature selection determines the services to be composed.
			\end{itemize}
		}	
		\pause
		\mynote{However}{
			``Standardized'' service composition instead of highly individual glue code.
		}	
		\myexample{}{
			\pic[width=.27\linewidth,height=1.75cm]{lego_components} 
				\vspace*{\fill}
					$+$ 
				\vspace*{\fill}	
			\pic[width=.27\linewidth,height=1.75cm]{lego_orchestration}
				\vspace*{\fill}
					$=$ 
				\vspace*{\fill}	
			\pic[width=.35\linewidth,height=1.75cm]{lego_product}
		}	
	\end{mycolumns}	
\end{frame}

\subsection{Service Composition}
\begin{frame}{Service ``Composition''}
	\begin{mycolumns}[widths={50,50}]
		\mydefinition{Orchestration}{
			Description of an executable (business-)process as a combination of services (centralized perspective, e.g., using WS-BPEL).
		}
		\myexampletight{}{
			\pic[width=\linewidth]{trafficlights}
		}
	\mynextcolumn
		\mydefinition{Choreography}{
			Each service describes its own task within a service composition (decentralized perspective, e.g, using WS-CDL).
		}
		\myexampletight{}{
			\pic[width=\linewidth]{crossroad}
		}
	\end{mycolumns}	
\end{frame}