%\subsection{Expression Problem? Preplanning? Tyrany}
%\subsection{Multiple Inheritance}
%\subsection{Mixins}
%\subsection{Design Patterns}
%\subsection{Default Methods}
%\subsection{\ldots}

\subsection{Recap on Modularity} % TODO all titles in this subsection have used the prefix recap before with clone-and-own of slides, but not sure how to do this with againframe
\againframe{ObjectOrientation}

\againframe{DesignPatterns}

\begin{frame}{Recap: Modularity - Components, Services, and Frameworks} % TODO such an overview slide could be helpful in Lecture 6 already
	\begin{fancycolumns}[widths={40,60},animation=none]
		\begin{example}{Component-/Service-Based SPLs} % TODO separate visualization for components?
			\raisebox{-0.4cm}{\pic[width=.23\linewidth,height=1.0cm]{lego_components}}
					$+$ 
			\raisebox{-0.4cm}{\pic[width=.23\linewidth,height=1.0cm]{lego_orchestration}}
					$=$ 
			\raisebox{-0.4cm}{\pic[width=.3\linewidth,height=1.0cm]{lego_product}}
		\end{example}	
		\begin{example}{Specification of ``composition'' (glue code, orchestration)}
			\centering
			\pic[width=.45\linewidth,height=1.7cm]{lego_orchestration}				
		\end{example}
	\nextcolumn		
		\begin{example}{Framework-Based SPLs}
			\raisebox{-0.8cm}{\pic[width=.29\linewidth,height=1.75cm]{lego_product_partial}}
					$+$ 
			\raisebox{-0.8cm}{\pic[width=.27\linewidth,height=1.75cm]{lego_components}}
					$=$ 
			\raisebox{-0.8cm}{\pic[width=.29\linewidth,height=1.75cm]{lego_product}}
		\end{example}
		\begin{note}{}
			Neither glue code nor service composition required.
		\end{note}			
	\end{fancycolumns}	
\end{frame}

\againframe<2>{GraphsWithPlugins}

\againframe<2>{ChallengesOfPlugins}

\subsection{Preplanning Problem}

\begin{frame}{Preplanning Problem}
	\begin{fancycolumns}[t,widths={45}]
		\begin{note}{Components, Services, and Frameworks}
			Extensions are not possible ad-hoc, but must be foreseen and planned in advance:
			\begin{itemize}
				\item Frameworks: Hot spots / extension points
				\item Components/services: Provided and required interfaces 
			\end{itemize}
			\emph{$\Rightarrow$ No modular extension without a suitable extension point / interface!}
		\end{note}
	\nextcolumn
		\begin{note}{And classical OO language concepts?}
			Subtyping and polymorphism allow for ad-hoc extensions to some extent, but:
			\begin{itemize}
				\item Often, client code and/or basic implementation need to be adapted, too (non-modular)
				\item Design patterns require preplanning of potential future extensions
				\item Limited flexibility of inheritance hierarchies (no ``mix and match'' supporting arbitrary feature combinations)
			\end{itemize}
			\emph{$\Rightarrow$ No variable extension without loosing modularity!}
		\end{note}
	\end{fancycolumns}
\end{frame}

\subsection{Crosscutting Concerns}

\begin{frame}{Crosscutting Concerns \deutschertitel{Querschneidende Belange}} % TODO 1:1 quotes? use mycite then?
	\begin{fancycolumns}[t]
		\begin{definition}{Concern \mysource{\fospl\mypages{55}}}
			A concern is an area of interest or focus in a system, and features are the concerns of primary interest in product-line engineering.
		\end{definition}
	\nextcolumn
		\begin{definition}{Crosscutting (Concern) \mysource{\fospl\mypages{55}}}
			Crosscutting is a structural relationship between the representations of two concerns. 
			It is an alternative to hierarchical and block structure.
		\end{definition}
	\end{fancycolumns}
\end{frame}

\subsection{Tyranny of the Dominant Decomposition}

\begin{frame}{Tyranny of the Dominant Decomposition \deutschertitel{Tyrannei der dominanten Dekomposition}} % TODO citation to ICSE'99?
	\begin{fancycolumns}
		\begin{definition}{Tyranny of the Dominant Decomposition}
			\begin{itemize}
				\item Many concerns can be modularized, but not always at the same time.
				\item Developers choose a decomposition, but some other concerns cut across.
				\item Simultaneous modularization along different dimensions is not possible.
			\end{itemize}
		\end{definition}
	\nextcolumn
		\begin{note}{} % TODO use example environment instead? split?
			Crosscutting concerns are inherently difficult to separate using traditional mechanisms.
			\begin{itemize}
				\item Logging: Each time a method is called.
				\item Caching/Pooling: Each time an object is created.
				\item Synchronization/Locking: Extension of many methods with lock/unlock calls.
			\end{itemize}
			Features in a software product line are often crosscutting (e.g., color and weight in our graph example).
		\end{note}
	\end{fancycolumns}
\end{frame}

\subsection{Example: Arithmetic Expressions}

\begin{frame}{\myframetitle}
	\begin{fancycolumns}[widths={60}]
		\begin{exampletight}{$(5 + ln(50)) * 1.73$}
			\centering
			\picDark[width=.45\linewidth]{expressions}
		\end{exampletight}
	\nextcolumn
		\begin{example}{}
			\begin{itemize}
				\item Arithmetic expressions are stored in a tree structure.
				\item Terms (i.e., sub-trees) can be evaluated and printed.
			\end{itemize}
		\end{example}
		\begin{note}{Question}
			How to separate data structures and operations such that both can be extended independently of each other?
		\end{note}
	\end{fancycolumns}
\end{frame}

\begin{frame}{Arithmetic Expressions -- Data-Centric Decomposition}
	\begin{fancycolumns}[widths={40},animation=none]
		\begin{definition}{Implementation Variant 1:\\Data-Centric}
			\begin{itemize}
				\item Recursive class structure (composite pattern)
				\item For each operation (eval, print, ...) there is a dedicated method in each class (Number, Plus, ...) 				
			\end{itemize}
		\end{definition}
		\uncover<2->{\begin{note}{Terms are modular, but...}
			\begin{itemize}
				\item New operations, e.g. drawTree or simplify, cannot simply be added
				\item All existing classes must be adjusted!
				\item Operations cut across the expressions.
			\end{itemize}
		\end{note}}
	\nextcolumn
		\begin{exampletight}{}
			\picDark[width=\linewidth]{data_centered}
		\end{exampletight}
	\end{fancycolumns}
\end{frame}

\begin{frame}{Arithmetic Expressions -- Operation-Centric Decomposition}
	\begin{fancycolumns}[widths={40}]
		\begin{definition}{Implementation Variant 2:\\Operation-Centric}
			\begin{itemize}
				\item Just a single accept method per class (visitor pattern).
				\item Each operation is implemented by a dedicated visitor.				
			\end{itemize}
		\end{definition}
		\uncover<2->{\begin{note}{Operations are modular, but...}
			\begin{itemize}
				\item New expressions, e.g. Min or Power, cannot simply be added
				\item For each new class, all visitor classes must be adjusted
				\item Expressions cut across operations
			\end{itemize}
		\end{note}}
	\nextcolumn
		\begin{exampletight}{}
			\picDark[width=\linewidth]{method_centered}
		\end{exampletight}
	\end{fancycolumns}
\end{frame}

\begin{frame}{Lessons Learned from the Simple Example}
	\begin{note}{}
		Hardly possible to modularize expressions and operations at the same time!
	\end{note}
	\pause
	\begin{fancycolumns}
		\begin{note}{Data-Centric Decomposition}
			\begin{itemize}
				\item New expressions can be added directly: modular.
				\item New operations must be added to all classes: not modular.
			\end{itemize}
		\end{note}
	\nextcolumn
		\begin{note}{Operation-Centric Decomposition}
			\begin{itemize}
				\item New operations can be added as another visitor: modular.
				\item For new expressions, all existing visitors must be extended: not modular.
			\end{itemize}
		\end{note}	
	\end{fancycolumns}
\end{frame}