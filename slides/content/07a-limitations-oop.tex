\subsection{Recap: Object-Oriented Key Concepts}

\begin{frame}{Recap: Object-Oriented Key Concepts}
	\leftorright{
		\mydefinition{Encapsulation}{Abstraction \& Information Hiding}		
		\mydefinition{Composition}{Nested objects}	
		\mydefinition{Message Passing}{Delegating responsibility}
		\mydefinition{Distribution of Responsibility}{Separation of concerns}	
		\mydefinition{Inheritance}{Conceptual hierarchy, polymorphism, reuse}
	}{
		\pic[width=0.8\linewidth]{oo-concepts-illustration} % was illustriert was? warum ist square kein rectangle?
	}
\end{frame}

\begin{frame}{Recap: Design Patterns (for Variability) \mytitlesource{\gof}}
	\leftorright{
		\vspace{-10mm}
		\mynote{Design patterns \deutsch{Entwurfsmuster}}{
			\begin{itemize}
				\item Document common solutions to concrete yet frequently occurring design problems.
				\item Suggest a concrete implementation for a specific object-oriented programming problem.
			\end{itemize}
		}		
		\mynote{Design patterns for variability}{
			\begin{itemize}
				\item Many Gang of Four (GoF) design patterns for designing software around stable abstractions and interchangeable (i.e., variable) parts, e.g.
				\begin{itemize}
					\item Template Method
					\item Abstract Factory
					\item Decorator
				\end{itemize}
			\end{itemize}
		}				
	}{
		\pic[width=\linewidth]{gof}
	}
\end{frame}

\begin{frame}{Recap: Modularity - Components, Services, and Frameworks}
	\begin{mycolumns}[widths={40,60},animation=none]
		\myexample{Component-/Service-Based SPLs}{
			\pic[width=.23\linewidth,height=1.0cm]{lego_components} 
				\vspace*{\fill}
					$+$ 
				\vspace*{\fill}	
			\pic[width=.23\linewidth,height=1.0cm]{lego_orchestration}
				\vspace*{\fill}
					$=$ 
				\vspace*{\fill}	
			\pic[width=.3\linewidth,height=1.0cm]{lego_product}
		}		
		\myexample{Specification of ``composition'' (glue code, orchestration)}{
			\centering
			\pic[width=.45\linewidth,height=1.7cm]{lego_orchestration}				
		}				
	\mynextcolumn		
		\myexample{Framework-Based SPLs}{
			\pic[width=.31\linewidth,height=1.75cm]{lego_product_partial} 
				\vspace*{\fill}
					$+$ 
				\vspace*{\fill}	
			\pic[width=.27\linewidth,height=1.75cm]{lego_components}
				\vspace*{\fill}
					$=$ 
				\vspace*{\fill}	
			\pic[width=.31\linewidth,height=1.75cm]{lego_product}
		}	
		\mynote{}{
			Neither glue code nor service composition required.
		}			
	\end{mycolumns}	
\end{frame}

\begin{frame}[fragile]{Recap: Extending Basic Graphs by Plug-Ins?}
	\tiny\begin{mycolumns}[columns=2,widths={50,50}]
\begin{codetight}{}
public class Graph {
	@private List<GraphPlugin> plugins = new ArrayList<GraphPlugin>();@
	// ...	
	@public void registerPlugin(GraphPlugin p){
		plugins.add(p);
	}@
	public void addNode(int id, Color c){
		Node n = new Node(id);
		@notifyAdd(n, c);@
		nodes.add(n);
	}
	public void print() {
		for (Node n : nodes) {
			@notifyPrint(n);@
			// ...
		}
		// ...
	}
	@private void notifyAdd(Node n, Color c) {
		for (GraphPlugin p : plugins) {
			p.aboutToAdd(n, c);
		}
	}
	private void notifyPrint(Node n) {
		for (GraphPlugin p : plugins) {
			p.aboutToPrint(n);
		}
	}@
	// ...
}
\end{codetight}
		\mynextcolumn
\begin{codetight}{}
public interface GraphPlugin {
	public void aboutToAdd(Node n, Color c);
	public void aboutToAdd(Edge e, Weight w);
	public void aboutToPrint(Node n);
	public void aboutToPrint(Edge e);
}
\end{codetight}
\begin{codetight}{}
public class ColorPlugin implements GraphPlugin {
	private Map<Node, Color> map = new HashMap<Node, Color>();

	public void aboutToAdd(Node n, Color c) {
		map.put(n, c);
	}
	
	public void aboutToAdd(Edge e, Weight w) {
		~// do nothing~
	}
	
	public void aboutToPrint(Node n) {
		Color c = map.get(n);
		Color.setDisplayColor(c);
	}
	
	public void aboutToPrint(Edge e) {
		~// do nothing~
	}
}
\end{codetight}
	\end{mycolumns}
\end{frame}

\begin{frame}{Recap: Challenges and Problems}
	\begin{mycolumns}[widths={50,50}]
		\myexample{}{
			In our example, we can observe that:
			\begin{itemize}
				\item There are lots of empty methods in the ColorPlugin 
				\item The framework consults all registered plug-ins before printing a node or edge
			\end{itemize}
		}	
		\pause
		\mydefinition{General Challenge: Cross-cutting Concerns}{
			Implementing cross-cutting concerns as plug-ins 
			\begin{itemize}				
				\item typically leads to huge interfaces, large parts of which are irrelevant for a dedicated plug-in 
				\item causes lots of communication overhead between plug-ins and framework
			\end{itemize}
		}
		\pause
	\mynextcolumn
		\myexample{}{
			If we were not familiar with our graph library, would we anticipate that:
			\begin{itemize}
				\item Colors and weights should be part of the Plugin interface?
				\item Every plug-in needs to be notified that the framework is about to print a node or edge? 
			\end{itemize}
		}
		\pause
		\mydefinition{Generally known as Preplanning Problem}{
			\begin{itemize}
				\item Hard to identify and foresee the relevant hot spots and nature of extensions
				\item Developing a framework needs lots of expertise and excellent domain knowledge 
			\end{itemize}
		}	
	\end{mycolumns}
\end{frame}

\subsection{Preplanning Problem}

\begin{frame}{Preplanning Problem}
	\begin{mycolumns}[widths={45,55},animation=none]
		\mynote{Components, Services, and Frameworks}{
			Extensions are not possible ad-hoc, but must be foreseen and planned in advance:
			\begin{itemize}
				\item Frameworks: Hot spots / extension points
				\item Components/services: Provided and required interfaces 
			\end{itemize}
			\emph{$\Rightarrow$ No modular extension without a suitable extension point / interface!}
		}
	\mynextcolumn
		\mynote{And classical OO language concepts?}{
			Subtyping and polymorphism allow for ad-hoc extensions to some extent, but:
			\begin{itemize}
				\item Often, client code and/or basic implementation need to be adapted, too (non-modular)
				\item Limited flexibility of inheritance hierarchies (no ``mix and match'' supporting arbitrary feature combinations)
			\end{itemize}
			\emph{$\Rightarrow$ No variable extension without loosing modularity!}
		}
	\end{mycolumns}
\end{frame}

\subsection{Crosscutting Concerns}

\begin{frame}{Crosscutting Concerns}
	\begin{mycolumns}[widths={50,50},animation=none]
		\mydefinition{Concern \mysource{\fospl\mypages{55}}}{
			A concern is an area of interest or focus in a system, and features are the concerns of primary interest in product-line engineering.
		}
	\mynextcolumn
		\mydefinition{Crosscutting (Concern) \mysource{\fospl\mypages{55}}}{
			Crosscutting is a structural relationship between the representations of two concerns. 
			It is an alternative to hierarchical and block structure.
		}
	\end{mycolumns}
\end{frame}

\subsection{Tyranny of the Dominant Decomposition}

\begin{frame}{Tyranny of the Dominant Decomposition}
	\begin{mycolumns}[widths={50,50},animation=none]
		\mydefinition{Tyranny of the Dominant Decomposition}{
			\begin{itemize}
				\item Many concerns can be modularized, but not always at the same time.
				\item Developers choose a decomposition, but some other concerns cut across.
				\item Simultaneous modularization along different dimensions is not possible.
			\end{itemize}
		}
	\mynextcolumn
		\mynote{}{
			Crosscutting concerns are inherently difficult to separate using traditional mechanisms.
			\begin{itemize}
				\item Logging: Each time a method is called.
				\item Chaching/Pooling: Each time an object is created.
				\item Synchronization/Locking: Extension of many methods with lock/unlock calls.
			\end{itemize}
			Features in a software product line are often crosscutting (e.g., color and weight in our graph example).
		}
	\end{mycolumns}
\end{frame}

\begin{frame}{Another Example: Arithmetic Expressions}
	\begin{mycolumns}[widths={50,50},animation=none]
		\myexample{}{
			\begin{itemize}
				\item Arithemtic expressions are stored in a tree structure.
				\item Terms (i.e., sub-trees) can be evaluated and printed.
			\end{itemize}
		}
		\mynote{Question}{
			How to separate data structures and operations such that both can be extended independently of each other?
		}
	\mynextcolumn
		\myexampletight{}{
			\centering
			\pic[width=0.5\linewidth]{expressions}
		}
	\end{mycolumns}
\end{frame}

\begin{frame}{Arithmetic Expressions - Data-centric Implementation}
	\begin{mycolumns}[widths={40,60},animation=none]
		\mydefinition{Implementation Variant 1: Data-centric}{
			\begin{itemize}
				\item Recursive class structure (composite pattern)
				\item For each operation (eval, print, ...) there is a dedicated method in each class (Number, Plus, ...) 				
			\end{itemize}
		}
		\pause
		\mynote{Terms are modular, but...}{
			\begin{itemize}
				\item New operations, e.g. drawTree or simplify, cannot simply be added
				\item All existing classes must be adjusted!
				\item Operations cut across the expressions.
			\end{itemize}
		}
	\mynextcolumn
		\myexampletight{}{
			\pic[width=\linewidth]{data_centered}
		}
	\end{mycolumns}
\end{frame}

\begin{frame}{Arithmetic Expressions - Operation-centric Implementation}
	\begin{mycolumns}[widths={40,60},animation=none]
		\mydefinition{Implementation Variant 2: Operation-centric}{
			\begin{itemize}
				\item Just a single accept method per class (visitor pattern).
				\item Each operation is implemented by a dedicated visitor.				
			\end{itemize}
		}
		\pause
		\mynote{Operations are modular, but...}{
			\begin{itemize}
				\item New expressions, e.g. Min or Power, cannot simply be added
				\item For each new class, all visitor classes must be adjusted
				\item Expressions cut across operations
			\end{itemize}
		}
	\mynextcolumn
		\myexampletight{}{
			\pic[width=\linewidth]{method_centered}
		}
	\end{mycolumns}
\end{frame}

\begin{frame}{Lessons Learned from the Simple Example}
	\mynote{}{
		Hardly possible to modularize expressions and operations at the same time!
	}
	\pause
	\begin{mycolumns}[widths={50,50},animation=none]
		\mynote{Data-centric approach}{
			\begin{itemize}
				\item New expressions can be added directly: modular.
				\item New operations must be added to all classes: not modular.
			\end{itemize}
		}
	\mynextcolumn
		\mynote{Method-centric approach}{
			\begin{itemize}
				\item New operations can be added as another visitor: modular.
				\item For new expressions, all existing visitors must be extended: not modular.
			\end{itemize}
		}			
	\end{mycolumns}
\end{frame}