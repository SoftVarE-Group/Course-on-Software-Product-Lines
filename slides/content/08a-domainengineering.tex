\subsection{Recap: Process Models}
\begin{frame}{\myframetitle}
	\begin{definition}{Recap: The Software Life Cycle}
		\waterfallcartoon
	\end{definition}

	\uncover<2->{\begin{mycolumns}[animation=none]
		\begin{example}{Process Models for Single-System Engineering}
			waterfall model, V model, scrum, \ldots
		\end{example}
	\mynextcolumn
		\uncover<3->{\begin{note}{Process Models for Product-Line Engineering}
			???
		\end{note}}
	\end{mycolumns}}
\end{frame}

\subsection{Domain Engineering}

\begin{frame}{Domain and Application Engineering}
	\begin{note}{A Process Model for Product-Line Engineering}
		idea: split development into two phases, one for product line and one for products
	\end{note}
	\begin{mycolumns}[animation=none,T]
		\uncover<2->{\begin{definition}{Domain Engineering\mysource{\fospl\mypage{21}}}
			\mycite{\emph{Domain engineering} is the process of analyzing the domain of a product line and developing reusable artifacts.}
		\end{definition}
		\begin{note}{Domain Engineering\mysource{\fospl\mypage{21}}}
			\begin{itemize}
				\item development for reuse
				\item prepares artifacts to be used in products
			\end{itemize}
		\end{note}}
	\mynextcolumn
		\uncover<3->{\begin{definition}{Application Engineering\mysource{\fospl\mypage{21}}}
			\mycite{\emph{Application engineering} has the goal of developing a specific product for the needs of a particular customer (or other stakeholder).}
		\end{definition}
		\begin{note}{Application Engineering\mysource{\fospl\mypage{21}}}
			\begin{itemize}
				\item development with reuse
				\item build products using artifacts from domain engineering
				\item repeated for every product
			\end{itemize}
		\end{note}}
	\end{mycolumns}
\end{frame}

%\begin{frame}{\myframetitle}
%	\begin{mycolumns}
%		\mydefinition{Product-Line Engineering\mysource{\sple\mypage{14}}}{
%			\mycite{\emph{Software product-line engineering} is a paradigm to develop software applications (software-intensive systems and software products) using software platforms and mass customization.}
%		}
%	\mynextcolumn
%	\end{mycolumns}
%\end{frame}

%\projectcartoon{01}{Requirements}
%\projectcartoon{13}{Product}
%
%\hprojectcartoon{02}{Analysis}
%\hprojectcartoon{03}{Design}
%\hprojectcartoon{04}{Implementation}
%\hprojectcartoon{05}{Testing}
%\hprojectcartoon{10}{Maintenance}

\begin{frame}{Recap: Domain}
	\begin{mycolumns}[animation=none]
		\begin{definition}{Recap: Domain \deutsch{Domäne} \hfill\tiny\lectureintroduction}
			\mycitebegin A \emph{domain} is an area of knowledge that:
			\begin{itemize}
				\item is scoped to maximize the satisfaction of the requirements of its stakeholders,
				\item includes a set of concepts and terminology understood by practitioners in that area,
				\item and includes the knowledge of how to build software systems (or parts of
				software systems) in that area.\myciteend
			\end{itemize}
		\end{definition}
	\mynextcolumn
	\end{mycolumns}
\end{frame}

\subsubsection{Domain Analysis}
\begin{frame}{\myframetitle}
	\begin{mycolumns}[T,columns=3,widths={10}]
		\renewcommand{\projectcartoonwidth}{1}\hprojectcartoon{02}{Analysis}
	\mynextcolumn
		\begin{definition}{Domain Analysis\mysource{\fospl\mypage{21}}}
			\mycite{\emph{Domain analysis} is a form of requirements engineering for an entire product line. Here, we need to decide the scope of the domain, that is, decide which products should be covered by the product line and, consequently, which features are relevant and should be implemented as reusable artifacts. The results of domain analysis are usually documented in a feature model.}
		\end{definition}
	\mynextcolumn
		\begin{definition}{Domain Scoping\mysource{\fospl\mypage{21}}}
			\mycite{\emph{Domain analysis} is a form of requirements engineering for an entire product line. Here, we need to decide the scope of the domain, that is, decide which products should be covered by the product line and, consequently, which features are relevant and should be implemented as reusable artifacts. The results of domain analysis are usually documented in a feature model.}
		\end{definition}
	\end{mycolumns}
\end{frame}

\begin{frame}{Domain Scoping in Practice}
	\centering\includegraphics[width=.8\linewidth]{toyota-aygo-mirrorcovers}
\end{frame}

\subsubsection{Domain Design and Implementation}
% reusable parts indespensible, reusable architecture optional
%\subsubsection{Domain Testing/Verification}
% systematic testing, sometimes skipped (e.g., if product derivation not automatic)

\subsection{Application Engineering}

\subsubsection{Product Configuration}
% application-specific requirements analysis
\subsubsection{Application Design and Implementation}
% fully automatic or semi-automatic with custom design and implementation
%\subsubsection{Application Testing/Verification}
% testing of a single product, sometimes optional

\subsection{Problem and Solution Space}
\begin{frame}{\myframetitle}
	\begin{mycolumns}[T]
		\begin{definition}{Problem Space\mysource{\fospl\mypage{21}}}
			\mycite{The \emph{problem space} takes the perspective of stakeholders and their problems, requirements, and views of the entire domain and individual products. Features are, in fact, domain abstractions that characterize the problem space.}
		\end{definition}
		\todo{add feature model + configuration}
	\mynextcolumn
		\begin{definition}{Solution Space\mysource{\fospl\mypage{21}}}
			\mycite{The \emph{solution space} represents the developer’s and vendor’s perspectives. It is characterized by the terminology of the developer, which includes names of functions, classes, and program parameters. The solution space covers the design, implementation, and validation and verification of features and their combinations in suitable ways to facilitate systematic reuse.}
		\end{definition}
		\todo{add source code and generated code}
	\end{mycolumns}
\end{frame}

\subsection{Overview on Domain and Application Engineering}
\begin{frame}%{\myframetitle}
	\footnotesize%
	\begin{mycolumns}[columns=3,widths={10,70,10},animation=none]
		\renewcommand{\projectcartoonwidth}{1}\hprojectcartoon{01}{Product-Line Requirements}
	\mynextcolumn
		\begin{note}{Domain Engineering}
			\renewcommand{\projectcartoonwidth}{.15}%
			\hprojectcartoon{02}{Domain Analysis}%
			\hprojectcartoon{03}{Domain Design}%
			\hprojectcartoon{04}{Domain Implementation}%
			\hprojectcartoon{05}{Domain Testing}
		\end{note}
	\mynextcolumn
	\end{mycolumns}
	\pause
	\begin{mycolumns}[columns=3,widths={10,70,10},animation=none]
		\renewcommand{\projectcartoonwidth}{1}\hprojectcartoon{01}{Product Requirements}
	\mynextcolumn
		\begin{note}{Application Engineering}
			\renewcommand{\projectcartoonwidth}{.15}%
			\hprojectcartoon{02}{Product Configuration}%
			\hprojectcartoon{03}{Application Design}%
			\hprojectcartoon{04}{Application Implementation}%
			\hprojectcartoon{05}{Application Testing}
		\end{note}
	\mynextcolumn
		\renewcommand{\projectcartoonwidth}{1}\hprojectcartoon{13}{Product}
	\end{mycolumns}
\end{frame}
% lecture 8: show how linux in lecture 4+5 relates to problem space (kconfig)/mapping (kbuild)/solution space (cpp) - put this into the four-quadrant model

