% discuss each technique wrt. its properties

\subsection{Runtime Variability}
\subsection{Clone-and-Own}
% probably three slides: ad-hoc, with version control, with build systems
\subsection{Conditional Compilation}
\subsection{Components and Services}
% probably feasible on the same slide
\subsection{White-Box and Black-Box Frameworks}
% probably requires two slides
\subsection{X-Oriented Programming}
% X=feature/aspect/delta




























% TODO \subsection{Design Patterns?}

\subsection{Comparison of Implementation Techniques}
\begin{frame}{\myframetitle}
	\centering
	\begin{tabular}{|p{32mm}|p{32mm}|p{14mm}|p{27mm}|p{14mm}|}
		\hline
		 & Compile-Time Variability & Features & Product \mbox{Generation} & Feature \mbox{Traceability} \\
		\hline\pause
		Runtime \mbox{Variability} & no (very limited for immutable global variables) & yes & yes (except for preference dialogs) & no \\
		% automated generation not for preference dialogs
		% 
		\hline\pause
		Clone-and-Own & yes (only for implemented products) & no & no (limited generation with build systems) & no \\
		\hline\pause
		Conditional Compilation & yes & yes & yes & with tool support \\
		\hline\pause
		Components \linebreak Services & yes & only coarse grained & no (except pure exchange) & only coarse grained \\
		\hline\pause
		Frameworks with Plug-Ins & yes & only coarse grained & yes & only coarse grained \\
		\hline\pause
		FOP \linebreak AOP & yes & yes & yes & yes \\
		\hline
	\end{tabular}
	\pause
	\begin{note}{Further Criteria}
		interfaces between features? code duplication necessary? modularization of homogeneous extensions? \ldots
	\end{note}
\end{frame}
% TODO animate table adding only columns relevant so far
% TODO conditional compilation: only coarse grained for build systems and only fine grained for preprocessors

% large table with all properties: automatic generation from feature selection, feature modularization, ...

