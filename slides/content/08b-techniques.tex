% discuss each technique wrt. its properties

\subsection{Runtime Variability}
\subsection{Clone-and-Own}
% probably three slides: ad-hoc, with version control, with build systems
\subsection{Conditional Compilation}
\subsection{Components and Services}
% probably feasible on the same slide
\subsection{White-Box and Black-Box Frameworks}
% probably requires two slides
\subsection{X-Oriented Programming}
% X=feature/aspect/delta




























% TODO \subsection{Design Patterns?}

\subsection{Comparison of Implementation Techniques}
\begin{frame}
	\centering
	\begin{tabular}{|p{30mm}|p{20mm}|p{20mm}|p{20mm}|p{20mm}|}
		\hline
		 & Compile-Time Variability & Features & Product \mbox{Generation} & Feature \mbox{Traceability} \\
		\hline
		Runtime \mbox{Variability} & no (very limited for immutable global variables) & yes & yes (except for preference dialogs) & no \\
		% automated generation not for preference dialogs
		% 
		\hline
		Clone-and-Own & yes (only for implemented products) & no & no (limited generation with build systems) & no \\
		\hline
		Conditional Compilation & yes & yes & yes & with tool support \\
		\hline
		Components and Services & yes & only coarse grained & no (except pure exchange) & only coarse grained \\
		\hline
		Frameworks with Plug-Ins & yes & only coarse grained & yes & only coarse grained \\
		\hline
		FOP and AOP & yes & yes & yes & yes \\
		\hline
	\end{tabular}
\end{frame}

% large table with all properties: automatic generation from feature selection, feature modularization, ...

