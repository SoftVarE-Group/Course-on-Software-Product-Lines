\subsection{Product-Line Adoption}
\begin{frame}{\myframetitle}
	\Huge\centering How to introduce a product line in practice?
\end{frame}

\subsection{Proactive Adoption Strategy}
\begin{frame}{\myframetitle\ \mytitlesource{\fospl\mypage{40}}}
	\begin{mycolumns}[animation=none]
		\begin{definition}{Proactive Adoption}
			\begin{itemize}
				\item development of a product line from scratch
				\item process model as presented in Part~1
				\item often seen as idealistic, academic
				\item comparable to the waterfall model
			\end{itemize}
		\end{definition}
	\mynextcolumn
		\begin{note}{Advantages}
			\begin{itemize}
				\item desired variability planned first
				\item potentially higher code quality
			\end{itemize}
		\end{note}
		\begin{note}{Disadvantages}
			\begin{itemize}
				\item high up-front investment and risks
				\item late time-to-market
				\item production stop: developers develop product line rather products
				\item no reuse of existing products
			\end{itemize}
		\end{note}
	\end{mycolumns}
\end{frame}

\subsection{Extractive Adoption Strategy}
\begin{frame}{\myframetitle\ \mytitlesource{\fospl\mypages{40--41}}}
	\begin{mycolumns}[animation=none]
		\begin{definition}{Extractive Adoption}
			\begin{itemize}
				\item migrate one existing product into a product line (with or without runtime variability)
				\item or: migrate several cloned products into a product line (cf.\ clone-and-own)
				\item often motivated by maintenance problems after inconsistent evolution
				\item requires identification of commonalities and variabilities
				\item extraction of reusable artifacts
				\item very common in practice
			\end{itemize}
		\end{definition}
	\mynextcolumn
		\begin{note}{Advantages}
			\begin{itemize}
				\item lower risks and up-front investment
				\item all products remain in production
			\end{itemize}
		\end{note}
		\begin{note}{Disadvantages}
			\begin{itemize}
				\item code quality depends on tools for extraction
				\item limited choice of implementation techniques
			\end{itemize}
		\end{note}
	\end{mycolumns}
\end{frame}

\subsection{Reactive Adoption Strategy}
\begin{frame}{\myframetitle\ \mytitlesource{\fospl\mypages{41--42}}}
	\begin{mycolumns}[animation=none]
		\begin{definition}{Reactive Adoption}
			\begin{itemize}
				\item start with one or a few products
				\item incrementally develop more features resulting in more products
				\item gradually reach the ideal product line
				\item requires to identify order among features (products)
				\item comparable to agile methods
			\end{itemize}
		\end{definition}
	\mynextcolumn
		\begin{note}{Advantages}
			\begin{itemize}
				\item less up-front investment that the proactive strategy
				\item also applicable for evolution of a product line (not only adoption)
			\end{itemize}
		\end{note}
		\begin{note}{Disadvantages}
			\begin{itemize}
				\item more changes to architecture and design necessary as not all features planned up-front
			\end{itemize}
		\end{note}
	\end{mycolumns}
\end{frame}

% TODO \subsection{Strategies in Practice}
% typically combinations of the above strategies
% reports on the use of those strategies

% TODO \subsection{Virtual Platform?}
% continuum between clone-and-own and product lines
% classification between ad-hoc clone-and-own and product line with fully automatic generation

