% TODO \subsection{Recap: Size of Configuration Spaces}

% TODO \subsection{Recap: Increase of Features and Variants}
% features: Linux
% variants: Automotive02-05, KfW?

% TODO number of potential interactions?

% TODO variant reduction, prevent the explosion. marketing wants them all. engineering and quality assurance too expensive.

%\subsection{Domain Scoping Revisited}

%\subsection{Costs of Variability}

\subsection{The Choice of Features}
\begin{frame}{\myframetitle}
	\begin{fancycolumns}
		\begin{exampletight}{John Ferguson Smart (2017)}
			\picDark[width=.98\linewidth,angle=2,trim=0 0 5 0,clip]{unnecessary-features}
		\end{exampletight}
		% 
	\nextcolumn
		\centering\pic[width=.47\linewidth]{john-carmack}
		\vspace{-7mm}
		
		\begin{note}{John Carmack (born 1970) \mysource{\href{https://www.ics.uci.edu/~pattis/quotations.html\#C}{uci.edu}}}
			\mycite{The important point is that the cost of adding a feature isn't just the time it takes to code it. The cost also includes the addition of an obstacle to future expansion. %Sure, any given feature list can be implemented, given enough coding time. But in addition to coming out late, you will usually wind up with a codebase that is so fragile that new ideas that should be dead-simple wind up taking longer and longer to work into the tangled existing web. 
		[...] The trick is to pick the features that don't fight each other.}
		\end{note}
		% video game developer, co-founder of a video game company
	\end{fancycolumns}
\end{frame}

\begin{frame}{\myframetitle}
	\begin{fancycolumns}[animation=none]
		\pic[width=\linewidth]{ford-t-1910}
	\nextcolumn
		\begin{note}{Henry Ford, 1909} % TODO source for this quote missing
			\mycite{Any customer can have a car painted any color that he wants so long as it is black.}
		\end{note}
		\uncover<2->{\begin{example}{Why only black?\mysource{\fospl}}
			\begin{itemize}
				\item black color dried faster
				\item faster production
				\item more products and cheaper production
			\end{itemize}
		\end{example}}
	\end{fancycolumns}
\end{frame}

\subsection{Documentation of Interactions}

\subsubsection*{Incompatibilities of Lenovo Hardware}
\begin{frame}{\myframetitle}
	\begin{fancycolumns}[widths={59}]
		\begin{note}{Documentation of Remaining Interactions}
			\begin{itemize}
				\item not all interactions can be prevented/fixed
				\item how to apply strategy S1 (see Part II) if there is no feature model?
				\item what to document?
			\end{itemize}
		\end{note}
		\uncover<2->{\begin{example}{Lenovo's Option Compatibility Matrices}
			\begin{itemize}
				\item 7 Excel files for current products\\+ archive for old products
				\item Excel file for computers contains 32 tables (series)
				\item table for ThinkPad X has 28 columns (models) and $>500$ rows (accessories)
				\item 14k cells contain $>400$ different footnotes
				\item a footnote explains one incompatibility
			\end{itemize}
		\end{example}}
	\nextcolumn
		\myexampletight{}{
			\picDark[width=\linewidth,trim=0 1900 0 0,clip]{lenovo-compatibility-matrices}
			\picDark[width=\linewidth,trim=0 1130 0 600,clip]{lenovo-compatibility-matrices}
		}
	\end{fancycolumns}
\end{frame}
\begin{frame}{\myframetitle}
	\begin{fancycolumns}[widths={80},animation=none]
		\picDark[width=\linewidth]{lenovo-compatibility-matrix-1}
	\nextcolumn
		\begin{example}{}\setlength\leftmargini{3mm} 
			\begin{itemize}
				\item columns contain notebooks
				\item rows contain accessories
				\item X indicates compatibility
				\item numbers indicate a known incompatibility
			\end{itemize}
		\end{example}
	\end{fancycolumns}
\end{frame}
\begin{frame}{\myframetitle}
	\begin{fancycolumns}[widths={80},animation=none]
		\picDark[width=\linewidth]{lenovo-compatibility-matrix-3}
	\nextcolumn
		\begin{example}{}\setlength\leftmargini{3mm} 
			\begin{itemize}
				\item 314: pen requires touch screen (cf.\ S1)
				\item 315/319/320: extra module needed (cf.\ S6)
				\item 316/317: fixed in newer BIOS versions
				\item 318/321: two modules with a different power supply (cf.\ S3)
			\end{itemize}
		\end{example}
	\end{fancycolumns}
\end{frame}
