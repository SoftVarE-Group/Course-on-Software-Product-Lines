
\newcommand{\notleftright}{\mathrel{\ooalign{$\Leftrightarrow$\cr\hidewidth$/$\hidewidth}}}

\subsection{The Use of Feature-Mapping Analyses}

\begin{frame}{\myframetitle}
	\begin{mycolumns}[widths={45,55}]
		\mynote{Recap: A Typical Product Line}{
			\begin{itemize}
				\item embedded or systems programming (e.g., Linux)
				\item implemented with conditional compilation
				\begin{itemize}
					\item build systems (e.g., KBuild)
					\item preprocessors (e.g., CPP)
				\end{itemize}
				\item feature traceability only implicit\\
					$\Rightarrow$ there is code scattering and tangling
			\end{itemize}
		}
		\mydefinition{Recap: Feature Mapping}{
			\begin{itemize}
				\item \todots
			\end{itemize}
		}
		\mynextcolumn
		\myexample{Asking Questions About the Feature Mapping}{
			\begin{itemize}
				\item Are there contradictory or unnecessary preprocessor annotations in the code?
				\item Is the code even included in any product?
				\item If so, in how many products is the code included?
				\item \ldots
			\end{itemize}
		}
		\myexampletight{Running Example}{
			\centering
			\featureDiagram{Graph,concrete[Node,concrete,mandatory[Labeled,concrete,optional][Colored,concrete,optional]][Edge,concrete,mandatory[Directed,concrete,optional][Undirected,concrete,optional][Hyper,concrete,optional]]}
			$\pnot (Directed \pand Undirected)$\\
			$Hyper \pimplies Undirected$\\
			$Directed \notleftright (Undirected \pand Hyper)$\\
			\todo{revise CTCs}
		}
	\end{mycolumns}
\end{frame}

% mit quadranten starten
%wir gucken uns jetzt Source Code an (evtl. linke zwei (VL 4, Problem Space) + rechte zwei Quadranten (VL 10+, Solution Space))
% wechselwirkung zw. sol und problem space (mapping+FM)

\subsection{Presence Conditions}

\begin{frame}[fragile]{\myframetitle}
	\begin{mycolumns}[columns=3,widths={40,23,37},animation=none]
		\mydefinition{Presence Condition}{
			A \emph{presence condition (PC)} for a code location (i.e., a line/chunk/file) is a formula that describes the circumstances under which the code location is included in a product.
		}
		\mynote{}{
			\begin{itemize}
				\item useful for implementation techniques with code scattering and tangling
				\item e.g., build systems (file PCs) or preprocessors (line/chunk PCs)
				\item here: C preprocessor
			\end{itemize}
		}
	\mynextcolumn
		\myexampletight{Presence Conditions}{
			\small
			\begin{flushright}
				$\top$\\
				$Labeled$\\
				$Labeled$\\
				$Labeled$\\
				$Colored$\\
				$Colored$\\
				$Colored$\\
				$\top$\\
				$\top$\\
				$\top$\\
				$Directed$\\
				$Directed$\\
				$\pnot Dir \pand Hyper$\\
				$\pnot Dir \pand Hy \pand Un$\\
				$\pnot Dir \pand Hy \pand Un$\\
				$\pnot Dir \pand Hy \pand \pnot Un \pand Dir$\\
				$\pnot Dir \pand Hy \pand \pnot Un \pand Dir$\\
				$\pnot Dir \pand Hy \pand \pnot Un \pand Dir$\\
				$\pnot Dir \pand \pnot Hy \pand \pnot Dir$\\
				$\pnot Dir \pand \pnot Hy \pand \pnot Dir$\\
				$\pnot Dir \pand \pnot Hy \pand \pnot Dir$\\
				$\top$
			\end{flushright}
		}
	\mynextcolumn
		\begin{cpptight}[basicstyle=\small]{Product-Line Implementation}
class Node {
#ifdef LABELED
	string label;
#endif
#ifdef COLORED
	string color;
#endif
};

class Edge {
#ifdef DIRECTED
	Node fromNode, toNode;
#elifdef HYPER
#ifdef UNDIRECTED
	set<Node> nodeSet;
#elifdef DIRECTED
	map<Node, set<Node>> nodeMap;
#endif
#elifndef DIRECTED
#error Unsupported edge type.
#endif
};
		\end{cpptight}
	\end{mycolumns}
\end{frame}

\subsection{Detecting Dead Code}

\begin{frame}[fragile]{\myframetitle}
	\begin{mycolumns}[columns=3,widths={40,23,37},animation=none]
		\mydefinition{Dead Code}{
			A line/chunk/file of code is \emph{dead} when

			\begin{itemize}
				\item no product includes it.
				\item or, equivalently:\\
					its presence condition $PC$ is contradictory (i.e., $PC \mequals \bot$).
			\end{itemize}
		}
		\mynote{}{
			calculated by querying a \emph{satisfiability solver} whether $PC$ is not satisfiable (i.e., $\pnot SAT(PC)$)
		}
		\mynote{What causes dead code?}{
			\begin{itemize}
				\item confusion due to nested \texttt{\#ifdef}
				\item domain modeling mistakes
				\item can be intended! \mysource{\href{https://dl.acm.org/doi/10.1145/3442391.3442406}{Hentze~et~al.~2021}}
			\end{itemize}
		}
	\mynextcolumn
		\myexampletight{Presence Conditions}{
			\small\vspace*{-1.7ex}
			\begin{flushright}
				{\color{gray}$\top$}\\
				{\color{gray}$Labeled$}\\
				{\color{gray}$Labeled$}\\
				{\color{gray}$Labeled$}\\
				{\color{gray}$Colored$}\\
				{\color{gray}$Colored$}\\
				{\color{gray}$Colored$}\\
				{\color{gray}$\top$}\\
				{\color{gray}$\top$}\\
				{\color{gray}$\top$}\\
				{\color{red}$Directed$}\\
				{\color{gray}$Directed$}\\
				{\color{red}$\pnot Dir \pand Hyper$}\\
				{\color{gray}$\pnot Dir \pand Hy \pand Un$}\\
				{\color{gray}$\pnot Dir \pand Hy \pand Un$}\\
				{\color{red}$\pnot Dir \pand Hy \pand \pnot Un \pand Dir$}\\
				{\color{red}$\pnot Dir \pand Hy \pand \pnot Un \pand Dir$}\\
				{\color{red}$\pnot Dir \pand Hy \pand \pnot Un \pand Dir$}\\
				{\color{gray}$\pnot Dir \pand \pnot Hy \pand \pnot Dir$}\\
				{\color{gray}$\pnot Dir \pand \pnot Hy \pand \pnot Dir$}\\
				{\color{gray}$\pnot Dir \pand \pnot Hy \pand \pnot Dir$}\\
				{\color{gray}$\top$}
			\end{flushright}
		}
	\mynextcolumn
		\begin{cpptight}[basicstyle=\small]{Product-Line Implementation}
class Node {
#ifdef LABELED
	string label;
#endif
#ifdef COLORED
	string color;
#endif
};

class Edge {
#ifdef DIRECTED
	Node fromNode, toNode;
#elifdef HYPER
#ifdef UNDIRECTED
	set<Node> nodeSet;
#elifdef DIRECTED
	@map<Node, set<Node>> nodeMap;@
#endif
#elifndef DIRECTED
#error Unsupported edge type.
#endif
};
		\end{cpptight}
	\end{mycolumns}
\end{frame}

% dead code paper (tartler beteiligt, erlangen evtl., TDS+:ATC14/EuroSys11) in linux? technik zum code auskommentieren, nicht alles ist ungewollt

\subsection{Detecting Superfluous Annotations}

\begin{frame}[fragile]{\myframetitle}
	\begin{mycolumns}[columns=3,widths={40,23,37},animation=none]
		\mydefinition{Superfluous Annotation}{
			An annotation is (partly) \emph{superfluous}

			\begin{itemize}
				\item when it can be omitted (or simplified) without consequences.
				\item or, equivalently:\\
					its presence condition $PC$ is contradictory (i.e., $PC \mequals \bot$).
			\end{itemize}
		}
		\mynote{}{
			calculated by querying a \emph{satisfiability solver} whether $PC$ is not satisfiable (i.e., $\pnot SAT(PC)$)
		}
		\mynote{What causes dead code?}{
			\begin{itemize}
				\item confusion due to nested \texttt{\#ifdef}
				\item domain modeling mistakes
				\item can be intended! \mysource{\href{https://dl.acm.org/doi/10.1145/3442391.3442406}{Hentze~et~al.~2021}}
			\end{itemize}
		}
	\mynextcolumn
		\myexampletight{Presence Conditions}{
			\small\vspace*{-1.7ex}
			\begin{flushright}
				{\color{gray}$\top$}\\
				{\color{gray}$Labeled$}\\
				{\color{gray}$Labeled$}\\
				{\color{gray}$Labeled$}\\
				{\color{gray}$Colored$}\\
				{\color{gray}$Colored$}\\
				{\color{gray}$Colored$}\\
				{\color{gray}$\top$}\\
				{\color{gray}$\top$}\\
				{\color{gray}$\top$}\\
				{\color{gray}$Directed$}\\
				{\color{gray}$Directed$}\\
				{\color{gray}$\pnot Dir \pand Hyper$}\\
				{\color{gray}$\pnot Dir \pand Hy \pand Un$}\\
				{\color{gray}$\pnot Dir \pand Hy \pand Un$}\\
				{\color{gray}$\pnot Dir \pand Hy \pand \pnot Un \pand Dir$}\\
				{\color{gray}$\pnot Dir \pand Hy \pand \pnot Un \pand Dir$}\\
				{\color{gray}$\pnot Dir \pand Hy \pand \pnot Un \pand Dir$}\\
				{\color{red}$\pnot Dir \pand \pnot Hy \pand \pnot Dir$}\\
				{\color{red}$\pnot Dir \pand \pnot Hy \pand \pnot Dir$}\\
				{\color{red}$\pnot Dir \pand \pnot Hy \pand \pnot Dir$}\\
				{\color{gray}$\top$}
			\end{flushright}
		}
	\mynextcolumn
		\begin{cpptight}[basicstyle=\small]{Product-Line Implementation}
class Node {
#ifdef LABELED
	string label;
#endif
#ifdef COLORED
	string color;
#endif
};

class Edge {
#ifdef DIRECTED
	Node fromNode, toNode;
#elifdef HYPER
#ifdef UNDIRECTED
	set<Node> nodeSet;
#elifdef DIRECTED
	map<Node, set<Node>> nodeMap;
#endif
#elifndef @DIRECTED@
#error Unsupported edge type.
#endif
};
		\end{cpptight}
	\end{mycolumns}
\end{frame}

\subsection{Considering the Feature Model}

%bezug auf FM-analyse
% + variability smells erwähnen

% show how linux in lecture 4+5 relates to problem space (kconfig)/mapping (kbuild)/solution space (cpp)
% maybe consider FM + KBuild? the whole chain?

\begin{frame}{\myframetitle}
	\begin{mycolumns}
	\mynextcolumn
	\end{mycolumns}
\end{frame}

\subsection{Feature-Mapping Analyses in FeatureIDE}

\begin{frame}{\myframetitle}
	\begin{mycolumns}
		\href{https://youtu.be/jVe7f32mLCQ?t=125}{demo video available} (minute 3 and 4): dead code block, superfluous annotations, generation of all products, error propagation, unit testing
	\mynextcolumn
		\mynote{Discussion}{
			\begin{itemize}
				\item only the feature mapping is analyzed, the source code is ignored
				\begin{itemize}
					\item pro: simple, language-independent
					\item con: can only find simple anomalies
				\end{itemize}
			\end{itemize}
		}
	\end{mycolumns}
\end{frame}
