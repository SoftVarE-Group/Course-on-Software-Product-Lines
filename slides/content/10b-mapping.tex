
\subsection{The Use of Feature-Mapping Analyses}

% ignoriert völlig Source-Code, nur Mapping, sprach-unabhängig, eher simpel umzusetzen
% kann nur einfache anomalien finden

% mit quadranten starten
%wir gucken uns jetzt Source Code an (evtl. linke zwei (VL 4, Problem Space) + rechte zwei Quadranten (VL 10+, Solution Space))
% wechselwirkung zw. sol und problem space (mapping+FM)

% show how linux in lecture 4+5 relates to problem space (kconfig)/mapping (kbuild)/solution space (cpp)

\subsection{Presence Conditions}

\begin{frame}{\myframetitle}
	\begin{mycolumns}
	\mynextcolumn
	\end{mycolumns}
\end{frame}

\subsection{Detecting Dead Code}

% dead code paper (tartler beteiligt, erlangen evtl., TDS+:ATC14/EuroSys11) in linux? technik zum code auskommentieren, nicht alles ist ungewollt

\begin{frame}{\myframetitle}
	\begin{mycolumns}
	\mynextcolumn
	\end{mycolumns}
\end{frame}

\subsection{Detecting Superfluous Annotations}

% Mapping + FM

\begin{frame}{\myframetitle}
	\begin{mycolumns}
		%general definition
	\mynextcolumn
		%small example
	\end{mycolumns}
\end{frame}

\subsection{Considering the Feature Model}

%bezug auf FM-analyse
% + variability smells erwähnen

\begin{frame}{\myframetitle}
	\begin{mycolumns}
	\mynextcolumn
	\end{mycolumns}
\end{frame}

\subsection{Feature-Mapping Analyses in FeatureIDE}

\begin{frame}{\myframetitle}
	\begin{mycolumns}
		\href{https://youtu.be/jVe7f32mLCQ?t=125}{demo video available} (minute 3 and 4): dead code block, superfluous annotations, generation of all products, error propagation, unit testing
	\mynextcolumn
	\end{mycolumns}
\end{frame}
