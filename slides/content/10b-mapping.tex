% mit quadranten starten
%bezug auf FM-analyse
%wir gucken uns jetzt Source Code an (evtl. linke zwei (VL 4, Problem Space) + rechte zwei Quadranten (VL 10+, Solution Space))
% wechselwirkung zw. sol und problem space (mapping+FM)




% show how linux in lecture 4+5 relates to problem space (kconfig)/mapping (kbuild)/solution space (cpp)

%ignoriert völlig Source-Code
% nur Mapping

\subsection{Presence Conditions}

\subsection{Dead Code} % family-based static analysis

% dead code paper (tartler beteiligt, erlangen evtl., TDS+:ATC14/EuroSys11) in linux? technik zum code auskommentieren, nicht alles ist ungewollt

\subsection{Superfluous Annotations}

% Mapping + FM

\subsection{Considering the Feature Model}

% + variability smells erwähnen

\subsection{Analyses in FeatureIDE}
% \href{https://youtu.be/jVe7f32mLCQ?t=125}{demo video available} (minute 3 and 4): dead code block, superfluous annotations, generation of all products, error propagation, unit testing
