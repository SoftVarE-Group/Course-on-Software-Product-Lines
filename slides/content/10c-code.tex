\subsection{Automated Analysis of Variable Code}

\begin{frame}{\myframetitle}
	\begin{mycolumns}[t,widths={46}]
		\begin{example}{Asking Questions About\\the Feature Mapping \ldots}
			\begin{itemize}
				\item Are there contradictory or unnecessary preprocessor annotations in the code?
				\item Is the code even included in any product?
				\item If so, in how many products is the code included?
				\item \ldots
			\end{itemize}
		\end{example}
		\begin{note}{}
			only finds code-agnostic anomalies
		\end{note}
		\mynextcolumn
		\begin{example}{\ldots\ and the Variable Code}
			\begin{itemize}
				\item Can every product be generated (e.g., compiled)?\\
					$\Rightarrow$ to find all \emph{syntax and type errors}
				\item Do all tests succeed for every product?\\
					$\Rightarrow$ to find some \emph{runtime and logic errors}
				\item Does every product adhere to its specification?\\
					$\Rightarrow$ to \emph{rule out} runtime and logic errors
				\item \ldots
			\end{itemize}
		\end{example}
		\begin{note}{}
			now: analyze (non-)functional properties of all products
		\end{note}
		\begin{example}{Today's Example}
			type checking for FOP and conditional compilation
		\end{example}
	\end{mycolumns}
\end{frame}

\subsection{Variability-Aware Type Checking}

\subsubsection{Analyzing Feature Modules}

\begin{frame}{\myframetitle}
	\begin{mycolumns}[widths={37,63},animation=none]
		\begin{exampletight}{}
			\featureDiagram{Store,abstract[Type,abstract,mandatory[SingleStore,concrete,alternative][MultiStore,concrete]][AccessControl,concrete,optional]}
		\end{exampletight}
		\begin{example}{Feature-Model Formula}
			\small
			$\Phi(FM) = Store \pand Type \pand (SS \por MS) \pand (\pnot SS \por \pnot MS)$
		\end{example}
		\begin{example}{Valid Configurations}
			\vspace*{-2ex}
			\small
			\begin{mycolumns}[animation=none]
				{\color{black}$\{SS\}$}\\
				$\{SS, AC\}$
			\mynextcolumn
				{\color{black}$\{MS\}$}\\
				$\{MS, AC\}$\\
			\end{mycolumns}
		\end{example}
		\uncover<3->{
			\begin{note}{}
				Is there a type error in \emph{any} product?
				\uncover<4->{What about $\{SS, AC\}$?}
			\end{note}
		}
	\mynextcolumn
		\only<2->{\pic[width=\linewidth]{fop-csur-example-error-blind}}
	\end{mycolumns}
\end{frame}

\begin{frame}{\myframetitle}
	\begin{mycolumns}[widths={37,63},animation=none]
		\begin{definition}{Reachability Condition of $id$}
			guarantees that a given \emph{reference} to $id$ is also \emph{defined} somewhere:\\[1ex]
			$\Phi(FM) \mimplies (PC_{ref}^{id} \pimplies \bigvee_{def} PC_{def}^{id})$\\[1ex]
			or, with a SAT solver:\\
			\small
			$\pnot SAT(\Phi(FM) \pand PC_{ref} \pand \bigwedge_{def} \pnot PC_{def})$
		\end{definition}
		\uncover<2->{
			\begin{example}{}
				$\Phi(FM) \mimplies (AC \pimplies SS \por MS)$ holds,\\
				$\Phi(FM) \mimplies (AC \pimplies MS)$ does not\\[1ex]
				$\Rightarrow$ $\{SS, AC\}$ has no \texttt{readAll}!
			\end{example}
		}
		\uncover<3->{
			\begin{definition}{Type-Safe Product-Line}
				in a \emph{type-safe} SPL, all references must always be defined (i.e., \emph{all} reachability conditions must hold)\\
				{\color{gray}\ldots\ and many more conditions \ldots}
			\end{definition}
		}
	\mynextcolumn
		\pic[width=\linewidth]{fop-csur-example-error}
	\end{mycolumns}
\end{frame}

\subsubsection{Analyzing Conditional Compilation}

\begin{frame}[fragile]{\myframetitle}
	\begin{mycolumns}[columns=3,widths={37,29,34},animation=none]
		\begin{exampletight}{}
			\centering
			\featureDiagram{Graph,concrete[Directed,concrete,optional][Undirected,concrete,optional][Hyper,concrete,optional]}
			{\color{blue}$\pnot (Directed \pand Undirected)$}\\
			{\color{green}$Hyper \pimplies Undirected$}\\
			{\color{red}$Directed \por Hyper$}
		\end{exampletight}
		\uncover<2->{
			\begin{definition}{Reachability Condition of $id$}
				$\Phi(FM) \mimplies (PC_{ref}^{id} \pimplies \bigvee_{def} PC_{def}^{id})$
			\end{definition}
		}
		\uncover<3->{
			\begin{definition}{Conflict Condition of $id$, def's $d_i$}
				guarantees that no definition of $id$ \emph{conflicts} with another:\\[1ex]
				$\Phi(FM) \mimplies \bigwedge_{d_1 \neq d_2} \pnot (PC_{d_1}^{id} \pand PC_{d_2}^{id}))$
			\end{definition}
		}
	\mynextcolumn
	\uncover<2->{
		\begin{example}{Is $e.nodes$ reachable?}
			\small
			$\Phi(FM) \mimplies (\top \pimplies$\\
			$~~Dir \por (Hy \pand Un) \por (Hy \pand Dir))$\\[1ex]
			holds, because each graph is {\color{red}directed} or an ({\color{green}undirected}) {\color{red}hypergraph}
		\end{example}
	}
		\uncover<3->{
			\begin{example}{Does $e.nodes$ conflict?}
				\small
				$\Phi(FM) \mimplies ($\\
				$~~\pnot (Dir \pand (Hy \pand Un))$\\
				$~\pand \pnot (Dir \pand (Hy \pand Dir))$\\
				$~~\pand \pnot ((Hy \pand Un) \pand (Hy \pand Dir)))$\\[1ex]
				holds, because a graph is {\color{blue}never} {\color{red}directed} and an ({\color{green}undirected}) {\color{red}hypergraph} {\color{blue}at the same time}
			\end{example}
		}
		\uncover<4->{
			\begin{example}{}
				all reachable, no conflicts
			\end{example}
		}
	\mynextcolumn
		\begin{cpptight}[basicstyle=\small]{\texttt{graph.cpp}}
class Node { ... };

class Edge {
#ifdef DIRECTED
	pair<Node, Node> nodes;
#endif
#ifdef HYPER
#ifdef UNDIRECTED
	set<Node> nodes;
#endif
#ifdef DIRECTED
	map<Node, set<Node>> nodes;
#endif
#endif
};

std::ostream& operator<<(
	std::ostream &s, const Edge &e) {
	return s << e.nodes;
}
		\end{cpptight}
	\end{mycolumns}
\end{frame}

\subsubsection{Discussion}

\begin{frame}{\myframetitle}
	\begin{mycolumns}
		\begin{note}{Just the Tip of the Iceberg}
			\begin{itemize}
				\item here, we only discussed reachability and conflict conditions
				\item but: actual type checking requires a table of all identifiers, their types, and their PCs (and a lot more SAT queries)
				\item the practical difficulty depends:
				\begin{itemize}
					\item FOP (due to superimposition)\\
						$\Rightarrow$ no conflict conditions required
					\item good feature traceability (e.g., FOP)\\
						$\Rightarrow$ trivial PCs, simpler implementation
					\item ignoring the feature model\\
						$\Rightarrow$ better performance (false positives!)
				\end{itemize}
			\end{itemize}
		\end{note}
	\mynextcolumn
		\begin{example}{The TypeChef Project \mysource{\typechef}}
			\begin{itemize}
				\item a variability-aware lexer, parser framework, and type system for C code with \texttt{\#ifdef}'s
				\item skips preprocessing, instead builds an abstract syntax tree (AST) annotated with presence conditions
				\item \href{https://ckaestne.github.io/TypeChef/typechef-poster.png}{poster} with examples
				\item does it scale?\\[1ex]
				{\small Busybox (811 features): \mycite{We need 57 minutes to type check all modules.}} \mysource{\href{https://dl.acm.org/doi/10.1145/2384616.2384673}{ref}}\\[1ex]
				{\small Linux (6065 features): \mycite{We successfully parsed [it in] roughly 85 hours on a single machine.}} \mysource{\href{https://dl.acm.org/doi/10.1145/2048066.2048128}{ref}} % not practically possible, can be used to motivate testing in next lecture
			\end{itemize}
		\end{example}
	\end{mycolumns}
\end{frame}

\subsection{Product-Line Analyses in the Wild}

\subsubsection*{Product-Line Complexity}

\begin{frame}{\myframetitle}
	\begin{mycolumns}[forget]
		\begin{definition}{Six Classes of Product-Line Complexity \mytitlesource{\href{https://youtu.be/qUuRp7_d0rU?t=1651}{Thüm~2021}}}
			In a timeframe of 24h \ldots
			\begin{enumerate}
				\item[\color{gray}\textbf{NC}] {\color{gray}Products cannot be generated automatically}
				\item[\textbf{C1}] All products can be generated and \emph{tested}
				\item[\textbf{C2}] Not C1, but all \emph{products} can be \emph{generated}
				\item[\textbf{C3}] Not C2, but all \emph{configurations} can be \emph{generated} (AllSAT)
				\item[\textbf{C4}] Not C3, but the \emph{number of valid configurations} can be computed (\ssat{})
				\item[\textbf{C5}] Not C4, but \emph{whether there is a valid configuration} can be computed (SAT)
				\item[\textbf{C6}] It cannot be computed whether there is a valid configuration
			\end{enumerate}
		\end{definition}
	\mynextcolumn
		\begin{example}{Examples}
			% also put examples of analysis strategies here, and which kind of mapping/code analysis scales for which class?
			\begin{enumerate}
				\item[\color{gray}\textbf{NC}] {\color{gray}all product lines with custom development in application engineering\\(e.g., components and services with glue code, white-box frameworks)}
				\item[\textbf{C1}] $< 2000$ products for 1 min per product
				\item[\textbf{C2}] $< 90000$ products for 1 s per product
				\item[\textbf{C3}] $< 10^{13}$ configurations for 1 ns per configuration
				\item[\textbf{C4}] older versions of Linux/Automotive05
				\item[\textbf{C5}] newer versions of Linux/Automotive05\\(see \evaluatingsharpsatsolvers)
				\item[\textbf{C6}] No example known
			\end{enumerate}
		\end{example}
	\end{mycolumns}
\end{frame}

\subsubsection*{Automated Analysis \ldots}

\begin{frame}{\myframetitle}
	\begin{mycolumns}[t,columns=3]
		\textbf{\lecturemodeling\partc}
		\begin{definition}{\ldots\ of Feature Models}
			analyze only the feature model
		\end{definition}
	\mynextcolumn
		\textbf{\lectureanalyses\partb}
		\begin{definition}{\ldots\ of Feature Mappings}
			analyze the feature mapping (considering the feature model)
		\end{definition}
	\mynextcolumn
		\textbf{\lectureanalyses\partc}
		\begin{definition}{\ldots\ of Variable Code}
			analyze the variable code (considering the feature model and feature mapping)
		\end{definition}
	\end{mycolumns}
	\begin{mycolumns}[t,columns=3]
		\begin{example}{}
			\begin{itemize}
				\item void, core/dead features
				\item decision propagation
				\item atomic sets, redundant constraints
				\item \ldots
			\end{itemize}
		\end{example}
	\mynextcolumn
		\begin{example}{}
			\begin{itemize}
				\item dead code
				\item superfluous annotations
				\item degree of code scattering and tangling
				\item \ldots
			\end{itemize}
		\end{example}
	\mynextcolumn
		\begin{example}{}
			\begin{itemize}
				\item parsing, type checking
				\item static analysis
				\item model checking, theorem proving
				\item \ldots
			\end{itemize}
		\end{example}
	\end{mycolumns}
	\begin{mycolumns}[t,columns=2,widths={32}]
	\mynextcolumn
		\begin{note}{}
			here: \emph{family-based} analysis strategies for \emph{conditional compilation} and \emph{feature-oriented programming}
		\end{note}
	\end{mycolumns}
\end{frame}