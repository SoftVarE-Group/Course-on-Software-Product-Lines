\subsection{Coverage in Single-System Engineering}
\begin{frame}{Recap: Coverage in White-Box Testing{} \deutschertitel{Testüberdeckung für Strukturtests} \mytitlesource{\ludewiglichter}}
	\begin{fancycolumns}
		\begin{definition}{White-Box Testing \deutsch{Strukturtest}}
			\begin{itemize}
%				\setlength\itemsep{.1em}
				\item inner structure of test object is used
				\item idea: coverage of structural elements
				\begin{itemize}
					\item code translated into control flow graph
					\item specific test case (concrete inputs)\\derived from logical test case (conditions)\\derived from path in control flow graph
				\end{itemize}	
			\end{itemize}
		\end{definition}
	\nextcolumn
		\begin{definition}{Coverage Criteria \deutsch{Überdeckungskriterien}}
			\begin{itemize}
%				\setlength\itemsep{.1em}
				\item[1.] \emph{statement coverage} \deutsch{Anweisungsüberdeck.}:\\all statements are executed for at least one test case
				\item<3->[2.] \emph{branching coverage} \deutsch{Zweigüberdeckung}: statement coverage and all branches of branching statement are executed %TODO not so easy to define as percentage
				\item<4->[3.] \emph{term coverage} \deutsch{Termüberdeckung}:\\branching coverage and all terms used in a branching statement ($n$) are combined exhaustively ($2^n$)\hfill(simplified)
				% TODO discuss path coverage?
			\end{itemize}
		\end{definition}
	\end{fancycolumns}
\end{frame}

\subsection{Coverage of Ifdef Blocks}
\begin{frame}[b]
	\mywhite{Can You Spot Problems in the Elevator Product Line?\mysource{\samplingsurvey}}{
		\centering\pic[width=.8\linewidth,page=3,trim=40 530 70 100,clip]{2018/2018-SPLC-Varshosaz}
	}
	\pause
	\begin{itemize}
		\item<+-> Line~29: compiler error (i.e., field \texttt{dreq} undefined) %for partial configuration $(\{FIFO\},\{DirectedCall\})$ / 
		when $FIFO \pand \pnot DirectedCall$
		\item<+-> Line~8: null pointer exception %for partial configuration $(\{DirectedCall\},\{FIFO\})$ / 
		when $DirectedCall \pand \pnot FIFO$
		\item<+-> both problems detectable with pairwise coverage, but presence conditions are more complicated in practice
		\item<+-> also: pairwise coverage often too much effort for large configuration spaces / continuous integration
	\end{itemize}
	\onslide<+->{\begin{definition}{\myframetitle{} \deutschertitel{Überdeckung von Ifdef-Blöcken} \mysource{\tartlerconfigurationcoverage}}
	\begin{itemize}
		\item every block selected for at least one configuration in the sample (cf.\ statement coverage)
	\end{itemize}
	\end{definition}}
\end{frame}

\subsection{Presence-Condition Coverage}
\begin{frame}{\myframetitle{} \deutschertitel{Presence-Condition-Überdeckung}}
	\begin{fancycolumns}[widths={48}]
		\begin{definition}{Presence-Condition Coverage\mysource{\krieterpresenceconditioncoverage}}
			\begin{itemize}
				\item application of t-wise interaction testing to presence conditions
				\item recap presence condition: formula specifying exactly those configurations under which a block is present
				\item t-wise presence condition coverage: every t-wise interaction of presence conditions is covered by at least one configuration in the sample
				\item $t=1$: every block is selected and also deselected (i.e., more than Tartler's coverage of ifdef blocks)
				\item $t=2$: every combination of two blocks covered
				\item $t=3$: every combination of three blocks covered
			\end{itemize}
		\end{definition}
	\nextcolumn
		\pause
		\begin{example}{{T=3 Presence-Condition Interactions}}
			for the blocks $a$, $b$, and $c$ with presence conditions $A$, $B$, and $C$:

			\begin{fancycolumns}[animation=none]
				$A \wedge B \wedge C$\\
				$A \wedge B \wedge \neg C$\\
				$A \wedge \neg B \wedge C$\\
				$A \wedge \neg B \wedge \neg C$
			\nextcolumn
				$\neg A \wedge B \wedge C$\\
				$\neg A \wedge B \wedge \neg C$\\
				$\neg A \wedge \neg B \wedge C$\\
				$\neg A \wedge \neg B \wedge \neg C$
			\end{fancycolumns}
		\end{example}
		\pause
		\begin{note}{Presence-Condition Coverage\mysource{\krieterpresenceconditioncoverage}}
			\begin{itemize}
				\item coverage of solution space (not problem space)
				\item aka.\ solution-space sampling
				\item for same t: often fewer configurations and similar effectiveness than feature interaction coverage
				\item also feasible by translating presence conditions into feature model \mysource{\hentzesolutionspacesampling}
			\end{itemize}
		\end{note}
	\end{fancycolumns}
\end{frame}

\subsection{Overview on Coverage Criteria}
\begin{frame}{\myframetitle{} \deutschertitel{Überblick über Überdeckungskriterien}}
	\begin{definition}{Techniques \& Coverage Criteria \mysource{\samplingsurvey}}
		\pic[width=\linewidth,page=4,trim=50 100 310 610,clip]{2018/2018-SPLC-Varshosaz}
	\end{definition}
\end{frame}

\subsection{Input for Sampling Algorithms}
\begin{frame}{\myframetitle{}}
	\begin{fancycolumns}
		\begin{definition}{Input Data \mysource{\samplingsurvey}}
			\pic[width=\linewidth,page=3,trim=350 430 90 310,clip]{2018/2018-SPLC-Varshosaz}
		\end{definition}
		\pause
		\begin{example}{Further Domain Knowledge \mysource{\samplingsurvey}}
			\begin{itemize}
				\item in addition to feature model
				\item e.g., configurations chosen by experts
				\item e.g., specialized feature model for sampling
			\end{itemize}
		\end{example}
	\nextcolumn
		\pause
		\vspace{-5mm}
		\begin{example}{Part 2: Combinatorial Interaction Testing}
			\begin{itemize}
				\item (Problem-Space Sampling)
				\item \emph{feature model} used to consider only valid configurations
			\end{itemize}
		\end{example}
		\pause
		\begin{example}{Part 3: Solution-Space Sampling}
			\begin{itemize}
				\item mapping from features to \emph{implementation artifacts}
				\item \emph{feature model} used to consider only valid configurations
			\end{itemize}
		\end{example}
		\pause
		\begin{example}{Combinatorial Reduction of Tests \mysource{\reducingconfigurations}}
			\begin{itemize}
				\item which configurations matter for each test?
				\item analyze \emph{unit tests} and \emph{impl. artifacts}
				\item \emph{feature model} used to consider only valid configurations
			\end{itemize}
		\end{example}
	\end{fancycolumns}
\end{frame}
