\subsection{Recap and Motivation}
\begin{frame}{\insertsubsection}
	\begin{mycolumns}[columns=3,widths={25,50},animation=none]
	\mynextcolumn
		\begin{note}{Jimmy Koppel, 2019\mysource{\href{https://corecursive.com/036-jimmy-koppel-advanced-software-design/}{corecursive.com}}}
			\mycite{Software maintenance is important because the world runs on software, and changing the world means changing the software.}
		\end{note}
	\mynextcolumn
	\end{mycolumns}
\end{frame}

\againframe<2>{ContinuingChangeAndGrowth}

\subsection{Recap: Evolution of the Linux Kernel}
\againframe<2>{CommitsOfLinux}
\againframe<2>{FeaturesOfLinux}
\againframe<2>{ProductsOfLinux}

\subsection{Comparison of Feature Models}
\begin{frame}{\myframetitle{} \mytitlesource{\reasoningfme}}
	\begin{mycolumns}
		\mywhite{}{\href{https://github.com/SoftVarE-Group/Papers/blob/main/2009/2009-ICSE-Thuem.pdf}{\pic[width=\linewidth,page=3,trim=70 200 330 460,clip]{2009/2009-ICSE-Thuem}}}
	\mynextcolumn
		\begin{note}{Advantages for Quality Assurance}
			assumption: only feature model has changed
			\begin{itemize}
				\item refactoring: no retest needed
				\item specialization: cannot produce new faults
				\item generalization: cannot fix known faults
				\item arbitrary edit: retest needed
			\end{itemize}
		\end{note}
	\end{mycolumns}
\end{frame}

\begin{frame}{\myframetitle{} \mytitlesource{...}}
	\begin{mycolumns}
		\begin{mycolumns}[t]
			\begin{example}{Version 1}
				\centering\featureDiagram{
					A,concrete
					[B,concrete,mandatory]
					[C,concrete,optional]
					[D,concrete,optional]
					[E,concrete,mandatory]
				}
				$B \pequals C \por D$
			\end{example}
		\mynextcolumn
			\begin{example}{Version 2}
				\pause\centering\featureDiagram{
					A,concrete
					[B,concrete,mandatory
						[C,concrete,or]
						[D,concrete]
					]
					[E,concrete,mandatory]
				}
			\end{example}
		\end{mycolumns}
		\begin{mycolumns}
			\begin{example}{Version 3}
				\pause\centering\featureDiagram{
					A,concrete
					[B,concrete,mandatory
						[C,concrete,or]
						[D,concrete]
					]
					[E,concrete,optional]
				}
			\end{example}
		\mynextcolumn
			\begin{example}{Version 4}
				\pause\centering\featureDiagram{
					A,concrete
					[B,concrete,mandatory
						[C,concrete,optional]
						[D,concrete,optional]
					]
					[E,concrete,mandatory]
				}
			\end{example}
		\end{mycolumns}
	\mynextcolumn
		\pause\begin{example}{Refactoring}
			1 to 2, 2 to 1
		\end{example}
		\pause\begin{example}{Generalization}
			1/2 to 3/4
		\end{example}
		\pause\begin{example}{Specialization}
			3/4 to 1/2
		\end{example}
		\pause\begin{example}{Arbitrary Edit}
			3 to 4, 4 to 3
		\end{example}
	\end{mycolumns}
\end{frame}

% TODO \subsection{Automated Classification}
% how to use a SAT solver for this
% derive formulas, from tautology to SAT query (mention laws on equivalent feature models?)
% only mention: added/removed features, abstract features, scalability issues

\subsection{Frequency of Feature Model Changes}
% in comparison to number of commits, how many commits change feature model, mapping, code (Hildesheim? Pett?)

\subsection{Outlook on Co-Evolution}
% outlook on related topics: co-evolution patterns (Fever, refinement theory), refactorings including artifacts, co-evolution of feature model and configurations (GuyDance)
% Sampling/Analysis/Testing under Evolution?
% Continuous Integration for Product Lines?
