\subsection{Recap and Motivation}
\begin{frame}{\insertsubsection}
	\begin{mycolumns}[columns=3,widths={25,50},animation=none]
	\mynextcolumn
		\begin{note}{Jimmy Koppel, 2019\mysource{\href{https://corecursive.com/036-jimmy-koppel-advanced-software-design/}{corecursive.com}}}
			\mycite{Software maintenance is important because the world runs on software, and changing the world means changing the software.}
		\end{note}
	\mynextcolumn
	\end{mycolumns}
\end{frame}

\againframe<2>{ContinuingChangeAndGrowth}

\subsection{Recap: Evolution of the Linux Kernel}
\againframe<2>{CommitsOfLinux}
\againframe<2>{FeaturesOfLinux}
\againframe<2>{ProductsOfLinux}

\subsection{Evolution of Feature Models}
\begin{frame}{\myframetitle{} \mytitlesource{\reasoningfme}}
	\begin{mycolumns}
		\mywhite{Classification of Changes}{\href{https://github.com/SoftVarE-Group/Papers/blob/main/2009/2009-ICSE-Thuem.pdf}{\pic[width=\linewidth,page=3,trim=70 200 330 460,clip]{2009/2009-ICSE-Thuem}}}
		\pause\begin{definition}{Automated Classification}
			classification can be reduced to SAT problems
		\end{definition}
	\mynextcolumn
		\pause\begin{note}{Advantages for Quality Assurance}
			assumption: only feature model has changed
			\begin{itemize}
				\item refactoring: no retest needed
				\item specialization: cannot produce new faults
				\item generalization: cannot fix known faults
				\item arbitrary edit: retest needed
			\end{itemize}
		\end{note}
		\pause\begin{note}{Advantages for Modelers}
			did the configuration space change as intended?
		\end{note}
	\end{mycolumns}
\end{frame}

\begin{frame}{\myframetitle{}}
	\begin{mycolumns}
		\begin{mycolumns}[t]
			\begin{example}{Version 1}
				\centering\featureDiagram{
					A,concrete
					[B,concrete,mandatory]
					[C,concrete,optional]
					[D,concrete,optional]
					[E,concrete,mandatory]
				}
				$B \pequals C \por D$
			\end{example}
		\mynextcolumn
			\pause\begin{example}{Version 2}
				\centering\featureDiagram{
					A,concrete
					[B,concrete,mandatory
						[C,concrete,or]
						[D,concrete]
					]
					[E,concrete,mandatory]
				}
			\end{example}
		\end{mycolumns}
		\begin{mycolumns}
			\pause\begin{example}{Version 3}
				\centering\featureDiagram{
					A,concrete
					[B,concrete,mandatory
						[C,concrete,or]
						[D,concrete]
					]
					[E,concrete,optional]
				}
			\end{example}
		\mynextcolumn
			\pause\begin{example}{Version 4}
				\centering\featureDiagram{
					A,concrete
					[B,concrete,mandatory
						[C,concrete,optional]
						[D,concrete,optional]
					]
					[E,concrete,mandatory]
				}
			\end{example}
		\end{mycolumns}
	\mynextcolumn
		\pause\begin{example}{Refactoring}
			1 to 2, 2 to 1
		\end{example}
		\pause\begin{example}{Generalization}
			1/2 to 3/4
		\end{example}
		\pause\begin{example}{Specialization}
			3/4 to 1/2
		\end{example}
		\pause\begin{example}{Arbitrary Edit}
			3 to 4, 4 to 3
		\end{example}
	\end{mycolumns}
\end{frame}

\subsubsection{Refactorings}
\newcommand{\addpattern}[1]{
	\mywhite{}{\pic[width=\linewidth,page=#1]{fmrefactoring}}
}
\begin{frame}{\myframetitle{} \mytitlesource{\fmrefactoring}}
	\begin{mycolumns}[columns=3,animation=none]
		\addpattern{1}
	\mynextcolumn
		\addpattern{2}
	\mynextcolumn
		\addpattern{3}
	\end{mycolumns}

	\begin{mycolumns}[columns=3,animation=none]
		\pause\addpattern{4}
	\mynextcolumn
		\addpattern{5}
	\mynextcolumn
		\centering\Huge\ldots
	\end{mycolumns}
\end{frame}

\subsubsection{Generalizations}
\begin{frame}{\myframetitle{} \mytitlesource{\fmrefactoring}}
	\begin{mycolumns}[columns=3,animation=none]
		\addpattern{6}
	\mynextcolumn
		\addpattern{7}
	\mynextcolumn
		\addpattern{8}
	\end{mycolumns}

	\begin{mycolumns}[columns=3,animation=none]
		\pause\addpattern{9}
	\mynextcolumn
		\addpattern{10}
	\mynextcolumn
		\addpattern{11}
	\end{mycolumns}
\end{frame}
\begin{frame}{\myframetitle{} \mytitlesource{\fmrefactoring}}
	\begin{mycolumns}[columns=3,animation=none]
		\addpattern{12}
	\mynextcolumn
		\addpattern{13}
	\mynextcolumn
		\addpattern{14}
	\end{mycolumns}

	\begin{mycolumns}[columns=4,widths={17,33,33},animation=none]
	\mynextcolumn
		\pause\addpattern{15}
	\mynextcolumn
		\centering\Huge\ldots
	\mynextcolumn
	\end{mycolumns}
\end{frame}

% TODO feature model diffing: by Acher
% TODO staged configuration: CHE:SPIP05staged
% TODO from diffs to edit operations: BKL+:AUSE15

% TODO \subsection{Automated Classification}
% how to use a SAT solver for this
% derive formulas, from tautology to SAT query (mention laws on equivalent feature models?)
% only mention: added/removed features, abstract features, scalability issues

% TODO \subsection{Frequency of Feature Model Changes}
% in comparison to number of commits, how many commits change feature model, mapping, code (Hildesheim? Pett?)
% 
\subsection{Co-Evolution of Product Lines}
\newcommand{\addevolution}[2]{
	\begin{mycolumns}[widths={67},animation=none]
		\mywhite{}{\pic[width=\linewidth,page=#1]{splevolution}}
	\mynextcolumn
		\begin{note}{}
			\begin{itemize}
				#2
			\end{itemize}
		\end{note}
	\end{mycolumns}
}
\begin{frame}{\myframetitle{} \mytitlesource{\splevolution}}
	\addevolution{1}{
		\item not only feature models evolve
		\item but also source code
		\item and mapping from features to source code
		\item aka.\ \emph{co-evolution}
		\item for conditional compilation:
			\begin{enumerate}
				\item feature model
				\item build scripts with features
				\item source code with preprocessor statements
			\end{enumerate}
		\item how frequent are those changes?
	}
\end{frame}
\begin{frame}{\myframetitle{} \mytitlesource{\splevolution}}
	\addevolution{2}{
		\item variability information evolves
		\item artifact-specific information evolves
		\item how frequent are those changes?
	}
\end{frame}
\begin{frame}{\myframetitle{} \mytitlesource{\splevolution}}
	\addevolution{3}{
		\item most changes only affect artifact-specific information
		\item fewest changes only affect variability information
	}
\end{frame}
\begin{frame}[label=CoEvolutionOfLinux]{\myframetitle{} \mytitlesource{\splevolution}}
	\addevolution{4}{
		\item 1\,\% of commits in Linux \emph{only} change feature model
		\item 4\,\% of commits in Linux change feature model\\~
		\item $<$1\,\% of commits in Linux \emph{only} change build scripts
		\item 3\,\% of commits in Linux change build scripts\\~
		\item 79\,\% of commits in Linux \emph{only} change source code
		\item 87\,\% of commits in Linux change source code\\~
		\item different percentages for other systems
	}
\end{frame}
%\begin{frame}{\myframetitle{} \mytitlesource{\splevolution}}
%	\addevolution{5}{
%		\item ...
%	}
%\end{frame}
%\begin{frame}{\myframetitle{} \mytitlesource{\splevolution}}
%	\addevolution{6}{
%		\item ...
%	}
%\end{frame}
%\begin{frame}{\myframetitle{} \mytitlesource{\splevolution}}
%	\addevolution{7}{
%		\item ...
%	}
%\end{frame}
% TODO there are more plots in this paper, but probably too much content for this lecture

% TODO Bittner's solution space classification

% TODO \subsection{Outlook on Co-Evolution}
% TODO refinement theory (BTG12) including code examples in Figure 3+4
% TODO refactoring including code and partial fefactorings (SBT:JSS19, Page 6+7 Figure 2+3)
% outlook on related topics: co-evolution patterns (Fever, refinement theory), refactorings including artifacts, partial refactorings, co-evolution of feature model and configurations (GuyDance)
% Sampling/Analysis/Testing under Evolution?
% Continuous Integration for Product Lines?
% regression testing

