\subsection{Recap on Software Maintenance}
\begin{frame}{\myframetitle\ \mytitlesource{\ludewiglichter}}
	\begin{fancycolumns}
		\begin{note}{Motivation}
			\begin{itemize}
				\item for software: no compensation of deterioration, repair, spare parts
				\item corrections (especially shortly after first delivery)
				\item modification and reconstruction
			\end{itemize}
		\end{note}
	\nextcolumn
		\begin{definition}{Operation and Maintenance Phase \mysource{\ieeesixten}}
			\mycite{The period of time in the software life cycle during which a software product is employed in its operational environment, monitored for satisfactory performance, and modified as necessary to correct problems or to respond to changing requirements.}
		\end{definition}
		\begin{definition}{Maintenance \mysource{\ieeesixten}}
			\mycite{The process of modifying a software system or component after delivery to correct faults, improve performance or other attributes, or adapt to a changed environment.}
		\end{definition}
	\end{fancycolumns}
\end{frame}

\begin{frame}{\myframetitle\ \mytitlesource{\ludewiglichter}}
	\begin{fancycolumns}[t]
		\begin{definition}{Evolution}
			\begin{itemize}
				\item new or removed functionality
				\item larger changes
				\item often foreseen changes
				\item results in upgrades, service packs, or cumulative updates
			\end{itemize}
		\end{definition}
		\begin{example}{Minor Release}
			new minor version: 2.3.1 $\Rightarrow$ 2.4.0
		\end{example}
		\begin{example}{Major Release}
			new major version: 2.3.1 $\Rightarrow$ 3.0.0
		\end{example}
	\nextcolumn
		\begin{definition}{Maintenance}
			\begin{itemize}
				\item mostly corrections
				\item smaller changes
				\item often unforeseen changes
				\item results in patches and hot fixes
			\end{itemize}
		\end{definition}
		\begin{example}{Patch Release}
			new patch version: 2.3.1 $\Rightarrow$ 2.3.2
		\end{example}
	\end{fancycolumns}
	\pause\pause\begin{note}{}
		\centering not easy to distinguish -- there is a continuum between evolution and maintenance
	\end{note}
\end{frame}

\subsection{Kinds of Maintenance}
\begin{frame}{\myframetitle\ \mytitlesource{\ludewiglichter}}
	\begin{fancycolumns}
		\begin{definition}{Adaptive Maintenance \mysource{\lientzswanson}}
			\mycite{Software maintenance performed to make a computer program usable in a changed environment.} \hfill \deutsch{adaptive Wartung}
		\end{definition}
		\begin{example}{}
			desktop application for a new version of an operating system (e.g., from Windows 10 to 11)
		\end{example}
		\begin{definition}{Corrective Maintenance \mysource{\lientzswanson}}
			\mycite{Maintenance performed to correct faults in software.} \hfill \deutsch{korrektive Wartung}
		\end{definition}
		\begin{example}{}
			feature interaction of Lenovo products fixed with BIOS update
		\end{example}
	\nextcolumn
		% TODO from SE course: it is really annoying that this category has an overlap with preventive maintenance, as better maintainability typically also results in fewer problems
		\begin{definition}{Perfective Maintenance \mysource{\lientzswanson}}
			\mycite{Software maintenance performed to improve the performance, maintainability, or other attributes of a computer program.} \hfill \deutsch{perfektive Wartung}
		\end{definition}
		\begin{example}{}
			improve start-up time of the Linux kernel
		\end{example}
		\begin{definition}{Preventive Maintenance \mysource{\lientzswanson}}
			\mycite{Maintenance performed for the purpose of preventing problems before they occur.} \hfill \deutsch{präventive Wartung}
		\end{definition}
		\begin{example}{}
			code audit of car software before next leap second
		\end{example}
	\end{fancycolumns}
\end{frame}

\subsection{Recap on Feature Model Transformations}
\againframe<2>{FeatureModelTransformations}

\subsection{Reengineering Tasks}
\begin{frame}{\insertsubsection\ \mytitlesource{\ludewiglichter}}
	\begin{fancycolumns}
		\begin{definition}{Reverse Engineering \mysource{Chikofsky und Cross}} % TODO link to literature missing
			\mycite{Reverse engineering is the process of analyzing a system to identify the system’s components and their interrelationships and create representations of the system in another form or at a higher level of abstraction.}
		\end{definition}
		\begin{example}{}
			create feature model from existing configurations
		\end{example}
		\begin{definition}{Forward Engineering \mysource{Chikofsky und Cross}}
			\mycite{Forward engineering is the traditional process of moving from high-level abstractions and logical, implementation-independent designs to the physical implementation of a system.}
		\end{definition}
		\begin{example}{}
			domain engineering \mysource{\lectureprocess\parta}
		\end{example}
	\nextcolumn
		\vspace{-3.5mm}
		\begin{definition}{Refactoring \mysource{Chikofsky und Cross}}
			\mycite{Refactoring is a transformation from one form of representation to another at the same relative level of abstraction. The new representation is meant to preserve the semantics and external behavior of the original.}
		\end{definition}
		\begin{example}{}
			feature-model refactorings \mysource{\lectureevonance\parta}
		\end{example}
		\begin{definition}{Reengineering \mysource{Chikofsky und Cross}}
			\mycite{Reengineering is the examination and alteration of a subject system to reconstitute it in a new form and the subsequent implementation of the new form.}
		\end{definition}
		\begin{example}{}
			combination of reverse engineering, refactoring, and forward engineering
		\end{example}
	\end{fancycolumns}
\end{frame}

% TODO \subsection{Refactoring of Product Lines}

% TODO \subsection{Reverse Engineering of Product Lines}

% TODO \subsection{Reengineering of Product Lines}
% TODO Fenske's reengineering classification

%\subsection{Outlook on Reverse Engineering Solution Space}
% Migration to New Implementation Technique?
% some transitions can be automated, others not, use for taxonomy in continuum between clone-and-own and product lines?
% TODO managed clone-and-own? VariantSync?
% clones levels?
