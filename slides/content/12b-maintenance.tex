\begin{frame}{\insertsubsection\ \mytitlesource{\ludewiglichter}}
	\begin{mycolumns}[t]
		\begin{note}{Evolution}
			\begin{itemize}
				\item new or removed functionality
				\item larger changes
				\item often foreseen changes
				\item results in upgrades, service packs, or cumulative updates
			\end{itemize}
		\end{note}
		\begin{example}{Minor Release}
			new minor version: 2.3.1 $\Rightarrow$ 2.4.0
		\end{example}
		\begin{example}{Major Release}
			new major version: 2.3.1 $\Rightarrow$ 3.0.0
		\end{example}
	\mynextcolumn
		\begin{note}{Maintenance}
			\begin{itemize}
				\item mostly corrections
				\item smaller changes
				\item often unforeseen changes
				\item results in patches and hot fixes
			\end{itemize}
		\end{note}
		\begin{example}{Patch Release}
			new patch version: 2.3.1 $\Rightarrow$ 2.3.2
		\end{example}
	\end{mycolumns}
\end{frame}



%\subsection{Software Maintenance}
%\begin{frame}{\insertsubsection\ \mytitlesource{\ludewiglichter}}
%	\begin{mycolumns}
%		\begin{note}{Motivation}
%			\begin{itemize}
%				\item for software: no compensation of deterioration, repair, spare parts
%				\item corrections (especially shortly after first delivery)
%				\item modification and reconstruction
%			\end{itemize}
%		\end{note}
%	\mynextcolumn
%		\begin{definition}{Operation and Maintenance Phase \mysource{\ieeesixten}}
%			\mycite{The period of time in the software life cycle during which a software product is employed in its operational environment, monitored for satisfactory performance, and modified as necessary to correct problems or to respond to changing requirements.}
%		\end{definition}
%		\begin{definition}{Maintenance \mysource{\ieeesixten}}
%			\mycite{The process of modifying a software system or component after delivery to correct faults, improve performance or other attributes, or adapt to a changed environment.}
%		\end{definition}
%	\end{mycolumns}
%\end{frame}
% TODO Operations

%\subsection{Kinds of Maintenance}
%\begin{frame}{\insertsubsection\ \mytitlesource{\ludewiglichter}}
%	\begin{mycolumns}
%		\begin{definition}{Adaptive Maintenance \mysource{\lientzswanson}}
%			\mycite{Software maintenance performed to make a computer program usable in a changed environment.} \hfill \deutsch{adaptive Wartung}
%		\end{definition}
%		\begin{example}{}
%			desktop application for a new version of an operating system (e.g., from Windows 8.1 to 10)
%		\end{example}
%		\begin{definition}{Corrective Maintenance \mysource{\lientzswanson}}
%			\mycite{Maintenance performed to correct faults in software.} \hfill \deutsch{korrektive Wartung}
%		\end{definition}
%		\begin{example}{}
%			Windows calculator showing wrong formulas
%		\end{example}
%	\mynextcolumn
%		% TODO it is really annoying that this category has an overlap with preventive maintenance, as better maintainability typically also results in fewer problems
%		\begin{definition}{Perfective Maintenance \mysource{\lientzswanson}}
%			\mycite{Software maintenance performed to improve the performance, maintainability, or other attributes of a computer program.} \hfill \deutsch{perfektive Wartung}
%		\end{definition}
%		\begin{example}{}
%			better handling of very large files in a text editor
%		\end{example}
%		\begin{definition}{Preventive Maintenance \mysource{\lientzswanson}}
%			\mycite{Maintenance performed for the purpose of preventing problems before they occur.} \hfill \deutsch{präventive Wartung}
%		\end{definition}
%		\begin{example}{}
%			2,000 year problem, leap seconds/years
%		\end{example}
%	\end{mycolumns}
%\end{frame}

%\subsection{Reengineering Tasks}
%\begin{frame}{\insertsubsection\ \mytitlesource{\ludewiglichter}}
%	\begin{mycolumns}
%		\begin{definition}{Reverse Engineering \mysource{Chikofsky und Cross}}
%			\mycite{Reverse engineering is the process of analyzing a system to identify the system’s components and their interrelationships and create representations of the system in another form or at a higher level of abstraction.}
%		\end{definition}
%		\begin{example}{}
%			updating UML diagrams from source code
%		\end{example}
%		\begin{definition}{Forward Engineering \mysource{Chikofsky und Cross}}
%			\mycite{Forward engineering is the traditional process of moving from high-level abstractions and logical, implementation-independent designs to the physical implementation of a system.}
%		\end{definition}
%		\begin{example}{}
%			see Software Engineering I
%		\end{example}
%	\mynextcolumn
%		\vspace{-3.5mm}
%		\begin{definition}{Restructuring \mysource{Chikofsky und Cross}}
%			\mycite{Restructuring is a transformation from one form of representation to another at the same relative level of abstraction. The new representation is meant to preserve the semantics and external behavior of the original.}
%		\end{definition}
%		\begin{example}{}
%			refactorings as introduced in lecture on evolution
%		\end{example}
%		\begin{definition}{Reengineering \mysource{Chikofsky und Cross}}
%			\mycite{Reengineering is the examination and alteration of a subject system to reconstitute it in a new form and the subsequent implementation of the new form.}
%		\end{definition}
%		\begin{example}{}
%			combination of reverse engineering, refactoring, and forward engineering
%		\end{example}
%	\end{mycolumns}
%\end{frame}


\subsection{Reengineering and Reverse Engineering}
\subsection{Recap: Adoption Strategies}
\subsection{Reverse Engineering from Configurations}
\subsection{Reverse Engineering from Formulas}
\subsection{Outlook on Reverse Engineering Solution Space}
% Migration to New Implementation Technique?
% some transitions can be automated, others not, use for taxonomy in continuum between clone-and-own and product lines?
% TODO managed clone-and-own? VariantSync?
% clones levels?
% Fenske's reengineering classification
