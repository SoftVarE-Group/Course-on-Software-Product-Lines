% this section should teach all usual FM/cfg analyses - some in more detail, some only high-level (other SPL analyses come in analyses.tex)

% revisit what the class of programs named "solver" does, and why it is important that we use OTS solvers and profit from their optimizations/contests - idea: reduce practical problems to solvers (sim. to TheoInf/Logics)
% only with Boolean formulas can we get really performant results today

% extensions to simple SAT solvers (maybe explain when needed):
% - when SAT = true, some give a satisfying assignment (for optimizations)
% - wen SAT = false, some give a MUS (for explanations)
% - some can count, called sharp SAT (or keep them completely separated from SAT?) (for model counting)

% ask questions:
% am i developing unused code?
% how to find out whether a partial configuration is valid? (show decision propagation in FeatureIDE)

% for each analysis, also list the sharp SAT encoding

% as interaction at the end: how can we leverage when a SAT solver also gives us a sat-assignment?

% proposal (explain with one running example):

\subsection{Questions on Feature Models}

list of all questions

\subsection{SAT and \#SAT Solver}
explain simple SAT solver ($SAT(\phi) \in \{\top, \bot\}$)

as a black-box/abstraction (may be implemented as BDD)

\begin{frame}{-- Minimum Unsatisfiable Set}
    cannot be answered (easily) with simple SAT solver

    explain extended SAT-MUS solver
\end{frame}

\begin{frame}{-- \#SAT Solver}
    ...
\end{frame}

\subsection{Void Feature Model}
% can be combined on one slide, possibly
\begin{frame}{-- Problem}
    is the FM even consistent? does it have errors? % can we even get a valid configuration? does a configurator allow anything? a clever configurator would see that
\end{frame}
\begin{frame}{-- Solution}
    void iff $SAT(\phi) = \bot$
\end{frame}

\subsection{Core and Dead Features}
\begin{frame}{-- Problem}
    can a feature be chosen at all? is it false-optional?
\end{frame}
\begin{frame}{-- Solution}
    f core iff $SAT(\phi and not f) = \bot$
    
    f dead iff $SAT(\phi and f) = \bot$
\end{frame}

\subsection{Partial Configurations}
\begin{frame}{-- Problem}
    can a partial configuration be completed?
    \end{frame}
\begin{frame}{-- Solution}
    note that this is a generalization of void/core/dead
\end{frame}

\subsection{Redundant Constraints}
\begin{frame}{-- Problem}
    ...
\end{frame}
\begin{frame}{-- Solution}
    ...
\end{frame}

\subsection{Explanations for Inconsistencies} % Thomas: always explain this with the analysis so we can show a graphical example? (show an example of an inconsistency and then reveal the issue)
\begin{frame}{-- Problem}
    WHY is a FM inconsistent or a feature core/dead?
\end{frame}
\begin{frame}{-- Solution}
    ...
\end{frame}

\subsection{Cardinalities}
\begin{frame}{-- Problem}
    to WHICH DEGREE is a FM consistent (i.e., degree of freedom / variability factor) or a feature core/dead?

    explain FM cardinality (extension of void), feature cardinality (extension of dead) and partial cfg cardinality (ext. of partial cfg)
\end{frame}
\begin{frame}{-- Solution}
    ...
\end{frame}

\subsection{Other Analyses}
\begin{frame}{~}
    list some more analyses/questions

    atomic sets, determinate features
\end{frame}

\subsection{Challenges}
\begin{frame}{~}
    how is SAT (before: a black box) implemented?

    solvers have parameters, heuristics, ...

    there are SAT solvers that do not use CNF

    CNF/DIMACS
\end{frame}

\subsection{Experiences} % performance / efficiency
\begin{frame}{~}
    show a few diagrams with performance on large models

    (may even be implemented concretely as BDD)
\end{frame}

\subsection{Summary}
\begin{frame}{~}
    make a table: each row is an analysis and each column says how to encode it logically

    maybe state that sharp SAT $>$ 0 implies SAT, SAT-MUS iff SAT
\end{frame}

\subsection{Interaction}
\begin{frame}{~}
    then, as an interaction, explain what SAT (with example solution) is and ask how it can be used

    another interaction: given a feature diagram, name all inconsistencies you can find (but maybe this is more of an exercise)

    or, demonstrate FeatureIDE
\end{frame}

% students are taught that SAT* solvers are tools for answering questions, and that if they support more features, more/faster analyses are possible
% master theses T. Günther, S. Ananieva

- old structure -

\subsection{Automated Reasoning}

solver!

SAT:
* DIMACS / CNF
* sat: example solution
* unsat: MUS (explanations)

SMT? \#SAT?

demo?

\subsection{Void Feature Model}

slide on SAT solving

including explanations

\subsection{Core and Dead Features}

\subsection{Validity of Configurations}

partial configurations

%\subsection{Other Analyses}
%
%e.g., partial configurations, model counting
%
%other questions about feature models

\subsection{Tool Support}

FeatureIDE configurator

show how FeatureIDE automatically detects the anomalies from the beginning

%\subsection{Edits to Feature Models}
% put this in a evolution + maintenance chapter

