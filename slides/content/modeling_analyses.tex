\subsection{Asking Questions About Feature Models}

% do everything with one running example (preferably the one from before)
% ie, show an example of an inconsistency and then reveal the issue - can be combined with the MUS explanations

\begin{frame}{~}
    this section should teach all usual FM/cfg analyses - some in more detail, some only high-level (other SPL analyses come in analyses.tex)

    list questions we might want to ask a feature model (maybe grouped in the order in which they are discussed)

    include some simple \#SAT questions (e.g., feature prioritization)

    am i developing unused code?

    how to find out whether a partial configuration is valid? (show decision propagation in FeatureIDE)

    etc.
\end{frame}

\subsection{Automated Reasoning with SAT and \#SAT}

\begin{frame}{-- Problem}
    recapitulate (very basic) theory behind both problems

    revisit what the class of programs named "solver" does, and why it is important that we use OTS solvers and profit from their optimizations/contests - idea: reduce practical problems to solvers (sim. to TheoInf/Logics)
\end{frame}

\begin{frame}{-- Solvers}
    explain simple SAT and \#SAT solver ($SAT(\phi) \in \{\top, \bot\}$)

    as a black-box/abstraction (may be implemented in many ways, e.g. as BDD)

    extensions to SAT solver interface (harder to implement):
    
    when SAT = true, some give a satisfying assignment (for optimizations)
    
    wen SAT = false, some give a MUS (for explanations) % see master theses of T. Günther, S. Ananieva

    \#SAT subsumes SAT, but is harder to compute
\end{frame}

\subsection{Void Feature Model}
\begin{frame}{~}
    is the FM even consistent? does it have errors? % can we even get a valid configuration? does a configurator allow anything? a clever configurator would see that

    void iff $SAT(\phi) = \bot$

    for each analysis, also list \#SAT encoding
    to WHICH DEGREE is a FM consistent (i.e., degree of freedom / variability factor)?

    also list explanations / MUS (very shortly)

    maybe for each analysis, list applications/scenarios, here: find grave modeling errors, check wether a configurator even allows to create ANY valid configuration
\end{frame}

\subsection{Core and Dead Features} % + these are variability smells
\begin{frame}{~}
    can a feature be chosen at all? is it false-optional?
    to WHICH DEGREE is a feature core/dead? feature prioritization
   
    f core iff $SAT(\phi and not f) = \bot$
    
    f dead iff $SAT(\phi and f) = \bot$

    applications: find anomalies/inconsistencies (false-optional), commonality, feature prioritization
\end{frame}

\subsection{Partial Configurations}
\begin{frame}{~}
    can a partial configuration be completed?
    can also be connected to Chico's \#SAT applications

    note that this is a generalization of void/core/dead

    total configurations were defined before, now expand this definition to a tuple $(sel, desel)$ and explain why

    applications: ...
\end{frame}

\subsection{Redundant Constraints (?)}
\begin{frame}{~}
    ... % include this? it has no real #SAT equivalent
\end{frame}

\subsection{Other Analyses}
\begin{frame}{~}
    list some more analyses/questions

    atomic sets, determinate features

    probably too much:
    FM edits?
    redundant constraints?
\end{frame}

% maybe combine Challenges/Experiences? or flip them?
\subsection{Challenges}
\begin{frame}{~}
    how is SAT (before: a black box) implemented (very broadly)?

    solvers have parameters, heuristics (variable ordering/assignment, restarts, ...)

    there are SAT solvers that do not use CNF, there are other techniques entirely (BDD, d-DNNF)

    CNF/DIMACS transformation

    non-Boolean requires richer theories (SMT, CSP)
\end{frame}

\subsection{Experiences} % performance / efficiency
\begin{frame}{~}
    show a few diagrams with performance on large models

    (may even be implemented concretely as BDD)

    only with Boolean formulas can we get really performant results today

    BDDs are hard to build, but have large payoff (maybe one slide on the abstract concept of knowledge compilation, without too many details?)

    most of this is implemented in FeatureIDE, the configurator is a clever combination of the techniques explained above (Boolean constraint/decision propagation)
\end{frame}