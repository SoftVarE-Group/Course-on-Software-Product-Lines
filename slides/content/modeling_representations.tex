this notation is already nice for communication, but semantics matter (for large models, it does not suffice to look sharply)

this section shall teach the relationship between FMs and formulas and FMs and sets (i.e., Damiani 2020, Batory 2005), so: FM semantics
also, (valid) total configurations should be explained here (how a computer can check them, this can be checked easily when an FM is encoded eg as runtime variability, but all other SAT-based questions are hard to answer)

first explain product lists / sets and configuration validity (ie, set membership):
$ext-sem (FM) = \{ C \mid C in 2^F | C satisfies all FM rules \}$
this is a nice (readable) semantics but impractical to check, so we need Formulas


at the end (what else is there?): in practice, we also have non-Boolean features/attributes/constraints over attributes (more details on efficiency in third block)





\subsection{Feature Diagrams}

show running example as a feature diagram with cross-tree-constraints

explain notation

lego example?

discuss pros/cons

\subsection{Propositional Formulas}

why is this needed? (forward ref?)

show running example as a formula

explain intuition behind elements of formula

CNF

\subsection{Transforming Diagrams into Formulas}

formal algorithm for transformation into FOL (and then CNF?)

+ example

% CNF is a universal language for saving formulas, maybe explain it here?

\subsection{Other Representations}

list of products, excel sheet, no explicit model (in motivation?), grammars, \dots

variations of feature models (e.g. cardinalities, non-boolean)

discuss pros/cons

Linux? KConfig? tri-state features

pure::variants?

%(state BDD?)
% probably not - (knowledge compilation: there are many nuances between CNF and BDD, maybe discuss?)