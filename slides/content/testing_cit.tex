\subsection{Test-Case Design in Single-System Engineering}
\begin{frame}{Recap: Test-Case Design \deutschertitel{Testfallentwurf}}
	\leftorright{
		\uncover<1>{\mydefinition{Systematic Test \mysource{\ludewiglichter}}{A systematic test is a test, in which
			\begin{itemize}
				\setlength\itemsep{.1em}
				\item[1.] the setup is defined,
				\item[2.] the inputs are chosen systematically,
				\item[3.] the results are documented and evaluated by criteria being defined prior to the test. 
			\end{itemize}
		}{}}
		\uncover<2>{\mydefinition{Test Case \mysource{\ludewiglichter}}{In a test, a number of test cases are executed, whereas each test case consists \emph{input values} for a single execution and \emph{expected outputs}. An \emph{exhaustive test} refers a test in which the test cases exercise all the possible inputs.}{}}
	}{
		\mynote{Goal \mysource{\ludewiglichter}}{Detect a large number of failures with a low number of test cases. A test case (execution) is \emph{positive}, if it detects a failure, and \emph{successful} if it detects an unknown failure.}
		\mydefinition{An ideal test case is \ldots \mysource{\ludewiglichter}}{
			\begin{itemize}
				\setlength\itemsep{.1em}
				\item representative: represents a large number of feasible test cases
				\item failure sensitive: has a high probability to detect a failure
				\item non-redundant: does not check what other test cases already check
			\end{itemize}
		}{}
	}
\end{frame}

\begin{frame}{Recap: Black-Box Testing \deutschertitel{Funktionstest}}
	\leftandright{
		\mynote{Motivation \mysource{\ludewiglichter}}{
			\begin{itemize}
				\setlength\itemsep{.1em}
				\item source code not always available (e.g., outsourced components, obfuscated code)
				\item specific test cases derived from logical ones using arbitrary values
				\item specification not incorporated so far (only for expected results)
				\item invalid inputs not tested
				\item errors are not equally distributed
			\end{itemize}
		}
		\vspace{-1mm}
		\mydefinition{Black-Box Testing \mysource{\ludewiglichter}}{
			\begin{itemize}
				\setlength\itemsep{.1em}
				\item test-case design based on specification
				\item source code and its inner structure is ignored (assumed as a black-box)
			\end{itemize}
		}{}
	}{
		\todo{discuss difference between test cases and sample configurations}
		
		\todo{table with chances how to combine white-box and black-box}
	}
\end{frame}

\subsection{Pairwise Interaction Testing}
\begin{frame}{\insertsubsection}
	\leftorright{
		\mydefinition{Pairwise Interaction Testing}{
			\begin{itemize}
				\setlength\itemsep{.5em}
				\item test a sample set $S \subseteq C$ of all valid configurations $C$ with pairwise coverage
				\item every pairwise interaction is covered by at least one configuration in the sample $S$
			\end{itemize}
		}
		\myexample{Configurations with the Interaction Get $\wedge$ Put}{
			\footnotesize
			\leftandright{
				$\{B,G,W\}$\\
				$\{B,P,W\}$\\
				\emph{$\{B,G,P,W\}$}\\
				$\{B,D,W\}$\\
				$\{B,G,D,W\}$\\
				$\{B,P,D,W\}$\\
				\eemph{$\{B,G,P,D,W\}$}\\
				$\{B,P,T,W\}$\\
				\eemph{$\{B,G,P,T,W\}$}\\
				$\{B,D,T,W\}$\\
				$\{B,G,D,T,W\}$\\
				$\{B,P,D,T,W\}$\\
				\eemph{$\{B,G,P,D,T,W\}$}
			}{
				$\{B,G,U\}$\\
				$\{B,P,U\}$\\
				\eemph{$\{B,G,P,U\}$}\\
				$\{B,D,U\}$\\
				$\{B,G,D,U\}$\\
				$\{B,P,D,U\}$\\
				\eemph{$\{B,G,P,D,U\}$}\\
				$\{B,P,T,U\}$\\
				\eemph{$\{B,G,P,T,U\}$}\\
				$\{B,D,T,U\}$\\
				$\{B,G,D,T,U\}$\\
				$\{B,P,D,T,U\}$\\
				\eemph{$\{B,G,P,D,T,U\}$}
			}
		}
	}{
		\mynote{Discussion}{
			\begin{itemize}
				\setlength\itemsep{.4em}
				\item applicable to large product lines
				\item reduced redundant effort compared to testing all configurations
				\item coverage guarantee (opposed to random configurations)
				\item still requires good test cases (program inputs)
				\item hard to compute small sample sets
			\end{itemize}
		}
		\mydefinition{Pairwise Interactions}{
			\begin{itemize}
				\setlength\itemsep{.5em}
				\item \emph{up-to} four interactions between $A$ and $B$
				\item both selected: $A \wedge B$
				\item one selected: $\neg A \wedge B$, $A \wedge \neg B$
				\item none selected: $\neg A \wedge \neg B$
			\end{itemize}
		}
	}
\end{frame}

\newcommand{\pair}[2]{$#1 \wedge #2$ & $#1 \wedge \neg #2$ & $\neg #1 \wedge #2$ & $\neg #1 \wedge \neg #2$\\}
\newcommand{\redandgray}[1]{\only<#1-| handout:#1->{\color{black}}\only<#1| handout:#1>{\color{blue}}}
\newcommand{\epair}[6]{
	{\redandgray{#3}$#1 \wedge #2$} & 
	{\redandgray{#4}$#1 \wedge \neg #2$} & 
	{\redandgray{#5}$\neg #1 \wedge #2$} & 
	{\redandgray{#6}$\neg #1 \wedge \neg #2$}\\
}

\begin{frame}{Pairwise Coverage}
	\leftorright{
		\vspace{5mm}
		\mydefinition{Interactions to Cover}{
			\begin{itemize}
				\setlength\itemsep{.5em}
				\item exclude invalid combinations (e.g., $W \wedge U$)
				\item exclude abstract features (e.g., $API$)
				\item exclude features contained in every configuration (e.g., $B$)
			\end{itemize}
			\todo{add formal definitions based on \lecturemodeling}
		}

		\centering\featureDiagramConfigurableDatabase
	}{
		\vspace{-5mm}
		\myexample{Pairwise Interactions}{
			\centering\footnotesize\color{lightgray}
			\begin{tabular}{llll}
				\epair{G}{P}{3}{2}{1}{6}
				\epair{G}{D}{2}{3}{1}{5}
				\epair{G}{T}{3}{2}{1}{5}
				\epair{G}{W}{4}{2}{1}{6}
				\epair{G}{U}{2}{4}{6}{1}
				\epair{P}{D}{1}{3}{2}{4}
				\epair{P}{T}{1}{5}{6}{2}
				\epair{P}{W}{1}{3}{4}{2}
				\epair{P}{U}{3}{1}{2}{4}
				\epair{D}{T}{1}{2}{3}{4}
				\epair{D}{W}{1}{2}{4}{3}
				\epair{D}{U}{2}{1}{3}{4}
				\epair{T}{W}{1}{3}{4}{2}
				\epair{T}{U}{3}{1}{2}{4}
				& {\redandgray{1}$W \wedge \neg U$} & {\redandgray{2}$\neg W \wedge U$} & \\
			\end{tabular} 
		}
		\myexample{Pairwise Coverage with Six Configurations}{
			\footnotesize\color{lightgray}
			{\redandgray{1}$\{B,P,D,T,W\}$}\\
			{\redandgray{2}$\{B,G,D,U\}$}\\
			{\redandgray{3}$\{B,G,P,T,U\}$}\\
			{\redandgray{4}$\{B,G,W\}$}\\
			{\redandgray{5}$\{B,P,W\}$}\\
			{\redandgray{6}$\{B,D,T,U\}$}\\
		}
	}
\end{frame}
% TODO use different colors for the different configurations (instead of separate handouts)

\subsection{T-Wise Interaction Testing}
\begin{frame}{\insertsubsection}
	\leftorright{
		\mydefinition{T-Wise Interaction Testing}{
			\begin{itemize}
				\setlength\itemsep{.5em}
				\item generalization of pairwise interaction testing
				\item t-wise coverage: every t-wise interaction is covered by at least one configuration in the sample
				\item t=1: every feature is selected and also deselected
				\item t=2: pairwise interaction coverage
				\item t=3: every combination of three features
			\end{itemize}
		}
	}{
		\myexample{{T=3 Interactions for the Features $G$, $P$, and $D$}}{
			$G \wedge P \wedge D$ \hfill $G \wedge P \wedge \neg D$ \hfill $G \wedge \neg P \wedge D$
			
			$G \wedge \neg P \wedge \neg D$ \hfill $\neg G \wedge P \wedge D$ \hfill $\neg G \wedge P \wedge \neg D$
			
			$\neg G \wedge \neg P \wedge D$ \hfill $\neg G \wedge \neg P \wedge \neg D$
		}
	}
\end{frame}

\subsection{Algorithms for Combinatorial Interaction Testing}

% TODO \subsection{A Greedy Algorithm}
\begin{frame}{\insertsubsection}
%\begin{frame}{Einfache Heuristik für Pairwise Interaction Testing}
%	Greedy-Algorithmus: Wähle immer die Konfiguration als nächstes, die die meisten fehlenden Interaktionen abdeckt
%	\vspace{2mm}\pause
%	\begin{itemize}
%		\item Erste Konfiguration frei wählbar (jede deckt am Anfang die gleiche Anzahl von Interaktionen ab: eine pro Paar)
%		\item Stoppe wenn keine Interaktionen übrig
%		\item Findet ggfs.\ nicht die kleinste Teilmenge an Konfigurationen
%		\item Bessere Algorithmen wurden in den letzten Jahren vorgeschlagen (z.B. ICPL)
%	\end{itemize}
\end{frame}
% TODO introduce ICPL

% TODO \subsection{Meta-Heuristic Search}

\subsection{Efficiency of Combinatorial Interaction Testing}
%\subsection{Combinatorial Interaction Testing with ICPL}
\begin{frame}{\insertsubsection\ \mytitlesource{\icpl}}
	\leftorright{
		\myexampletight{Assumption: All Features are Optional}{
			\centering\footnotesize\featureDiagramEightOptionalFeatures
		}
		
		\myexampletight{Number of Configurations in Pairwise Sample}{
			\includegraphics[width=\linewidth,page=4]{citplots}
		}
	}{%
		\myexampletight{Assumption: All Features are Optional}{
			\centering\footnotesize\featureDiagramEightOptionalFeatures
		}
		
		\myexampletight{Number of Configurations in T-Wise Sample}{
			\includegraphics[width=\linewidth,page=5]{citplots}
		}
	}
\end{frame}

\begin{frame}{\insertsubsection\ \mytitlesource{\icpl}}
	\leftorright{
		\myexampletight{Time in Minutes to Compute Sample}{
			\includegraphics[width=\linewidth,page=2]{citplots}

			\begin{itemize}
				\setlength\itemsep{.5em}
				\item about 9h for Linux
				\item 480 configuration in pairwise sample
			\end{itemize}
		}
	}{
		\myexampletight{Number of Configurations in Sample}{
			\includegraphics[width=\linewidth,page=3]{citplots}

			\begin{itemize}
				\setlength\itemsep{.5em}
				\item Linux kernel v2.6.28.6 (February 2009)
				\item 6,888 features
				\item 187,193 clauses in conjunctive normal form
			\end{itemize}
		}
	}
\end{frame}

% TODO distinguish testing efficiency and sampling efficiency

\subsection{Effectiveness of Combinatorial Interaction Testing}
\begin{frame}{\insertsubsection}
	\leftorright{
		\myexampletight{Effectivity of Interaction Testing \mysource{\href{https://ieeexplore.ieee.org/document/1321063}{Kuhn et al.\ 2004}}}{
			\includegraphics[width=\linewidth,page=1]{citplots}
		}
	}{
		\mynote{Trade-Off}{large t: high coverage (more effective)
			
			small t: low testing effort (more efficient)}
	}
\end{frame}
