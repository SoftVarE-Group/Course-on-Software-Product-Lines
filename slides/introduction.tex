\documentclass[
	aspectratio=169, % default is 43
	8pt, % font size, default is 11pt
	handout, % handout mode without animations, comment out to add animations
	%sectiontitleslides,
	sectionoverviews,
]{beamer}

\documentclass[
	aspectratio=169, % default is 43
	8pt, % font size, default is 11pt
	handout, % handout mode without animations, comment out to add animations
]{beamer}

\usepackage{../template/beamerthemeuulm} % use the inofficial uulm beamer theme
\setfaculty{infIngPsy} % set the color scheme for your faculty here [med/infIngPsy/math/nat]

% requires symbolic links
% git clone git@github.com:SoftVarE-Group/SlideTemplate.git C:\Users\...\SlideTemplate
% mklink /J template C:\Users\...\SlideTemplate
% git clone git@spgit.informatik.uni-ulm.de:thuem/slides.git C:\Users\...\ThomasSlides
% mklink /J thomasslides C:\Users\...\ThomasSlides
\graphicspath{{../template/pics/logos}{../template/pics/nature}{../template/pics/uulm}{../thomasslides/}{../pics/}}

%\usepackage[ngerman]{babel} % use this line for slides in German
%\recordingtrue % special recording mode for use with a greenscreen, gives you space to show yourself in a layer in front of the slides, has no effect in the handout mode

\title{Software Product Lines} % short title is used for the slide footer but optional

%
%
%% IMPORTED PACKAGES
%
%\usepackage{adjustbox} % used for partofpage
%\usepackage{tcolorbox} % used for mydefinition, mynote, myexample
\usepackage{multicol} % used temporarily for the lecture overview
%\usepackage{mathtools} % required for absolute value in modeling lecture
%
%% SLIDE TEMPLATE
%
%\beamertemplatenavigationsymbolsempty 
%
%% COMMANDS TO LAYOUT AND ANNIMATE SLIDES
%
\newcommand{\lessonslearned}[3]{
	\subsection{Summary}
	\begin{frame}{\insertsection -- \insertsubsection}
		\leftorright{
			\mydefinition{Lessons Learned}{
				\begin{itemize}
					#1
				\end{itemize}
			}
			\mynote{Further Reading}{
				\small % references take space, can be a little smaller
				\begin{itemize}
					#2
				\end{itemize}
			}
		}{
			\myexample{Practice}{
				#3
			}
		}
	\end{frame}
}

\renewcommand{\lectureoverview}{
%	\section*{Overview}
%	\subsection*{Overview}
	\begin{frame}{\insertsubtitle}
		\begin{multicols}{2}
			\tableofcontents
		\end{multicols}
	\end{frame}
}

%
%\newcommand{\onlyleft}[1]{
%	\halfpage{#1}
%}
%
%\newcommand{\onlyright}[1]{
%	~\hfill
%	\halfpage{#1}
%}
%
%\newcommand{\leftorright}[2]{
%	\uncover<1>{\halfpage{#1}}
%	\hfill
%	\uncover<3->{\halfpage{#2}}
%}
%
%\newcommand{\rightorleft}[2]{
%	\uncover<3->{\halfpage{#1}}
%	\hfill
%	\uncover<1>{\halfpage{#2}}
%}
%
%\newcommand{\leftthenright}[2]{
%	\halfpage{#1}
%	\hfill\pause
%	\halfpage{#2}
%}
%
%\newcommand{\leftandright}[2]{
%	\halfpage{#1}
%	\hfill
%	\halfpage{#2}
%}
%
%\newcommand{\leftmiddleandright}[3]{
%	\thirdpage{#1}
%	\hfill
%	\thirdpage{#2}
%	\hfill
%	\thirdpage{#3}
%}
%
%\newcommand{\leftmiddleorright}[3]{
%	\uncover<1>{\thirdpage{#1}}
%	\hfill
%	\uncover<3>{\thirdpage{#2}}
%	\hfill
%	\uncover<5->{\thirdpage{#3}}
%}
%
%\newcommand{\halfpage}[1]{\partofpage{48}{#1}}
%
%\newcommand{\thirdpage}[1]{\partofpage{31}{#1}}
%
%\newcommand{\partofpage}[2]{
%	\adjustbox{valign=t}{\begin{minipage}{0.#1\textwidth}
%			\begin{flushleft}
%				#2
%			\end{flushleft}
%	\end{minipage}}
%}
%
%\newcommand{\mydefinition}[2]{
%	\begin{tcolorbox}[title=#1,colback=orange!10,colframe=orange!30,coltitle=black,fonttitle=\bfseries,left=1mm,right=1mm,top=1mm,bottom=1mm]
%		\begin{flushleft}
%			#2
%		\end{flushleft}
%	\end{tcolorbox}
%}
%
%\newcommand{\mydefinitiontight}[2]{
%	\begin{tcolorbox}[title=#1,colback=white,colframe=orange!30,coltitle=black,fonttitle=\bfseries,left=0mm,right=0mm,top=0mm,bottom=0mm]
%		\begin{flushleft}
%			#2
%		\end{flushleft}
%	\end{tcolorbox}
%}
%
%\newcommand{\mynote}[2]{
%	\begin{tcolorbox}[title=#1,colback=red!10,colframe=red!30,coltitle=black,fonttitle=\bfseries,left=1mm,right=1mm,top=1mm,bottom=1mm]
%		\begin{flushleft}
%			#2
%		\end{flushleft}
%	\end{tcolorbox}
%}
%
%\newcommand{\myexample}[2]{
%	\begin{tcolorbox}[title=#1,colback=blue!10,colframe=blue!30,coltitle=black,fonttitle=\bfseries,left=1mm,right=1mm,top=1mm,bottom=1mm]
%		\begin{flushleft}
%			#2
%		\end{flushleft}
%	\end{tcolorbox}
%}
%
%\newcommand{\myexampletight}[2]{
%	\begin{tcolorbox}[title=#1,colback=white,colframe=blue!30,coltitle=black,fonttitle=\bfseries,left=0mm,right=0mm,top=0mm,bottom=0mm]
%		\begin{flushleft}
%			#2
%		\end{flushleft}
%	\end{tcolorbox}
%}

\subtitle{1. Introduction}
\author{Thomas Thüm, Timo Kehrer, Elias Kuiter}

\begin{document}

% TITLE SLIDE

\maketitle

% SLIDE TEMPLATE

%\setbeamercolor{title}{fg=black}
%\setbeamercolor{frametitle}{fg=black}
\setbeamertemplate{frametitle}{{\huge~\\\insertsubsection~\insertframetitle}}
\setbeamertemplate{footline}[text line]{\parbox{\linewidth}{\vspace*{-10pt}\hspace{0pt}%
	\insertshortauthor\phantom{g\insertpagenumber}%
	\hfill%
	\inserttitle%
	\ifx \insertsubtitle \empty \else \ -- \insertsubtitle\fi%
	\ifx \insertsectionhead \empty \else \ -- \insertsectionhead\fi%
	\hfill%
	\phantom{g\insertshortauthor}\insertpagenumber%
}}
%\defbeamertemplate{footline}{\begin{beamercolorbox}[sep=1em]{author in head/foot}\insertshortauthor\hfill\insertsection\hfill\insertframenumber\end{beamercolorbox}}
%\defbeamertemplate*{footline}{mytheme}{\begin{beamercolorbox}[sep=1em]{author in head/foot}\insertshortauthor\hfill\insertsection\hfill\insertframenumber\end{beamercolorbox}}

% OVERVIEW SLIDES

\newcommand{\overview}{
	\section*{Overview}
	\subsection*{Overview}
	\begin{frame}{-- \insertsubtitle}
		\begin{multicols}{3}
			\tableofcontents
		\end{multicols}
	
		\begin{flushright}
			\footnotesize
			Author: \insertauthor
			
			Date: \insertdate
		\end{flushright}
	\end{frame}
}
% temporarily added slide to have a lecture overview 
\overview

% temporarily removed
%\begin{frame}{Lecture Overview -- \insertsubtitle}
%	\tableofcontents[hideallsubsections]
%\end{frame}

\AtBeginSection[]{%
	\begin{frame}{Lecture Overview -- \insertsubtitle}
		\tableofcontents[currentsection,hideothersubsections]
	\end{frame}
}

\newcommand{\sectionend}{\addtocontents{toc}{\newpage}}


\section{Introduction to Product Lines}

\subsection{Customization}
% why?
% \deutsch{Maßschneiderung}
% variants / product variants

\subsection{Mass Production}
% why?
% \deutsch{Massenproduktion}
% industrial revolution/era:
% - John Hall, exchangable parts 1826, 25 years of trials (source?)
% - Henry Ford/Ransom Olds, production/assembly line, 1901 (source?)
% - 1961 first industrial roboter at General Motors
% - 1980s automatic assembly lines
% swiss-army knife \deutsch{Eierlegende Wollmilchsau}

\subsection{Mass Customization}
% why?
%\subsection{Motivation}
% why product lines?
% resource limitations: energy, performance, memory
% variability in hardware, laws, ...
% expensive customization

\subsection{Features and Products of a Domain}
\begin{frame}{\insertsubsection}
	\leftorright{
		\mydefinition{Feature \deutsch{Feature} \mysource{\fospl}}{\mycite{A \emph{feature} is a characteristic or end-user-visible behavior of a software system.}}
		\mynote{Feature in a Product Line \mysource{\fospl}}{\mycite{Features are used in product-line engineering to specify and communicate commonalities and differences of the products between stakeholders, and to guide structure, reuse, and variation across all phases of the software life cycle.}}
		\mydefinition{Product \deutsch{Produkt} \mysource{\fospl}}{\mycite{A \emph{product} of a product line is specified by a valid feature selection (a subset of the features of the product line). A feature selection is \emph{valid} if and only if it fulfills all feature dependencies.}}
	}{
		\mydefinition{Domain \deutsch{Domäne} \mysource{\fospl}}{ % TODO cite CE00 here too?
			\mycitebegin A \emph{domain} is an area of knowledge that:
			\begin{itemize}
				\item is scoped to maximize the satisfaction of the requirements of its stakeholders,
				\item includes a set of concepts and terminology understood by practitioners in
				that area,
				\item and includes the knowledge of how to build software systems (or parts of
				software systems) in that area.\myciteend
			\end{itemize}
		}
		\mynote{}{
			\begin{itemize}
				\item a feature is a domain abstraction
				\item in this course:
				
					product = product variant = variant = program variant
			\end{itemize}
		}
	}
\end{frame}
% features = domain abstractions. def of apel book
% goals of features
%a distinctively identifiable functional abstraction that must be implemented, tested, delivered, and maintained” (Kang et al.
%a product characteristic from user or customer views, which essentially consists of a cohesive set of individual requirements” (Chen et al.
%an optional or incremental unit of Zave 2003) 

\subsection{Software Product Line}
\begin{frame}{\insertsubsection}
	\leftandright{
		\mydefinition{Software Product Line \mysource{\seiwhitepaperspl}}{\mycitebegin A \emph{software product line} is 
			\begin{itemize}
				\item a set of software-intensive systems \only<2|handout:0>{\myexample{}{aka.\ products or variants}}
				\item that share a common, managed set of features \only<3|handout:0>{\myexample{}{common set, but not all products have all features in common}}
				\item satisfying the specific needs of a particular market segment or mission \only<4|handout:0>{\myexample{}{aka.\ domain \deutsch{Domäne}}}
				\item and that are developed from a common set of core assets in a prescribed way.\myciteend \only<5|handout:0>{\myexample{}{aka.\ planned, structured reuse \deutsch{Wiederverwendung}}}
			\end{itemize}
			\mysource{Software Engineering Institute, Carnegie Mellon University}
		}
	}{}
\end{frame}
% variants / platforms / domain / variant generation

% configurable software, highly-configurable software, variable software, software variation

\subsection{Product-Line Engineering}
% opposed to single-system engineering

% product-line hall of fame

% product (aka. variant)

% how many examples in first lecture? move certain examples in later lectures? if so, which ones?
%\subsection{Automotive Systems}
% car configurators
% history? number of variants over time?
%\subsection{Notebooks}
% lenovo, microsoft, apple
%\subsection{Printer (Firmware)}
% real printers, 3d printers
% 30 printers per year, more examples
% XKCD: all-in-one paper processor
%\subsection{Operating Systems}
% windows, linux!!!, android
% apps?
%\subsection{Integrated Development Environments}
% eclipse
%\subsection{Browsers}
% plug-ins

% apps! in market store android, iOS
% ecos, packages in debian

%\subsection{Beyond Software}
% financial products by KfW, bikes, shoes, muesli, Subway, headphones, lego, detergents
% brompton: picture in ulm slides

% TODO zoo of animals/tools

% historical development? exchangable parts, production lines, automated product lines, ...
% all-in-one solution vs custom development? software for German fire departments
% all-in-one application software vs embedded software
% reasons for custom development: 

% goal of the lecture

%\subsection{Features}

%\subsection{Software Product Lines}
%% product-line engineering
%\begin{frame}{Who Produces Only One Product?}
%	\href{https://pxhere.com/en/photo/920906}{\includegraphics[width=.6\linewidth]{car-tower}}
%\end{frame}

%\subsection{Single System}
%% single-system engineering
%
%\begin{frame}{Greenfield Development? \deutschertitel{Auf der grünen Wiese?}}
%	\href{https://github.com/SoftVarE-Group/SlideTemplate/blob/main/pics/nature/may21-ulm.jpg}{\includegraphics[width=.6\linewidth]{may21-ulm}}
%\end{frame}



% add illustration for variants/versions (space/time) by icons of word/excel/powerpoint/one note/... over the years




% Apel 2013, Page 49
%Compile-time variability is decided before or at compile time.
%Load-time variability is decided after compilation when the program is started.
%With run-time variability, decisions can be made and changed during program
%execution.


\lessonslearned{
	\item \ldots
}{
	\item \ldots
}{
	\item What other examples of product lines do you know?
	\item Exemplify the differences between feature, product, domain, and product line for these examples.
	\item Are these product lines related to software?
}

\sectionend

\section{Challenges of Product Lines}

\subsection{Code Clones}
\begin{frame}{\insertsubsection}
	\begin{mycolumns}
		\todots
	\mynextcolumn
		\todots
	\end{mycolumns}
\end{frame}
% cloning in single-system engineering, levels of clones?
% cloning in the large, considered harmful
% clone-and-own: even more problematic

\subsection{Feature Traceability}
\begin{frame}{\insertsubsection}
	\begin{mycolumns}[widths={40}]
		\mydefinition{Feature Traceability \mysource{\fospl\mypage{54}}}{\mycite{Feature traceability is the ability to trace a feature from the problem space (for example, the feature model) to the solution space (that is, its manifestation in design and code artifacts).}}
		\todo{all terms introduced?}
	\mynextcolumn
		\myexampletight{Feature Traceability with Colored Source Code}{\includegraphics[width=\linewidth]{feature-traceability}}
	\end{mycolumns}
\end{frame}

\subsection{Automated Generation}
\begin{frame}{\insertsubsection}
	\begin{mycolumns}
		\todo{lego pictures}
	\mynextcolumn
		\href{https://pxhere.com/en/photo/920906}{\includegraphics[width=\linewidth]{car-tower}}
	\end{mycolumns}
\end{frame}
% how to automatically generate products given a descriptive selection of features?

\subsection{Combinatorial Explosion}
\begin{frame}{\insertsubsection}
	\begin{mycolumns}[widths={49}]
		\myexampletight{Combinatorial Explosion}{
			\includegraphics[width=\linewidth,page=6]{cit-plots}%
			\small%
			\begin{itemize}
				\item assumption: all combinations of features are valid
				\item 33 features: a unique combination for every human
				\item 320 features: more combinations than atoms in the universe
			\end{itemize}
		}
	\mynextcolumn
		\myexampletight{Industrial Configuration Spaces \mysource{\evaluatingsharpsatsolvers}}{
			\evaluatingsharpsatsolverslink{\includegraphics[width=\linewidth,page=6,trim=50 210 320 440,clip]{2020/2020-VaMoS-Sundermann}}%
			\small%
			\begin{itemize}
				\item in practice: not all combinations of features valid
				\item many industrial product lines too large to specify all valid combinations manually
				\item largest automotive product line has about $1.7 \cdot 10^{1534}$ products
			\end{itemize}
		}
	\end{mycolumns}
\end{frame}
\begin{frame}{\insertsubsection}
	\centering\href{https://github.com/SoftVarE-Group/Slides/blob/main/2021/2021-02-10-VaMoS-SharpSATApplications.pdf}{\includegraphics[height=\textheightwithtitle,page=9,trim=60 15 15 5,clip]{2021/2021-02-10-VaMoS-SharpSATApplications}}
\end{frame}

\subsection{Feature Interactions}
\begin{frame}{\insertsubsection}
	\begin{mycolumns}
		\todots
	\mynextcolumn
		\myexampletight{Invalid Car Configurations}{\includegraphics[width=\linewidth]{bmw-series1-confassistant-bluetooth}}
	\end{mycolumns}
\end{frame}
% typically unknown in advance
% quality assurance/correctness necessary

\subsection{Continuing Change and Growth}
\begin{frame}{\insertsubsection}
	\begin{mycolumns}[widths={42}]
		\mydefinition{Lehman's Laws of Software Evolution (excerpt) \mysource{\lehmanslaws}}{
			\begin{itemize}
				\item Continuing Change: systems must be continually adapted to stay satisfactory % E-type systems must be continually adapted else they become progressively less satisfactorv.
				\item Increasing Complexity: complexity increases during evolution unless work is done to maintain or reduce it % As an E-type system evolves its complexity increases unless work is done to maintain or reduce it.
				%\item Self Regulation: %E-type system evolution process is self regulating with distribution of product and process measures close to normal.
				%\item Conservation of Organizational Stability (invariant work rate): %The average effective global activity rate in an evolving E-type system is invariant over product lifetime.
				%\item Conservation of Familiarity: satisfactory evolution excludes excessive growth %As an E-type system evolves all associated with it, developers, sales personnel, users, for example, must maintain mastery of its content and behaviour to achieve satisfactory evolution. Excessive growth diminishes that mastery. Hence the average incremental growth remains invariant as the system evolves.
				\item Continuing Growth: functionality must be continually increased to maintain user satisfaction %The functional content of E-type systems must be continually increased to maintain user satisfaction over their lifetime.
				\item Declining Quality: quality will decline unless rigorously maintained and adapted to operational environment changes %The quality of E-type systems will appear to be declining unless they are rigorously maintained and adapted to operational environment changes.
				%\item Feedback System: %E-type evolution processes constitute multi-level, multi-loop, multi-agent feedback systems and must be treated as such to achieve significant improvement over any reasonable base.
			\end{itemize}
		}
	\mynextcolumn
		\mynote{Essence of the Laws}{
			\begin{itemize}
				\item software that is used will be modified
				\item when modified, its complexity will increase (unless one does actively work against it)
			\end{itemize}
		}
		\myexample{Consequences for Product Lines}{
			\begin{itemize}
				\item number of features and size of implementation increases over time
				\item discussed challenges increase over time
					\begin{itemize}
						\item more code clones
						\item harder to trace features
						\item automated generation more urgent
						\item increasing combinatorial explosion
						\item more feature interactions
					\end{itemize}
			\end{itemize}
		}
	\end{mycolumns}
\end{frame}

\begin{frame}{Evolution of the Linux Kernel}
	\begin{mycolumns}
		\myexample{}{
			\begin{itemize}
				\item about $60,000$ commits per year
				\item in peak weeks: new commit every 5 minutes
				\item in average weeks: every 9 minutes
			\end{itemize}
		}
	\mynextcolumn
		\vspace{-12mm} % TODO Benno: hack. not sure how to improve this.
		\href{https://github.com/erikbern/git-of-theseus}{\includegraphics[width=\linewidth,trim=100 70 110 80,clip]{linux-stack-plot}}
	\end{mycolumns}
	\twodimensionalanalysislink{\includegraphics[width=.8\linewidth,page=1,trim=80 420 80 260,clip]{2019/2019-VariVolution-Thuem}}
\end{frame}

\begin{frame}{Evolution of the Linux Kernel}
	\begin{mycolumns}
		\myexampletight{Size of the Code Base}{
			\includegraphics[width=\linewidth,page=1]{linux-plots}%
			\begin{itemize}
				\item from 4 to 24 millions in 17 years
				\item about one million LOC added every year
				\item about 3,000 LOC per day
			\end{itemize}
		}
	\mynextcolumn
		\myexampletight{Number of Features}{
			\includegraphics[width=\linewidth,page=2]{linux-plots}%
			\begin{itemize}
				\item about 800 new features every year
				\item about 15 new features every week
				\item in 2018 four times more features than in 2005
			\end{itemize}
		}
	\end{mycolumns}
\end{frame}

\begin{frame}{Evolution of the Linux Kernel}
	\begin{mycolumns}
		\myexampletight{Number of Products}{
			\includegraphics[width=\linewidth,page=3]{linux-plots}%
			\begin{itemize}
				\item number of products is doubled more than three times a day
				\item the current kernel is likely to have between $10^{6000}$ and $10^{8000}$ products
			\end{itemize}
		}
	\mynextcolumn
		\myexampletight{Time to Count Products}{
			\includegraphics[width=\linewidth,page=4]{linux-plots}%
			\begin{itemize}
				\item most kernel versions before 2015 can be computed within 1 minute
				\item most kernel versions after 2015 cannot be computed within 1 hour
			\end{itemize}
		}
	\end{mycolumns}
\end{frame}

% costs: development, investment, maintenance, logistic, production
% customer needs?



\lessonslearned{
	\item focus: how to implement features
	\item how to model valid combinations
	\item how to do quality assurance
	\item challenges: code clones, feature traceability, automated generation, combinatorial explosion, feature interactions, continuous growth
}{
	\item \ldots
}{
	\item Form groups of 2--3 students
	\item Explain 2--3 of the six challenges to your colleagues
	\item Can you find own examples for these challenges?
}

\sectionend

\section{Course Organization}

%\subsection{Preliminaries}
\subsection{What You Should Know}

\begin{frame}{\insertsubsection{}}
	\leftorright{
		\mynote{Fundamentals of Software Engineering}{
			\begin{itemize}
				\item development processes
				\item object-oriented programming
				\item design patterns
				\item UML class diagrams
				\item modularity
			\end{itemize}
		}
	}{
		\mynote{Fundamentals of Theoretical Computer Science}{
			\begin{itemize}
				\item set theory
				\item propositional logic
				\item complexity theory
			\end{itemize}
		}

		\mynote{Exercise}{
			solid programming skills in Java
		}
	}
\end{frame}

%\subsection{Course Overview}
\subsection{What You Will Learn}

\begin{frame}{\insertsubsection{}}
	\leftmiddleorright{
		\mydefinition{Part I: Ad-Hoc Approaches for Variability}{
			\begin{enumerate}
				\item Introduction
				\item Runtime Variability and Design Patterns
				\item Compile-Time Variability with Clone-and-Own
			\end{enumerate}
		}
	}{
		\mydefinition{\small Part II: Modeling and Implementing Features}{
			\begin{enumerate}
				\setcounter{enumi}{3}
				\item Feature Modeling
				\item Compile-Time Features
				\item Modular Features
				\item Languages for Features
				\item Development Process
			\end{enumerate}
		}
	}{
		\mydefinition{Part III: Quality Assurance and Maintenance}{
			\begin{enumerate}
				\setcounter{enumi}{8}
				\item Feature Interactions
				\item Product-Line Analyses
				\item Product-Line Testing
				\item Evolution and Maintenance
			\end{enumerate}
		}
	}
\end{frame}

%\subsection{Literature and Tool Support}
\subsection{What You Might Need}

\begin{frame}{\insertsubsection{}}
	\leftorright{
		\myexampletight{Recommended Literature for Lecture \& Exercise}{
			\centering
			\parbox{0.49\linewidth}{
				\centering
				\href{http://link.springer.com/book/10.1007/978-3-642-37521-7}{\includegraphics[width=\linewidth]{cover-fospl}}
				\emph{theory-focused}
			}
			\parbox{0.475\linewidth}{
				\centering
				\href{http://www.springer.com/de/book/9783319614427}{\includegraphics[width=\linewidth]{cover-featureide}}
				\emph{practice-oriented}
			}
		}
	}{
		\myexampletight{Recommended Tool Support for the Exercise}{
			\centering
			\href{https://featureide.github.io/}{\includegraphics[width=\linewidth]{featureide-feature-model-editor.png}}\\[.5ex]
			\href{https://featureide.github.io/}{\includegraphics[width=0.25\linewidth]{featureide-logo.png}}
		}
	}
\end{frame}

\subsection{Credit for the Slides}

\begin{frame}{\insertsubsection{}}
	\vspace{-10mm}\hfill\href{https://github.com/SoftVarE-Group/Course-on-Software-Product-Lines}{\includegraphics[scale=.5]{cc-by-sa}}\vspace{2mm}
	\begin{mycolumns}[columns=3,animation=none]
		\mynote{Thomas Thüm}{
			\centering
			\href{https://www.uni-ulm.de/en/in/sp/team/thuem/}{\adjincludegraphics[height=.45\textheight,trim={.125\width} 0 {.125\width} 0,clip]{thomas-thuem}}

			\small Professor at University of Ulm

			product-line engineering

			FeatureIDE team leader
		}
	\mynextcolumn
		\mynote{Timo Kehrer}{
			\centering
			\href{https://seg.inf.unibe.ch/people/timo/}{\includegraphics[height=.45\textheight]{timo-kehrer}}

			\small Professor at University of Bern

			model-based engineering

			~
		}
	\mynextcolumn
		\mynote{Elias Kuiter}{
			\centering
			\href{https://www.dbse.ovgu.de/en/Staff/Elias+Kuiter.html}{\includegraphics[height=.45\textheight]{elias-kuiter}}

			\small PhD student in Magdeburg

			feature-model analysis

			FeatureIDE core developer
		}
	\end{mycolumns}
\end{frame}
% TODO include relevant university-specific content here
%\subsection{People}

\begin{frame}{\insertsubsection}
	\leftorright{
		\mynote{Who Are You?}{
			\begin{itemize}
				\item A student in BeNeFri's \emph{Joint Master in Computer Science}, interested in
				\begin{itemize}
					\item the topics of the \emph{Advanced Software Engineering} track (T2),
					\item \emph{learning} about the basic principles of systematically managing software variability,
					\item \emph{experimenting} with novel software engineering methods and tools,
					\item getting in touch with current \emph{research} on software product lines.
				\end{itemize}
			\end{itemize}
		}
	}{
		\mynote{Who Are We?}{
			\centering
			\parbox{0.45\linewidth}{
				\centering
				\href{https://seg.inf.unibe.ch/people/timo/}{\includegraphics[width=0.9\linewidth]{timo-kehrer}}\\[.5ex]
				\href{https://seg.inf.unibe.ch/people/timo/}{\emph{Timo Kehrer}}\\[.5ex]
				\small Professor for Software Engineering \\[.5ex]
				\href{https://seg.inf.unibe.ch/}{\small \emph{SEG @ UniBE}}
			}
			\parbox{0.45\linewidth}{
				\centering
				\href{https://seg.inf.unibe.ch/people/sandra/}{\includegraphics[width=0.9\linewidth]{sandra-greiner}}\\[.5ex]
				\href{https://seg.inf.unibe.ch/people/sandra/}{\emph{Sandra Greiner}}\\[.5ex]
				\small Research assistant finishing her PhD on Model-Driven SPLs \\[.5ex]
				\href{https://seg.inf.unibe.ch/}{\small \emph{SEG @ UniBE}}
			}
		}
	}
\end{frame}

\begin{frame}{Where and When?}
	\leftorright{
		\mynote{Lecture}{
			\begin{itemize}
				\item once per week
				\begin{itemize}
					\item on \emph{Thursday}, 13:15--15:00
					\item Seminarraum 111, Engehalde E8 (UniBE)
					\item starts on Thursday, September 22, 2022
				\end{itemize}
				\item usually held by Timo
				\item \emph{slides} are available on \href{https://ilias.unibe.ch/goto_ilias3_unibe_crs_2469231.html}{ILIAS}
				\item lectures are meant to be \emph{interactive}, you are kindly invited to \emph{actively participate}!
				%\item guest lectures planned:
				%\begin{itemize}
					%\item industry talk around Christmas
					%\item research talk at end of January
				%\end{itemize}
			\end{itemize}
		}
	}{
		\mynote{Exercise}{
			\begin{itemize}
				\item once per week 
				\begin{itemize}
					\item on \emph{Thursday}, 15:15--16:00
					\item Seminarraum 111, Engehalde E8 (UniBE)
					\item starts on Thursday, September 29, 2022
				\end{itemize}
				\item usually held by Sandra
				\item \emph{exercise sheets} are available on \href{https://ilias.unibe.ch/goto_ilias3_unibe_crs_2469231.html}{ILIAS}
				\item typically 2 weeks time to work on \emph{practical tasks}
				%\item lab exercise planned for end of January
			\end{itemize}
		}
	}
\end{frame}

\begin{frame}{Taking the Exam, and Then?}
	\leftorright{
		%\mynote{Exam Eligibility \deutsch{Pr�fungszulassung}}{
			%\begin{itemize}
				%\item 60\% of all normal tasks \deutsch{Votierungspunkte}
				%\item 70\% for Master students
				%\item 3 presentation points \deutsch{Vortragspunkte}
				%\item all practical tasks (in teams of 2--3 students)
			%\end{itemize}
		%}
		\mynote{Exam}{
			\begin{itemize}
				\item \emph{oral exam} ($\approx 20$ minutes)
				\item 1--2 exam days in January (details will follow)
				\item \emph{registration through Academia}!
			\end{itemize}
		}
	}{
		\mynote{Further Studies}{
			\begin{itemize}
				\item \emph{Seminar Software Engineering} (offered every semester)
				\item \emph{Master Thesis} (several open topics on SPLE)
				\item \emph{PhD Thesis?} (SPLs are still actively researched)
			\end{itemize}
			\ldots{} Just contact us and let us know \ldots
		}
	}
\end{frame}
\subsection{Formalities}

\begin{frame}{\insertsubsection}
	\leftorright{
		\mynote{Who Are You?}{
			\begin{itemize}
				\item Bachelor student (5 ECTS)
				\item Master student (6 ECTS)
				\item enrolled in one of these courses of study:
				\begin{itemize}
					\item Informatik
					\item Computervisualistik
					\item Wirtschaftsinformatik
					\item Ingenieurinformatik
					\item Digital Engineering
				\end{itemize}
				\item looking for an elective subject \deutsch{Wahlpflichtfach (WPF)}
			\end{itemize}
		}
	}{
		\mynote{Who Are We?}{
			\centering
			\parbox{0.45\linewidth}{
				\centering
				\href{https://www.dbse.ovgu.de/Mitarbeiter/Gunter+Saake.html}{\includegraphics[width=\linewidth]{gunter-saake}}\\[.5ex]
				\href{https://www.dbse.ovgu.de/Mitarbeiter/Gunter+Saake.html}{\emph{Gunter Saake}}\\[.5ex]
				\small professor for databases and software engineering\\[.5ex]
				FeatureIDE project manager
			}
			\parbox{0.45\linewidth}{
				\centering
				\href{https://www.dbse.ovgu.de/Mitarbeiter/Elias+Kuiter.html}{\includegraphics[width=0.75\linewidth]{elias-kuiter}}\\[.5ex]
				\href{https://www.dbse.ovgu.de/Mitarbeiter/Elias+Kuiter.html}{\emph{Elias Kuiter}}\\[.5ex]
				\small PhD student in feature-model analysis\\[.5ex]
				FeatureIDE core developer
			}
		}
	}
\end{frame}

\begin{frame}{\insertsubsection}
	\leftorright{
		\mynote{Lecture}{
			\begin{itemize}
				\item once per week (2 SWS)
				\begin{itemize}
					\item on \emph{Wednesday}, 09.15am--10.45am
					\item in room G40B-326
					\item starts on October 11
				\end{itemize}
				\item usually held by Gunter
				\item \emph{slides} are handed out in \href{https://elearning.ovgu.de/course/view.php?id=13228}{Moodle}
				\item guest lectures planned:
				\begin{itemize}
					\item industry talk around Christmas
					\item research talk at end of January
				\end{itemize}
			\end{itemize}
		}
	}{
		\mynote{Exercise}{
			\begin{itemize}
				\item once per week (2 SWS)
				\begin{itemize}
					\item on \emph{Tuesday}, 09.15am--10.45am
					\item in room G40B-326
					\item starts on October 18
				\end{itemize}
				\item usually held by Elias
				\item \emph{exercise sheets} are handed out 1 week in advance in \href{https://elearning.ovgu.de/course/view.php?id=13228}{Moodle}
				\item 2 weeks time to work on \emph{practical tasks}
				\item lab exercise planned for end of January
			\end{itemize}
		}
	}
\end{frame}

\begin{frame}{\insertsubsection}
	\leftorright{
		\mynote{Exam Eligibility \deutsch{Prüfungszulassung}}{
			\begin{itemize}
				\item 60\% of all normal tasks \deutsch{Votierungspunkte}
				\item 70\% for Master students
				\item 3 presentation points \deutsch{Vortragspunkte}
				\item all practical tasks (in teams of 2--3 students)
			\end{itemize}
		}
		\mynote{Exam}{
			\begin{itemize}
				\item oral exam ($\approx 20$ minutes)
				\item 1--2 exam days in February or March
				\item to get an ungraded performance \deutsch{Schein}, you have to pass the exam
				\item consultation in the last lecture
			\end{itemize}
		}
	}{
		\mynote{Further Studies}{
			\begin{itemize}
				\item Bachelor/Master thesis
				\item individual project
				\item seminar or software project in summer term % FeatJAR software project or proseminar on advanced language concepts
			\end{itemize}
			\ldots{} consult with us!
		}
	}
\end{frame}
%\input{content/introduction_course_ulm}

\lessonslearned{
	\item Prerequisites, course overview, formalities
}{
	\item \fospl
 	\item \featureide
}{
	Ask questions on the course organization!
}

\mode<beamer>{
	\begin{frame}{\inserttitle}
		\lectureseriesoverview
	\end{frame}

	\contentoverview
}


\end{document}
