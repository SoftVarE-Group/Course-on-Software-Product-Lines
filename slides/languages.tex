\documentclass[
	aspectratio=169, % default is 43
	8pt, % font size, default is 11pt
	handout, % handout mode without animations, comment out to add animations
]{beamer}

\usepackage{../template/beamerthemeuulm} % use the inofficial uulm beamer theme
\setfaculty{infIngPsy} % set the color scheme for your faculty here [med/infIngPsy/math/nat]

% requires symbolic links
% git clone git@github.com:SoftVarE-Group/SlideTemplate.git C:\Users\...\SlideTemplate
% mklink /J template C:\Users\...\SlideTemplate
% git clone git@spgit.informatik.uni-ulm.de:thuem/slides.git C:\Users\...\ThomasSlides
% mklink /J thomasslides C:\Users\...\ThomasSlides
\graphicspath{{../template/pics/logos}{../template/pics/nature}{../template/pics/uulm}{../thomasslides/}{../pics/people/}{../pics/xkcd/}}

%\usepackage[ngerman]{babel} % use this line for slides in German
%\recordingtrue % special recording mode for use with a greenscreen, gives you space to show yourself in a layer in front of the slides, has no effect in the handout mode

\title{Software Product Lines} % short title is used for the slide footer but optional

% LINKED LITERATURE

\newcommand{\ludewiglichter}{\href{https://learning.oreilly.com/library/view/-/9781457184932/?ar}{Ludewig and Lichter}}
\newcommand{\seeconomics}{\href{https://rds-ulm.ibs-bw.de/link?kid=027381854}{SE Economics}}
\newcommand{\sommervillelink}[1]{\href{https://ulm.ibs-bw.de/aDISWeb/app?service=direct/0/Home/$DirectLink\&sp=SOPAC00\&sp=SAKSWB-IdNr1615420983}{#1}}
\newcommand{\sommerville}{\sommervillelink{Sommerville}}
\newcommand{\thehumbleprogrammer}{\href{https://dl.acm.org/doi/10.1145/1283920.1283927}{The Humble Programmer}}
\newcommand{\thepragmaticprogrammer}{\href{https://learning.oreilly.com/library/view/the-pragmatic-programmer/9780135956977/}{The Pragmatic Programmer}}

% TYPICAL COMMANDS FOR LECTURES

\renewcommand{\emph}[1]{{\color{blue}\textbf{#1}}}

\newcommand{\deutsch}[1]{{\color{blue}(#1)}}
\newcommand{\deutschertitel}[1]{{\tiny\deutsch{#1}}}

\newcommand{\mycite}[1]{``#1''}
\newcommand{\mytitlesource}[1]{{\tiny\normalfont\mbox{[#1]}}}
\newcommand{\mysource}[1]{\ifthenelse{\equal{#1}{}}{}{\phantom{.}~\hfill~\mytitlesource{#1}}}

\newcommand{\todo}[1]{{\color{red}\textbf{[#1]}}}
\newcommand{\fodo}[1]{\todo{\footnote{\todo{#1}}}}
\newcommand{\todots}{\todo{\ldots}}

% IMPORTED PACKAGES

%\usepackage{adjustbox} % used for partofpage
%\usepackage{tcolorbox} % used for mydefinition, mynote, myexample
\usepackage{multicol} % used temporarily for the lecture overview
\usepackage{mathtools} % required for absolute value in modeling lecture

% COMMANDS TO LAYOUT AND ANNIMATE SLIDES

\newcommand{\lessonslearned}[3]{
	\subsection{Summary}
	\begin{frame}{\insertsection -- \insertsubsection}
		\leftorright{
			\mydefinition{Lessons Learned}{
				\begin{itemize}
					#1
				\end{itemize}
			}
			\mynote{Further Reading}{
				\small % references take space, can be a little smaller
				\begin{itemize}
					#2
				\end{itemize}
			}
		}{
			\myexample{Practice}{
				#3
			}
		}
	\end{frame}
}

% TODO temporary hack to layout the slide overview in two colums
\renewcommand{\lectureoverview}{
%	\section*{Overview}
%	\subsection*{Overview}
	\begin{frame}{\insertsubtitle}
		\begin{multicols}{2}
			\tableofcontents
		\end{multicols}
	\end{frame}
}

\renewcommandx{\maketitle}[2][1=apr21-o25a,2=150]{
    {
	\usebackgroundtemplate{} % TODO temporary hack to enable missing pictures at title slide
	%\ifx {#1} \empty \else {\usebackgroundtemplate{\includegraphics[trim=0 0 0 #2,clip,width=\paperwidth]{#1}}} \fi     
	%\usebackgroundtemplate{\includegraphics[trim=0 0 0 #2,clip,width=\paperwidth]{#1}}
    \begin{frame}[plain]
        \vskip0pt plus 1filll
        \begin{beamercolorbox}[wd=\paperwidth,ht=4.5ex,dp=2ex,right]{titlebox}
            \LARGE\textbf{\inserttitle}\hspace*{20pt}
        \end{beamercolorbox}%
        \nointerlineskip%
        \begin{beamercolorbox}[wd=\paperwidth,ht=2.25ex,dp=1ex,right]{subtitlebox}
            \small 
            \ifx \insertsubtitle \empty \else \insertsubtitle\ $\vert$ \fi
            \insertauthor\
            \ifx \insertdate \empty \else $\vert$ \insertdate \fi
            \hspace*{20pt}
        \end{beamercolorbox}%
        \nointerlineskip%
        \begin{beamercolorbox}[wd=\paperwidth,ht=4.5ex,dp=2ex,left]{logobox}
            \centering
            \vspace{-1ex}
            \hspace{10pt}
            \includegraphics[height=4.5ex]{sp} % SPECIFY INSTITUTE LOGO HERE
            \hfill
            \includegraphics[height=4.5ex]{uulm}
            \hspace{10pt}
        \end{beamercolorbox}%
    \end{frame}
    }  
}

%
%\newcommand{\onlyleft}[1]{
%	\halfpage{#1}
%}
%
%\newcommand{\onlyright}[1]{
%	~\hfill
%	\halfpage{#1}
%}
%
%\newcommand{\leftorright}[2]{
%	\uncover<1>{\halfpage{#1}}
%	\hfill
%	\uncover<3->{\halfpage{#2}}
%}
%
%\newcommand{\rightorleft}[2]{
%	\uncover<3->{\halfpage{#1}}
%	\hfill
%	\uncover<1>{\halfpage{#2}}
%}
%
%\newcommand{\leftthenright}[2]{
%	\halfpage{#1}
%	\hfill\pause
%	\halfpage{#2}
%}
%
%\newcommand{\leftandright}[2]{
%	\halfpage{#1}
%	\hfill
%	\halfpage{#2}
%}
%
%\newcommand{\leftmiddleandright}[3]{
%	\thirdpage{#1}
%	\hfill
%	\thirdpage{#2}
%	\hfill
%	\thirdpage{#3}
%}
%
%\newcommand{\leftmiddleorright}[3]{
%	\uncover<1>{\thirdpage{#1}}
%	\hfill
%	\uncover<3>{\thirdpage{#2}}
%	\hfill
%	\uncover<5->{\thirdpage{#3}}
%}
%
%\newcommand{\halfpage}[1]{\partofpage{48}{#1}}
%
%\newcommand{\thirdpage}[1]{\partofpage{31}{#1}}
%
%\newcommand{\partofpage}[2]{
%	\adjustbox{valign=t}{\begin{minipage}{0.#1\textwidth}
%			\begin{flushleft}
%				#2
%			\end{flushleft}
%	\end{minipage}}
%}
%
%\newcommand{\mydefinition}[2]{
%	\begin{tcolorbox}[title=#1,colback=orange!10,colframe=orange!30,coltitle=black,fonttitle=\bfseries,left=1mm,right=1mm,top=1mm,bottom=1mm]
%		\begin{flushleft}
%			#2
%		\end{flushleft}
%	\end{tcolorbox}
%}
%
%\newcommand{\mydefinitiontight}[2]{
%	\begin{tcolorbox}[title=#1,colback=white,colframe=orange!30,coltitle=black,fonttitle=\bfseries,left=0mm,right=0mm,top=0mm,bottom=0mm]
%		\begin{flushleft}
%			#2
%		\end{flushleft}
%	\end{tcolorbox}
%}
%
%\newcommand{\mynote}[2]{
%	\begin{tcolorbox}[title=#1,colback=red!10,colframe=red!30,coltitle=black,fonttitle=\bfseries,left=1mm,right=1mm,top=1mm,bottom=1mm]
%		\begin{flushleft}
%			#2
%		\end{flushleft}
%	\end{tcolorbox}
%}
%
%\newcommand{\myexample}[2]{
%	\begin{tcolorbox}[title=#1,colback=blue!10,colframe=blue!30,coltitle=black,fonttitle=\bfseries,left=1mm,right=1mm,top=1mm,bottom=1mm]
%		\begin{flushleft}
%			#2
%		\end{flushleft}
%	\end{tcolorbox}
%}
%
%\newcommand{\myexampletight}[2]{
%	\begin{tcolorbox}[title=#1,colback=white,colframe=blue!30,coltitle=black,fonttitle=\bfseries,left=0mm,right=0mm,top=0mm,bottom=0mm]
%		\begin{flushleft}
%			#2
%		\end{flushleft}
%	\end{tcolorbox}
%}

\subtitle{7. Language-Based Techniques}
\author{Thomas Thüm}

\begin{document}

\mode<handout>{\contentoverview}

\mode<beamer>{
	\ifdefined\thepicture
		\maketitle[\thepicture][\thepictureoffset]
	\else
		\maketitle[]
	\fi
}

% shared slide content

% introduced: 02a-configuration
% reused: 03a-intro
\newcommand{\frameImplementSPLs}{
	\begin{mycolumns}[widths={45},animation=none]
		\pic[width=\linewidth]{metaproduct2}
	\mynextcolumn
		\begin{note}{Key Issues}
			\begin{itemize}
			\item Systematic reuse of implementation artifacts
			\item Explicit handling of variability
			\end{itemize}
		\end{note}
		\uncover<2->{\begin{definition}{Variability\mysource{\fospl\mypage{48}}}
			\mycite{\emph{Variability} is the ability to derive different products from a common set of artifacts.}
		\end{definition}}
		~
		\uncover<3->{\begin{note}{Variability-Intensive System}
			Any software product line is a variability-intensive system. % TODO Timo: do we really need this term? where does this definition come from?
		\end{note}}
	\end{mycolumns}
}

% introduced: 02a-configuration
% reused: 02b-implementation, 03a-intro
\newcommand{\frameVariabilityAndBindingTimes}{
	\begin{mycolumns}[widths={55},animation=none]
		\begin{definition}{Binding Time \deutsch{Bindungszeitpunkt}\mysource{\fospl\mypage{48}}}
			\begin{itemize}
				\item Variability offers choices
				\item Derivation of a product requires to make decisions (aka. binding)
				\item Decisions may be bound at different binding times
			\end{itemize}
		\end{definition}
		~
		\uncover<2->{\begin{note}{When? By whom? How?}
			\lectureruntime\parta: \emph{when} and \emph{by whom}

			\lectureruntime\partb: \emph{how}
		\end{note}}
	\mynextcolumn
		\pic[width=\linewidth]{metaproduct2}
	\end{mycolumns}
}

% introduced: 03a-intro
% reused: 03a-intro
\newcommand{\frameRuntimeVariabilityProblems}{
	\begin{note}{Problems of Runtime Variability}
		{\bf Conditional Statements:}
		\begin{itemize}
			\item Code scattering, tangling, and replication
		\end{itemize}
		{\bf Design Patterns for Variability:}
		\begin{itemize}
			\item Trade-offs and potential negative side effects
			\item Constraints that may restrict their usage
		\end{itemize}
		{\bf In General:}
		\begin{itemize}
			\item Variable parts are always delivered
			\item Not well-suited for compile-time binding
		\end{itemize}
	\end{note}
}

% introduced: 03a-intro
% reused: 03a-intro
\newcommand{\frameSoftwareConfigurationManagement}{
	\begin{mycolumns}
		\begin{definition}{Software Configuration Management} % TODO source missing
			Policies, processes, and tools for managing evolving software systems:
			\begin{itemize}
				\item Version control
				\item System building
				\item Release management
				\item Change management
				\item Collaborative work
			\end{itemize}
		\end{definition}
	\mynextcolumn
		\begin{note}{No Software Configuration Management}
			\lecturecloneandown\parta: Ad-Hoc Clone-and-Own

			aka.\ unmanaged clone-and-own
		\end{note}
		\begin{note}{Version Control}
			\lecturecloneandown\partb: Clone-and-Own with Version Control

			instance of managed clone-and-own
		\end{note}
		\begin{note}{System Building}
			\lecturecloneandown\partc: Clone-and-Own with Build Systems

			instance of managed clone-and-own
		\end{note}
	\end{mycolumns}
}


\section{Limits of Object-Oriented Programming}
% of Classical Programming Paradigms
% of Teachniques for Application Software

\subsection{Expression Problem? Preplanning? Tyrany}
\subsection{Multipe Inheritance}
\subsection{Mixins}
%\subsection{Design Patterns}
\subsection{Default Methods}
\subsection{\ldots}

%\subsection{Slide Title 1}
\begin{frame}{~}
	\ldots
\end{frame}

\subsection{Slide Title 2}
\begin{frame}{-- Example Subtitle}
	\ldots
\end{frame}

\begin{frame}{-- Second Example Subtitle}
	\ldots
\end{frame}

\subsection{Slide Title 3}
%\begin{frame}{~}
%	\ldots
%\end{frame}



\lessonslearned{
	\item \ldots
}{
	\item \ldots
}{
	\ldots
}

\sectionend

\section{Feature-Oriented Programming}

\subsection{Motivation}
% Feature Traceability, Crosscutting Concerns, Preplanning Problem, Unflexible Inheritance Hierarchies
\subsection{Feature-Oriented Programming}
% idea, collaborations, roles, feature modules
\subsection{Feature Composition}
% with FeatureHouse, AHEAD?, FeatureC++, different realizations (mixins, jampack)?
\subsection{Principle of Uniformity}
% 

%\subsection{Slide Title 1}
\begin{frame}{~}
	\ldots
\end{frame}

\subsection{Slide Title 2}
\begin{frame}{-- Example Subtitle}
	\ldots
\end{frame}

\begin{frame}{-- Second Example Subtitle}
	\ldots
\end{frame}

\subsection{Slide Title 3}
%\begin{frame}{~}
%	\ldots
%\end{frame}



\lessonslearned{
	\item \ldots
}{
	\item \ldots
}{
	\ldots
}

\sectionend

\section{Delta-Oriented Programming}

\subsection{Slide Title 1}
\begin{frame}{~}
	\ldots
\end{frame}

\subsection{Slide Title 2}
\begin{frame}{-- Example Subtitle}
	\ldots
\end{frame}

\begin{frame}{-- Second Example Subtitle}
	\ldots
\end{frame}

\subsection{Slide Title 3}
%\begin{frame}{~}
%	\ldots
%\end{frame}



\lessonslearned{
	\item \ldots
}{
	\item \ldots
}{
	\ldots
}

\sectionend

\section{Aspect-Oriented Programming}

\subsection{Slide Title 1}
\begin{frame}{~}
	\ldots
\end{frame}

\subsection{Slide Title 2}
\begin{frame}{-- Example Subtitle}
	\ldots
\end{frame}

\begin{frame}{-- Second Example Subtitle}
	\ldots
\end{frame}

\subsection{Slide Title 3}
%\begin{frame}{~}
%	\ldots
%\end{frame}



\lessonslearned{
	\item \ldots
}{
	\item \ldots
}{
	\ldots
}

\mode<beamer>{
	\begin{frame}{\inserttitle}
		\lectureseriesoverview
	\end{frame}

	\contentoverview
}


\end{document}
