\documentclass[
	aspectratio=169, % default is 43
	8pt, % font size, default is 11pt
	handout, % handout mode without animations, comment out to add animations
]{beamer}

\usepackage{../template/beamerthemeuulm} % use the inofficial uulm beamer theme
\setfaculty{infIngPsy} % set the color scheme for your faculty here [med/infIngPsy/math/nat]

% requires symbolic links
% git clone git@github.com:SoftVarE-Group/SlideTemplate.git C:\Users\...\SlideTemplate
% mklink /J template C:\Users\...\SlideTemplate
% git clone git@spgit.informatik.uni-ulm.de:thuem/slides.git C:\Users\...\ThomasSlides
% mklink /J thomasslides C:\Users\...\ThomasSlides
\graphicspath{{../template/pics/logos}{../template/pics/nature}{../template/pics/uulm}{../thomasslides/}{../pics/}}

%\usepackage[ngerman]{babel} % use this line for slides in German
%\recordingtrue % special recording mode for use with a greenscreen, gives you space to show yourself in a layer in front of the slides, has no effect in the handout mode

\title{Software Product Lines} % short title is used for the slide footer but optional

%
%
%% IMPORTED PACKAGES
%
%\usepackage{adjustbox} % used for partofpage
%\usepackage{tcolorbox} % used for mydefinition, mynote, myexample
\usepackage{multicol} % used temporarily for the lecture overview
%\usepackage{mathtools} % required for absolute value in modeling lecture
%
%% SLIDE TEMPLATE
%
%\beamertemplatenavigationsymbolsempty 
%
%% COMMANDS TO LAYOUT AND ANNIMATE SLIDES
%
\newcommand{\lessonslearned}[3]{
	\subsection{Summary}
	\begin{frame}{\insertsection -- \insertsubsection}
		\leftorright{
			\mydefinition{Lessons Learned}{
				\begin{itemize}
					#1
				\end{itemize}
			}
			\mynote{Further Reading}{
				\small % references take space, can be a little smaller
				\begin{itemize}
					#2
				\end{itemize}
			}
		}{
			\myexample{Practice}{
				#3
			}
		}
	\end{frame}
}

\renewcommand{\lectureoverview}{
%	\section*{Overview}
%	\subsection*{Overview}
	\begin{frame}{\insertsubtitle}
		\begin{multicols}{2}
			\tableofcontents
		\end{multicols}
	\end{frame}
}

%
%\newcommand{\onlyleft}[1]{
%	\halfpage{#1}
%}
%
%\newcommand{\onlyright}[1]{
%	~\hfill
%	\halfpage{#1}
%}
%
%\newcommand{\leftorright}[2]{
%	\uncover<1>{\halfpage{#1}}
%	\hfill
%	\uncover<3->{\halfpage{#2}}
%}
%
%\newcommand{\rightorleft}[2]{
%	\uncover<3->{\halfpage{#1}}
%	\hfill
%	\uncover<1>{\halfpage{#2}}
%}
%
%\newcommand{\leftthenright}[2]{
%	\halfpage{#1}
%	\hfill\pause
%	\halfpage{#2}
%}
%
%\newcommand{\leftandright}[2]{
%	\halfpage{#1}
%	\hfill
%	\halfpage{#2}
%}
%
%\newcommand{\leftmiddleandright}[3]{
%	\thirdpage{#1}
%	\hfill
%	\thirdpage{#2}
%	\hfill
%	\thirdpage{#3}
%}
%
%\newcommand{\leftmiddleorright}[3]{
%	\uncover<1>{\thirdpage{#1}}
%	\hfill
%	\uncover<3>{\thirdpage{#2}}
%	\hfill
%	\uncover<5->{\thirdpage{#3}}
%}
%
%\newcommand{\halfpage}[1]{\partofpage{48}{#1}}
%
%\newcommand{\thirdpage}[1]{\partofpage{31}{#1}}
%
%\newcommand{\partofpage}[2]{
%	\adjustbox{valign=t}{\begin{minipage}{0.#1\textwidth}
%			\begin{flushleft}
%				#2
%			\end{flushleft}
%	\end{minipage}}
%}
%
%\newcommand{\mydefinition}[2]{
%	\begin{tcolorbox}[title=#1,colback=orange!10,colframe=orange!30,coltitle=black,fonttitle=\bfseries,left=1mm,right=1mm,top=1mm,bottom=1mm]
%		\begin{flushleft}
%			#2
%		\end{flushleft}
%	\end{tcolorbox}
%}
%
%\newcommand{\mydefinitiontight}[2]{
%	\begin{tcolorbox}[title=#1,colback=white,colframe=orange!30,coltitle=black,fonttitle=\bfseries,left=0mm,right=0mm,top=0mm,bottom=0mm]
%		\begin{flushleft}
%			#2
%		\end{flushleft}
%	\end{tcolorbox}
%}
%
%\newcommand{\mynote}[2]{
%	\begin{tcolorbox}[title=#1,colback=red!10,colframe=red!30,coltitle=black,fonttitle=\bfseries,left=1mm,right=1mm,top=1mm,bottom=1mm]
%		\begin{flushleft}
%			#2
%		\end{flushleft}
%	\end{tcolorbox}
%}
%
%\newcommand{\myexample}[2]{
%	\begin{tcolorbox}[title=#1,colback=blue!10,colframe=blue!30,coltitle=black,fonttitle=\bfseries,left=1mm,right=1mm,top=1mm,bottom=1mm]
%		\begin{flushleft}
%			#2
%		\end{flushleft}
%	\end{tcolorbox}
%}
%
%\newcommand{\myexampletight}[2]{
%	\begin{tcolorbox}[title=#1,colback=white,colframe=blue!30,coltitle=black,fonttitle=\bfseries,left=0mm,right=0mm,top=0mm,bottom=0mm]
%		\begin{flushleft}
%			#2
%		\end{flushleft}
%	\end{tcolorbox}
%}

% todo: move these commands somewhere sensible?
\newcommand{\sem}[1]{\ensuremath{\llbracket #1 \rrbracket}} % semantics brackets
\newcommand{\pand}{\wedge} % conjunction
\newcommand{\por}{\vee} % disjunction
\newcommand{\pnot}{\neg} % negation
\newcommand{\pequals}{\leftrightarrow} % biconditional
\newcommand{\npequals}{\nleftrightarrow} % exclusive disjunction
\newcommand{\mequals}{\Leftrightarrow} % equivalence (meta-level)
\newcommand{\pimplies}{\rightarrow} % conditional
\newcommand{\mimplies}{\Rightarrow} % implication (meta-level)
\newcommand{\defeq}{\vcentcolon=} % defining equals
\DeclarePairedDelimiter\abs{\lvert}{\rvert} % absolute value

\subtitle{4. Feature Modeling}
\author{Elias Kuiter}

\begin{document}

% TITLE SLIDE

\maketitle

% SLIDE TEMPLATE

%\setbeamercolor{title}{fg=black}
%\setbeamercolor{frametitle}{fg=black}
\setbeamertemplate{frametitle}{{\huge~\\\insertsubsection~\insertframetitle}}
\setbeamertemplate{footline}[text line]{\parbox{\linewidth}{\vspace*{-10pt}\hspace{0pt}%
	\insertshortauthor\phantom{g\insertpagenumber}%
	\hfill%
	\inserttitle%
	\ifx \insertsubtitle \empty \else \ -- \insertsubtitle\fi%
	\ifx \insertsectionhead \empty \else \ -- \insertsectionhead\fi%
	\hfill%
	\phantom{g\insertshortauthor}\insertpagenumber%
}}
%\defbeamertemplate{footline}{\begin{beamercolorbox}[sep=1em]{author in head/foot}\insertshortauthor\hfill\insertsection\hfill\insertframenumber\end{beamercolorbox}}
%\defbeamertemplate*{footline}{mytheme}{\begin{beamercolorbox}[sep=1em]{author in head/foot}\insertshortauthor\hfill\insertsection\hfill\insertframenumber\end{beamercolorbox}}

% OVERVIEW SLIDES

\newcommand{\overview}{
	\section*{Overview}
	\subsection*{Overview}
	\begin{frame}{-- \insertsubtitle}
		\begin{multicols}{3}
			\tableofcontents
		\end{multicols}
	
		\begin{flushright}
			\footnotesize
			Author: \insertauthor
			
			Date: \insertdate
		\end{flushright}
	\end{frame}
}
% temporarily added slide to have a lecture overview 
\overview

% temporarily removed
%\begin{frame}{Lecture Overview -- \insertsubtitle}
%	\tableofcontents[hideallsubsections]
%\end{frame}

\AtBeginSection[]{%
	\begin{frame}{Lecture Overview -- \insertsubtitle}
		\tableofcontents[currentsection,hideothersubsections]
	\end{frame}
}

\newcommand{\sectionend}{\addtocontents{toc}{\newpage}}


\section{Feature Models}

% first: motivation and syntax of feature models/diagrams (how can a human model features sensibly?)
\subsection{Recap: Software Product Lines and Features}
\begin{frame}{\insertsubsection\ \mytitlesource{\fospl}}
	\leftorright{
		\mydefinition{Software Product Line}{
			\begin{itemize}
				\item set of software-intensive systems (aka products)
				\item sharing a common, managed set of features
				\item satisfying the needs of a domain
				\item developed from a common set of core assets (reuse)
			\end{itemize}
		}
		\href{https://commons.wikimedia.org/wiki/File:Geely_assembly_line_in_Beilun,_Ningbo.JPG}{\includegraphics[width=\linewidth,trim=0 50 0 240,clip]{car-manufacturing}}
	}{
		\mydefinition{Feature}{
			\begin{itemize}
				\item domain abstraction
				\item used for communication by stakeholders (e.g., manager, developer, tester, customer, marketing)
				\item specifies differences between products
			\end{itemize}
		}
		\myexample{Examples}{
			\begin{itemize}
				\item \todots
			\end{itemize}
		}
		\todo{scoping?}
		\todo{SPL has been explained, HCS, feature also. only recap here}
	}
\end{frame}

\subsection{Configuring Features}
\begin{frame}{\insertsubsection}
	\todo{examples for simple feature models without (or with few) dependencies}
	
	\todo{Configuring a Sandwich}

	\todo{Configuring ...}
\end{frame}

\xkcd{2369}

\subsection{Configuring Features with Dependencies}
\begin{frame}{\insertsubsection}
	\partofpage{70}{\myexampletight{Configuring a Notebook}{\only<1,3->{\includegraphics[width=\linewidth]{thinkpad-x1-yoga-display}}\only<2|handout:0>{\includegraphics[width=\linewidth]{thinkpad-x1-yoga-display-invalidconf}}}}
\end{frame}

\begin{frame}{\insertsubsection}
	\partofpage{70}{\myexampletight{Still Configuring a Notebook}{\includegraphics[width=\linewidth]{thinkpad-x1-yoga-office}}}
\end{frame}

\begin{frame}{\insertsubsection}
	~\hfill\partofpage{60}{\myexampletight{Configuring a Car}{\includegraphics[width=\linewidth]{toyota-aygo-wheels}}}
\end{frame}

\begin{frame}{\insertsubsection}
	~\hfill\partofpage{60}{\myexampletight{Configuring a Car with a Weird Price}{\centering\includegraphics[width=.55\linewidth]{toyota-aygo-costs}}}
\end{frame}

\begin{frame}{\insertsubsection}
	~\hfill\partofpage{60}{\myexampletight{Configuring a Car with 8 Wheels!}{\includegraphics[width=\linewidth]{toyota-aygo-costs3}}}
\end{frame}

\begin{frame}{\insertsubsection}
	\partofpage{70}{\myexampletight{Configuring a German Car}{\includegraphics[width=\linewidth]{bmw-series1-confassistant-bluetooth}}}
\end{frame}

\subsection{Configurations}

\newcommand{\feat}[1]{{\emph{#1}}}

\begin{frame}{\insertsubsection}
	\leftorright{
		\mydefinition{Configuration}{
			\begin{itemize}
				\item a \emph{configuration} \deutsch{Konfiguration} over a set of features $F$ selects and deselects features in $F$
				\item formally: a pair $(S, D)$ such that $S, D \subseteq F$ and $S, D$ are disjoint ($S \cap D = \varnothing$)
				\item is \emph{complete} \deutsch{vollständig} if all features are covered ($S \cup D = F$) and \emph{partial} \deutsch{partiell} otherwise
				\item a complete configuration is \emph{valid} \deutsch{gültig} if it ``makes sense'' in the domain and \emph{invalid} \deutsch{ungültig} otherwise
    			\item we often abbreviate complete configurations with $S \equiv (S, F \setminus S)$
			\end{itemize}
		}
	}{
		\myexample{A Configurable Database}{
			Feature set $F = \{ConfigDB, Get, Put, Delete,$
			\hspace*{22mm}$Transactions, Windows, Linux\}$
			
			Examples for \emph{complete} configurations:
			\begin{itemize}
				\item \emph{Valid} (read-only database on Windows):
					$(\{C, G, W\}, \{P, D, T, L\})$
				\item \emph{Valid} (fully functional database on Linux):
					$(\{C, G, P, D, T, L\}, \{W\}\}$
				\item \emph{Invalid} ($\lightning$ no operating system):
					$(\{C, G\}, \{P, D, T, W, L\})$
				\item \emph{Invalid} (transactions $\lightning$ read-only database):
					$(\{C, G, T, L\}, \{P, D, W\})$
			\end{itemize}
			Examples for \emph{partial} configurations:
			
			$(\{C, G\}, \{P, D\})$, $(\varnothing, \varnothing)$, \ldots
		}
		
	}
\end{frame}

\subsection{Characterizing Valid Configurations}

\begin{frame}{\insertsubsection}
	\leftorright{
		\mynote{Valid Configuration}{
			A complete configuration over $F$ is valid if it ``makes sense'' in the domain.
			\emph{\color{red}{$\leadsto$ ``makes sense''?}}
		}

		\mydefinition{First Approach: Natural Language}{
			\begin{itemize}
				\item informal description of relationships between features in $F$
				\item a complete configuration $S$ is \emph{valid} if it conforms to the description
    			\item[+] succinct
    			\item[--] sometimes ambiguous, not machine-readable
			\end{itemize}
		}
	}{
		\myexample{A Configurable Database}{
			``A \feat{configurable database} has an API that allows for at least one of the request types \feat{Get}, \feat{Put}, or \feat{Delete}.
			Optionally, the database can support \feat{transactions}, provided that the API allows for Put or Delete requests.
			Also, the database targets a supported operating system, which is either \feat{Windows} or \feat{Linux}.''
		}
	}
\end{frame}

\begin{frame}{\insertsubsection}
	\leftorright{
		\mynote{Valid Configuration}{
			A complete configuration over $F$ is valid if it ``makes sense'' in the domain.
			\emph{\color{red}{$\leadsto$ ``makes sense''?}}
		}

		\mydefinition{Second Approach: Configuration Map}{
			\begin{itemize}
				\item a \emph{configuration map} over $F$ is a set of complete configurations $C \subseteq F$
				\item a complete configuration $S$ is \emph{valid} if it occurs in the configuration map ($S \in C$)
				\item also known as product map
    			\item[+] precise
				\item[--] unreadable, redundant
				\item[--] explodes in size ($0 \leq \abs{C} \leq 2^{\abs{F}}$)
			\end{itemize}
		}
	}{
		\myexample{A Configurable Database}{
			Feature set $F = \{ConfigDB, Get, Put, Delete,$
			\hspace*{22mm}$Transactions, Windows, Linux\}$
			
			Configuration map:
			
			\small
			\leftandright{
				{\color{blue}$\{C,G,W\}$}\\
				$\{C,P,W\}$\\
				$\{C,G,P,W\}$\\
				$\{C,D,W\}$\\
				$\{C,G,D,W\}$\\
				$\{C,P,D,W\}$\\
				$\{C,G,P,D,W\}$\\
				$\{C,P,T,W\}$\\
				$\{C,G,P,T,W\}$\\
				$\{C,D,T,W\}$\\
				$\{C,G,D,T,W\}$\\
				$\{C,P,D,T,W\}$\\
				$\{C,G,P,D,T,W\}$
			}{
				$\{C,G,L\}$\\
				$\{C,P,L\}$\\
				$\{C,G,P,L\}$\\
				$\{C,D,L\}$\\
				$\{C,G,D,L\}$\\
				$\{C,P,D,L\}$\\
				$\{C,G,P,D,L\}$\\
				$\{C,P,T,L\}$\\
				$\{C,G,P,T,L\}$\\
				$\{C,D,T,L\}$\\
				$\{C,G,D,T,L\}$\\
				$\{C,P,D,T,L\}$\\
				{\color{blue}$\{C,G,P,D,T,L\}$}
			}
		}
	}
\end{frame}

\begin{frame}{\insertsubsection}
	\centering
	\includegraphics[width=0.48\linewidth]{products-in-excel}

	\textbf{Can we do better?}
\end{frame}

\subsection{Feature Models}

\newcommand{\dbmpl}{
	\myexampletight{A Configurable Database}{
		\centering
		\featureDiagramConfigurableDatabase

		\featureDiagramLegend
	}
}

\begin{frame}{\insertsubsection\ \mytitlesource{\fospl}}
	\leftorright{
		\dbmpl
	}{
		\mydefinition{Feature Model \deutsch{Feature-Modell}}{
			\begin{itemize}
				\item hierarchy of features
				\item dependencies between features modeled by tree and cross-tree constraints
				\item \emph{tree constraints}: defined by the hierarchy
				\item \emph{cross-tree constraints}: propositional formulas over features
				\item \emph{abstract features} are used to group other features
				\item \emph{concrete features} have an implementation
    			\item also known as feature diagram or feature tree
			\end{itemize}
		}
	}
\end{frame}

\begin{frame}{\insertsubsection\ \mytitlesource{\fospl}}
	\leftorright{
		\dbmpl
	}{
		\mydefinition{Tree Constraints}{
			\begin{itemize}
				\item each feature requires its parent
				\item an \emph{optional feature} can be (de-)selected freely when its parent is selected
				\item a \emph{mandatory feature} is required by its parent
				\item \emph{or group}: at least one feature must be selected when the parent is selected
				\item \emph{alternative group}: exactly one feature must be selected when the parent is selected
			\end{itemize}
		}
		\mydefinition{Cross-Tree Constraints}{
			\begin{itemize}
				\item a list of propositional formulas expressing further dependencies between features
			\end{itemize}
		}
	}
\end{frame}

\begin{frame}{\insertsubsection\ \mytitlesource{\fospl}}
	\leftorright{
		\dbmpl
	}{
		\myexample{Valid Configurations \todo{animate}}{
			\begin{itemize}
				\item \emph{Valid} (read-only database on Windows):
					$(\{C, G, W\}, \{P, D, T, L\})$
				\item \emph{Valid} (fully functional database on Linux):
					$(\{C, G, P, D, T, L\}, \{W\}\}$
				\item \emph{Invalid} ($\lightning$ no operating system):
					$(\{C, G\}, \{P, D, T, W, L\})$
				\item \emph{Invalid} (transactions $\lightning$ read-only database):
					$(\{C, G, T, L\}, \{P, D, W\})$
			\end{itemize}
			
		}
	}
\end{frame}

\begin{frame}{\insertsubsection\ \mytitlesource{\fospl}}
	\leftorright{
		\todo{other feature diagram example(s) which shows off how notation can also be used}
	}{
		\mynote{Notation}{
			\begin{itemize}
				\item abstract and concrete features can be assigned arbitrarily
    			\item groups can be used anywhere
				\item directly below groups, no optional or mandatory markers are allowed
			\end{itemize}
		}
	}
\end{frame}

\subsection{Advantages of Feature Modeling}
\begin{frame}{\insertsubsection}
	\leftandright{
		\textbf{Making Tacit Knowledge Explicit}
		\mynote{Interview with Practitioners \mysource{\href{https://link.springer.com/chapter/10.1007/978-3-319-11653-2_19}{Berger~et~al.~2014}}}{
			\mycite{I think the best [about feature modeling] is you can see relationships, to actually know what configurations are allowed and what are not allowed. That was also not so easy to express in the past [\ldots] This is from the developer’s point of view. But it’s also, we can see that from the, say project development, it’s also important, because before we noticed that \emph{the same functionality was implemented twice} within the same project, basically they haven’t realized that. They implemented the same features.}
		}
	}{
		\textbf{Tool Support for Automation}

		\todo{FeatureIDE, pure::variants screenshot?}
	}
\end{frame}

% advantages over Natural Language, List of Configurations

\lessonslearned{
	\item Explicit knowledge about features matters.
	\item Feature diagrams are a visual language for communicating feature models.
}{
	\item Thorsten Berger et al. (2013): \href{https://doi.org/10.1145/2430502.2430513}{A Survey of Variability Modeling in Industrial Practice}
	\item Damir Nešić et al. (2019): \href{https://doi.org/10.1145/3338906.3338974}{Principles of Feature Modeling}
}{
	Think of a soft- or hardware domain, then model its variability in a small feature diagram (with pen and paper or in FeatureIDE).

	What do you (dis-)like about the feature diagram notation? Do you need cross-tree constraints to model your domain?
}

\sectionend

% second: semantics of feature diagrams and other modeling notations (how can we encode features machine-readably?)
\section{Representations and Translations}

this notation is already nice for communication, but semantics matter (for large models, it does not suffice to look sharply)

this section shall teach the relationship between FMs and formulas and FMs and sets (i.e., Damiani 2020, Batory 2005), so: FM semantics
also, (valid) total configurations should be explained here (how a computer can check them, this can be checked easily when an FM is encoded eg as runtime variability, but all other SAT-based questions are hard to answer)

first explain product lists / sets and configuration validity (ie, set membership):
$ext-sem (FM) = \{ C \mid C in 2^F | C satisfies all FM rules \}$
this is a nice (readable) semantics but impractical to check, so we need Formulas


at the end (what else is there?): in practice, we also have non-Boolean features/attributes/constraints over attributes (more details on efficiency in third block)





\subsection{Feature Diagrams}

show running example as a feature diagram with cross-tree-constraints

explain notation

lego example?

discuss pros/cons

\subsection{Universal Variability Language}

\subsection{Propositional Formulas}

why is this needed? (forward ref?)

show running example as a formula

explain intuition behind elements of formula

CNF

\subsection{Transforming Diagrams into Formulas}

formal algorithm for transformation into FOL (and then CNF?)

+ example

% CNF is a universal language for saving formulas, maybe explain it here?

\subsection{Other Representations}

list of products, excel sheet, no explicit model (in motivation?), grammars, \dots

variations of feature models (e.g. cardinalities, non-boolean)

discuss pros/cons

Linux? KConfig? tri-state features

pure::variants?

%(state BDD?)
% probably not - (knowledge compilation: there are many nuances between CNF and BDD, maybe discuss?)

\lessonslearned{
	\item To understand large configuration spaces, formal semantics and machine-readable representations matter.
	\item Propositional formulas satisfy many (though not all) needs for such a representation.
}{
	\item Don Batory (2005): \href{https://doi.org/10.1007/11554844_3}{Feature Models, Grammars, and Propositional Formulas}
	\item Alexander Knüppel et al. (2017): \href{https://doi.org/10.1145/3106237.3106252}{Is There a Mismatch Between Real-World Feature Models and Product-Line Research?}
}{
	Figure out how many valid configurations your feature diagram represents.
	When is this easy?

	Translate (a part of) your feature diagram into a formula, then into CNF, and finally into a DIMACS file.
	How might this be useful?
}

\sectionend

% third: analyses of feature models (what can a computer tell us, which we cannot tell just by looking?)
\section{Automated Analyses}

this section should teach all usual FM/cfg analyses - some in more detail, some only high-level (other SPL analyses come in analyses.tex)

revisit what the class of programs named "solver" does, and why it is important that we use OTS solvers and profit from their optimizations/contests - idea: reduce practical problems to solvers (sim. to TheoInf/Logics)
only with Boolean formulas can we get really performant results today

extensions to simple SAT solvers (maybe explain when needed):
- when SAT = true, some give a satisfying assignment (for optimizations)
- wen SAT = false, some give a MUS (for explanations)
- some can count, called sharp SAT (or keep them completely separated from SAT?) (for model counting)

ask questions:
am i developing unused code?
how to find out whether a partial configuration is valid? (show decision propagation in FeatureIDE)

for each analysis, also list the sharp SAT encoding

as interaction at the end: how can we leverage when a SAT solver also gives us a sat-assignment?

proposal (explain with one running example):

Q: is the FM consistent?
Tool: SAT (explain)
A: SAT(FM)

Q: is a feature core/dead?
Tool: SAT
A: not SAT(FM and not F) etc.

Q: WHY is this the case?
Tool: SAT-MUS (explain)
A: ...

Q: to WHICH DEGREE is a FM consistent (i.e., degree of freedom / variability factor) or a feature core/dead?
Tool: sharp SAT (explain)
A: ...

(then list some other analyses quickly)

on a summary page, state that sharp SAT > 0 implies SAT, SAT-MUS iff SAT
then, as an interaction, explain what SAT-ASS is and ask how it can be used
this way, students are taught that SAT* solvers are tools for answering questions, and that if they support more features, more/faster analyses are possible

% master theses T. Günther, S. Ananieva


\subsection{Inconsistencies in Feature Models}

\subsection{Slide Title 1}
\begin{frame}{~}
	\ldots
\end{frame}

show examples of inconsistencies/anomalies in feature models (interaction?)

questions?

\subsection{Automated Reasoning}

solver!

SAT:
* DIMACS / CNF
* sat: example solution
* unsat: MUS (explanations)

SMT? \#SAT?

demo?

\subsection{Void Feature Model}

slide on SAT solving

including explanations

\subsection{Core and Dead Features}

\subsection{Validity of Configurations}

partial configurations

%\subsection{Other Analyses}
%
%e.g., partial configurations, model counting
%
%other questions about feature models

\subsection{Tool Support}

FeatureIDE configurator

show how FeatureIDE automatically detects the anomalies from the beginning

%\subsection{Edits to Feature Models}
% put this in a evolution + maintenance chapter



\lessonslearned{
	\item Automated SAT-based analyses help in understanding large configuration spaces.
	\item There are still many open problems regarding encodings and applications (e.g., \#SAT).
}{
	\item David Benavides et al. (2010): \href{https://doi.org/10.1016/j.is.2010.01.001}{Automated Analysis of Feature Models 20 Years Later: A Literature Review}
	\item Thomas Thüm et al. (2009): \href{https://doi.org/10.1109/ICSE.2009.5070526}{Reasoning About Edits to Feature Models}
	\item Chico Sundermann et al. (2021): \href{https://doi.org/10.1145/3442391.3442404}{Applications of \#SAT Solvers on Feature Models}
}{
	% alternative: find inconsistencies in your feature diagram
	Recall that some $f \in F$ is dead iff $\pnot SAT(\phi \pand f)$.
	Suppose you want to know \emph{all} dead features of $\phi$.
	Naively, you may formulate $\abs{F}$ SAT queries.
	
	Can this be improved?
	
	(Hint: Some SAT solvers return an example model $M_\psi$ when $\psi$ is satisfiable.)
}

\mode<beamer>{
	\begin{frame}{\inserttitle}
		\lectureseriesoverview
	\end{frame}

	\contentoverview
}


\end{document}
