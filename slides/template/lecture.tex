\usepackage{../template/beamerthemeuulm} % use the inofficial uulm beamer theme
\setfaculty{infIngPsy} % set the color scheme for your faculty here [med/infIngPsy/math/nat]
\uulmlogos{,uulm,,unibe-flat-crop,,ovgu-blue,}
\setmycolumnsdefault{animation=forget} % animate all columns per default
%\recordingtrue % special recording mode for use with a greenscreen, gives you space to show yourself in a layer in front of the slides, has no effect in the handout mode

\title{Software Product Lines} % short title is used for the slide footer but optional

% IMPORTED PACKAGES

\usepackage{multicol} % used temporarily for the lecture overview
\usepackage{stmaryrd} % \lightning in modeling lecture

% INCLUDED TEMPLATE FILES

% default colors
% used in standalone pictures and when no university is selected
\definecolor{black}{HTML}{000000}
\definecolor{green}{HTML}{33BB33}
\definecolor{red}{HTML}{BB3333}
\definecolor{orange}{HTML}{BB6600}
\definecolor{blue}{HTML}{3333BB}
\definecolor{uulmaccent}{HTML}{999999}

\definecolor{uulmlogoblue}{named}{blue}
\definecolor{uulmblue}{named}{blue}

\ifdarkmode
	\colorlet{green}{green!85!white}
	\colorlet{red}{red!85!white}
	\colorlet{orange}{orange!85!white}
	\colorlet{blue}{blue!85!white}
	\setbeamercolor{section in toc shaded}{fg=black}
	\setbeamertemplate{section in toc shaded}[default][50]
\fi

% TYPICAL COMMANDS FOR LECTURES

\renewcommand{\emph}[1]{{\color{blue}\textbf{#1}}}

\newcommand{\deutsch}[1]{{\color{blue}(#1)}}
\newcommand{\deutschertitel}[1]{{\tiny\deutsch{#1}}}

\newcommand{\mycite}[1]{``#1''}
\newcommand{\mytitlesource}[1]{{\tiny\normalfont\mbox{[#1]}}}
\newcommand{\mysource}[1]{\ifthenelse{\equal{#1}{}}{}{\phantom{.}~\hfill~\mytitlesource{#1}}}

\newcommand{\todo}[1]{{\color{red}\textbf{[#1]}}}
\newcommand{\fodo}[1]{\todo{\footnote{\todo{#1}}}}
\newcommand{\todots}{\todo{\ldots}}

\newcommand{\textheightwithtitle}{.85\textheight}
\newcommand{\textheightwithouttitle}{.975\textheight}

% COMMANDS FOR PROPOSITIONAL FORMULAS AND MATHEMATICAL NOTATIONS

\newcommand{\sem}[1]{\ensuremath{\llbracket #1 \rrbracket}} % semantics brackets
\newcommand{\pand}{\wedge} % conjunction
\newcommand{\por}{\vee} % disjunction
\newcommand{\pnot}{\neg} % negation
\newcommand{\pequals}{\leftrightarrow} % biconditional
\newcommand{\npequals}{\nleftrightarrow} % exclusive disjunction
\newcommand{\mequals}{\Leftrightarrow} % equivalence (meta-level)
\newcommand{\pimplies}{\rightarrow} % conditional
\newcommand{\mimplies}{\Rightarrow} % implication (meta-level)
\newcommand{\defeq}{\vcentcolon=} % defining equals
\newcommand{\power}[1]{\mathcal{P}(#1)} % power set
\newcommand{\refslide}[1]{\hyperlink{#1}{(see Slide \autoref{#1})}} % link to slide

\usepackage{mathtools} % required for absolute value in modeling lecture
\DeclarePairedDelimiter\abs{\lvert}{\rvert} % absolute value

% COMMANDS TO INCLUDE XKCDs

\newcommand{\xkcd}[1]{
	\begin{frame}
		\centering%
		\href{https://xkcd.com/#1/}{\includegraphics[height=80mm]{#1}}
	\end{frame}
}
\newcommand{\widexkcd}[1]{
	\begin{frame}
		\centering%
		\href{https://xkcd.com/#1/}{\includegraphics[width=\linewidth]{xkcd/#1}}
	\end{frame}
}

\usepackage{xstring}

\newcommand{\featureDiagramConfigurableDatabase}[1][]{
	\featureDiagram{
		\subnode{ConfigDB#1}{ConfigDB},concrete
		[\subnode{API#1}{API},abstract,mandatory
			[\subnode{Get#1}{Get},concrete,or]
			[\subnode{Put#1}{Put},concrete]
			[\subnode{Delete#1}{Delete},concrete]
		]
		[\subnode{Transactions#1}{Transactions},concrete,optional]
		[\subnode{OS#1}{OS},abstract,mandatory
			[\subnode{Windows#1}{Windows},concrete,alternative]
			[\subnode{Linux#1}{Linux},concrete]
		]
	}
	
	$Transactions \pimplies Put \por Delete$
}

\newcommand{\featureDiagramConfigurableDatabaseNoAbstract}[1][]{
	\featureDiagram{
		\subnode{ConfigDB#1}{ConfigDB},concrete
		[\subnode{API#1}{\textcolor{black!30}{API}},abstract,mandatory
			[\subnode{Get#1}{Get},concrete,or]
			[\subnode{Put#1}{Put},concrete]
			[\subnode{Delete#1}{Delete},concrete]
		]
		[\subnode{Transactions#1}{Transactions},concrete,optional]
		[\subnode{OS#1}{\textcolor{black!30}{OS}},abstract,mandatory
			[\subnode{Windows#1}{Windows},concrete,alternative]
			[\subnode{Linux#1}{Linux},concrete]
		]
	}

	$Transactions \pimplies Put \por Delete$
}

\newcommand{\featureDiagramConfigurableDatabaseNoAbstractNoCore}[1][]{
	\featureDiagram{
		\subnode{ConfigDB#1}{\textcolor{black!30}{ConfigDB}},concrete
		[\subnode{API#1}{\textcolor{black!30}{API}},abstract,mandatory
			[\subnode{Get#1}{Get},concrete,or]
			[\subnode{Put#1}{Put},concrete]
			[\subnode{Delete#1}{Delete},concrete]
		]
		[\subnode{Transactions#1}{Transactions},concrete,optional]
		[\subnode{OS#1}{\textcolor{black!30}{OS}},abstract,mandatory
			[\subnode{Windows#1}{Windows},concrete,alternative]
			[\subnode{Linux#1}{Linux},concrete]
		]
	}

	$Transactions \pimplies Put \por Delete$
}

\newcommand{\featureDiagramWaffle}{
	\featureDiagram{
		Waffle,concrete
		[Topping,optional,abstract
			[Sugar,concrete,or]
			[Cream,concrete]
			[Cherries,concrete]
			[Nutella,concrete]
			[Crumbles,concrete
				[Chocolate,concrete,alternative]
				[Colored,concrete]]]
		[Accessories,abstract,mandatory
			[Plate,concrete,mandatory]
			[Fork,concrete,optional
				[Plastic,concrete,alternative]
				[Wood,concrete]]]
		[Customer,abstract,mandatory
			[Adult,abstract,alternative]
			[Child,abstract]]
	}

	$Sugar$\\
	$Cherries \pimplies Sugar \pand Fork$\\
	$Nutella \por Crumbles \pimplies Child$\\
	$Fork \pimplies Adult$
}

\newcommand{\featureDiagramGraphs}{
	\featureDiagram{
		Graph,concrete
		[Nodes,mandatory,abstract
			[Colored,optional,concrete]]
		[Edges,mandatory,abstract
			[Directed,optional,concrete]
			[Weighted,optional,concrete]]
		[Algorithms,mandatory,abstract,
			[ShortestPath,optional,concrete]]
	}

	$ShortestPath \pimplies Weighted$
}

% has been used earlier for practical task in Lecture 4b. not needed anymore?
\newcommand{\featureDiagramABCDEFG}{
	\featureDiagram{
		Root,concrete
		[A,concrete,optional
			[C,concrete,or]
			[D,concrete]
		]
		[B,concrete,mandatory
			[E,concrete,alternative]
			[F,concrete]
		]
	}
}

\newcommand{\featureDiagramABCDE}{
	\featureDiagram{
		A,concrete
		[B,concrete,optional]
		[C,concrete,mandatory
			[D,concrete,or]
			[E,concrete]
		]
	}
}

\newcommand{\featureDiagramEightOptionalFeatures}{
	\featureDiagram{
		SPL,abstract
		[F1,concrete,optional]
		[F2,concrete,optional]
		[F3,concrete,optional]
		[F4,concrete,optional]
		[F5,concrete,optional]
		[F6,concrete,optional]
		[F7,concrete,optional]
		[F8,concrete,optional]
	}
}

\newcommand{\featureDiagramLego}{
	\featureDiagram{
		Lego Manikin,abstract
			[Headpiece,abstract,optional
				[Helmet,concrete,alternative]
				[Hat,concrete]]
			[Head,concrete,mandatory]
			[Item,abstract,optional
				[Brush,concrete,or]
				[Phone,concrete]]
			[Shirt,concrete,mandatory]
			[Pants,abstract,mandatory
				[Red,concrete,alternative]
				[Blue,concrete]]
	}
	%\\$Helmet \pimplies \pnot Phone$
}


% STYLE COMMANDS FOR FEATURE DIAGRAMS

\usepackage{forest}
\usepackage{xcolor}
\usetikzlibrary{angles}
\usetikzlibrary{tikzmark}
\definecolor{drawColor}{RGB}{128 128 128}
\newcommand{\circleSize}{0.175em}
\newcommand{\angleSize}{0.75em}

\forestset{
	/tikz/mandatory/.style={
		circle,fill=drawColor,
		draw=drawColor,
		inner sep=\circleSize
	},
	/tikz/optional/.style={
		circle,
		fill=white,
		draw=drawColor,
		inner sep=\circleSize
	},
	featureDiagram/.style={
		for tree={
			text = black, % required for dark mode
			text depth = 0,
			parent anchor = south,
			child anchor = north,
			draw = drawColor,
			edge = {draw=drawColor},
		}
	},
	featureDiagramEmpty/.style={
	},
	/tikz/abstract/.style={
		fill = blue!85!cyan!5,
		draw = drawColor
	},
	/tikz/concrete/.style={
		fill = blue!85!cyan!20,
		draw = drawColor
	},
	mandatory/.style={
		edge label={node [mandatory] {} }
	},
	optional/.style={
		edge label={node [optional] {} }
	},
	or/.style={
		tikz+={
			\path (.parent) coordinate (A) -- (!u.children) coordinate (B) -- (!ul.parent) coordinate (C) pic[fill=drawColor, angle radius=\angleSize]{angle};
		}	
	},
	/tikz/or/.style={
	},
	alternative/.style={
		tikz+={
			\path (.parent) coordinate (A) -- (!u.children) coordinate (B) -- (!ul.parent) coordinate (C) pic[draw=drawColor, angle radius=\angleSize]{angle};
		}	
	},
	/tikz/alternative/.style={
	},
	/tikz/placeholder/.style={
	},
	collapsed/.style={
		rounded corners,
		no edge,
		for tree={
			fill opacity=0,
			draw opacity=0,
			l = 0em,
		}
	},
	/tikz/hiddenNodes/.style={
		midway,
		rounded corners,
		draw=drawColor,
		fill=white,
		minimum size = 1.2em,
		minimum width = 0.8em,
		scale=0.9
	},
}

\newcommand{\legend}{
	\matrix [draw=drawColor,anchor=north west] at (current bounding box.north east) {
		\node [label=center:\underline{Legend:}] {}; \\
		\node [abstract,label=right:Abstract Feature] {}; \\
		\node [concrete,label=right:Concrete Feature] {}; \\
		\node [mandatory,label=right:Mandatory] {}; \\
		\node [optional,label=right:Optional] {}; \\
			\filldraw[drawColor] (0.1,0) - +(-0,-0.2) - +(0.2,-0.2)- +(0.1,0);
			\draw[drawColor] (0.1,0) -- +(-0.2, -0.4);
			\draw[drawColor] (0.1,0) -- +(0.2,-0.4);
			\fill[drawColor] (0,-0.2) arc (240:300:0.2);
		\node [or,label=right:Or Group] {}; \\
		\draw[drawColor] (0.1,0) -- +(-0.2, -0.4);
			\draw[drawColor] (0.1,0) -- +(0.2,-0.4);
			\draw[drawColor] (0,-0.2) arc (240:300:0.2);
		\node [alternative,label=right:Alternative Group] {}; \\
	};
}

\newcommand{\featureDiagramWithLegend}[1]{
	\begin{forest}
		featureDiagram[#1]
		\matrix [anchor=north west] at (current bounding box.north east) {
			\node [placeholder] {}; \\
		};
		{\small\legend}
	\end{forest}
}

\newcommand{\featureDiagram}[1]{
	\begin{forest}
		featureDiagram[#1]
	\end{forest}
}

\newcommand{\featureDiagramLegend}{
	\begin{forest}
		featureDiagramEmpty[]
		{\small\legend}
	\end{forest}
}

\newcommand{\featureDiagramOverlay}[1]{
	\begin{tikzpicture}[overlay,remember picture]
		#1
	\end{tikzpicture}
}

\newcommand{\featureEmph}[2][draw=black]{
		\foreach \f in {#2} {
			\node [fit=\f,fill opacity=0.4,line width=0.6pt,rounded corners,#1] {};
		}
}

\newcommand{\featureDeemph}[2][fill=white]{
	\foreach \f in {#2} {
		\node [fit=\f,fill opacity=0.4,line width=0.6pt,rounded corners,#1] {};
	}
}

\newcommand{\featureSelected}[1]{
	\featureEmph[fill=green,draw=green,fill opacity=0.1,draw opacity=0.2]{#1}
}

\newcommand{\featureDeselected}[1]{
	\featureEmph[fill=red,draw=red,fill opacity=0.1,draw opacity=0.2]{#1}
}



% GRAPHICSPATH

% TODO use git module for public repositories
% requires symbolic links
% git clone git@github.com:SoftVarE-Group/SlideTemplate.git C:\Users\...\SlideTemplate
% mklink /J template C:\Users\...\SlideTemplate
% git clone git@spgit.informatik.uni-ulm.de:thuem/slides.git C:\Users\...\ThomasSlides
% mklink /J thomasslides C:\Users\...\ThomasSlides
% git clone git@github.com:SoftVarE-Group/Papers.git C:\Users\...\SoftVarE-Papers
% mklink /J softvarepapers C:\Users\...\SoftVarE-Papers
% git clone git@github.com:SoftVarE-Group/Slides.git C:\Users\...\SoftVarE-Slides
% mklink /J softvareslides C:\Users\...\SoftVarE-Slides
\graphicspath{
	{../template/pics/logos/}
	{../template/pics/nature/}
	{../template/pics/uulm/}
	{../thomasslides/}
	{../softvarepapers/}
	{../softvareslides/}
	{../pics/}
	{../pics/cars/}
	{../pics/configurators/}
	{../pics/frameworks/}
	{../pics/graphs/}
	{../pics/interactions/}
	{../pics/linux/}
	{../pics/literature/}
	{../pics/lego/}
	{../pics/logos/}
	{../pics/mindmaps/}
	{../pics/people/}
	{../pics/preprocessors/}
	{../pics/products/}
	{../pics/projectcartoon/}
	{../pics/runtimevariability/}
	{../pics/testing/}
	{../pics/variabilitymodels/}
	{../pics/versioncontrol/}
	{../pics/xkcd/}
}

% LINKED LITERATURE

\newcommand{\bakarnaturallanguage}{\href{https://doi.org/10.1016/j.jss.2015.05.006}{Bakar~et~al.~2015}}
\newcommand{\czarneckithereandbackagain}{\href{https://ieeexplore.ieee.org/document/4339252}{Czarnecki~and~Wasowski~2007}}
\newcommand{\evaluatingsharpsatsolverslink}[1]{\href{https://github.com/SoftVarE-Group/Papers/raw/main/2020/2020-VaMoS-Sundermann.pdf}{#1}}
\newcommand{\evaluatingsharpsatsolvers}{\evaluatingsharpsatsolverslink{Sundermann~et~al.~2020}}
\newcommand{\essentialconfigurationcomplexitylink}[1]{\href{https://github.com/SoftVarE-Group/Papers/raw/main/2016/2016-ASE-Meinicke.pdf}{#1}}
\newcommand{\essentialconfigurationcomplexity}{\essentialconfigurationcomplexitylink{Meinicke~et~al.~2016}}
\newcommand{\featureide}{\href{https://link.springer.com/book/10.1007\%2F978-3-319-61443-4}{Meinicke~et~al.~2013}}
\newcommand{\fospl}{\href{http://link.springer.com/book/10.1007/978-3-642-37521-7}{Apel~et~al.~2013}}
\newcommand{\gof}{\href{https://learning.oreilly.com/library/view/design-patterns-elements/0201633612/}{Gang of Four}}
\newcommand{\icpl}{\href{https://dl.acm.org/doi/10.1145/2362536.2362547}{Johansen~et~al.~2012}}
\newcommand{\lehmanslaws}{\href{https://ieeexplore.ieee.org/iel3/5031/13795/00637156.pdf}{Lehman et al.\ 1997}}
\newcommand{\ludewiglichter}{\href{https://learning.oreilly.com/library/view/-/9781457184932/?ar}{Ludewig and Lichter}}
\newcommand{\reducingconfigurations}{\href{https://doi.org/10.1145/1960275.1960284}{Kim~et~al.~2011}}
\newcommand{\samplingsurvey}{\href{https://github.com/SoftVarE-Group/Papers/raw/main/2018/2018-SPLC-Varshosaz.pdf}{Varshosaz~et~al.~2018}}
\newcommand{\seeconomics}{\href{https://rds-ulm.ibs-bw.de/link?kid=027381854}{SE Economics}}
\newcommand{\seiwhitepaperspl}{\href{https://resources.sei.cmu.edu/asset_files/WhitePaper/2012_019_001_495381.pdf}{Northrop~et~al. 2012}}
\newcommand{\shereverseengineering}{\href{https://dl.acm.org/doi/abs/10.1145/1985793.1985856}{She~et~al.~2011}}
\newcommand{\softwareclonedetection}{\href{https://www.sciencedirect.com/science/article/abs/pii/S0950584913000323}{Rattan~et~al.~2013}}
\newcommand{\sommervillelink}[1]{\href{https://ulm.ibs-bw.de/aDISWeb/app?service=direct/0/Home/$DirectLink\&sp=SOPAC00\&sp=SAKSWB-IdNr1615420983}{#1}}
\newcommand{\sommerville}{\sommervillelink{Sommerville}}
\newcommand{\sple}{\href{https://link.springer.com/book/10.1007/3-540-28901-1}{Pohl~et~al.~2005}}
\newcommand{\thehumbleprogrammer}{\href{https://dl.acm.org/doi/10.1145/1283920.1283927}{The Humble Programmer}}
\newcommand{\thepragmaticprogrammer}{\href{https://learning.oreilly.com/library/view/the-pragmatic-programmer/9780135956977/}{The Pragmatic Programmer}}
\newcommand{\twodimensionalanalysislink}[1]{\href{https://github.com/SoftVarE-Group/Papers/raw/main/2019/2019-VariVolution-Thuem.pdf}{#1}}
\newcommand{\twodimensionalanalysis}{\twodimensionalanalysislink{Thüm~et~al.~2019}}
\newcommand{\universitycourses}{\href{https://link.springer.com/chapter/10.1007/978-3-030-30446-1_7}{Bittner~et~al.~2019}}

% REFERENCES TO OTHER LECTURES

\newcommand{\reflecture}[1]{(see Lecture~#1)}
\newcommand{\lectureintroduction}{\reflecture{1}}
\newcommand{\lectureruntime}{\reflecture{2}}
\newcommand{\lecturecloneandown}{\reflecture{3}}
\newcommand{\lecturemodeling}{\reflecture{4}}
\newcommand{\lectureembedded}{\reflecture{5}}
\newcommand{\lectureapps}{\reflecture{6}}
\newcommand{\lecturelanguages}{\reflecture{7}}
\newcommand{\lectureprocess}{\reflecture{8}}
\newcommand{\lectureinteractions}{\reflecture{9}}
\newcommand{\lectureanalyses}{\reflecture{10}}
\newcommand{\lecturetesting}{\reflecture{11}}
\newcommand{\lectureevonance}{\reflecture{12}}

% HACKS TO WORKAROUND PROBLEMS WITH THE TEMPLATE

% TODO command to add extra space after each section manually
\newcommand{\sectionend}{\addtocontents{toc}{\vspace{5mm}}}

% TODO temporary hack to layout the slide overview in two colums
\renewcommand{\lectureoverview}{
	\begin{frame}{\insertsubtitle}
		\begin{multicols}{2}
        	\setlength{\parskip}{0ex}
			\tableofcontents
		\end{multicols}
	\end{frame}
}

% TODO hack to make \leftandright work again in example boxes
\renewcommand{\myexample}[2]{
	\begin{tcolorbox}[title={#1}, before title={\setlength{\parskip}{0ex}\vphantom{/}},colback=blue!10,colframe=blue!30,coltitle=black,fonttitle=\bfseries,left=1mm,right=1mm,top=1mm,bottom=1mm]
		\begin{flushleft}
			#2
		\end{flushleft}
	\end{tcolorbox}
}

% UNIVERSITY-SPECIFIC ADJUSTMENTS
% or: a product line of product-line lectures ;-)

\newif\ifbern
\newif\ifmagdeburg
\newif\ifulm

\newcommand{\foruniversity}[4][]{\ifbern#2\else\ifmagdeburg#3\else\ifulm#4\else#1\fi\fi\fi}

\let\originalauthor\author
\renewcommand{\author}[1]{
	\foruniversity[\originalauthor{#1}]
		{\originalauthor{Timo Kehrer, Sandra Greiner}}
		{\originalauthor{Gunter Saake, Elias Kuiter}}
		{\originalauthor{Thomas Thüm, Chico Sundermann}}
}

\newcommand{\inputforuniversity}[1]{
	\foruniversity
		{\input{#1_bern}}
		{\input{#1_magdeburg}}
		{\input{#1_ulm}}
}