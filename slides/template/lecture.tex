\documentclass[
	aspectratio=169, % default is 43
	8pt, % font size, default is 11pt
	handout, % handout mode without animations, comment out to add animations
]{beamer}

\usepackage{../template/beamerthemeuulm} % use the inofficial uulm beamer theme
\setfaculty{infIngPsy} % set the color scheme for your faculty here [med/infIngPsy/math/nat]

% TODO use git module for public repositories
% requires symbolic links
% git clone git@github.com:SoftVarE-Group/SlideTemplate.git C:\Users\...\SlideTemplate
% mklink /J template C:\Users\...\SlideTemplate
% git clone git@spgit.informatik.uni-ulm.de:thuem/slides.git C:\Users\...\ThomasSlides
% mklink /J thomasslides C:\Users\...\ThomasSlides
% git clone git@github.com:SoftVarE-Group/Papers.git C:\Users\...\SoftVarE-Papers
% mklink /J softvarepapers C:\Users\...\SoftVarE-Papers
% git clone git@github.com:SoftVarE-Group/Slides.git C:\Users\...\SoftVarE-Slides
% mklink /J softvareslides C:\Users\...\SoftVarE-Slides
\graphicspath{{../template/pics/logos/}{../template/pics/nature/}{../template/pics/uulm/}{../thomasslides/}{../softvarepapers/}{../softvareslides/}{../pics/}{../pics/configurators/}{../pics/people/}{../pics/products/}{../pics/variabilitymodels/}{../pics/versioncontrol/}{../pics/xkcd/}}

%\usepackage[ngerman]{babel} % use this line for slides in German
%\recordingtrue % special recording mode for use with a greenscreen, gives you space to show yourself in a layer in front of the slides, has no effect in the handout mode

\title{Software Product Lines} % short title is used for the slide footer but optional

% LINKED LITERATURE

\newcommand{\fospl}{\href{http://link.springer.com/book/10.1007/978-3-642-37521-7}{Apel~et~al.~2013}}
\newcommand{\ludewiglichter}{\href{https://learning.oreilly.com/library/view/-/9781457184932/?ar}{Ludewig and Lichter}}
\newcommand{\seeconomics}{\href{https://rds-ulm.ibs-bw.de/link?kid=027381854}{SE Economics}}
\newcommand{\sommervillelink}[1]{\href{https://ulm.ibs-bw.de/aDISWeb/app?service=direct/0/Home/$DirectLink\&sp=SOPAC00\&sp=SAKSWB-IdNr1615420983}{#1}}
\newcommand{\sommerville}{\sommervillelink{Sommerville}}
\newcommand{\thehumbleprogrammer}{\href{https://dl.acm.org/doi/10.1145/1283920.1283927}{The Humble Programmer}}
\newcommand{\thepragmaticprogrammer}{\href{https://learning.oreilly.com/library/view/the-pragmatic-programmer/9780135956977/}{The Pragmatic Programmer}}

% TYPICAL COMMANDS FOR LECTURES

\renewcommand{\emph}[1]{{\color{blue}\textbf{#1}}}

\newcommand{\deutsch}[1]{{\color{blue}(#1)}}
\newcommand{\deutschertitel}[1]{{\tiny\deutsch{#1}}}

\newcommand{\mycite}[1]{``#1''}
\newcommand{\mytitlesource}[1]{{\tiny\normalfont\mbox{[#1]}}}
\newcommand{\mysource}[1]{\ifthenelse{\equal{#1}{}}{}{\phantom{.}~\hfill~\mytitlesource{#1}}}

\newcommand{\todo}[1]{{\color{red}\textbf{[#1]}}}
\newcommand{\fodo}[1]{\todo{\footnote{\todo{#1}}}}
\newcommand{\todots}{\todo{\ldots}}

\newcommand{\xkcd}[1]{
	\begin{frame}
		\centering%
		\href{https://xkcd.com/#1/}{\includegraphics[height=80mm]{#1}}
	\end{frame}
}

% IMPORTED PACKAGES

%\usepackage{adjustbox} % used for partofpage
%\usepackage{tcolorbox} % used for mydefinition, mynote, myexample
\usepackage{multicol} % used temporarily for the lecture overview
\usepackage{mathtools} % required for absolute value in modeling lecture

% INCLUDE COLOR SCHEME
% default colors
% used in standalone pictures and when no university is selected
\definecolor{black}{HTML}{000000}
\definecolor{green}{HTML}{33BB33}
\definecolor{red}{HTML}{BB3333}
\definecolor{orange}{HTML}{BB6600}
\definecolor{blue}{HTML}{3333BB}
\definecolor{uulmaccent}{HTML}{999999}

\definecolor{uulmlogoblue}{named}{blue}
\definecolor{uulmblue}{named}{blue}

\ifdarkmode
	\colorlet{green}{green!85!white}
	\colorlet{red}{red!85!white}
	\colorlet{orange}{orange!85!white}
	\colorlet{blue}{blue!85!white}
	\setbeamercolor{section in toc shaded}{fg=black}
	\setbeamertemplate{section in toc shaded}[default][50]
\fi


% COMMANDS TO LAYOUT AND ANNIMATE SLIDES

\newcommand{\lessonslearned}[3]{
	\subsection{Summary}
	\begin{frame}{\insertsection{} -- \insertsubsection}
		\leftorright{
			\mydefinition{Lessons Learned}{
				\begin{itemize}
					#1
				\end{itemize}
			}
			\mynote{Further Reading}{
				\small % references take space, can be a little smaller
				\begin{itemize}
					#2
				\end{itemize}
			}
		}{
			\myexample{Practice}{
				\begin{itemize}
					#3
				\end{itemize}
			}
		}
	\end{frame}
}

% TODO temporary hack to layout the slide overview in two colums
\renewcommand{\lectureoverview}{
%	\section*{Overview}
%	\subsection*{Overview}
	\begin{frame}{\insertsubtitle}
		\begin{multicols}{2}
			\tableofcontents
		\end{multicols}
	\end{frame}
}

\renewcommandx{\maketitle}[2][1=apr21-o25a,2=150]{
	{
	\usebackgroundtemplate{} % TODO temporary hack to enable missing pictures at title slide
	%\ifx {#1} \empty \else {\usebackgroundtemplate{\includegraphics[trim=0 0 0 #2,clip,width=\paperwidth]{#1}}} \fi     
	%\usebackgroundtemplate{\includegraphics[trim=0 0 0 #2,clip,width=\paperwidth]{#1}}
	\begin{frame}[plain]
		\vskip0pt plus 1filll
		\begin{beamercolorbox}[wd=\paperwidth,ht=4.5ex,dp=2ex,right]{titlebox}
			\LARGE\textbf{\inserttitle}\hspace*{20pt}
		\end{beamercolorbox}%
		\nointerlineskip%
		\begin{beamercolorbox}[wd=\paperwidth,ht=2.25ex,dp=1ex,right]{subtitlebox}
			\small 
			\ifx \insertsubtitle \empty \else \insertsubtitle\ $\vert$ \fi
			\insertauthor\
			\ifx \insertdate \empty \else $\vert$ \insertdate \fi
			\hspace*{20pt}
		\end{beamercolorbox}%
		\nointerlineskip%
		\begin{beamercolorbox}[wd=\paperwidth,ht=4.5ex,dp=2ex,left]{logobox}
			\centering
			\vspace{-1ex}
			\hspace{10pt}
			\includegraphics[height=4.5ex]{sp} % SPECIFY INSTITUTE LOGO HERE
			\hfill
			\includegraphics[height=4.5ex]{uulm}
			\hspace{10pt}
		\end{beamercolorbox}%
	\end{frame}
	}  
}

%
%\newcommand{\onlyleft}[1]{
%	\halfpage{#1}
%}
%
%\newcommand{\onlyright}[1]{
%	~\hfill
%	\halfpage{#1}
%}
%
%\newcommand{\leftorright}[2]{
%	\uncover<1>{\halfpage{#1}}
%	\hfill
%	\uncover<3->{\halfpage{#2}}
%}
%
%\newcommand{\rightorleft}[2]{
%	\uncover<3->{\halfpage{#1}}
%	\hfill
%	\uncover<1>{\halfpage{#2}}
%}
%
%\newcommand{\leftthenright}[2]{
%	\halfpage{#1}
%	\hfill\pause
%	\halfpage{#2}
%}
%
%\newcommand{\leftandright}[2]{
%	\halfpage{#1}
%	\hfill
%	\halfpage{#2}
%}
%
%\newcommand{\leftmiddleandright}[3]{
%	\thirdpage{#1}
%	\hfill
%	\thirdpage{#2}
%	\hfill
%	\thirdpage{#3}
%}
%
%\newcommand{\leftmiddleorright}[3]{
%	\uncover<1>{\thirdpage{#1}}
%	\hfill
%	\uncover<3>{\thirdpage{#2}}
%	\hfill
%	\uncover<5->{\thirdpage{#3}}
%}
%
%\newcommand{\halfpage}[1]{\partofpage{48}{#1}}
%
%\newcommand{\thirdpage}[1]{\partofpage{31}{#1}}
%
%\newcommand{\partofpage}[2]{
%	\adjustbox{valign=t}{\begin{minipage}{0.#1\textwidth}
%			\begin{flushleft}
%				#2
%			\end{flushleft}
%	\end{minipage}}
%}
%
%\newcommand{\mydefinition}[2]{
%	\begin{tcolorbox}[title=#1,colback=orange!10,colframe=orange!30,coltitle=black,fonttitle=\bfseries,left=1mm,right=1mm,top=1mm,bottom=1mm]
%		\begin{flushleft}
%			#2
%		\end{flushleft}
%	\end{tcolorbox}
%}
%
%\newcommand{\mydefinitiontight}[2]{
%	\begin{tcolorbox}[title=#1,colback=white,colframe=orange!30,coltitle=black,fonttitle=\bfseries,left=0mm,right=0mm,top=0mm,bottom=0mm]
%		\begin{flushleft}
%			#2
%		\end{flushleft}
%	\end{tcolorbox}
%}
%
%\newcommand{\mynote}[2]{
%	\begin{tcolorbox}[title=#1,colback=red!10,colframe=red!30,coltitle=black,fonttitle=\bfseries,left=1mm,right=1mm,top=1mm,bottom=1mm]
%		\begin{flushleft}
%			#2
%		\end{flushleft}
%	\end{tcolorbox}
%}
%
%\newcommand{\myexample}[2]{
%	\begin{tcolorbox}[title=#1,colback=blue!10,colframe=blue!30,coltitle=black,fonttitle=\bfseries,left=1mm,right=1mm,top=1mm,bottom=1mm]
%		\begin{flushleft}
%			#2
%		\end{flushleft}
%	\end{tcolorbox}
%}
%
%\newcommand{\myexampletight}[2]{
%	\begin{tcolorbox}[title=#1,colback=white,colframe=blue!30,coltitle=black,fonttitle=\bfseries,left=0mm,right=0mm,top=0mm,bottom=0mm]
%		\begin{flushleft}
%			#2
%		\end{flushleft}
%	\end{tcolorbox}
%}

\usepackage{forest}
\usepackage{xcolor}
\usetikzlibrary{angles}
\definecolor{drawColor}{RGB}{128 128 128}
\newcommand{\circleSize}{0.25em}
\newcommand{\angleSize}{0.8em}

\forestset{
	/tikz/mandatory/.style={
		circle,fill=drawColor,
		draw=drawColor,
		inner sep=\circleSize
	},
	/tikz/optional/.style={
		circle,
		fill=white,
		draw=drawColor,
		inner sep=\circleSize
	},
	featureDiagram/.style={
		for tree={
			text depth = 0,
			parent anchor = south,
			child anchor = north,
			draw = drawColor,
			edge = {draw=drawColor},
		}
	},
	featureDiagramEmpty/.style={
	},
	/tikz/abstract/.style={
		fill = blue!85!cyan!5,
		draw = drawColor
	},
	/tikz/concrete/.style={
		fill = blue!85!cyan!20,
		draw = drawColor
	},
	mandatory/.style={
		edge label={node [mandatory] {} }
	},
	optional/.style={
		edge label={node [optional] {} }
	},
	or/.style={
		tikz+={
			\path (.parent) coordinate (A) -- (!u.children) coordinate (B) -- (!ul.parent) coordinate (C) pic[fill=drawColor, angle radius=\angleSize]{angle};
		}	
	},
	/tikz/or/.style={
	},
	alternative/.style={
		tikz+={
			\path (.parent) coordinate (A) -- (!u.children) coordinate (B) -- (!ul.parent) coordinate (C) pic[draw=drawColor, angle radius=\angleSize]{angle};
		}	
	},
	/tikz/alternative/.style={
	},
	/tikz/placeholder/.style={
	},
	collapsed/.style={
		rounded corners,
		no edge,
		for tree={
			fill opacity=0,
			draw opacity=0,
			l = 0em,
		}
	},
	/tikz/hiddenNodes/.style={
		midway,
		rounded corners,
		draw=drawColor,
		fill=white,
		minimum size = 1.2em,
		minimum width = 0.8em,
		scale=0.9
	},
}

\newcommand{\legend}{
	\matrix [draw=drawColor,anchor=north west] at (current bounding box.north east) {
		\node [label=center:\underline{Legend:}] {}; \\
		\node [abstract,label=right:Abstract Feature] {}; \\
		\node [concrete,label=right:Concrete Feature] {}; \\
		\node [mandatory,label=right:Mandatory] {}; \\
		\node [optional,label=right:Optional] {}; \\
			\filldraw[drawColor] (0.1,0) - +(-0,-0.2) - +(0.2,-0.2)- +(0.1,0);
			\draw[drawColor] (0.1,0) -- +(-0.2, -0.4);
			\draw[drawColor] (0.1,0) -- +(0.2,-0.4);
			\fill[drawColor] (0,-0.2) arc (240:300:0.2);
		\node [or,label=right:Or Group] {}; \\			\draw[drawColor] (0.1,0) -- +(-0.2, -0.4);
			\draw[drawColor] (0.1,0) -- +(0.2,-0.4);
			\draw[drawColor] (0,-0.2) arc (240:300:0.2);
		\node [alternative,label=right:Alternative Group] {}; \\
	};
}

\newcommand{\featureDiagramWithLegend}[1]{
	\begin{forest}
		featureDiagram[#1]
		\matrix [anchor=north west] at (current bounding box.north east) {
			\node [placeholder] {}; \\
		};
		{\small\legend}
	\end{forest}
}

\newcommand{\featureDiagram}[1]{
	\begin{forest}
		featureDiagram[#1]
	\end{forest}
}

\newcommand{\featureDiagramLegend}{
	\begin{forest}
		featureDiagramEmpty[]
		{\small\legend}
	\end{forest}
}

\newcommand{\featureDiagramDbmgmt}{\featureDiagram{ConfigurableDatabase,abstract[Base,concrete,mandatory][API,concrete,mandatory[Get,concrete,or][Put,concrete][Delete,concrete]][Transactions,concrete,optional][OS,abstract,mandatory[Windows,concrete,alternative][Linux,concrete]]}}