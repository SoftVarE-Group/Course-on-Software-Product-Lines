\documentclass[8pt, aspectratio=169, handout]{beamer} % hyperref={colorlinks=true}

\title{Software Product Lines}

% IMPORTED PACKAGES

% proper encoding and typography
\usepackage[T1]{fontenc}
\usepackage[utf8]{inputenc}
\usepackage[english]{babel}
\usepackage{microtype}

\usepackage{adjustbox} % used for partofpage
\usepackage{tcolorbox} % used for mydefinition, mynote, myexample
\usepackage{multicol} % used temporarily for the lecture overview

% SLIDE TEMPLATE

\beamertemplatenavigationsymbolsempty 

% COMMANDS TO LAYOUT AND ANNIMATE SLIDES

\newcommand{\lessonslearned}[3]{
	\subsection{Summary}
	\begin{frame}{-- \insertsection}
		\leftorright{
			\mydefinition{Lessons Learned}{
				\begin{itemize}
					#1
				\end{itemize}
			}
			\mynote{Further Reading}{
				\small % references take space, can be a little smaller
				\begin{itemize}
					#2
				\end{itemize}
			}
		}{
			\myexample{Practice}{
				#3
			}
		}
	\end{frame}
}

\newcommand{\onlyleft}[1]{
	\halfpage{#1}
}

\newcommand{\onlyright}[1]{
	~\hfill
	\halfpage{#1}
}

\newcommand{\leftorright}[2]{
	\uncover<1>{\halfpage{#1}}
	\hfill
	\uncover<3->{\halfpage{#2}}
}

\newcommand{\rightorleft}[2]{
	\uncover<3->{\halfpage{#1}}
	\hfill
	\uncover<1>{\halfpage{#2}}
}

\newcommand{\leftthenright}[2]{
	\halfpage{#1}
	\hfill\pause
	\halfpage{#2}
}

\newcommand{\leftandright}[2]{
	\halfpage{#1}
	\hfill
	\halfpage{#2}
}

\newcommand{\leftmiddleandright}[3]{
	\thirdpage{#1}
	\hfill
	\thirdpage{#2}
	\hfill
	\thirdpage{#3}
}

\newcommand{\leftmiddleorright}[3]{
	\uncover<1>{\thirdpage{#1}}
	\hfill
	\uncover<3>{\thirdpage{#2}}
	\hfill
	\uncover<5->{\thirdpage{#3}}
}

\newcommand{\halfpage}[1]{\partofpage{48}{#1}}

\newcommand{\thirdpage}[1]{\partofpage{31}{#1}}

\newcommand{\partofpage}[2]{
	\adjustbox{valign=t}{\begin{minipage}{0.#1\textwidth}
			\begin{flushleft}
				#2
			\end{flushleft}
	\end{minipage}}
}

\newcommand{\mydefinition}[2]{
	\begin{tcolorbox}[title=#1,colback=orange!10,colframe=orange!30,coltitle=black,fonttitle=\bfseries,left=1mm,right=1mm,top=1mm,bottom=1mm]
		\begin{flushleft}
			#2
		\end{flushleft}
	\end{tcolorbox}
}

\newcommand{\mydefinitiontight}[2]{
	\begin{tcolorbox}[title=#1,colback=white,colframe=orange!30,coltitle=black,fonttitle=\bfseries,left=0mm,right=0mm,top=0mm,bottom=0mm]
		\begin{flushleft}
			#2
		\end{flushleft}
	\end{tcolorbox}
}

\newcommand{\mynote}[2]{
	\begin{tcolorbox}[title=#1,colback=red!10,colframe=red!30,coltitle=black,fonttitle=\bfseries,left=1mm,right=1mm,top=1mm,bottom=1mm]
		\begin{flushleft}
			#2
		\end{flushleft}
	\end{tcolorbox}
}

\newcommand{\myexample}[2]{
	\begin{tcolorbox}[title=#1,colback=blue!10,colframe=blue!30,coltitle=black,fonttitle=\bfseries,left=1mm,right=1mm,top=1mm,bottom=1mm]
		\begin{flushleft}
			#2
		\end{flushleft}
	\end{tcolorbox}
}

\newcommand{\myexampletight}[2]{
	\begin{tcolorbox}[title=#1,colback=white,colframe=blue!30,coltitle=black,fonttitle=\bfseries,left=0mm,right=0mm,top=0mm,bottom=0mm]
		\begin{flushleft}
			#2
		\end{flushleft}
	\end{tcolorbox}
}

