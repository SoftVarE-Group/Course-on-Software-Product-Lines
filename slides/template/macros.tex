% TYPICAL COMMANDS FOR LECTURES

\newcommand{\deutsch}[1]{{\color{blue}(#1)}}
\newcommand{\deutschertitel}[1]{{\tiny\deutsch{#1}}}

\newcommand{\mycite}[1]{``#1''}
\newcommand{\mycitebegin}{``}
\newcommand{\myciteend}{''}
\newcommand{\mytitlesource}[1]{{\tiny\normalfont\mbox{[#1]}}}
\newcommand{\mysource}[1]{\ifthenelse{\equal{#1}{}}{}{\phantom{.}~\hfill~\mytitlesource{#1}}}
\newcommand{\mypage}[1]{, p.~#1}
\newcommand{\mypages}[1]{, pp.~#1}

\newcommand{\todo}[1]{{\color{red}\textbf{[#1]}}}
\newcommand{\fodo}[1]{\todo{\footnote{\todo{#1}}}}
\newcommand{\todots}{\todo{\ldots}}

\newcommand{\textheightwithtitle}{.85\textheight}
\newcommand{\textheightwithouttitle}{.975\textheight}

% COMMANDS FOR PROPOSITIONAL FORMULAS AND MATHEMATICAL NOTATIONS

\newcommand{\sem}[1]{\ensuremath{\llbracket #1 \rrbracket}} % semantics brackets
\newcommand{\pand}{\wedge} % conjunction
\newcommand{\por}{\vee} % disjunction
\newcommand{\pnot}{\neg} % negation
\newcommand{\pequals}{\leftrightarrow} % biconditional
\newcommand{\npequals}{\nleftrightarrow} % exclusive disjunction
\newcommand{\mequals}{\Leftrightarrow} % equivalence (meta-level)
\newcommand{\pimplies}{\rightarrow} % conditional
\newcommand{\mimplies}{\Rightarrow} % implication (meta-level)
\newcommand{\defeq}{\vcentcolon=} % defining equals
\newcommand{\power}[1]{\mathcal{P}(#1)} % power set
\newcommand{\refslide}[1]{\hyperlink{#1}{(see Slide \autoref{#1})}} % link to slide

\usepackage{mathtools} % required for absolute value in modeling lecture
\DeclarePairedDelimiter\abs{\lvert}{\rvert} % absolute value

% COMMANDS TO INCLUDE XKCDs

\newcommand{\xkcd}[1]{
	\begin{frame}
		\centering%
		\href{https://xkcd.com/#1/}{\includegraphics[height=80mm]{#1}}
	\end{frame}
}
\newcommand{\widexkcd}[1]{
	\begin{frame}
		\centering%
		\href{https://xkcd.com/#1/}{\includegraphics[width=\linewidth]{xkcd/#1}}
	\end{frame}
}
