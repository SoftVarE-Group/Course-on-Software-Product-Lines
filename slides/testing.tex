\documentclass[
	aspectratio=169, % default is 43
	8pt, % font size, default is 11pt
	handout, % handout mode without animations, comment out to add animations
]{beamer}

\usepackage{../template/beamerthemeuulm} % use the inofficial uulm beamer theme
\setfaculty{infIngPsy} % set the color scheme for your faculty here [med/infIngPsy/math/nat]

% requires symbolic links
% git clone git@github.com:SoftVarE-Group/SlideTemplate.git C:\Users\...\SlideTemplate
% mklink /J template C:\Users\...\SlideTemplate
% git clone git@spgit.informatik.uni-ulm.de:thuem/slides.git C:\Users\...\ThomasSlides
% mklink /J thomasslides C:\Users\...\ThomasSlides
\graphicspath{{../template/pics/logos}{../template/pics/nature}{../template/pics/uulm}{../thomasslides/}{../pics/}}

%\usepackage[ngerman]{babel} % use this line for slides in German
%\recordingtrue % special recording mode for use with a greenscreen, gives you space to show yourself in a layer in front of the slides, has no effect in the handout mode

\title{Software Product Lines} % short title is used for the slide footer but optional

%
%
%% IMPORTED PACKAGES
%
%\usepackage{adjustbox} % used for partofpage
%\usepackage{tcolorbox} % used for mydefinition, mynote, myexample
\usepackage{multicol} % used temporarily for the lecture overview
%\usepackage{mathtools} % required for absolute value in modeling lecture
%
%% SLIDE TEMPLATE
%
%\beamertemplatenavigationsymbolsempty 
%
%% COMMANDS TO LAYOUT AND ANNIMATE SLIDES
%
\newcommand{\lessonslearned}[3]{
	\subsection{Summary}
	\begin{frame}{\insertsection -- \insertsubsection}
		\leftorright{
			\mydefinition{Lessons Learned}{
				\begin{itemize}
					#1
				\end{itemize}
			}
			\mynote{Further Reading}{
				\small % references take space, can be a little smaller
				\begin{itemize}
					#2
				\end{itemize}
			}
		}{
			\myexample{Practice}{
				#3
			}
		}
	\end{frame}
}

\renewcommand{\lectureoverview}{
%	\section*{Overview}
%	\subsection*{Overview}
	\begin{frame}{\insertsubtitle}
		\begin{multicols}{2}
			\tableofcontents
		\end{multicols}
	\end{frame}
}

%
%\newcommand{\onlyleft}[1]{
%	\halfpage{#1}
%}
%
%\newcommand{\onlyright}[1]{
%	~\hfill
%	\halfpage{#1}
%}
%
%\newcommand{\leftorright}[2]{
%	\uncover<1>{\halfpage{#1}}
%	\hfill
%	\uncover<3->{\halfpage{#2}}
%}
%
%\newcommand{\rightorleft}[2]{
%	\uncover<3->{\halfpage{#1}}
%	\hfill
%	\uncover<1>{\halfpage{#2}}
%}
%
%\newcommand{\leftthenright}[2]{
%	\halfpage{#1}
%	\hfill\pause
%	\halfpage{#2}
%}
%
%\newcommand{\leftandright}[2]{
%	\halfpage{#1}
%	\hfill
%	\halfpage{#2}
%}
%
%\newcommand{\leftmiddleandright}[3]{
%	\thirdpage{#1}
%	\hfill
%	\thirdpage{#2}
%	\hfill
%	\thirdpage{#3}
%}
%
%\newcommand{\leftmiddleorright}[3]{
%	\uncover<1>{\thirdpage{#1}}
%	\hfill
%	\uncover<3>{\thirdpage{#2}}
%	\hfill
%	\uncover<5->{\thirdpage{#3}}
%}
%
%\newcommand{\halfpage}[1]{\partofpage{48}{#1}}
%
%\newcommand{\thirdpage}[1]{\partofpage{31}{#1}}
%
%\newcommand{\partofpage}[2]{
%	\adjustbox{valign=t}{\begin{minipage}{0.#1\textwidth}
%			\begin{flushleft}
%				#2
%			\end{flushleft}
%	\end{minipage}}
%}
%
%\newcommand{\mydefinition}[2]{
%	\begin{tcolorbox}[title=#1,colback=orange!10,colframe=orange!30,coltitle=black,fonttitle=\bfseries,left=1mm,right=1mm,top=1mm,bottom=1mm]
%		\begin{flushleft}
%			#2
%		\end{flushleft}
%	\end{tcolorbox}
%}
%
%\newcommand{\mydefinitiontight}[2]{
%	\begin{tcolorbox}[title=#1,colback=white,colframe=orange!30,coltitle=black,fonttitle=\bfseries,left=0mm,right=0mm,top=0mm,bottom=0mm]
%		\begin{flushleft}
%			#2
%		\end{flushleft}
%	\end{tcolorbox}
%}
%
%\newcommand{\mynote}[2]{
%	\begin{tcolorbox}[title=#1,colback=red!10,colframe=red!30,coltitle=black,fonttitle=\bfseries,left=1mm,right=1mm,top=1mm,bottom=1mm]
%		\begin{flushleft}
%			#2
%		\end{flushleft}
%	\end{tcolorbox}
%}
%
%\newcommand{\myexample}[2]{
%	\begin{tcolorbox}[title=#1,colback=blue!10,colframe=blue!30,coltitle=black,fonttitle=\bfseries,left=1mm,right=1mm,top=1mm,bottom=1mm]
%		\begin{flushleft}
%			#2
%		\end{flushleft}
%	\end{tcolorbox}
%}
%
%\newcommand{\myexampletight}[2]{
%	\begin{tcolorbox}[title=#1,colback=white,colframe=blue!30,coltitle=black,fonttitle=\bfseries,left=0mm,right=0mm,top=0mm,bottom=0mm]
%		\begin{flushleft}
%			#2
%		\end{flushleft}
%	\end{tcolorbox}
%}

\subtitle{11. Product-Line Testing}
\author{Thomas Thüm}

\begin{document}

% TITLE SLIDE

\maketitle

% SLIDE TEMPLATE

%\setbeamercolor{title}{fg=black}
%\setbeamercolor{frametitle}{fg=black}
\setbeamertemplate{frametitle}{{\huge~\\\insertsubsection~\insertframetitle}}
\setbeamertemplate{footline}[text line]{\parbox{\linewidth}{\vspace*{-10pt}\hspace{0pt}%
	\insertshortauthor\phantom{g\insertpagenumber}%
	\hfill%
	\inserttitle%
	\ifx \insertsubtitle \empty \else \ -- \insertsubtitle\fi%
	\ifx \insertsectionhead \empty \else \ -- \insertsectionhead\fi%
	\hfill%
	\phantom{g\insertshortauthor}\insertpagenumber%
}}
%\defbeamertemplate{footline}{\begin{beamercolorbox}[sep=1em]{author in head/foot}\insertshortauthor\hfill\insertsection\hfill\insertframenumber\end{beamercolorbox}}
%\defbeamertemplate*{footline}{mytheme}{\begin{beamercolorbox}[sep=1em]{author in head/foot}\insertshortauthor\hfill\insertsection\hfill\insertframenumber\end{beamercolorbox}}

% OVERVIEW SLIDES

\newcommand{\overview}{
	\section*{Overview}
	\subsection*{Overview}
	\begin{frame}{-- \insertsubtitle}
		\begin{multicols}{3}
			\tableofcontents
		\end{multicols}
	
		\begin{flushright}
			\footnotesize
			Author: \insertauthor
			
			Date: \insertdate
		\end{flushright}
	\end{frame}
}
% temporarily added slide to have a lecture overview 
\overview

% temporarily removed
%\begin{frame}{Lecture Overview -- \insertsubtitle}
%	\tableofcontents[hideallsubsections]
%\end{frame}

\AtBeginSection[]{%
	\begin{frame}{Lecture Overview -- \insertsubtitle}
		\tableofcontents[currentsection,hideothersubsections]
	\end{frame}
}

\newcommand{\sectionend}{\addtocontents{toc}{\newpage}}


\section{Product-Line Testing in Practice}

\subsection{Motivation}
\begin{frame}
	\leftorright{
		\href{https://commons.wikimedia.org/wiki/File:Andy_Hunt_programmer.jpg}{\includegraphics[width=\linewidth,trim=0 240 0 300,clip]{andy-hunt}}
		\vspace{-7mm}
		
		\mynote{Andy Hunt \mysource{\thepragmaticprogrammer}}{\mycite{No one in the brief history of computing has ever written a piece of perfect software. It's unlikely that you'll be the first.}}
		% co-authored The Pragmatic Programmer, known for the Agile Manifesto
	}{
		\href{https://commons.wikimedia.org/wiki/File:Edsger_Wybe_Dijkstra.jpg}{\includegraphics[width=\linewidth,trim=0 425 0 75,clip]{edsger-dijkstra}}
		\vspace{-7mm}
		
		\mynote{Edsger W. Dijkstra (1972) \mysource{\thehumbleprogrammer}}{\mycite{Program testing can be a very effective way to show the presence of bugs, but it is hopelessly inadequate for showing their absence.}}
		% 1930-2002, ACM Turing Award winner
	}
\end{frame}

\begin{frame}{Recap: Software Quality}
	\vspace{-7mm}
	\leftorright{
		\mydefinition{Quality \mysource{\ludewiglichter}}{Quality is the entirety of properties and characteristics of a product or process that indicate adequacy with respect to given requirements.}
		\mydefinition{Quality Assurance \mysource{\ludewiglichter}}{Quality assurance \deutsch{Qualitätssicherung} are all activities with the goal to improve the quality.}
	}{
	}
\end{frame}

\begin{frame}{Recap: Software Testing}
	\leftandright{
		\vspace{7mm}
		\uncover<1-3>{\mydefinition{Software Testing \mysource{\sommerville}}{\mycite{Testing is intended to show that a program does what it is intended to do and to discover program defects before it is put into use.}}{}}
		\uncover<2-3>{\mydefinition{Validation Testing \mysource{\sommerville}}{\mycite{Demonstrate to the developer and the customer that the software meets its requirements.}}{}}
		\uncover<3-3>{\mydefinition{Defect Testing \mysource{\sommerville}}{\mycite{Find inputs or input sequences where the behavior of the software is incorrect, undesirable, or does not conform to its specification.}}{}}
	}{
		\vspace{-12mm}
		\uncover<4>{\mynote{V\&V \mysource{\seeconomics}}{\mycite{\emph{Validation}: Are we building the right product?\\\emph{Verification}: Are we building the product right?}}} 
		% TODO better visualization of V&V (see Inas slides). move to V model?
		\vspace{-2mm}
		\uncover<5>{\mynote{Stages of Testing \mysource{\sommerville}}{
			\begin{itemize}
				\setlength\itemsep{.1em}
				\item[1.] \mycite{\emph{Development testing}, where the system is tested during development to discover
				bugs and defects}
				\item[2.] \mycite{\emph{Release testing}, where a separate testing team tests a complete version of the
				system before it is released to users}
				\item[3.] \mycite{\emph{User testing}, where users or potential users of a system test the system in their
				own environment}
			\end{itemize}
		}}
		\vspace{-2mm}
		\uncover<6>{\mynote{}{\mycite{In \emph{manual testing}, a tester runs the program with some test data and
				compares the results to their expectations. [...] In \emph{automated testing}, the tests are encoded in a program that is run each time the system under development is to be tested.} \mysource{\sommerville}}}
	}
\end{frame}

% Dijkstra: we cannot test everything, and not all configurations
% complete test: all inputs, for all products/variants/configurations
% sampling technique (Stichprobenverfahren)
% recap: Testing All Configurations (remember numbers from intro? take 1ms per config)
% recap: Testing One Configuration (cp. allyesconfig, feature interactions may not be visible)

\subsection{Expert Knowledge in Sampling}

\subsection{Random Sampling}

\subsection{Excursus: Uniform Random Sampling}

\subsection{Testing the Linux Kernel}
% allyesconfig

%\subsection{Automation in Product Sampling} ???

%\subsection{Missing: Test-Case Selection/Generation}
% What to test for those configurations? Variable unit tests? Avoid redundant testing?

%\subsection{Slide Title 1}
\begin{frame}{~}
	\ldots
\end{frame}

\subsection{Slide Title 2}
\begin{frame}{-- Example Subtitle}
	\ldots
\end{frame}

\subsection{Slide Title 3}
\begin{frame}{~}
	\ldots
\end{frame}



\lessonslearned{
	\item \ldots
}{
	\item \ldots
}{
	\ldots
}

\sectionend

\section{Combinatorial Interaction Testing}

\subsection{Pairwise Interaction Testing}

\subsection{A Greedy Algorithm}

\subsection{Meta-Heuristic Search}

\subsection{T-Wise Interaction Testing}

\subsection{Effectiveness of Combinatorial Interaction Testing}

\subsection{Efficiency of Combinatorial Interaction Testing}
% testing efficiency and sampling efficiency

%\subsection{Slide Title 1}
\begin{frame}{~}
	\ldots
\end{frame}

\subsection{Slide Title 2}
\begin{frame}{-- Example Subtitle}
	\ldots
\end{frame}

\subsection{Slide Title 3}
\begin{frame}{~}
	\ldots
\end{frame}



\lessonslearned{
	\item \ldots
}{
	\item \ldots
}{
	\ldots
}

\sectionend

\section{Solution-Space Sampling}

\subsection{Coverage in Single-System Engineering}

\subsection{Coverage of Ifdef Blocks}

\subsection{Presence-Condition Coverage}

%\subsection{Encoding Solution Space in Feature Models}

\subsection{Overview on Coverage Criteria}

\subsection{Overview on Input for Sampling Algorithms}
% include test cases as input

%\subsection{Slide Title 1}
\begin{frame}{~}
	\ldots
\end{frame}

\subsection{Slide Title 2}
\begin{frame}{-- Example Subtitle}
	\ldots
\end{frame}

\subsection{Slide Title 3}
\begin{frame}{~}
	\ldots
\end{frame}



\lessonslearned{
	\item \ldots
}{
	\item \ldots
}{
	\ldots
}

\mode<beamer>{
	\begin{frame}{\inserttitle}
		\lectureseriesoverview
	\end{frame}

	\contentoverview
}


%\begin{frame}{Example}
%	\includegraphics[width=\linewidth,page=3,trim=0 40 0 40,clip]{2D Analysis/2019-10-17 SPP1593 2Danalysis.pdf}
%\end{frame}

\end{document}
